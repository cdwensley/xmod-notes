%% xpair.tex,  version 01/05/17


%%%%%%%%%%%%%%%%%%%%%%%%%%%%%%%%%%%%%%%%%%%%%%%%%%%%%%%%%%%%%%%%%%%%%%%%%%%
\section{Crossed Pairings and Nonabelian Tensor Products} \label{sect:natp}

The \emph{nonabelian tensor product} 
\index{tensor product} \index{nonabelian tensor product}
was introduced by Brown and Loday in \cite{brow:lod}
and developed in Brown, Johnson, Robertson \cite{brow:jo:ro}.

When  $G,H$  are both abelian,  $\ugth$  is the usual tensor product.

Many computations of the \emph{nonabelain tensor square} 
\index{tensor square} 
$\ugtg$  of a group  $G$  have been made.
Here is a small sample of known results:

\begin{center}
\begin{tabular}{|l|c|c|}
\hline
symmetric group        &  $S_3$              &  $C_6$              \\
alternating group      &  $A_4$              &  $Q_8 \times C_3$   \\
dihedral groups        &  $D_{2n}$, $n$ odd  &  $\mathbf{Z}_{2n}$  \\
Heisenberg group       &  $\calH$            &  ${\mathbf Z}^6$    \\
\hline
\end{tabular}
\end{center}

The nonabelian tensor product is a special case of a crossed pairing.

%%%%%%%%%%%%%%%%%%%%%%%%%%%%%%%%%%%%%
\subsection{Compatible Group Actions}

\begin{defn}
Let  $G$  and  $H$  be groups which act on themselves by conjugation, 
and also act on each other.  
These four actions are said to be \emph{compatible} if
$$
{g_1}^{(h^g)} \;=\; (({g_1}^{g^{-1}})^h)^g~, \qquad
{h_1}^{(g^h)} \;=\; (({h_1}^{h^{-1}})^g)^h~.
$$
\end{defn}

\begin{example}
\emph{If $G,H$ are normal subgroups of a group $\Gamma$, then each acts on the 
other by conjugation and the actions are compatible.}
\end{example}

\begin{example} \label{ex:xmod-actions}
\emph{Let $\calX = (\partial : S \to R)$ be a crossed module.
If $r^s$ is defined to be $(\partial s^{-1})r(\partial s)$ 
then both $R$ and $S$ act on each other and on themselves.
Compatibility is easily checked:}
\begin{itemize}
\item~
${s_1}^{(r^s)} \;=\; {s_1}^{(\partial s^{-1})r(\partial s)}
\;=\;  {s_1}^{s^{-1}r(\partial s)}
\;=\;  {s_1}^{s^{-1}rs}$\,,
\item~
${r_1}^{(s^r)} \;=\; (\partial {s^r})^{-1} r_1 (\partial s^r) 
\;=\; (r^{-1}(\partial s)r)^{-1} r_1 (r^{-1}(\partial s)r)
\;=\; (({r_1}^{r^{-1}})^s)^r$\,.
\end{itemize}
\end{example}


\bigskip
%%%%%%%%%%%%%%%%%%%%%%%%%%%%%%%%%%%%%%%%%%%%%%%%%%%%%%%%%%%%%%%%%%%%%%%%%%
\subsection{Crossed Pairings} \label{subs:xp}
\index{crossed pairing}

There are two standard definitions of a \emph{crossed pairing}.
Here is the one which we shall use.\\
(There is a more general definition when the two actions are
not compatible.)

\begin{defn} \label{def:xpair} 
\index{compatible action} \index{action!compatible}
Let  $G,H$  be groups which act compatibly on each other 
and on a group  $L$.\\
A map  $\bt : \ugxh \to L,\; (g,h) \mapsto g \bt h$,
is a crossed pairing if
\begin{enumerate}[{\rm (a)}]
\item\quad
$(g_1g_2 \bt h) \;=\; (g_1 \bt h)^{g_2}\;(g_2 \bt h)$~,
\item\quad
$(g \bt h_1h_2) \;=\; (g \bt h_2)\;(g \bt h_1)^{h_2}$~,
\item\quad
$(g \bt h)^x  \;=\; g^x \bt h^x \quad \mbox{for all} 
   \;\; x \in G \cup H$~. 
\end{enumerate}
\end{defn}

The alternative definition does not require actions on  $L$
and omits axiom (c).
It is then observed that   
$\im \bt \leq L$  inherits  $G$-  and  $H$-actions given by (c).

\begin{example}
If $N \unlhd G$ then a crossed pairing is provided by commutators:
\begin{eqnarray*}
\mbox{\emph{(a)}}~~ \bt ~:~ G \times N \to N,
 && g \bt n = [g,n] = (n^{-1})^g \; n, \\
\mbox{\emph{(b)}}~~ \bt ~:~ N \times G \to N,
 && n \bt g = [n,g] = n^{-1} \; n^g. \\
\end{eqnarray*}
\end{example}

Here are some standard properties of crossed pairings
(see Proposition 3 of \cite{brow:jo:ro}).

\begin{prop} \label{prop:xpair}
The following relations hold for all  $g,g_1,g_2 \in G$
and for all  $h,h_1,h_2 \in H$~.
\begin{enumerate}[{\rm (a)}]
\item\quad
$(g \bt 1_H)
  \;=\; (1_G \bt h)
  \;=\; 1_L$~;
\item\quad
$(g \bt h)^{-1}
  \;=\; (g^{-1} \bt h)^g
  \;=\; (g \bt h^{-1})^h
  \;=\; (g^{-1} \bt h^g)
  \;=\; (g^h \bt h^{-1})$~;
\item\quad
$(g_1 \bt h_1)^{h_2g_2}\,(g_2 \bt h_2)
  \;=\; (g_2 \bt h_2)\,(g_1 \bt h_1)^{g_2h_2}$~;
\item\quad
$(g^h \bt h_1)
   \;=\; (g \bt h)^{-1}\,(g \bt h_1)\,(g \bt h)^{h_1}
 \quad\mbox{and}\quad
 (g_1 \bt h^g)
   \;=\; (g \bt h)^{g_1}\,(g_1 \bt h)\,(g \bt h)^{-1}$~;
\item\quad
$(g_1g_2 \bt h)
  \;=\; (g_2 \bt h^{g_1})(g_1 \bt h)
 \quad\mbox{and}\quad
 (g \bt h_1h_2)
  \;=\; (g \bt h_1)(g^{h_1} \bt h_2)$~;
\item\quad
$(g \bt h)^{[g_2,h_2]}
  \;=\; (g_2 \bt h_2)^{-1}\,(g \bt h)\,(g_2 \bt h_2)$~;
\item\quad
$(g^{-1}\,g^h \bt h_1)
  \;=\; (g \bt h)^{-1}\,(g \bt h)^{h_1}
 \quad\mbox{and}\quad
 (g_1 \bt (h^{-1})^g\,h)
  \;=\; ((g \bt h)^{-1})^{g_1}\,(g \bt h)$~;
\item\quad
$[(g_1 \bt h_1),\,(g_2 \bt h_2)]
  \;=\; ((g_1^{-1}\,{g_1}^{h_1}) \bt ((h_2^{-1})^{g_2}\,h_2))$~.
\end{enumerate}
\end{prop}
\begin{pf}
Where there are two formulae, the proof of the second
mirrors that of the first.
\begin{enumerate}[(a)]
\item\quad
$\gbh \;=\; g1 \bt h \;=\; (\gbh)^1\,(1 \bt h) $
\item\quad
$1\;=\; 1 \bt h \;=\; g^{-1}g \bt h \;=\; (g^{-1} \bt h)^g \,(\gbh)$
\item\quad\vspace{-7mm}
\begin{eqnarray*}
g_1g_2 \bt h_1h_2
  & = &  (g_1g_2 \bt h_2)\,(g_1g_2 \bt h_1)^{h_2} \\
  & = &  (g_1 \bt h_2)^{g_2}\,(\gbht)\,
          (\gbho)^{g_2h_2}\,(g_2 \bt h_1)^{h_2} \\
\mbox{and} \quad
g_1g_2 \bt h_1h_2
  & = &  (g_1 \bt h_1h_2)^{g_2}\,(g_2 \bt h_1h_2) \\
  & = &  (g_1 \bt h_2)^{g_2}\,(\gbho)^{h_2g_2}\,
          (\gbht)\,(g_2 \bt h_1)^{h_2} \\
\Rightarrow \quad
   (\gbho)^{h_2g_2}\,(\gbht) 
  & = &  (\gbht)\,(\gbho)^{g_2h_2}
\end{eqnarray*}
\item\quad\vspace{-7mm}
\begin{eqnarray*}
g^h \bt h_1
  & = &  (g \bt {h_1}^{h^{-1}})^h  \;=\;  (g \bt hh_1h^{-1})^h \\
  & = &  (g \bt h^{-1})^h\,(g \bt h_1)\,(\gbh)^{h_1} \\
  & = &  (\gbh)^{-1}\,(g \bt h_1)\,(\gbh)^{h_1} 
\end{eqnarray*}
\item\quad
These alternative forms for \ref{def:xpair} (a),(b)
follow immediately from (d).
\item\quad
Substitute  $\,g = {g_1}^{h_2g_2},\; h = {h_1}^{h_2g_2}\,$  in (c).
\item\quad\vspace{-7mm}
\begin{eqnarray*}
(g^{-1}g^h) \bt h_1
  & = &  (g^{-1} \bt h_1)^{g^h}\,(g^h \bt h_1)
  \;=\;  (g^{-1} \bt h_1^g)^{[g,h]}\,(g^h \bt h_1) \\
  & = &  (\gbh)^{-1}\,(g \bt h_1)^{-1}\,(\gbh)
          .(\gbh)^{-1}\,(g \bt h_1)\,(\gbh)^{h_1}
           \quad \mbox{by (b),(d),(e)} \\
  & = &  (\gbh)^{-1}\,(\gbh)^{h_1}
\end{eqnarray*}
\item\quad\vspace{-7mm}
\begin{eqnarray*}
[\gbho,\gbht]
  & = &  (\gbho)^{-1}\,(\gbht)^{-1}\,(\gbho)\,(\gbht) \\
  & = &  (\gbho)^{-1}\,(\gbho)^{[g_2,h_2]}
  \;=\;  (\gbho)^{-1}\,(\gbho)^{(h_2^{-1})^{g_2}h_2}
         \quad \mbox{by (e)} \\
  & = &  ((g_1^{-1}\,{g_1}^{h_1}) \bt ((h_2^{-1})^{g_2}\,h_2)) 
         \quad \mbox{by (f)}.
\end{eqnarray*}
\end{enumerate}
\end{pf}

\begin{lem}
\index{principal crossed pairing} \index{crossed pairing!principal}
The \emph{principal crossed pairing} of a crossed module
$\calX = (\partial : S \to R)$ is given by
$$
\bt : R \times S \to S,~ (r,s) \mapsto \eta_s(r) = (s^{-1})^rs\ .
$$
\end{lem}
\begin{pf}
We have seen in Example \ref{ex:xmod-actions}
that $R$ and $S$ have compatible actions.
The three axioms are easily checked.
\begin{enumerate}[(a)]
\item~
$$
(r_1 \bt s)^{r_2}(r_2 \bt s)
~=~ ((s^{-1})^{r_1}s)^{r_2}((s^{-1})^{r_2}s)
~=~ (s^{-1})^{r_1r_2}s
~=~ r_1r_2 \bt s
$$
\item~
$$
(r \bt s_1)(r \bt s_1)^{s_2}
~=~ ((s_2^{-1})^r s_2)(s_2^{-1}(s_1^{-1})^rs_1s_2)
~=~ ((s_1s_2)^{-1})^r (s_1s_2)
~=~ r \bt s_1s_2
$$
\item~
$$
r^{r_0} \bt s^{r_0}
~=~ ((s^{r_0})^{-1})^{r^{r_0}}s^{r_0}
~=~ (s^{-1})^{rr_0}s^{r_0}
~=~(r \bt s)^{r_0}
$$
$$
r^{\partial s_0} \bt s^{s_0}
~=~ ((s^{s_0})^{-1})^{s_0^{-1}rs_0} s^{s_0}
~=~ ((s^{-1})^r s)^{s_0}
~=~ (r \bt s)^{s_0}
$$
\end{enumerate}
\end{pf} 

\noindent
So we may write principal derivations as $\eta_s r = r \bt s$ 
and principal sections as $\kappa_s r = (r, r \bt s)$.

\medskip
A standard result concerning crossed pairings 
shows that the nonabelian tensor product is the universal
object for this construction.

\begin{defn}
Given groups  $G$  and  $H$  which act compatibly on each other,
the \emph{nonabelian tensor product}  $\ugth$ 
of  $G$  and  $H$  has generating set
$\{ \gth \;\mid\; g \in G,\, h \in H\}$
subject to relations
\begin{equation} \label{eq:tensrels}
(g_1g_2 \ot h) \;=\; (g_1 \ot h)^{g_2}\,(g_2 \ot h)~, \quad
(g \ot h_1h_2) \;=\; (g \ot h_2)\,(g \ot h_1)^{h_2}~, 
\end{equation}
where  
$$
(\gth)^x \;=\; (g^x \ot h^x)
\quad \mbox{for all} \; x \in G \cup H~.
$$\end{defn}


\begin{thm} \label{thm:uni-prop}
The nonabelian tensor product function
$$
\ot \;:\; \ugxh \to \ugth, \;\; (g,h) \mapsto \gth
$$
is a crossed pairing.
Moreover, given any crossed pairing  $\bt : \ugxh \to L$,
there is a unique homomorphism  $\bt_{\ot} : \ugth \to L$
satisfying  $\bt = \bt_{\ot} \circ \ot$
so that the following diagram commutes:
$$
\xymatrix{ 
   \ugxh \ar[dd]_{\ot} \ar[ddrr]^{\bt}
         &&   \\
     &&  \\
   \ugth \ar[rr]_{\bt_{\ot}} 
         &&  L ~.
}
$$
\end{thm}

Checking that a potential crossed pairing satisfies the axioms
of Definition \ref{def:xpair}  can be a tedious process.  
However we can convert this into checking that
maps to certain semidirect products are homomorphisms.

\begin{lem} \label{lem:xpair-to-homs}
Let  $\bt : \ugxh \to L$  be a crossed pairing.  Then
\begin{enumerate}[{\rm (a)}]
\item
given a fixed element   $h \in H$,  the map
$$
\theta_h \;:\; G \to G \ltimes L, \;\; g \mapsto (g,\,g \bt h)
$$
is a group homomorphism;
\item
given a fixed element   $g \in G$,  the map
$$
\theta_g \;:\; H \to H \ltimes L, \;\; h \mapsto (h,(g \bt h)^{-1})
$$
is a group homomorphism.
\end{enumerate}
\end{lem}
\begin{pf}
$$
\theta_h(g_1g_2)
  \;=\;  (g_1g_2,\,(g_1 \bt h)^{g_2}\;(g_2 \bt h)) 
  \;=\;  (g_1,\,g_1 \bt h)\;(g_2,\,g_2 \bt h)    
  \;=\;  (\theta_h g_1)\;(\theta_h g_2)~.
$$
$$
\theta_g(h_1h_2)
  \;=\;  (h_1h_2,\;\{(g \bt h_2)\;((g \bt h_1))^{h_2}\}^{-1}) 
  \;=\;  (h_1,\,(g \bt h_1)^{-1})\;(h_2,\;(g \bt h_2)^{-1})    
  \;=\;  (\theta_g h_1)\;(\theta_g h_2)~.
$$
\end{pf}

The converse proposition gives a way of checking
that a given map is a crossed pairing.

\begin{prop} \label{prop:homs-to-xpair}
Given a map  $\odot : \ugxh \to L$
and an action  $(\goh)^x = g^x \odot h^x$  for all  $x \in G \cup H$,
such that for all  $h \in H$ and $g \in G$
\begin{itemize}
\item
$\theta_h : G \to G \ltimes L,\; g \mapsto (g,\goh)$
is a homomorphism, and
\item
$\theta_g : H \to H \ltimes L,\; h \mapsto (h, (\goh)^{-1})$
is a homomorphism,
\end{itemize}
then  $\odot$  is a crossed pairing.
\end{prop}
\begin{pf}
\begin{eqnarray*}
(g_1g_2,\, g_1g_2 \odot h)
  & = &  \theta_h(g_1g_2)
  \;=\;  (\theta_h g_1)(\theta_h g_2)  \\
  & = &  (g_1,\, g_1 \odot h)(g_2,\, g_2 \odot h) \\
  & = &  (g_1g_2,\, (g_1 \odot h)^{g_2}\,(g_2 \odot h)) ~.
\end{eqnarray*}
A similar argument shows that (\ref{def:xpair})(b)  is also satisfied,
so  $\odot$  is a crossed pairing.
\end{pf}

