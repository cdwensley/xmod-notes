%% xpair-dev.tex,  version 22/04/17
\noindent
{\bf [This is now only in the development version.]}

\bigskip

%%%%%%%%%%%%%%%%%%%%%%%%%%%%%%%%%%%%%%%%%%%%%%%%%%%%%%%%%%%%%%%%%%%%%%%%
\subsection{Interchange Laws} \label{subs:intlaw}

The crossed pairing rules of Definition \ref{def:xpair}:
$$
(g_1g_2 \bt h) \;=\; 
(g_1 \bt h)^{g_2} (g_2 \bt h)~,
\qquad
(g \bt h_1h_2) \;=\;
(g \bt h_2)   (g \bt h_1)^{h_2}~,
$$
generalise to
\begin{equation}\label{eq:Ggen}
\qquad\quad
(g_1 \ldots g_{k}\bt h) \quad=\quad 
 (g_1 \bt h)^{g_2 \ldots g_{k}}   
  (g_2 \bt h)^{g_{3} \ldots g_{k}} \ldots 
   (g_{k-1}\bt h)^{g_{k}}   (g_{k}\bt h)~,
\end{equation}
\begin{equation}\label{eq:Hgen}
\mbox{and}\quad
(g \bt h_1 \ldots h_{l}) \quad=\quad (g \bt h_{l})  
 (g \bt h_{l-1})^{h_{l}} \ldots (g \bt h_2)^{h_{3} \ldots h_{l}}
  (g \bt h_1)^{h_2 \ldots h_{l}}~.
\end{equation}

\medskip
The element  $(g_1 g_2 \ldots g_k \bt h_1 h_2 \ldots h_{l})$
can be expanded in many ways, but by its $GH$-\emph{expansion}
we mean applying (\ref{eq:Ggen}) first,
and then applying (\ref{eq:Hgen}) to each of the  $k$  terms which result.
Similarly the $HG$-\emph{expansion} is obtained by
applying (\ref{eq:Hgen}) first, and then (\ref{eq:Ggen}) to the $l$ terms.

In Proposition \ref{prop:xpair}(c) we equated the two expansions of  
$(g_1g_2 \bt h_1h_2)$  
to obtain the basic \emph{interchange law}
\begin{equation}
(g_1 \bt h_1)^{h_2g_2}(g_2 \bt h_2)
\quad=\quad 
(g_2 \bt h_2)(g_1 \bt h_1)^{g_2h_2}~.
\end{equation}

We now consider the cases  $k=2, l=3$  and  $k=l=3$.

\vspace{2mm}\noindent
The $GH$-expansion of $(g_1g_2 \bt h_1h_2h_{3})$ is:
\begin{eqnarray*}
         g_1g_2 \bt h_1h_2h_{3} 
  & = &  (g_1 \bt h_1h_2h_{3})^{g_2}   
          (g_2 \bt h_1h_2h_{3}) \\
  & = &  (g_1 \bt h_{3})^{g_2}   (g_1 \bt h_2)^{h_{3}g_2} 
          (g_1 \bt h_1)^{h_2h_{3}g_2} (g_2 \bt h_{3}) (g_2 \bt h_2)^{h_{3}} 
           (g_2 \bt h_1)^{h_2h_{3}}
\end{eqnarray*}
The $HG$-expansion of  $(g_1g_2 \bt h_1h_2h_{3})$ is:
\begin{eqnarray*}
g_1g_2 \bt h_1h_2h_{3}  
  & = &  (g_1g_2 \bt h_{3})   (g_1g_2 \bt h_2)^{h_{3}} 
          (g_1g_2 \bt h_1)^{h_2h_{3}} \\  
  & = &  (g_1 \bt h_{3})^{g_2} (g_2 \bt h_{3}) (g_1 \bt h_2)^{g_2h_{3}}
           (g_2 \bt h_2)^{h_{3}} (g_1 \bt h_1)^{g_2h_2h_{3}} 
             (g_2 \bt h_1)^{h_2h_{3}}
\end{eqnarray*}

By applying a series of interchanges to the second expansion of
$(g_1g_2 \bt h_1h_2h_{3})$, 
moving each term involving  $g_2$  to the right, 
it is clear that these two expansions are equal:
\begin{eqnarray*}
 &   & (g_1 \bt h_{3})^{g_2}   \underbrace{(g_2 \bt h_{3})  
       (g_1 \bt h_2)^{g_2h_{3}}}   \underbrace{(g_2 \bt h_2)^{h_{3}} 
        (g_1 \bt h_1)^{g_2h_2h_{3}}} 
         (g_2 \bt h_1)^{h_2h_{3}} \\
 & = & (g_1 \bt h_{3})^{g_2}   (g_1 \bt h_2)^{h_{3}g_2} 
       \underbrace{(g_2 \bt h_{3})    (g_1 \bt h_1)^{h_2g_2h_{3}}} 
        (g_2 \bt h_2)^{h_{3}}   (g_2 \bt h_1)^{h_2h_{3}} \\
 & = & (g_1 \bt h_{3})^{g_2}    (g_1 \bt h_2)^{h_{3}g_2}  
       (g_1 \bt h_1)^{h_2h_{3}g_2}   (g_2 \bt h_{3}) 
        (g_2 \bt h_2)^{h_{3}}   (g_2 \bt h_1)^{h_2h_{3}}~.
\end{eqnarray*}


Similarly, by applying a series of interchanges to
the  $HG$-expansion of  $(g_1g_2g_{3}\bt h_1h_2h_{3})$,
moving terms involving  $g_3$  to the right, 
we obtain: 

\begin{eqnarray*}
  &   &  (g_1 \bt h_{3})^{g_2g_{3}}   (g_2 \bt h_{3})^{g_{3}} 
          \underbrace{(g_{3}\bt h_{3})   (g_1 \bt h_2)^{g_2g_{3}h_{3}}} 
           (g_2 \bt h_2)^{g_{3}h_{3}} \\
  &   &  \qquad
         \underbrace{(g_{3}\bt h_2)^{h_{3}}  
          (g_1 \bt h_1)^{g_2g_{3}h_2h_{3}}}  
           (g_2 \bt h_1)^{g_{3}h_2h_{3}}   
            (g_{3}\bt h_1)^{h_2h_{3}} \\
  & = &  (g_1 \bt h_{3})^{g_2g_{3}}   (g_2 \bt h_{3})^{g_{3}} 
          (g_1 \bt h_2)^{g_2h_{3}g_{3}}   \underbrace{(g_{3}\bt h_{3}) 
           (g_2 \bt h_2)^{g_{3}h_{3}}} \\
  &   &  \qquad
         (g_1 \bt h_1)^{g_2h_2g_{3}h_{3}}  
          \underbrace{(g_{3}\bt h_2)^{h_{3}}  
           (g_2 \bt h_1)^{g_{3}h_2h_{3}}}  
            (g_{3}\bt h_1)^{h_2h_{3}} \\
  & = &  (g_1 \bt h_{3})^{g_2g_{3}}   (g_2 \bt h_{3})^{g_{3}}  
          (g_1 \bt h_2)^{g_2h_{3}g_{3}} 
           (g_2 \bt h_2)^{h_{3}g_{3}} \\
  &   &  \qquad
         \underbrace{(g_{3}\bt h_{3})  
          (g_1 \bt h_1)^{g_2h_2g_{3}h_{3}}} 
           (g_2 \bt h_1)^{h_2g_{3}h_{3}}   (g_{3}\bt h_2)^{h_{3}} 
            (g_{3}\bt h_1)^{h_2h_{3}} \\
  & = &  (g_1 \bt h_{3})^{g_2g_{3}}   (g_2 \bt h_{3})^{g_{3}}  
          (g_1 \bt h_2)^{g_2h_{3}g_{3}}  
           (g_2 \bt h_2)^{h_{3}g_{3}} \\
  &   &  \qquad
         (g_1 \bt h_1)^{g_2h_2h_{3}g_{3}}  
          \underbrace{(g_{3}\bt h_{3})   (g_2 \bt h_1)^{h_2g_{3}h_{3}}} 
           (g_{3}\bt h_2)^{h_{3}}   (g_{3}\bt h_1)^{h_2h_{3}} \\
  & = &  (g_1 \bt h_{3})^{g_2g_{3}}   (g_2 \bt h_{3})^{g_{3}} 
          (g_1 \bt h_2)^{g_2h_{3}g_{3}} 
           (g_2 \bt h_2)^{h_{3}g_{3}} \\
  &   &  \qquad
         (g_1 \bt h_1)^{g_2h_2h_{3}g_{3}} 
          (g_2 \bt h_1)^{h_2h_{3}g_{3}}   (g_{3}\bt h_{3})  
           (g_{3}\bt h_2)^{h_{3}}   (g_{3}\bt h_1)^{h_2h_{3}} \\
  & = &  \Big(HG-\mbox{expansion of} \quad(g_1g_2
          \bt h_1h_2h_{3})\Big)^{g_{3}}\Big(g_{3}\bt h_1h_2h_{3}\Big) \\
  & = &  \Big(GH-\mbox{expansion of} \quad(g_1g_2
          \bt h_1h_2h_{3})\Big)^{g_{3}}\Big(g_{3}\bt h_1h_2h_{3}\Big) \\
  & = &  \Big(GH-\mbox{expansion of} 
          \quad(g_1g_2g_{3}\bt h_1h_2h_{3})\Big) ~.
\end{eqnarray*}

\noindent
These two examples indicate how to develop an inductive argument
to prove the following result.

\begin{prop}\label{prop:GHeqHG}
The $GH$-expansion and $HG$-expansion of
$(g_1 g_2 \ldots g_k \bt h_1 h_2 \ldots h_{l})$
are equal.
\end{prop}
\begin{pf}
Assume the result to be true for
$(g_1 g_2 \ldots g_{k-1} \bt h_1 h_2 \ldots h_{l})$~.\\
The first $l-1$ terms involving  $g_k$  in the $HG$-expansion
can each be interchanged with the term on the right:
$$
\{(g_{k}\bt h_{j}) (g_1 \bt h_{j+1})^{g_2 \ldots g_{k}h_{j}}\}
  ^{h_{j+1} \ldots h_{l}}
\quad \longrightarrow \quad
\{(g_1 \bt h_{j+1})^{g_2 \ldots g_{k-1}h_{j}g_{k}} (g_{k}\bt h_{j})\}
  ^{h_{j+1} \ldots h_{l}}~,
$$
and then with the next term on the right:
$$
\{(g_{k}\bt h_{j}) (g_2 \bt h_{j+1})^{g_3 \ldots g_{k}h_{j}}\}
  ^{h_{j+1} \ldots h_{l}}
\quad \longrightarrow \quad
\{(g_2 \bt h_{j+1})^{g_3 \ldots g_{k-1}h_{j}g_{k}} (g_{k}\bt h_{j})\}
  ^{h_{j+1} \ldots h_{l}}~,
$$
and so on.
After $k-1$ sets of interchanges the last two terms in the expansion are
$(g_k \bt h_2)^{h_3 \ldots h_{l}}\,(g_k \bt h_1)^{h_2 \ldots h_{l}}\,$,
and there remain  $l-2$  terms involving  $g_k$  
still to be moved to the right.
After a total of  $\frac{1}{2}(k-1)(l-1)l$  interchanges, we obtain
\begin{eqnarray*}
   &   &  \Big(HG-\mbox{expansion of} 
            \quad(g_1 \ldots g_{k}\bt h_1 \ldots h_{l})\Big) \\
   & = &  \Big(HG-\mbox{expansion of} 
            \quad(g_1 \ldots g_{k-1}\bt h_1 \ldots 
           h_{l})\Big)^{g_{k}}\Big(g_{k}\bt h_1 \ldots h_{l}\Big) \\
   & = &  \Big(GH-\mbox{expansion of} 
            \quad(g_1 \ldots g_{k-1}\bt h_1 \ldots 
           h_{l})\Big)^{g_{k}}\Big(g_{k}\bt h_1 \ldots h_{l}\Big) 
             \qquad \mbox{(by induction)} \\
   & = &  \Big(GH-\mbox{expansion of} 
            \quad(g_1 \ldots g_{k}\bt h_1 \ldots h_{l}\Big) ~.
\end{eqnarray*}
A similar argument can be used when starting with the $GH$-expansion.
\end{pf}



%%%%%%%%%%%%%%%%%%%%%%%%%%%%%%%%%%%%%%%%%%%%%%%%%%%%%%%%%%%%%%%%%%%%%%%%
\subsection{An alternative construction for $\ugth$}

This describes work of Rodrigues and Wensley.

\vspace{2mm}\noindent
Let  $G$  and  $H$  be groups which act on themselves and each other
in a compatible way.

\vspace{2mm}
\noindent
A presentation  $<C_G \,|\, R_G>$  for  $G$  is constructed as follows.
\begin{itemize}
\item
A generating set  $C_G$  for  $G$  is chosen, which is closed 
under the actions of both  $G$  and  $H$.
\item
A list of elements  $E_G$  for  $G$  is initialised
as  $[1_G]$  and  $R_G$  is initialised as an empty list.
\item
A fixed word in the generators for each element of  $G$
is chosen by the following iteration:
\begin{itemize}
 \item
 take the next element  $g$  from  $E_G$~;
 \item
 for each  $c \in C_G$  let  $g' = gc$~;
 \item
 if  $g' \notin E_G$  then add  $g'$  to  $E_G$,
 else add  $gc = g'$  to  $R_G$~;
\end{itemize}
\end{itemize}

A similar presentation  $<C_H\,|\,R_H>$  for  $H$  is also constructed.
We use the notation 
$$
C_G = \{c_1,c_2,\ldots\}, \quad
E_G = \{g_1,g_2,\ldots\}, \quad
C_H = \{d_1,d_2,\ldots\}, \quad
E_H = \{h_1,h_2,\ldots\}~.
$$

Define  $\ugoh = <C_{(\ugoh)}\,|\,R_{(\ugoh)}>$  
to be the group where 
\begin{itemize}
\item
\quad$C_{(\ugoh)}$  is the set of symbols  
$\{c \odot d \,|\, c \in C_G,\, d \in C_H\}$ \\
with actions  $(c \odot d)^x \;=\; (c^x \odot h^x)$
for all  $x \in G \cup H$,
\item
\quad$R_{(\ugoh)}$  is the relator set consisting of:
\begin{enumerate}[(1)]
\item
for each  $gc = g' \in R_G$  where
$g = c_1c_2 \ldots c_k,~ g' = c'_1c'_2 \ldots c'_{k'}$,
and for each  $d_j \in C_H$,
$$
(c_1 \odot d_j)^{c_2\ldots c_kc}\,\ldots\,
 (c_k \odot d_j)^c\,(c \odot d_j)
\;\;=\;\;
(c'_1 \odot d_j)^{c'_2\ldots c'_{k'}}\,\ldots\,(c'_{k'} \odot d_j)~;
$$
\item
for each  $hd = h' \in R_H$  where
$h = d_1d_2 \ldots d_k,~ h' = d'_1d'_2 \ldots d'_{k'}$,
and for each  $c_i \in C_G$,
$$
 (c_i \odot d)\,(c_i \odot d_k)^d\,\ldots\,(c_i \odot d_1)^{d_2\ldots d_kd}
\;\;=\;\;
 (c'_i \odot d'_{k'})\,\ldots\,(c'_i \odot d'_1)^{d'_2\ldots d'_{k'}}~;
$$
\item 
for each  $c_i,c_{i'} \in C_G$  and for each  $d_j,d_{j'} \in C_H$  
the interchange laws
$$
(c_{i'} \odot d_{j'})^{d_j c_i}\;(c_i \odot d_j)
\quad=\quad 
(c_i \odot d_j)\;(c_{i'} \odot d_{j'})^{c_i d_j}~.
$$
\noindent
\end{enumerate}
(\textbf{Note:} we hope to restrict to a subset of these in due course.)
\end{itemize}

\noindent
Define a map, which we shall show to be a crossed pairing, 
$\odot : \ugxh \to \ugoh,\; (g,h) \mapsto g \odot h$~,
extending the obvious map  $C_G \times C_H \to C_{(\ugoh)}$  
to the whole of  $\ugxh$  by 
\begin{enumerate}[{\rm (a)}]
\item\quad
if  $g_i \,=\, c_1c_2\ldots c_k$  ~then~
$g_i \odot d_j \,=\, 
 (c_1 \odot d_j)^{c_2\ldots c_k}\,\ldots\,
 (c_{k-1} \odot d_j)^{c_k}\,(c_k \odot d_j)$~,
\item\quad
if  $h_j \,=\, d_1d_2\ldots d_{l}$  ~then~
$c_i \odot h_j \,=\,
 (c_i \odot d_{l})\,(c_i \odot d_{l-1})^{d_{l}}\,\ldots\,
 (c_i \odot d_1)^{d_2\ldots d_{l}}$~,
\item\quad
$g_k \odot h_{l} \,=\, GH$-expansion of
$c_1c_2\ldots c_k \odot d_1d_2\ldots d_{l}$~.
\end{enumerate}

Recall the \emph{Substitution Test} (see Johnson, p.29),
a corollary of von Dyck's Theorem,
which we may use to verify that a given map is a homomorphism 
of finitely presented groups.

\begin{thm}
Suppose we are given a presentation  $\Gamma = <C\,|\,R>$,
a group  $\Delta$  and a mapping  $\theta' : C \to \Delta$~.
Then  $\theta'$  extends to a homomorphism
$\theta : \Gamma \to \Delta$  if and only if,
for all  $(u_i = v_i) \in R$,
the result of substituting  $\theta' c$  for each letter 
$c$  in  $u_i$  and  $v_i$  yeilds a relation for  $\Delta$~.
\end{thm}

\begin{lem} \label{lem:odot-gens}
The map
$$
\theta'_{d_j} : C_G \to G \ltimes \ugoh, \quad
c_i \mapsto (c_i,\,c_i \odot d_j)
$$
determines a homomorphism  $\theta_{d_j} : G \to G \ltimes (\ugoh)$~.
\end{lem}
\begin{pf}
Let  $(gc = g') \in R_G$  be a deduction for  $G$.  
Then, by definition of  $\theta_{d_j}$ 
and by the semidirect product multiplication rule,
\begin{eqnarray*}
\theta_{d_j}(gc)
  \;=\;  (\theta_{d_j}c_1)\ldots(\theta_{d_j}c_k)(\theta_{d_j}c)
  & = &  (c_1,c_1 \odot d_j)(c_2,c_2 \odot d_j) \ldots
           (c_k,c_k \odot d_j)(c,c \odot d_j) \\
  & = &  (c_1\ldots c_kc,\,(c_1 \odot d_j)^{c_2\ldots c_kc}\,
           \ldots (c_k \odot d_j)^{c}\,(c,c \odot d_j) ~, \\
\mbox{and}\qquad\qquad 
\theta_{d_j}(g')
  \;=\;  (\theta_{d_j}c'_1)\ldots(\theta_{d_j}c'_{k'})
  & = &  (c'_1\ldots c'_{k'},\,
          (c'_1 \odot d_j)^{c'_2\ldots c'_{k'}}\, \ldots (c'_{k'} \odot d_j) ~.
\end{eqnarray*}
The two $(\ugoh)$-components in these expansions are equal by 
a relator of type (1) in $R_{(\ugoh)}$~.
\end{pf}

\begin{lem} \label{lem:odot-elts}
The map
$$
\theta'_{d_{j_1} \ldots d_{j_{l}}} : C_G \to G \ltimes (\ugoh), \quad
c_i \mapsto (c_i,\,c_i \odot d_{j_1} \ldots d_{j_{l}})
$$
determines a homomorphism  
$\theta_{d_{j_1} \ldots d_{j_{l}}} : G \to G \ltimes (\ugoh)$~.
\end{lem}
\begin{pf}
We again use the Substitution Test to verify this result.

The proof of Proposition \ref{prop:GHeqHG} requires only 
the basic interchange laws for elements of  $\ugth$. 
Since the interchange laws for the generating elements of  $\ugoh$
have been imposed in  $R_{(\ugoh)}$, it is also true that
the $GH$-expansion and $HG$-expansion of
$~(c_1 c_2 \ldots c_k \odot d_1 d_2 \ldots d_{l})~$
are equal.

Let  $(gc = g') \in R_G$  be a deduction for  $G$.  
Then,
by definition of  $\theta_{d_{j_1} \ldots d_{j_{l}}}$~,
\begin{eqnarray*}
\theta_{d_{j_1} \ldots d_{j_{l}}}(c_1c_2\ldots c_kc)
  & = &  (c_1,c_1 \odot d_{j_1} \ldots d_{j_{l}}) \ldots
         (c_k,c_k \odot d_{j_1} \ldots d_{j_{l}})
         (c,c \odot d_{j_1} \ldots d_{j_{l}})~.
\end{eqnarray*}
Part (b) of the definition of  $\odot$ gives,
$$
(c',\,c' \odot d_{j_1} \ldots d_{j_{l}}) \;=\; 
(c',\,c' \odot d_{j_{l}})\,(c',\,c' \odot d_{j_{l-1}})^{d_{j_{l}}}
\ldots (c',\,c' \odot d_{j_1})^{d_{j_2}\ldots d_{j_{l}}}
\quad \mbox{for all} \quad  c' \in C_G~.
$$
Hence
\begin{eqnarray*}
\theta_{d_{j_1} \ldots d_{j_{l}}}(gc) 
  & = &  GH-\mbox{expansion of}\quad gc \odot d_{j_1} \ldots d_{j_{l}} \\
  & = &  HG-\mbox{expansion of}\quad gc \odot d_{j_1} \ldots d_{j_{l}} \\
  & = &  (\theta_{d_{j_1}}(gc) )\;(\theta_{d_{j_2}}(gc) )
         \ldots (\theta_{d_{j_{l}}}(gc) )    \\
  & = &  (\theta_{d_{j_1}}(g') )\;(\theta_{d_{j_2}}(g') )
         \ldots (\theta_{d_{j_{l}}}(g') ) 
        \qquad\qquad\qquad \mbox{by the previous Lemma,}   \\
  & = &  HG-\mbox{expansion of}\quad g' \odot d_{j_1} \ldots d_{j_{l}} \\
  & = &  GH-\mbox{expansion of}\quad g' \odot d_{j_1} \ldots d_{j_{l}} \\
  & = &  \theta_{d_{j_1} \ldots d_{j_{l}}}(g') ~.
\end{eqnarray*}
\end{pf}

\begin{thm}
The group  $\ugoh$  is isomorphic to  $\ugth$~.
\end{thm}
\begin{pf}
We have shown in Lemma \ref{lem:odot-elts} that  
$\theta_h : G \to G \ltimes (\ugoh)$  is a homomorphism,
and it can be shown in the same way that  
$\theta_g : H \to H \ltimes (\ugoh)$  is also a homomorphism.  
It follows from Proposition \ref{prop:homs-to-xpair}
that  $\odot$  is a crossed pairing.
The unique homomorphism  $\odot_{\ot} : \ugth \to \ugoh$
given by the universal property of Theorem \ref{thm:uni-prop}
maps  $\gth$  to  $\goh$.

In order to show that  $\ugoh$  is a universal crossed pairing
we require homomorphisms  $\bt_{\odot}$  and  $\ot_{\odot}$
making the following diagram commute:
$$
\xymatrix{ 
   &&  \ugxh \ar[ddll]_{\ot} \ar[dd]_{\odot} \ar[ddrr]^{\bt}
      &&  \\
   && &&  \\
 \ugth  \ar[rr]<+0.5ex>^{\odot_{\ot}}
   \ar `d[1,4] `[rrrr]^{\bt_{\ot}} [rrrr]
   &&  \ugoh  \ar[ll]<+0.5ex>^{\ot_{\odot}} \ar[rr]_{\bt_{\odot}}
      &&  L ~. \\
   && &&
}
$$
Commutativity forces the images of the generators to be given by maps
$$
\bt'_{\odot} : C_{(\ugoh)} \to L\,,\;      \cod \mapsto \cbd\;, \qquad
\ot'_{\odot} : C_{(\ugoh)} \to \ugth\,,\;  \cod \mapsto \ctd\;.
$$
It is easy to see that these two maps determine the required homomorphisms
and that these are unique.
Since  $\ugoh$  satisfies the universal property of the tensor product,
$\ugoh \cong \ugth$  and 
$\odot_{\ot}\,,\;\ot_{\odot}$  are inverse isomorphisms.
\end{pf}



%\newpage
%%%%%%%%%%%%%%%%%%%%%%%%%%%%%%%%%%%%%%%%%%%%%%%%%%%%%%%%%%%%%%%%%%%%%%%%%%%
\subsection{The Heisenberg group $\calH_2$}

This describes work of Bacon and Kappe.

Nilpotent groups generalise abelian groups in that multiple commutators
of a certain length are all trivial.
A nilpotent group  $G$  of class $2$ is such that its commutator subgroup
$G'$  is a subgroup of the centre of  $G$.

The Heisenberg group  $\calH_2$  is the free nilpotent group of class $2$
with presentation
$$
\calH_2 \quad = \quad <\,a,b,c \;|\; [a,b]=c,\,[a,c]=1,\,[b,c]=1\,>~.
$$

\begin{prop}
$$
\calH_2 \;\ot\; \calH_2 \quad\cong\quad \mathbf{Z}^6~.
$$
\end{prop}
\begin{pf}
Here is a brief outline of the proof.
\begin{itemize}
\item
show that the tensor square is abelian;
\item
show that every element of  $\calH_2$ can be written in the form
$a^m\,b^n\,c^l$~;
\item
show that every element of the tensor square is a sum of multiples of
$$
a \ot a,\quad
b \ot b,\quad
a \ot b,\quad
b \ot a,\quad
a \ot c,\quad \mbox{and}\quad
b \ot c~;
$$
\item
it follows that the tensor square is a quotient of $\mathbf{Z}^6$~;
\item
construct a surjective crossed pairing  
$\phi : \calH_2 \times \calH_2 \to \mathbf{Z}^6$~;
\item
it follows that there is a unique homomorphism from the tensor square
onto $\mathbf{Z}^6$  and hence that the tensor square is isomorphic
to $\mathbf{Z}^6$~.
\end{itemize}
\end{pf}
