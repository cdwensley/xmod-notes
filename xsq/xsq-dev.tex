% xsq-dev.tex,  version 21/05/21

%%%%%%%%%%%%%%%%%%%%%%%%%%%%%%%%%%%%%%%%%%%%%%%%%%%%%%%%%%%%%%%%%%%%%%%%%
\newpage
\noindent
{\bf [The final section in this chapter is experimental.]}

\bigskip
\noindent
{\bf These used to be Lemmas 9.22, 9.23 following the Proposition for $\tau$.} 

\bigskip
The isomorphism $\tau$ provides a neat formula for the inverse of a 
general element in $(P \ltimes N) \ltimes (M \ltimes L)$.
\begin{lem}
$$
((p,n),(m,\ell))^{-1} ~=~ 
\tau((p^{-1},(m^{-1})^{p^{-1}}),~ 
 ((n^{-1})^{p^{-1}},(\ell^{-1})^{m^{-1}n^{-1}p^{-1}}))
$$
\end{lem}
\begin{pf}
\begin{eqnarray*}
((p,n),(m,\ell))^{-1} 
 &=& ((p^{-1},(n^{-1})^{p^{-1}}),~
        (m^{-1},(\ell^{-1})^{m^{-1}})^{(p^{-1},(n^{-1})^{p^{-1}})})) \\
 &=& ((p^{-1},(n^{-1})^{p^{-1}}),~
        ((m^{-1})^{p^{-1}},
          ((m^{-1})^{p^{-1}} \bt (n^{-1})^{p^{-1}}) 
           (\ell^{-1})^{m^{-1}n^{-1}p^{-1}})) \\
 &=& \tau((p^{-1},(m^{-1})^{p^{-1}}),~ 
          ((n^{-1})^{p^{-1}},(\ell^{-1})^{m^{-1}n^{-1}p^{-1}}))) 
\end{eqnarray*}
\end{pf}

\begin{lem}
The commutator of elements in $\ddeeo(P \ltimes N)$ 
and $\ddeet(P \ltimes M)$ is 
$$
 [((p,n),(1,1)),~((q,1),(m,1))]
~=~
 (([p,q],n^{-1}n^q), 
      ((m^{-1})^{q^{-1}pq}m,((m^{-1})^{q^{-1}pq}\bt n^q)^m))
$$
\end{lem}
\begin{pf}
\vspace*{-8mm}
\begin{eqnarray*}
 & & [((p,n),(1,1)),~((q,1),(m,1))] \\
 &=& ((p^{-1},(n^{-1})^{p^{-1}}),(1,1))~
     ((q^{-1},1),((m^{-1})^{q^{-1}},1))~ 
     ((p,n),(1,1))~
     ((q,1),(m,1)) \\
 &=& ((p^{-1}q^{-1},(n^{-1})^{p^{-1}q^{-1}}),((m^{-1})^{q^{-1}},1))~ 
     ((pq,n^q),(m,1)) \\
 &=& ((p^{-1}q^{-1}pq,n^{-1}n^q), 
      ((m^{-1})^{q^{-1}pq},((m^{-1})^{q^{-1}pq}\bt n^q))(m,1)) \\
 &=& (([p,q],n^{-1}n^q), 
      ((m^{-1})^{q^{-1}pq}m,((m^{-1})^{q^{-1}pq}\bt n^q)^m)) 
\end{eqnarray*}
\end{pf}



%\begin{prop}[Loday \cite{loday1}, Ellis \cite{ellis-thesis}]
%\mbox{}\\
%Let ${\ddtto}^{\prime}, {\ddhho}^{\prime} : C_2 \to P_1$ 
%be the composite maps 
%${\ddtto}^{\prime} = \tau \ddtto \tau^{-1},~ 
% {\ddhho}^{\prime} = \tau \ddhho \tau^{-1}$. 
%The four maps ${\ddtto}^{\prime}, {\ddhho}^{\prime}, \ddttt, \ddhht$
%constitute a cat$^2$-group.
%\end{prop}


\bigskip
\subsection{Semidirect Factorisation of a cat$^2$-group}
\label{subs:sdpfact-catt}
\index{semidirect factorisation!for cat2-groups}

Starting with the cat$^2$-group 
$$
\xymatrix @C=2pc @R=2pc{
 & \GGot \ar[dd] <-1.2ex>  \ar[dd] <-2.0ex>_{\ddttt,\ddhht}
     \ar[rr] <+1.2ex>  \ar[rr] <+2.0ex>^{\ddtto,\ddhho}
    &&  \GGbt  \ar[ll]^{\ddeeo}
            \ar[dd]<+1.2ex>  \ar[dd] <+2.0ex>^{\dttt,\dhht}  \\
\calG \quad = \quad
 &  &&   \\
 & \GGbo \ar[uu]_{\ddeet}
     \ar[rr] <-1.2ex>  \ar[rr] <-2.0ex>_{\dtto,\dhho} 
    &&  \GGe \ar[uu]^{\deet}   \ar[ll]_{\deeo} 
 \\
}
$$
we wish to generalise the semidirect factorisation of 
Subsection \ref{subs:cat1} equation (\ref{eq:cat1-sdp-fact}).
Put
$$
\ddSSo = \ker\ddtto, \quad
\ddSSt = \ker\ddttt, \quad
 \dSSo = \ker\dtto, \quad
 \dSSt = \ker\dttt \quad\mbox{and}\quad
 \SSot = \ker\ttot,
$$
giving isomorphisms
\begin{eqnarray*}
\dpho ~:~ \GGbo ~\cong~ \GGe \ltimes \dSSo,
  \quad  b \mapsto (\dtto b, \duuo b), 
& \quad &
\ddpho ~:~ \GGot ~\cong~ \GGbt \ltimes \ddSSo,
  \quad  g \mapsto (\ddtto g, \dduuo g),      \\
\dpht ~:~ \GGbt ~\cong~ \GGe \ltimes \dSSt,
  \quad  c \mapsto (\dttt c, \duut c),   
& \quad &
\ddpht ~:~ \GGot ~\cong~ \GGbo \ltimes \ddSSt,
  \quad  g \mapsto (\ddttt g, \dduut g), 
\end{eqnarray*}
where, for example, $\duuo b = (\deeo\dtto b^{-1})b$.

\medskip\noindent
Note that~ $\SSot = (\ker \ddtto) \cap (\ker \ddttt)$.

\medskip\noindent
Where possible we keep to the convention that
$$
g \in \GGot,~~
b \in \GGbo,~~
c \in \GGbt,~~
p \in \GGe,~~
j \in \ddSSo,~~
k \in \ddSSt,~~
m \in \dSSo,~~
n \in \dSSt,~~
\ell \in \SSot.
$$
The following identities are easily verified:
\begin{equation} \label{eq:cat2gp-ids}
\duuo\ddttt = \ddttt\dduuo, \quad
\duuo\ddhht = \ddhht\dduuo, \quad
\dduuo\ddeet = \ddeet\duuo, \quad
\duut\ddtto = \ddtto\dduut, \quad
\duut\ddhho = \ddhho\dduut, \quad
\dduut\ddeeo = \ddeeo\duut. 
\end{equation} 

\noindent
We thus obtain the following result.

\begin{lem}
There is a cat$^1$-group isomorphism 
$\phi_1 = (\ddpho,\dpho) : \ddcalGt \to \barGGt$,
$$
\xymatrix @C=2pc @R=2pc{
 \GGot \ar[dd] <-1.2ex>  \ar[dd] <-2.0ex>_{\ddttt,\ddhht}
   \ar[rr]^(0.45){\ddpho}
  &&  \GGbt \ltimes \ddSSo 
       \ar[dd] <+1.2ex>  \ar[dd] <+2.0ex>^{\barttt,\barhht}  \\
  &&   \\
 \GGbo \ar[uu]_{\ddeet}
   \ar[rr]_(0.45){\dpho}
  &&  \GGe \ltimes \dSSo  \ar[uu]^{\bareet}
 \\
}
$$
with structure maps on $\barGGt$  given by
$$
\quad \barttt(c,j) = (\dttt c, \ddttt j),
\quad \barhht(c,j) = (\dhht c, \ddhht j),
\quad \bareet(p,m) = (\deet p, \ddeet m).
$$
\end{lem}
\begin{pf}
There are three equations to be verified. 
$$
\barttt(c,j)
~=~ \dpho\ddttt\ddpho^{-1}(c,j) \\
~=~ \dpho\ddttt(\ddeeo c)j \\
~=~ \dpho((\deeo\dttt c)(\ddttt j)) \\
~=~ (\dtto((\deeo\dttt c)(\ddttt j)), \duuo((\deeo\dttt c)(\ddttt j))).
$$
The first of these terms is~
$(\dttt c)(\dttt\ddtto j) ~=~ \dttt c$~ since ~$\ddtto j = 1$.

\medskip\noindent
The second term is~
$(\duuo(\ddttt j))(\duuo(\deeo\dttt c))^{\ddttt j}
~=~ (\deeo\dtto\ddttt j^{-1})(\ddttt j) ~=~ \duuo\ddttt j ~=~ \ddttt j$~ 
since ~$\dduuo j = j$. 

\medskip\noindent
{\bf [Parts 2,3 added 01/05/07, and need to be checked.]}

\medskip\noindent 
Similarly 
$$
\barhht(c,j)
~=~ \dpho\ddhht\ddpho^{-1}(c,j) \\
~=~ \dpho\ddhht(\ddeeo c)j \\
~=~ \dpho((\deeo\dhht c)(\ddhht j)) \\
~=~ (\dtto((\deeo\dhht c)(\ddhht j)), \duuo((\deeo\dhht c)(\ddhht j))).
$$
The first of these terms is~
$(\dhht c)(\dtto\ddhht j) ~=~ (\dhht c)(\dhht \ddtto j) 
                          ~=~ \dhht(c(\ddtto j))$. 

\medskip\noindent
The second term is~
$(\duuo(\ddhht j))(\duuo(\deeo\dhht c))^{\ddhht j}
~=~ (\deeo\dtto\ddhht j^{-1})(\ddhht j) ~=~ \duuo\ddhht j 
                                        ~=~ \ddhht(\dduuo j)$.
Hence
$$
\barhht(c,j) ~=~ (\dhht(c(\ddtto j)), \ddhht(\dduuo j)) ~=~ (\dhht c,\ddhht j).
$$

\medskip\noindent
Thirdly, 
$$
\bareet(p,m)
~=~ \ddpho\ddeet\ddpho^{-1}(p,m) \\
~=~ \ddpho\ddeet(\deeo p)m \\
~=~ \ddpho(\ddeeo\deet p)(\ddeet m) \\
~=~ (\ddtto((\ddeeo\deet p)(\ddeet m)), \dduuo((\ddeeo\deet p)(\ddeet m))).
$$
The first of these terms is~
$(\deet p)(\deet\dtto m) ~=~ \deet(p(\dtto m)$. 

\medskip\noindent
The second term is 
$(\dduuo(\ddeet m))(\dduuo(\ddeeo\deet p))^{\ddeet m} 
~=~ (\ddeeo\ddtto\ddeet m^{-1})(\ddeet m) 
~=~ (\ddeet\deeo\dtto m^{-1})(\ddeet m) 
~=~ \ddeet(\duuo m)$. 
Hence
$$
\bareet(p,m) ~=~ (\deet(p(\ddtto m)), \ddeet(\duuo m)) ~=~ (\deet p,\ddeet m).
$$
\end{pf}

\bigskip\noindent
Now $\ker\barttt = \ddpho(\ker\ddttt) = \ddpho\ddSSt$.
\hfill {\bf [Really?]} 

\medskip\noindent
A typical element in $\ddSSt$ is $\dduut g$, and
\begin{equation}
\ddpho\dduut g = (\ddtto\dduut g, \dduuo\dduut g) \in \dSSt \ltimes \SSot
% \qquad\mbox{where}\qquad \SSot ~=~ (\ker\ddtto) \cup (\ker\ddttt).
\qquad\mbox{so}\qquad
\ker\barttt ~=~ \dSSt \ltimes \SSot ~\unlhd~ \GGbt \ltimes \ddSSo.
\end{equation}

\begin{lem} \label{lem:cat1iso-prime}
There is a cat$^1$-group isomorphism~ 
$\phi_1{}^{\prime} = (\ddphod,1) : \barGGt \to \ddcalGtd$
$$
\xymatrix @C=2pc @R=2pc{
  \GGbt \ltimes \ddSSo 
   \ar[dd] <-1.2ex>  \ar[dd] <-2.0ex>_{\barttt,\barhht}
   \ar[rr]^(0.45){\ddphod}
  &&  (\GGe \ltimes \dSSo) \ltimes (\dSSt \ltimes \SSot) 
       \ar[dd] <+1.2ex>  \ar[dd] <+2.0ex>^{\ddtttd,\ddhhtd}  \\
  &&   \\
 \GGe \ltimes \dSSo  \ar[uu]_{\bareet}
   \ar[rr]_(0.45){1}
  &&  \GGe \ltimes \dSSo  \ar[uu]^{\ddeetd}
 \\
}
$$
where $\ddphod(c,j) \quad=\quad (\barttt(c,j),\baruut(c,j))$, 
and the structure maps on $\ddcalGd$ are given by
\begin{eqnarray*}
\ddtttd((p,m),(n,\ell)) &=& (p(\dttt n),m^{\dttt n}(\ddttt\ell)),\\
\ddhhtd((p,m),(n,\ell)) &=& (p(\dhht n),m^{\dhht n}(\ddhht\ell)),\\
\ddeetd(p,m) &=& ((p,m),(1,1)).
\end{eqnarray*}
\end{lem}
\begin{pf}
Since
$$
\baruut(c,j)
~=~ (\bareet\barttt(c^{-1},(j^{-1})^{c^{-1}}))(c,j)
~=~ (\deet\dttt c^{-1}, \ddeet\ddttt(j^{-1})^{c^{-1}})(c,j)
~=~ (\duut c, \dduut j),
$$
it follows that 
$$
\ddphod(c,j) \quad=\quad
(\,(\dttt c,~\ddttt j),~ (\duut c,~\dduut j)\,).
$$

\noindent
We verify the first of the structure map formulae -- the rest 
follow similarly:
\begin{eqnarray*}
\ddtttd((p,m),(n,\ell)) 
  &=&  \barttt \ddphod{}^{-1}((p,m),(n,\ell)) \\
  &=&  \barttt(\bareet(p,m)(n,\ell)) \\
  &=&  (p,m)\barttt(n,\ell) \\
  &=&  (p,m)(\dttt n, \ddttt \ell) \\
  &=&  (p,m)(1,1) ~=~ (p,m).
\end{eqnarray*}
\end{pf}

\noindent
The composite source group isomorphism $\phot = \ddphod \circ \ddpho$ 
is therefore
$$
\phot ~:~ \GGot \to (\GGe \ltimes \dSSo) \ltimes (\dSSt \ltimes \SSot), 
\quad g \mapsto 
      (\,(\ttot g,\duuo\ddttt g),~(\duut\ddtto g,~\dduut\dduuo g)\,).
$$

\bigskip\noindent
Combining $\phi_1$ with $\phi_1{}^{\prime}$ we obtain the following result.
\begin{thm}
There is an isomorphism of cat$^2$-groups
$$
\phi ~=~ (\phot,\dpht,\dpho,1) ~:~ \calG \to \calG'~.
$$
$$
\xymatrix @C=2pc @R=2pc{
   & \GGot \ar[dd] <-1.2ex>  \ar[dd] <-2.0ex>_{\ddttt,\ddhht}
     \ar[rr] <+1.2ex>  \ar[rr] <+2.0ex>^{\ddtto,\ddhho}
    && \GGbt  \ar[ll]^{\ddeeo}
       \ar[dd]<+1.2ex>  \ar[dd] <+2.0ex>^{\dttt,\dhht}  
      &&  (\GGe \ltimes \dSSo) \ltimes (\dSSt \ltimes \SSot) 
           \ar[dd] <+1.2ex>  \ar[dd] <+2.0ex>^{\ddtttd,\ddhhtd} 
           \ar[rr] <+1.2ex>  \ar[rr] <+2.0ex>^(0.6){\ddttod,\ddhhod}       
        && \GGe \ltimes \dSSt  \ar[ll]^(0.4){\ddeeod}
           \ar[dd]<+1.2ex>  \ar[dd] <+2.0ex>^{\dtttd,\dhhtd} \\
\phi~: 
   &&&& \qquad\to
       &&&                                                   \\
   & \GGbo \ar[uu]_{\ddeet}
     \ar[rr] <-1.2ex>  \ar[rr] <-2.0ex>_{\dtto,\dhho} 
    && \GGe \ar[uu]^{\deet}   \ar[ll]_{\deeo} 
      && \GGe \ltimes \dSSo  \ar[uu]^{\ddeetd} 
         \ar[rr] <-1.2ex>  \ar[rr] <-2.0ex>_{\dttod,\dhhod} 
        && \GGe  \ar[uu]^{\deetd}  \ar[ll]_{\ddeeod}         \\
}
$$
where
\begin{eqnarray*}
\qquad\qquad\qquad~
\phot(g) &=& (\,(\ttot g,\duuo\ddttt g),~(\duut\ddtto g,~\dduut\dduuo g)\,),\\
\dpht(c) &=& (\dttt c, \duut c),\\
\dpho(b) &=& (\dtto b, \duuo b),
\end{eqnarray*}
and the $12$ homomorphisms are given by
\begin{eqnarray*}
\ddtttd((p,m),(n,\ell)) 
  &=& (p(\dttt n),m^{\dttt n}(\ddttt\ell)),\\
\ddhhtd((p,m),(n,\ell)) 
  &=& (p(\dhht n),m^{\dhht n}(\ddhht\ell)),\\
\ddeetd(p,m) 
  &=& ((p,m),(1,1)), \\
\ddttod((p,m),(n,\ell)) 
  &=& (p(\dtto m)(\dttt n),(\duut n)(\duut\ddtto\ell)),\\
\ddhhod((p,m),(n,\ell)) 
  &=& ((?,?),(?,?)),\\
\ddeeod(p,n) 
  &=& ((p,1),(n,1)), \\
\dttt(p,n)  &=& p(\dttt n),\\
\dhht(p,n)  &=& p(\dhht n),\\
\deet(p) &=& (p,1) \\
\dtto(p,m)  &=& p(\dtto m),\\
\dhho(p,m)  &=& p(\dhho m),\\
\deeo(p) &=& (p,1).
\end{eqnarray*}
\end{thm}
\begin{pf}
To be added.
\end{pf}

