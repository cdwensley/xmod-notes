% xsq.tex,  version 03/05/17

%%%%%%%%%%%%%%%%%%%%%%%%%%%%%%%%%%%%%%%%%%
\section{Crossed Squares and cat$^2$-groups} \label{sect:xsq-cat2}

Crossed squares were introduced by Guin-Wal\'ery and Loday 
(see, for example, \cite{walery:loday,loday1,brow:lod})
as fundamental crossed squares of commutative squares of spaces,
but are also of purely algebraic interest.
We denote by $[n]$ the set $\{1,2,\ldots,n\}$.
We use the $n=2$ version of the definition of crossed $n$-cube
as given by Ellis and Steiner \cite{ell:st}.

\begin{defn} \label{def:xsq}  \index{crossed square}
A crossed square consists of the following:
\begin{itemize}
\item
a commutative diagram of group homomorphisms
\begin{equation} \label{eq:Rsquare}
\vcenter{\xymatrix{
       &   &  R_{[2]} \ar[rr]^{\ddbdyo} \ar[dd]_{\ddbdyt} 
              && R_{\{2\}} \ar[dd]^{\dbdyt} &   \\
\calR  & = &  &&                 & ; \\
       &   &  R_{\{1\}} \ar[rr]_{\dbdyo}  
              && R_{\emptyset} 
}} 
\end{equation}
\item
actions of $R_{\emptyset}$ on $R_{\{1\}}, R_{\{2\}}$ and $R_{[2]}$ 
which determine actions of 
$R_{\{1\}}$ on $R_{\{2\}}$ and $R_{[2]}$ via $\dbdyo$ 
and actions of 
$R_{\{2\}}$ on $R_{\{1\}}$ and $R_{[2]}$ via $\dbdyt$~;
\item
a function ~ $\bt : R_{\{1\}} \times R_{\{2\}} \to R_{[2]}$~.
\end{itemize}
The following axioms must be satisfied for all 
$l \in R_{[2]},\; m,m_1,m_2 \in R_{\{1\}},\; 
n,n_1,n_2 \in R_{\{2\}},\; p \in R_{\emptyset}$~:
\begin{enumerate}[{\rm (a)}]
\item 
the homomorphisms $\ddbdyo, \ddbdyt$ preserve the action of $R_{\emptyset}$~;
\item 
each of 
$~\ddcalRo = (\ddbdyo : R_{[2]} \to R_{\{2\}}),
 ~\ddcalRt = (\ddbdyt : R_{[2]} \to R_{\{1\}}), 
 ~\dcalRo  = (\dbdyo : R_{\{1\}} \to R_{\emptyset}),
 ~\dcalRt  = (\dbdyt : R_{\{2\}} \to R_{\emptyset})~$
and the diagonal  
$~\calRot = (\bdyot = \dbdyo\ddbdyt = \dbdyt\ddbdyo 
             : R_{[2]} \to R_{\emptyset})~$ 
are crossed modules (with actions via $R_{\emptyset}$);
\item 
$\bt$ is a \emph{crossed pairing}:
\begin{enumerate}[{\rm (i)}]
\item
$(m_1m_2 \bt n)\;=\;{(m_1 \bt n)}^{m_2}\;(m_2 \bt n)$~,
\item
$(m \bt n_1n_2) \;=\; (m \bt n_2)\;{(m \bt n_1)}^{n_2}$~,
\item 
$(m \bt n)^{p} \;=\; (m^p \bt n^p)$~;
\end{enumerate}
\item\quad
$\ddbdyo (m \bt n) \;=\; (n^{-1})^{m}\;n$
\quad \mbox{and} \quad
$\ddbdyt (m \bt n) \;=\; m^{-1}\;m^{n}$~,
\item\quad
$(m \bt \ddbdyo l) \;=\; (l^{-1})^{m}\;l$
\quad \mbox{and} \quad
$(\ddbdyt l \bt n) \;=\; l^{-1}\;l^n$~.
\end{enumerate}
\end{defn}

Note that the actions of  $R_{\{1\}}$  on  $R_{\{2\}}$  and  $R_{\{2\}}$  
on  $R_{\{1\}}$  via  $R_{\emptyset}$
are compatible since
$$
{m_1}^{(n^m)} \;=\; {m_1}^{\dbdyt(n^m)} \;=\; {m_1}^{m^{-1}(\dbdyt n)m}
\;=\; (({m_1}^{m^{-1}})^n)^m~.
$$

\begin{lem}
Pairs $\partial_1 = (\ddbdyo,\dbdyo)$ and  $\partial_2 = (\ddbdyt,\dbdyt)$ 
are morphisms of crossed modules, and so also are
$(1_{R_{[2]}},\dbdyo)$, $(1_{R_{[2]}},\dbdyt)$,
$(\ddbdyt,1_{R_{\emptyset}})$  and  $(\ddbdyo,1_{R_{\emptyset}})$.
\end{lem}
\begin{pf}
%This is immediate from axiom (a). \quad {\bf [Is this correct?]}
For $\partial_1$ we note that (\ref{eq:Rsquare}) commutes,
that  $\ddbdyo(\ell^m) = (\ddbdyo l)^m$  by (a),
and that  $(\ddbdyo \ell)^m = (\ddbdyo \ell)^{\dbdyo m}$  by (b).
The arguments for the other five morphisms are similar.
\end{pf}

\medskip\noindent
Note in particular that
$$
\ddbdyo(\ell^p) = (\ddbdyo \ell)^p
\qquad \mbox{and} \qquad
\ddbdyt(\ell^p) = (\ddbdyt \ell)^p.
$$

\begin{lem} \label{lem:xsq}
In the crossed square  $\calR$  above:
\begin{enumerate}[{\em (a)}]
\item~
$$
\ell^{(m \bt n)} \quad = \quad \ell^{[m,n]}\,,
$$
\item~
$$
\ell^{nm}\,(m \bt n) ~=~ (m \bt n)\,\ell^{mn}
\qquad\mbox{and}\qquad
\ell^{(n^m)}\,(m \bt n) ~=~ (m \bt n)\,\ell^{n}\,.
$$
\item~
$$
m^{\ddbdyo\ell} = m^{\ddbdyt\ell}
\qquad\mbox{and}\qquad
n^{\ddbdyt\ell} = n^{\ddbdyo\ell}\,.
$$
\end{enumerate}
\end{lem}
\begin{pf}
\begin{enumerate}[(a)]
\item
$$
\ell^{(m \bt n)}
  \;=\;  \ell^{\dbdyo\ddbdyt(m \bt n)}
  \;=\;  \ell^{m^{-1}\,m^n}
  \;=\;  \ell^{m^{-1}n^{-1}mn}
  \;=\;  \ell^{[m,n]}~.
$$
\item
The first identity is given by
$$
(m \bt n)^{-1}\,\ell^{nm}\,(m \bt n) \;=\;
\ell^{nm\ddbdyo(m \bt n)} \;=\;
\ell^{nm({n^{-1}}^m)n} \;=\;
\ell^{mn}~.
$$
Then replace $\ell$ by $\ell^{m^{-1}}$ to get the second.
\item
$$
m^{\ddbdyo\ell} ~=~ m^{\dbdyt\ddbdyo\ell} ~=~ m^{\dbdyo\ddbdyt\ell} 
~=~  m^{\ddbdyt\ell} 
$$
\end{enumerate}
\end{pf}


%%%%%%%%%%%%%%%%%%%%%%%%%%%%%%%%%%%%%%%%
\subsection{Examples of crossed squares}

\begin{example}
\emph{If $M, N$ are normal subgroups of the group $P$ 
then the diagram of inclusions}
$$
\xymatrix{
M \cap N \ar[rr]^(0.6){\ddiio} \ar[dd]_{\ddiit} \ar[rrdd]^{\iiot} 
  && N \ar[dd]^{\diit} \\
       \\
M \ar[rr]_{\diio}  
  && P 
}
$$
\emph{together with the actions of $P$ on $M, N$ 
and $M \cap N$ given by conjugation and the function}
$$
\bt \;:\; M \times N \to M\cap N, \quad (m,n) \mapsto [m,n]\,=\,m^{-1}n^{-1}mn
$$
\emph{is a crossed square.
We may check the axioms as follows:}
\begin{enumerate}[{\rm (a)}]
\item
The identity maps preserve $P$-actions.
\item
The five crossed modules are all conjugation crossed modules.
\item
We verify the three commutator identities:
\begin{enumerate}[{\rm (i)}]
\item
\begin{eqnarray*}
(m_1m_2 \bt n) \;=\; [m_1m_2,n] 
  & = &  (m_1m_2)^{-1} n^{-1} (m_1m_2) n  \\
  & = &  m_2^{-1}\{m_1^{-1}n^{-1}m_1n\}m_2\;m_2^{-1} n^{-1} m_2 n  \\
  & = &  [m_1,n]^{m_2}\;[m_2,n]~;  
\end{eqnarray*}
\item
\begin{eqnarray*}
(m \bt n_1n_2) \;=\; [m,n_1n_2] 
  & = &  m^{-1} (n_1n_2)^{-1} m (n_1n_2)  \\
  & = &  m^{-1} n_2^{-1} m n_2\;n_2^{-1}\{m^{-1}n_1^{-1}mn_1\}n_2  \\
  & = &  [m,n_2]\;[m,n_1]^{n_2}~;
\end{eqnarray*}
\item
\vspace{-5mm}
\begin{eqnarray*}
(m^p \bt n^p) \;=\; [m^p,n^p]
  & = &  \{p^{-1}mp\}^{-1}\,\{p^{-1}np\}^{-1}\,\{p^{-1}mp\}\{p^{-1}np\}  \\
  & = &  p^{-1}m^{-1}p\,p^{-1}n^{-1}p\,p^{-1}mp\,p^{-1}np \\
  & = &  p^{-1}[m,n]p \;=\; [m,n]^p ~.
\end{eqnarray*}
\end{enumerate}
\item\quad
$\ddiio (m \bt n) \;=\; (m^{-1}n^{-1}m)n \;=\; {n^{-1}}^m\,n$
\quad \emph{and} \quad
$\ddiit (m \bt n) \;=\; m^{-1}\,(n^{-1}mn) \;=\; m^{-1}\,m^n$~.
\item\quad
$(m \bt \ddiio l) \;=\; (m^{-1}l^{-1}m)l \;=\; {l^{-1}}^m\,l$
\quad \emph{and} \quad
$(\ddiit l \bt n) \;=\; l^{-1}\,(n^{-1}ln) \;=\; l^{-1}\,l^n$~.
\end{enumerate}
\end{example}

\medskip
\begin{example}
\emph{If $M, N$ are ordinary $P-$modules and $A$ is an arbitrary abelian group
on which $P$ is assumed to act trivially, 
then there is a crossed square}
$$
\xymatrix{
A \ar[rr]^{0} \ar[dd]_{0} \ar[rrdd]^{0}  
  && N \ar[dd]^{0} \\
  &&               \\
M \ar[rr]_{0} 
  && P 
}
$$
\emph{Note that $M$ acts trivially on $N$, and conversely, 
and that $m \bt n = 1_A$.} 
\end{example}

\medskip
\begin{example}
\emph{The diagram}
$$
\xymatrix{
M \ar[rr]^{\alpha} \ar[dd]_{\alpha}  && \Inn M \ar[dd]^{\iota} \\
       \\
\Inn M \ar[rr]_{\iota}  && \Aut M 
}
$$
\emph{is a crossed square, 
where $\alpha$ maps $m \in M$
to the inner automorphism  $\beta_m : M \to M,\; 
m^{\prime} \mapsto m^{-1}m^{\prime}m$~; 
where $\iota$ is the inclusion of $\Inn M$  in  $\Aut M$;
the actions are standard; and the crossed pairing is}
$$
\bt \;:\; \Inn M \times \Inn M \to M, \quad
(\beta_m, \beta_{m^{\prime}}) \;\mapsto\; [m, m^{\prime}]~.
$$
\end{example}

\medskip
\begin{example}
\emph{If $U, V$ are subspaces of a space $X$ with a point $x_0$ in common, 
then the diagram of boundary maps}
$$
\xymatrix{
\pi_3(X;U,V, x_0) \ar[rr] \ar[dd]  && \pi_2(V, U \cap V, x_0) \ar[dd] \\
       \\
\pi_2(U, U \cap V, x_0) \ar[rr] && \pi_1(U \cap V, x_0) 
}
$$
\emph{in which $\pi_3(X;U,V,x_0)$ is the triad homotopy group, 
together with the standard actions and the triad Whitehead product
$$
\bt \;:\; 
 \pi_2(U, U \cap V, x_0) \times \pi_2(V, U \cap V, x_0) 
 \;\to\; \pi_3(X;U, V, x_0)
$$
is a crossed square.}
\end{example}

\begin{lem} \label{lam:Stranspose}
The \emph{transpose}
$$
\xymatrix{
  &  R_{[2]} \ar[rr]^{\ddbdyt} \ar[dd]_{\ddbdyo} 
     &&  R_{\{1\}} \ar[dd]^{\dbdyo} \\
\tilde{\calR} \quad = 
  &  &&                             \\
  &  R_{\{2\}} \ar[rr]_{\dbdyt} 
     && R_{\emptyset} 
}
\xymatrix{
  &&  R_{[2]} \ar[rr]^{\ddbdyo} \ar[dd]_{\ddbdyt} 
     &&  R_{\{2\}} \ar[dd]^{\dbdyt} \\
  & \mbox{of} \qquad \calR \quad =  
   & &&                             \\
  &&  R_{\{1\}} \ar[rr]_{\dbdyo} 
     &&  R_{\emptyset} 
}
$$
is a crossed square with crossed pairing
\begin{equation} \label{eq:btt}
\btt \;:\; R_{\{2\}} \times R_{\{1\}} \to R_{[2]}, \quad 
(n,m) \;\mapsto\; n \btt m := (m \bt n)^{-1}~.
\end{equation}
\end{lem}
\begin{pf}
\begin{eqnarray*}
n_1n_2 \btt m
  & = & (m \bt n_1n_2)^{-1}
  \;=\; ((m \bt n_2)\,(m \bt n_1)^{n_2})^{-1} \\
  & = & ((m \bt n_1)^{n_2})^{-1}\,(m \bt n_2)^{-1}
  \;=\; (n_1 \btt m)^{n_2}\,(n_2 \btt m)~;\\
n \btt m_1m_2
  & = & (m_1m_2 \bt n)^{-1}
  \;=\; ((m_1 \bt n)^{m_2}\,(m_2 \bt n))^{-1} \\
  & = & ((m_2 \bt n)^{-1}\,((m_1 \bt n)^{m_2})^{-1}
  \;=\; (n \btt m_2)\,(n \btt m_1)^{m_2}~.
\end{eqnarray*}
\end{pf}

\newpage
\begin{example} \label{ex:actor-square} 
\emph{The actor $\calA(\calX)$ of a crossed module $\calX$ 
(see subsection \ref{subs:AX})}
\begin{equation} \label{eq:actor}
\vcenter{\xymatrix{ 
               &  S \ar[rr]^(0.45){\ddi} \ar[dd]_{\partial}
                  &&  W \ar[dd]^{\Delta}  \\
\calA \qquad =  &     &&  \\
               &  R \ar[rr]_(0.45){\di}
                  &&  A
}} 
\end{equation}
\emph{is a crossed square with crossed pairing}
$$
\bt \;:\;  R \times W \,\to\, S, \quad
(r,\chi) \,\mapsto\, \chi r~.
$$
\emph{We already know that the square  $\calA$  contains $5$ crossed modules,
but we still need to check the axioms which involve the crossed pairing:}
\begin{enumerate}[{\rm (a)}]
%\item
%(Find some way of passing by the first two items.)
%\item
\setcounter{enumi}{2}
\item
 \begin{enumerate}[{\rm (i)}]
 \item
  $(qr \bt \chi) 
  \;=\; \chi(qr)
  \;=\; (\chi q)^r\,(\chi r)
  \;=\; (q \bt \chi)^r\,(r \bt \chi)$~.
 \item
  $(r \bt \chi_1\star\chi_2)
  \;=\; (\chi_2 r)(\chi_1 r)(\chi_2\partial\chi_1 r)
  \;=\; (\chi_2 r)\,(\chi_1 r)^{\chi_2}
  \;=\; (r \bt \chi_2)\,(r \bt \chi_1)^{\chi_2}$~,
  \begin{flushright}
    \emph{using the action in Lemma \ref{lem:deriv-act}.}
  \end{flushright}
 \item
  $(r^{\beta} \bt \chi^{\beta})
   \; = \; \chi^{\beta}(r^{\beta}) 
   \; = \; (\ddb \chi \db^{-1})(\db r) 
   \; = \; \ddb(\chi r) 
   \; = \; (r \bt \chi)^{\beta}$~.
 \end{enumerate}
\item
\emph{Since $\ddi(r \bt \chi) = \ddi(\chi r) = \eta_{\chi r}$,
we wish to prove that}
$$
\eta_{\chi r} = (\chi^{-1})^r \star \chi
\quad\mbox{\emph{or, equivalently,}}\quad
\chi^r = \chi \star (\eta_{\chi r})^{-1} = \chi \star \eta_{(\chi r)^{-1}}\,.
$$
\emph{Starting with the right-hand side,}
\begin{eqnarray*}
(\chi \star \eta_{(\chi r)^{-1}})q
 & = & (\eta_{(\chi r)^{-1}}q)(\ddb_{\eta_{(\chi r)^{-1}}} \chi q) 
        \hspace{34mm} \mbox{\emph{ by Lemma \ref{lem:gamma-beta-chi} (c)}} \\
 & = & (\chi r)^q (\chi r)^{-1} (\chi q)^{(\chi r)^{-1}} 
        \hspace{33mm} \mbox{\emph{ by Lemma \ref{lem:princ-prop} (b)}} \\
 & = & (\chi r)^q (\chi q) (\chi r)^{-1} \\
 & = & (\chi r)^q (\chi q) (\chi r^{-1})^r 
        \hspace{40mm} \mbox{\emph{ by Lemma \ref{lem:invchir} (b)}}    \\
 & = & \ddb_r((\chi r)^{qr^{-1}} (\chi q)^{r^{-1}} (\chi r^{-1})) \\
 & = & \ddb_r \chi(rqr^{-1})
   ~=~ (\ddb_r\chi\db_r^{-1})q
        \hspace{25mm} \mbox{\emph{ by Lemma \ref{lem:gamma-beta-chi} (c)}} \\
 & = & \chi^{\di r}q
   ~=~ \chi^r q\,.
\end{eqnarray*}
\emph{The second formula follows by}\\
\hspace*{4mm}$\partial (r \bt \chi) 
  \;=\; \partial \chi(r) 
  \;=\; r^{-1} r (\partial \chi r) 
  \;=\; r^{-1} (\db_{\chi} r) 
  \;=\; r^{-1} r^{\chi}$~.
\item
\quad $(r \bt \ddi(s)) \;=\; \eta_{s}(r) \;=\; (s^{-1})^r s$
  \hspace{60mm} \emph{by Lemma \ref{lem:gamma-eta_s},} \\
\hspace*{4mm}$(\partial s \bt \chi) 
  \;=\; \chi(\partial s) 
  \;=\; s^{-1} s (\chi\partial s) 
  \;=\; s^{-1} (\ddb_{\chi} s) 
  \;=\; s^{-1} s^{\chi}$~, \\
\end{enumerate}
\end{example}


\newpage
%%%%%%%%%%%%%%%%%%%%%%%%%%%%%%%%%%%%%%%%%%%%%%%%%%%%%%%%%%%%%%%%%%%%%%%%%%%
\subsection{Morphisms of crossed squares} 
\index{morphism!of crossed squares} 

A morphism $\theta : \calR \to \calS$ of crossed squares 
is a $4$-tuple of group homomorphisms which commute with the
morphisms in $\calR$ and $\calS$ and preserve all the actions
and the crossed pairings.

\begin{defn}
A morphism $\theta : \calR \to \calS$ of crossed squares
consists of four group homomorphisms
$$
\theta_{[2]} : R_{[2]} \to S_{[2]}, \quad
\theta_{\{2\}} : R_{\{2\}} \to S_{\{2\}}, \quad
\theta_{\{1\}} : R_{\{1\}} \to S_{\{1\}}, \quad
\theta_{\emptyset} : R_{\emptyset} \to S_{\emptyset},
$$
forming a commutative cube with the morphisms 
$\ddbdy_j, \dbdy_j$ of $\calR$
and $\ddg_j, \dg_j$ of $\calS$ (for $j \in \{1,2\})$,
which pair off in appropriate ways to form crossed module morphisms
$$
(\theta_{[2]},\theta_{\{2\}}) : \ddcalRo \to \ddcalSo, \quad
(\theta_{[2]},\theta_{\{1\}}) : \ddcalRt \to \ddcalSt, \quad
(\theta_{\{2\}},\theta_{\emptyset}) : \dcalRt \to \dcalSt, \quad
(\theta_{\{1\}},\theta_{\emptyset}) : \dcalRo \to \dcalSo,
$$
and which preserve the crossed pairing: 
$$
\theta_{[2]}(m \bt_{\calR} n) ~=~ 
(\theta_{\{1\}} m) \bt_{\calS} (\theta_{\{2\}} n).
$$
\end{defn}

\begin{defn} \label{def:aut-xsq} 
\index{automorphism group!of a crossed square}
The group $\Aut(\calR)$ of automorphisms of the crossed square $\calR$ is 
$$
\Aut(\calR) ~=~ \{ \alpha = 
(\alpha_{[2]}, \alpha_{\{1\}}, \alpha_{\{2\}}, \alpha_{\emptyset}) 
 ~:~ \calR \to \calR \}
$$ 
such that 
$(\alpha_{[2]}, \alpha_{\{1\}}) $  is an automorphism of  $\ddcalRt$,
$(\alpha_{[2]}, \alpha_{\{2\}}) $  is an automorphism of  $\ddcalRo$,
$(\alpha_{\{2\}}, \alpha_{\emptyset}) $  is an automorphism of  $\dcalRt$, and 
$(\alpha_{\{1\}}, \alpha_{\emptyset}) $  is an automorphism of  $\dcalRo$.
\end{defn}

\vspace*{15mm}
\noindent
{\bf [Do we also require the following?]}
$$
\alpha_{[2]}(m \bt n) ~=~ (\alpha_{\{1\}} m) \bt (\alpha_{\{2\}} n)
~~{\bf ?}
$$

\vspace*{15mm}
\noindent
{\bf [Is there a crossed square version of Theorem 1.8~?]}

\begin{thm}
Every crossed square is a quotient of normal inclusion crossed squares~ 
{\bf ?}
\end{thm}

\noindent
(Note: notes 10/7/03 only go one way.)


%%%%%%%%%%%%%%%%%%%%%%%%%%%%%%%%%%%%%%%%%%%%%%%%%%%%%%%%%%%%%%%%%%%%%%%%%%%
\newpage
\subsection{Cat$^2$-groups}  \label{subs:cattwo}
\index{cat$^2$-group} 

We shall give three definitions of cat$^2$-groups 
and show that they are equivalent. 
When we come to define cat$^n$-groups we shall give a similar set of 
three definitions. 

Firstly, we take the definition of a cat$^2$-group from 
Section 5 of Brown and Loday \cite{brow:lod}, suitably modified.

\begin{defn} \label{defn:cat2a} 
\emph{A cat$^2$-group  $\calC = (\CCot,\CCbt,\CCbo,\CCe)$  
comprises $4$ groups (one for each of the subsets of $[2]$) 
and $15$ homomorphisms, as shown in the following diagram,}
\begin{equation} \label{eq:cat2-diag1}
\vcenter{\xymatrix{
 & C_{[2]} \ar[ddd] <-1.2ex>  \ar[ddd] <-2.0ex>_{\ddttt,\ddhht}
     \ar[rrr] <+1.2ex>  \ar[rrr] <+2.0ex>^{\ddtto,\ddhho}
     \ar[dddrrr] <-0.2ex>  \ar[dddrrr] <-1.0ex>_(0.55){\ttot,\hhot}
    &&&  C_{\{2\}}  \ar[lll]^{\ddeeo}
            \ar[ddd]<+1.2ex>  \ar[ddd] <+2.0ex>^{\dttt,\dhht}  \\
\calC \quad = \quad
 &  &&&   \\
 &  &&&   \\
 & C_{\{1\}} \ar[uuu]_{\ddeet}
     \ar[rrr] <-1.2ex>  \ar[rrr] <-2.0ex>_{\dtto,\dhho} 
    &&&  C_{\emptyset} \ar[uuu]^{\deet}   \ar[lll]_{\deeo} 
           \ar[uuulll] <-1.0ex>_{\eeot}
 \\
}}
\end{equation}

\noindent
\emph{subject to the following axioms:}
\begin{enumerate}[(a)]
\item~
\emph{the four sides of the square are cat$^1$-groups,}
\emph{denoted} $\ddcalCo, \ddcalCt, \dcalCo, \dcalCt$,
\item~
$
 \dtto\circ\ddhht = \dhht\circ\ddtto, ~
 \dttt\circ\ddhho = \dhho\circ\ddttt, ~
 \deeo\circ\dttt = \ddttt\circ\ddeeo, ~
 \deet\circ\dtto = \ddtto\circ\ddeet, ~
 \deeo\circ\dhht = \ddhht\circ\ddeeo, ~
 \deet\circ\dhho = \ddhho\circ\ddeet,$
\item~
$\dtto\circ\ddttt = \dttt\circ\ddtto = \ttot, ~ 
 \dhho\circ\ddhht = \dhht\circ\ddhho = \hhot, ~
 \deeo\circ\ddeet = \deet\circ\ddeeo = \eeot,$

\emph{making the diagonal a pre-cat$^1$-group
$(e_{[2]}; t_{[2]}, h_{[2]} : C_{[2]} \to C_{\emptyset})$.}
\end{enumerate}
\end{defn}

\noindent
It follows from these identities that 
$(\ddtto,\dtto),\,(\ddhho,\dhho)$ and $(\ddeeo,\deeo)$ 
are morphisms of cat$^1$-groups.

\bigskip\noindent
{\bf [It needs to be checked that the diagonal pre-cat$^1$-group 
\emph{is} a cat$^1$-group.]}
\index{Peiffer subgroup!of a cat$^1$-group} 

\medskip 
Secondly, we give the simplest of the three definitions, 
adapted from Ellis-Steiner \cite{ell:st}. 
\begin{defn} \label{defn:cat2b} 
\emph{A cat$^2$-group  $\calC$ consists of groups $G, R_1,R_2$ 
and six homomorphisms 
$t_1,h_1 : G \to R_2,~ e_1 : R_2 \to G,~
 t_2,h_2 : G \to R_1,~ e_2 : R_1 \to G$,  
satisfying the following axioms for all $1 \leqslant i \leqslant 2$,} 
\begin{enumerate}[(a)]
\item~
$
(t_i \circ e_i)r = r,~ (h_i \circ e_i)r = r,~ 
\forall r \in R_{[2] \setminus \{i\}}, \quad 
[\ker t_i, \ker h_i] = 1, 
$
\item~ 
$
(e_1 \circ t_1) \circ (e_2 \circ t_2) = (e_2 \circ t_2) \circ (e_1 \circ t_1), \quad
(e_1 \circ h_1) \circ (e_2 \circ h_2) = (e_2 \circ h_2) \circ (e_1 \circ h_1), 
$
\item~ 
$
(e_1 \circ t_1) \circ (e_2 \circ h_2) = (e_2 \circ h_2) \circ (e_1 \circ t_1), \quad 
(e_2 \circ t_2) \circ (e_1 \circ h_1) = (e_1 \circ h_1) \circ (e_2 \circ t_2). 
$
\end{enumerate}
\end{defn} 

To show that Definition \ref{defn:cat2b} is equivalent to 
Definition \ref{defn:cat2a} we take: 
\begin{itemize}
\item~ 
$
C_{[2]} = G,~  C_{\{1\}} = R_2,~ C_{\{2\}} = R_1, 
$
\item~ 
$
\ddtto = t_1,~ \ddhho = h_1,~ \ddeeo = e_1,\quad 
\ddttt = t_2,~ \ddhht = h_2,~ \ddeet = e_2.
$
\end{itemize} 
It is clear that the resulting $\ddcalCo, \ddcalCt$ are cat$^1$-groups. 
Then, since $e_1 \circ t_1$ and $e_2 \circ t_2$ commute etc., 
we are able to define: 
\begin{itemize} 
\item~
$
t_{[2]} = (e_1 \circ t_1) \circ (e_2 \circ t_2),\quad 
h_{[2]} = (e_1 \circ h_1) \circ (e_2 \circ h_2), 
$
\item~ 
$
C_{\emptyset} = \mbox{their common image},\quad 
e_{[2]} = \mbox{identity mapping on}\; C_{\emptyset}. 
$
\end{itemize}
We then need to check that $(e_{[2]};t_{[2]},h_{[2]} : G \to C_{\emptyset})$ 
is a cat$^1$-group by, for example, verifying that 
\begin{eqnarray*} 
t_{[2]} \circ e_{[2]} \circ h_{[2]} 
  & = &  
e_1 \circ t_1 \circ (e_2 \circ t_2) \circ (e_1 \circ h_1) \circ e_2 \circ h_2 \\ 
  & = & 
e_1 \circ t_1 \circ (e_1 \circ h_1) \circ (e_2 \circ t_2) \circ e_2 \circ h_2 \\ 
  & = & 
(e_1 \circ h_1) \circ (e_2 \circ h_2) \\ 
  & = & 
h_{[2]}.
\end{eqnarray*} 

\noindent
{\bf [Do we need more justification for the term "common image"?]}

\medskip\noindent
The remaining homomorphisms can then be defined: 
\begin{itemize}
\item~ 
$
\dtto = t_{[2]} \circ \ddeet = e_1 \circ t_1 \circ e_2,~~ 
\dhho = h_{[2]} \circ \ddeet = e_1 \circ h_1 \circ e_2,~~ 
\deeo = \ddttt \circ e_{[2]} 
      = \mbox{restriction of}\; t_2\; \mbox{to}\; C_{\emptyset}, 
$
\item~ 
$
\dttt = t_{[2]} \circ \ddeeo = e_2 \circ t_2 \circ e_1,~~ 
\dhht = h_{[2]} \circ \ddeeo = e_2 \circ h_2 \circ e_1,~~ 
\deet = \ddtto \circ e_{[2]} 
      = \mbox{restriction of}\; t_1\; \mbox{to}\; C_{\emptyset}. 
$
\end{itemize} 
It is then routine to check that 
$(\deeo;\dtto,\dhho : C_{\{1\}} \to C_{\emptyset})$
and
$(\deet;\dttt,\dhht : C_{\{2\}} \to C_{\emptyset})$
are cat$^1$-groups. 
There are also various axioms in Definition \ref{defn:cat2a} to be checked, 
for example: 
\begin{eqnarray*}
\dtto\circ\ddhht &=& 
  (e_1 \circ t_1 \circ e_2) \circ h_2 
~=~ (e_2 \circ h_2 \circ e_1) \circ t_1 
~=~ \dhht\circ\ddtto, \\
\dtto\circ\ddttt &=& 
  (e_1 \circ t_1 \circ e_2) \circ t_2 
~=~ t_{[2]} 
~=~ (e_2 \circ t_2 \circ e_1) \circ t_1 
~=~ \dttt\circ\ddtto. \\
\end{eqnarray*}

Our third definition defines a cat$^2$-group as a 
"cat$^1$-group of cat$^1$-groups" 
(compare this with Definition \ref{defn:cat1-group}). 

\begin{defn} \label{defn:cat2c} 
A cat$^2$-group $\calC$ consists of two cat$^1$-groups 
$\calC_1 = (e_1;t_1,h_1 : G_1 \to R_1)$ and 
$\calC_2 = (e_2;t_2,h_2 : G_2 \to R_2)$ and cat$^1$-morphisms
$t = (\ddtt,\dtt),\; h = (\ddhh,\dhh) : \calC_1 \to \calC_2,\; 
e = (\ddee,\dee) : \calC_2 \to \calC_1$,  
subject to the following conditions: 
\begin{center}
\begin{tabular}{r l}
\textbf{C1:}  &  $(t \circ e)$ and $(h \circ e)$~ 
                 are the identity mapping on~ $\calC_2$, \\
\textbf{C2:}  &  $[\ker t, \ker h] = \{ 1_{\calC_1} \}$.
\end{tabular}
\end{center}
\end{defn}

Kernels of cat$^1$-mappings have been considered in 
subsection \ref{subs:precat1}. 
The diagram for the situation here is 
\begin{equation*} \label{eq:cat2cker}
\vcenter{\xymatrix{
   \ker\ddtt,\ \ker\ddhh \ar[rr] \ar[dd]<1.0ex> \ar[dd]<0.2ex> 
     && G_1 \ar[dd]<1.0ex>^{t_1,h_1} \ar[dd]<0.2ex> 
            \ar[rr]<1.0ex>^{\ddtt,\ddhh} \ar[rr]<0.2ex> 
          && G_2 \ar[dd]<1.0ex>^{t_2,h_2} \ar[dd]<0.2ex>  
             \ar[ll]<1.0ex>^{\ddee}  \\
     &&  \\
   \ker\dtt,\ \ker\dhh \ar[rr] \ar[uu]<1.0ex> 
     && R_1 \ar[uu]<1.0ex>^{e_1} \ar[rr]<1.0ex>^{\dtt,\dhh} \ar[rr]<0.2ex>
          && R_2 \ar[uu]<1.0ex>^{e_2} 
             \ar[ll]<1.0ex>^{\dee}
}}
\end{equation*}
\medskip
We now need to show that this third definition is equivalent to the first two. 


\newpage
%%%%%%%%%%%%%%%%%%%%%%%%%%%%%%%%%%%%%%%%%%%%%%%%%%%%%%%%%%%%%
\subsection{The cat$^2$-group associated to a crossed square} 
\label{sect:cat2-xsq}


Given a crossed square
\begin{equation} \label{eq:RsquareR}
\vcenter{\xymatrix{
        &   &    R_{[2]} \ar[rr]^{\ddbdyo} \ar[dd]_{\ddbdyt}
             &&  R_{\{2\}} \ar[dd]^{\dbdyt} \\
  \calR & = &&&  \\
        &   &    R_{\{1\}} \ar[rr]_{\dbdyo}
             &&  R_{\emptyset}
}}
\end{equation}
with crossed pairing $\bt: R_{\{1\}} \times R_{\{2\}} \to R_{[2]}$, 
we wish to construct an associated cat$^2$-group.

\begin{prop} \label{prop:semidirect-actions}
For $\calR$ a crossed square (as in Definition \ref{def:xsq}) 
there are group actions of   
$R_{\emptyset} \ltimes R_{\{2\}}$  on $R_{\{1\}} \ltimes R_{[2]}$
and  $R_{\emptyset} \ltimes R_{\{1\}}$  on $R_{\{2\}} \ltimes R_{[2]}$
given by
\begin{eqnarray}
(m,\ell)^{(p,n)} & = &  (m^p, (m^p \bt n)\,\ell^{pn}) ~,  \label{eq:action1}\\
(n,\ell)^{(p,m)} & = &  (n^p, (m \bt n^p)^{-1}\,\ell^{pm}) \label{eq:action2}
\label{eq:semidirect-action}~.
\end{eqnarray}
\end{prop}
\begin{pf}
There are two axioms to be checked for the first identity: 

\medskip
\begin{eqnarray*}
(m_1,\ell_1)^{(p,n)} (m_2,\ell_2)^{(p,n)}  
& = &  (m_1^p, (m_1^p \bt n)\ell_1^{pn})(m_2^p, (m_2^p \bt n)\ell_2^{pn}) \\
& = &  (m_1^pm_2^p, (m_1^p \bt n)^{m_2^p} 
          [\ell_1^{pnm_2^p}(m_2^p \bt n)] \ell_2^{pn}) \\
& = &  (m_1^pm_2^p, (m_1^p \bt n)^{m_2^p} 
          [(m_2^p \bt n) \ell_1^{m_2pn}] \ell_2^{pn}) 
       \qquad\mbox{by Lemma \ref{lem:xsq}(b)} \\
& = &  ((m_1m_2)^p, ((m_1m_2)^p \bt n)(\ell_1^{m_2}\ell_2)^{pn}) \\
& = &  (m_1m_2, \ell_1^{m_2}\ell_2)^{(p,n)} \\
& = &  ((m_1,\ell_1)(m_2,\ell_2))^{(p,n)}~,  \\
&   &  \mbox{} \\
((m,\ell)^{(p_1,n_1)})^{(p_2,n_2)}
& = &  (m^{p_1}, (m^{p_1} \bt n_1)\ell^{p_1n_1})^{(p_2,n_2)} \\
& = &  ((m^{p_1})^{p_2}, (m^{p_1p_2} \bt n_2)
        ((m^{p_1} \bt n_1)\ell^{p_1n_1})^{p_2n_2}) \\
& = &  (m^{p_1p_2}, (m^{p_1p_2} \bt n_2)
        (m^{p_1p_2} \bt {n_1}^{p_2})^{n_2} \ell^{p_1n_1p_2n_2}) \\
& = &  (m^{p_1p_2}, (m^{p_1p_2} \bt {n_1}^{p_2}n_2) 
          \ell^{p_1p_2{n_1}^{p_2}n_2}) \\
& = &  (m,\ell)^{(p_1p_2,{n_1}^{p_2}n_2)} \\
& = &  (m,\ell)^{((p_1,n_1)(p_2,n_2))} ~.
\end{eqnarray*}
The second identity follows using the transpose crossed pairing (\ref{eq:btt}).
\end{pf}

%\bigskip\noindent
%{\bf [Is this the place to introduce a crossed module of cat$^1$-groups?]}

\medskip
In \cite{alp:wens-ijac} we noted that the cat$^1$-group associated to a
crossed module $\calX$ has homomorphisms
$$
t,h : R \ltimes S \to S, \quad t(r,s) = r, \quad h(r,s) = r(\partial s)~.
$$
Applying this construction to $\ddcalRo$ and $\dcalRo$
we obtain a \emph{crossed module of cat$^1$-groups}:

\begin{equation} \label{eq:xmod-of-cat1s}
\vcenter{\xymatrix{
   R_{\{2\}} \ltimes R_{[2]} 
     \ar[rr]<-1.2ex> \ar[rr]<-0.2ex> \ar[dd]_{\barbdyt}
   &&  R_{\{2\}} \ar[dd]^{\dbdyt} 
     \ar[ll]<-1.2ex> \\
   &&  \\
   R_{\emptyset} \ltimes R_{\{1\}} 
     \ar[rr]<+1.2ex> \ar[rr]<+0.2ex> 
   &&  R_{\emptyset}
     \ar[ll]<+1.2ex>
}} 
\end{equation}

\begin{lem}
The action given in Proposition \ref{prop:semidirect-actions} makes 
$(\barbdyt: R_{\{2\}} \ltimes R_{[2]} \to R_{\emptyset} \ltimes R_{\{1\}})$
a crossed module.
\end{lem}
\begin{pf}
{\bf X1:}
\vspace{-5mm}
\begin{eqnarray*}
\barbdyt((n, \ell)^{(p,m)}) 
  & = & \barbdyt(n^p, (m \boxtimes n^p)^{-1} \ell^{pm}) \\
  & = & (\dbdyt(n^p), \ddbdyt(m \boxtimes n^p)^{-1} \ddbdyt(\ell^{pm})) \\
  & = & (\dbdyt(n^p), (m^{-1} m^{\dbdyt n^p})^{-1} \ddbdyt(\ell^{pm})) \\
  & = & (\dbdyt(n^p), (m^{-1})^{(\dbdyt n^p)} m \ddbdyt(\ell^{pm})) \\
  & = & (\dbdyt(n^p), (m^{-1})^{\dbdyt (n^p)} (\ddbdyt \ell^p) m) \\
  & = & (p^{-1}(\dbdyt n)p, (m^{-1})^{p^{-1}(\dbdyt n)p} \ddbdyt (\ell^p) m) \\
  & = & (p^{-1}, (m^{-1})^{p^{-1}}) (\dbdyt n, \ddbdyt \ell) (p,m) \\
  & = & (p,m)^{-1}\;\barbdyt(n, \ell) (p,m)~.
\end{eqnarray*}
{\bf X2:}
\begin{eqnarray*}
(n_1, \ell_1)^{\barbdyt(n_2, \ell_2)} 
  & = & (n_1, \ell_1)^{(\dbdyt n_2, \ddbdyt \ell_2)} \\
  & = & ({n_1}^{\dbdyt n_2}, (\ddbdyt\ell_2 \boxtimes {n_1}^{\dbdyt n_2})^{-1} 
          {\ell_1}^{(\dbdyt n_2)(\ddbdyt \ell_2)}) \\
  & = & ({n_1}^{n_2}, \{(\ddbdyt \ell_2 \boxtimes n_2)
          (\ddbdyt \ell_2 \boxtimes n_1)^{n_2}
          (\ddbdyt \ell_2 \boxtimes n_2^{-1})^{n_1n_2}\}^{-1}  
          {\ell_1}^{(\dbdyt n_2)(\ddbdyt \ell_2)}) \\
  & = & ({n_1}^{n_2}, ((\ell_2^{-1} {\ell_2}^{\dbdyt n_2})
          (\ell_2^{-1} {\ell_2}^{\dbdyt n_1})^{\dbdyt n_2} 
          (\ell_2^{-1} {\ell_2}^{\dbdyt n_2^{-1}})^{\dbdyt(n_1n_2))})^{-1} 
          {\ell_1}^{(\dbdyt n_2)(\ddbdyt \ell_2)} \\
  & = & ({n_1}^{n_2}, ((\ell_2^{-1})^{n_2^{-1}n_1n_2} \ell_2) 
          (\ell_2^{-1} {\ell_1}^{\dbdyt n_2} \ell_2)) \\
  & = & ({n_1}^{n_2}, (\ell_2^{-1})^{n_2^{-1}n_1n_2} 
          {\ell_1}^{n_2} \ell_2 ) \\
  & = & (n_2^{-1}n_1n_2, (\ell_2^{-1})^{n_2^{-1} n_1} \ell_1)(n_2, \ell_2) \\
  & = & (n_2, \ell_2)^{-1}(n_1, \ell_1)(n_2, \ell_2) 
\end{eqnarray*} 
\end{pf}

\bigskip\noindent
We may then construct a \emph{cat$^1$-group of cat$^1$-groups}
where the required homomorphisms are:
\begin{eqnarray} 
\label{eq:cat2-th} 
\ddttt,\,\ddhht \;:\; 
(R_{\emptyset} \ltimes R_{\{1\}}) \ltimes (R_{\{2\}} \ltimes R_{[2]})
  & \to & R_{\emptyset} \ltimes R_{\{1\}} \\
\nonumber
\ddttt((p,m),(n,l))
  & \;=\; & (p,m) \\
\nonumber
\ddhht((p,m),(n,l))
  & \;=\; & (p,m)\,\barbdyt(n,l) \;=\; (p,m)(\dbdyt n,\ddbdyt\ell)
              \;=\;  (p(\dbdyt n),\,m^{\dbdyt n}(\ddbdyt\ell)) \\
\label{eq:cat2-e}
\mbox{and} \quad\quad \ddeet \;:\; R_{\emptyset} \ltimes R_{\{1\}}
  & \to &  (R_{\emptyset} \ltimes R_{\{1\}}) 
            \ltimes (R_{\{2\}} \ltimes R_{[2]}) \\
\nonumber
\ddeet(p,m)
  & \;=\; & ((p,m),(1,1))~.
\end{eqnarray}

\bigskip
We now check that
$(\ddeet;\ddttt,\ddhht :  
   (R_{\emptyset} \ltimes R_{\{1\}}) \ltimes (R_{\{2\}} 
   \ltimes R_{[2]}) \to R_{\emptyset} \ltimes R_{\{1\}} )$
is a cat$^1$-group.
Note that
\begin{eqnarray*}
\ker \ddttt  & = & 
\{~((1,1),(n,\ell))~\}   \\
\ker \ddhht  & = &
\{~((p,m),(n,\ell))~\}  \quad\mbox{where}\quad
p = (\dbdyt n)^{-1}       \;\;\mbox{and}\;\;
m = ((\ddot{\nu} \ell)^{-1})^{p},
\end{eqnarray*}
so that
$$
\ell^{pn} ~=~ \ell^{(\dbdyt n)^{-1}(\dbdyt n)} ~=~ \ell
\qquad\mbox{in}\quad \ker \ddhht.
$$
The formula for the action is given by equation (\ref{eq:semidirect-action}): 
$$
(n,\ell)^{(p,m)} ~:=~ (~n^p,\ (m \bt n^p)^{-1}\ \ell^{pm}~)\,.
$$

\medskip\noindent
We now check the cat$^1$-group axioms.

\medskip\noindent
\textbf{C1:}
\vspace{-8mm}
\begin{eqnarray*}
\ddttt \ddeet(p,m)
   & = & \ddttt((p,m),(1,1)) ~~ = ~~ (p,m) \\
\ddhht \ddeet(p,m) 
   & = & \ddhht((p,m),(1,1)) ~~ = ~~ (p,m) \\
\end{eqnarray*}

\medskip\noindent
\textbf{C2:} 
\vspace{-4mm}
\begin{eqnarray*} 
((1,1),(n_0, \ell_0))\,((p,m), (n, \ell)) 
  & = & ((p, m),(n_0,\ell_0)^{(p,m)}(n,\ell))\\
  & = & ((p, m),({n_0}^p,(m \bt {n_0}^p)^{-1} {\ell_0}^{pm}) (n,\ell))\\
  & = & ((p, m),({n_0}^{(\dbdyt n)^{-1}} n,((m \bt {n_0}^p)^{-1})^n
           {\ell_0}^{pmn} \ell))\\
  & = & ((p, m),(n n_0 n^{-1} n,((\ddbdyt\ell^{-1} \bt n_0)^{-1})^{pn}\,
           {\ell_0}^{pmn} \ell))\\
  & = & ((p, m),(n n_0, (((\ell^{-1})^{-1}(\ell^{-1})^{n_0})^{-1})^{pn})
           {\ell_0}^{pmn} \ell))\\
  & = & ((p, m),(n n_0,(\ell^{n_0} \ell^{-1})^{pn}\,{\ell_0}^{pmn} \ell))\\
  & = & ((p, m),(n n_0,(\ell^{n_0} \ell^{-1})\,
           {\ell_0}^{p(\ddbdyt \ell^{-1})^p n} \ell))\\
  & = & ((p, m),(n n_0,(\ell^{n_0} \ell^{-1})\,
           {\ell_0}^{(\ddbdyt \ell^{-1})} \ell))\\
  & = & ((p, m),(n n_0,(\ell^{n_0} \ell^{-1})\,\ell {\ell_0} \ell^{-1} \ell))\\
  & = & ((p, m),(n n_0,\ell^{n_0}\,{\ell_0})) \\
  & = & ((p, m),(n, \ell)(n_0, \ell_0)) \\
  & = & ((p, m),(n, \ell))\,((1,1),(n_0, \ell_0))  
\end{eqnarray*}

\medskip
\begin{thm}
The homomorphisms in (\ref{eq:cat2-th}) and (\ref{eq:cat2-e}) 
give a cat$^2$-group:
\begin{equation} \label{eq:cat2-sdp}
\vcenter{\xymatrix{
 & (R_{\emptyset} \ltimes R_{\{1\}}) \ltimes (R_{\{2\}} \ltimes R_{[2]})
      \ar[ddd] <-1.2ex>  \ar[ddd] <-2.0ex>_{\ddttt,\ddhht}
      \ar[rrr] <+1.2ex>  \ar[rrr] <+2.0ex>^(0.7){\ddtto,\ddhho}
      \ar[dddrrr] <-0.2ex>  \ar[dddrrr] <-1.0ex>_{\ttot,\hhot}
    &&&  R_{\emptyset} \ltimes R_{\{2\}}
            \ar[lll]^(0.3){\ddeeo}
            \ar[ddd]<+1.2ex>  \ar[ddd] <+2.0ex>^{\dttt,\dhht}  \\
\calC(\calR) \quad = 
 &   &&&   \\
 &   &&&   \\
 & R_{\emptyset} \ltimes R_{\{1\}}
    \ar[uuu]_{\ddeet}
    \ar[rrr] <-1.2ex>  \ar[rrr] <-2.0ex>_{\dtto,\dhho} 
    &&&  R_{\emptyset} \ar[uuu]^{\deet}   \ar[lll]_{\deeo} 
           \ar[uuulll] <-1.0ex>_{\eeot}
 \\
}} 
\end{equation}
\end{thm}
\begin{pf}
To be added.  (Is material required from Section 7.2~?)
\end{pf}


\vspace*{5mm} 
%%%%%%%%%%%%%%%%%%%%%%%%%%%%%%%%%%%%%%%%%%%%%%%%%%%%%%%%%%%%%%%%%%%%
\subsection{The other cat$^2$-structure}

Now the underlying diagram (\ref{eq:Rsquare}) of the crossed square $\calR$, 
together with the crossed pairing 
$$
\btt : R_{\{2\}} \times R_{\{1\}} \to R_{[2]}, \ 
               (n \btt m) = (m \bt n)^{-1}
$$
forms a second crossed square $\tilde{\calR}$. 
% (see Lemma \ref{prop:princ}).
Thus we can form a second cat$^2$-group $\calC(\tilde{\calR})$ with 
$$
\ddtto, \ddhho : (R_{\emptyset} \ltimes R_{\{2\}}) \ltimes (R_{\{1\}} 
            \ltimes R_{[2]}) \to (R_{\emptyset} \ltimes R_{\{2\}})~.
$$

Let
$$
\tilde{C} = (R_{\emptyset} \ltimes R_{\{2\}}) \ltimes (R_{\{1\}} 
       \ltimes R_{[2]}), \qquad
C = (R_{\emptyset} \ltimes R_{\{1\}}) \ltimes (R_{\{2\}} \ltimes R_{[2]}),
$$
and
$$
P  = R_{\emptyset} \ltimes R_{\{2\}}~, \quad 
\tilde{P} = R_{\emptyset} \ltimes R_{\{1\}}~.
$$


\begin{prop} \label{prop:tau}
There is an isomorphism between these two semidirect products: 
\begin{eqnarray}
\label{eq:cat1-iso}
\tau ~:~ C \to \tilde{C}, &&
((p,m),(n,\ell)) \mapsto ((p,n),(m,(m \bt n)\ell))~, \\
\nonumber
\tilde{\tau} ~:=~ \tau^{-1} ~:~ \tilde{C} \to C, &&
((p,n),(m,\ell)) \mapsto ((p,m),(n,(n \btt m) \ell))~.
\end{eqnarray}
\end{prop}
\begin{pf}
\vspace{-6mm}
\begin{eqnarray*}
 &   &  \tau((p_1,m_1),(n_1,\ell_1))((p_2,m_2),(n_2,\ell_2)) \\
 & = &  \tau((p_1,m_1)(p_2,m_2),
                  (n_1,\ell_1)^{(p_2,m_2)}(n_2,\ell_2)) \\
 & = &  \tau((p_1p_2,{m_1}^{p_2}m_2),
    ({n_1}^{p_2},(m_2 \bt {n_1}^{p_2})^{-1} {\ell_1}^{p_2m_2})(n_2,\ell_2)) \\
 & = &  \tau((p_1p_2,{m_1}^{p_2}m_2),
                  ({n_1}^{p_2}n_2,((m_2 \bt {n_1}^{p_2})^{-1})^{n_2} 
                    {\ell_1}^{p_2m_2n_2} \ell_2)) \\
 & = &  ((p_1p_2,{n_1}^{p_2}n_2),
         ({m_1}^{p_2}m_2,({m_1}^{p_2}m_2 \bt {n_1}^{p_2}n_2)
          ((m_2 \bt {n_1}^{p_2})^{-1})^{n_2}
           ({\ell_1}^{p_2})^{m_2n_2} \ell_2)) \\
 & = &  ((p_1p_2,{n_1}^{p_2}n_2), ({m_1}^{p_2},({m_1}^{p_2} \bt n_2)
          (m_1 \bt n_1)^{p_2n_2} {\ell_1}^{p_2n_2})
           (m_2,(m_2 \bt n_2)\ell_2)) \\
 & = &  ((p_1,n_1)(p_2,n_2), (m_1,(m_1 \bt n_1)\ell_1)^{(p_2,n_2)}
           (m_2,(m_2 \bt n_2)\ell_2)) \\
 & = &  ((p_1,n_1),(m_1,(m_1 \bt n_1)\ell_1))
          ((p_2,n_2),(m_2,(m_2 \bt n_2)\ell_2)) \\
 & = &  \tau((p_1,m_1),(n_1,\ell_1))\ \tau((p_2,m_2),(n_2,\ell_2))
\end{eqnarray*}
\end{pf}

\noindent
Note that the subgroup 
$(1 \ltimes R_{\{1\}}) \ltimes (R_{\{2\}} \ltimes 1)$
does \emph{not} in general get mapped by $\tau$ to the subgroup 
$(1 \ltimes R_{\{2\}}) \ltimes (R_{\{1\}} \ltimes 1)$. 


\bigskip
The isomorphism $\tau$ provides a neat formula for the inverse of a 
general element in $(P \ltimes N) \ltimes (M \ltimes L)$.
\begin{lem}
$$
((p,n),(m,\ell))^{-1} ~=~ 
\tau((p^{-1},(m^{-1})^{p^{-1}}),~ 
 ((n^{-1})^{p^{-1}},(\ell^{-1})^{m^{-1}n^{-1}p^{-1}}))
$$
\end{lem}
\begin{pf}
\begin{eqnarray*}
((p,n),(m,\ell))^{-1} 
 &=& ((p^{-1},(n^{-1})^{p^{-1}}),~
        (m^{-1},(\ell^{-1})^{m^{-1}})^{(p^{-1},(n^{-1})^{p^{-1}})})) \\
 &=& ((p^{-1},(n^{-1})^{p^{-1}}),~
        ((m^{-1})^{p^{-1}},
          ((m^{-1})^{p^{-1}} \bt (n^{-1})^{p^{-1}}) 
           (\ell^{-1})^{m^{-1}n^{-1}p^{-1}})) \\
 &=& \tau((p^{-1},(m^{-1})^{p^{-1}}),~ 
          ((n^{-1})^{p^{-1}},(\ell^{-1})^{m^{-1}n^{-1}p^{-1}}))) 
\end{eqnarray*}
\end{pf}

\begin{lem}
The commutator of elements in $\ddeeo(P \ltimes N)$ 
and $\ddeet(P \ltimes M)$ is 
$$
 [((p,n),(1,1)),~((q,1),(m,1))]
~=~
 (([p,q],n^{-1}n^q), 
      ((m^{-1})^{q^{-1}pq}m,((m^{-1})^{q^{-1}pq}\bt n^q)^m))
$$
\end{lem}
\begin{pf}
\vspace*{-8mm}
\begin{eqnarray*}
 & & [((p,n),(1,1)),~((q,1),(m,1))] \\
 &=& ((p^{-1},(n^{-1})^{p^{-1}}),(1,1))~
     ((q^{-1},1),((m^{-1})^{q^{-1}},1))~ 
     ((p,n),(1,1))~
     ((q,1),(m,1)) \\
 &=& ((p^{-1}q^{-1},(n^{-1})^{p^{-1}q^{-1}}),((m^{-1})^{q^{-1}},1))~ 
     ((pq,n^q),(m,1)) \\
 &=& ((p^{-1}q^{-1}pq,n^{-1}n^q), 
      ((m^{-1})^{q^{-1}pq},((m^{-1})^{q^{-1}pq}\bt n^q))(m,1)) \\
 &=& (([p,q],n^{-1}n^q), 
      ((m^{-1})^{q^{-1}pq}m,((m^{-1})^{q^{-1}pq}\bt n^q)^m)) 
\end{eqnarray*}
\end{pf}



%\begin{prop}[Loday \cite{loday1}, Ellis \cite{ellis-thesis}]
%\mbox{}\\
%Let ${\ddtto}^{\prime}, {\ddhho}^{\prime} : C_2 \to P_1$ 
%be the composite maps 
%${\ddtto}^{\prime} = \tau \ddtto \tau^{-1},~ 
% {\ddhho}^{\prime} = \tau \ddhho \tau^{-1}$. 
%The four maps ${\ddtto}^{\prime}, {\ddhho}^{\prime}, \ddttt, \ddhht$
%constitute a cat$^2$-group.
%\end{prop}

