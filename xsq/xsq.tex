% xsq.tex,  version 08/11/19

%%%%%%%%%%%%%%%%%%%%%%%%%%%%%%%%%%%%%%%%%%
\section{Crossed Squares and Cat$^2$-groups} \label{sect:xsq-cat2}

Crossed squares were introduced by Guin-Wal\'ery and Loday 
(see, for example, \cite{walery:loday,loday1,brow:lod})
as fundamental crossed squares of commutative squares of spaces,
but are also of purely algebraic interest.
%We denote by $[n]$ the set $\{1,2,\ldots,n\}$.
%We use the $n=2$ version of the definition of crossed $n$-cube
%as given by Ellis and Steiner \cite{ell:st}.

\begin{defn} \label{def:xsq}  \index{crossed square}
A crossed square consists of the following:
\begin{enumerate}[{\rm (i)}]
\item
a commutative diagram of group homomorphisms
\begin{equation} \label{eq:Ssquare}
\vcenter{\xymatrix{
       &   &  L \ar[rr]^{\kappa} \ar[dd]_{\lambda} 
              && M \ar[dd]^{\mu} &   \\
\calS  & = &  &&                 & ; \\
       &   &  N \ar[rr]_{\nu}  
              && P 
}} 
\end{equation}
\item
actions of $P$ on $N, M$ and $L$ 
which determine actions of 
$N$ on $M$ and $L$ via $\nu$ 
and actions of 
$M$ on $N$ and $L$ via $\mu$~;
\item
a function ~ $\bt : N \times M \to L$~.
\end{enumerate}
The following axioms must be satisfied for all 
$l \in L,\; m,m_1,m_2 \in M,\; 
n,n_1,n_2 \in N,\; p \in P$~:
\begin{enumerate}[{\rm (a)}]
\item 
the homomorphisms $\kappa, \lambda$ preserve the action of $P$~;
\item 
each of 
$~\calK = (\kappa : L \to M),
 ~\calL = (\lambda : L \to N), 
 ~\calM  = (\mu : M \to P),
 ~\calN  = (\nu : N \to P)~$
and the diagonal  
$~\calD = (\delta = \nu\circ\lambda = \mu\circ\kappa : L \to P)~$ 
are crossed modules;
\item 
$\bt$ is a \emph{crossed pairing}:
\begin{enumerate}[{\rm (i)}]
\item
$(n_1n_2 \bt m)\;=\;{(n_1 \bt m)}^{n_2}\;(n_2 \bt m)$~,
\item
$(n \bt m_1m_2) \;=\; (n \bt m_2)\;{(n \bt m_1)}^{m_2}$~,
\item 
$(n \bt m)^{p} \;=\; (n^p \bt m^p)$~;
\end{enumerate}
\item\quad
$\lambda (n \bt m) \;=\; n^{-1}\;n^m$
\quad \mbox{and} \quad
$\kappa (n \bt m) \;=\; (m^{-1})^n\;m$~, 
\item\quad
$(\lambda l \bt m) \;=\; l^{-1}\;l^m$ 
\quad \mbox{and} \quad
$(n \bt \kappa l) \;=\; (l^{-1})^n\;l$~. 
\end{enumerate}
\end{defn}

Note that the actions of  $N$  on  $M$  and  $M$  
on  $N$  via  $P$
are compatible since
$$
{m_1}^{(n^m)} \;=\; {m_1}^{\nu(n^m)} \;=\; {m_1}^{m^{-1}(\nu n)m}
\;=\; (({m_1}^{m^{-1}})^n)^m~.
$$
Note also that identities c(i) and c(ii) are based on the commutatort identities 
$$
[n,m] ~=~ n^{-1}m^{-1}nm, \qquad 
[n_1n_2,m] ~=~ [n_1,m]^{n_2}[n_2,m], \qquad 
[n,m_1m_2] ~=~ [n,m_2][n,m_1]^{m_2} .
$$

\begin{lem} 
$$
1 = (1 \bt m) = (n \bt 1 ) 
\qquad\mbox{and}\qquad 
(n \bt m)^{-1} = (n^{-1} \bt m)^n = (n \bt m^{-1})^m. 
$$
\end{lem}
\begin{pf}
These follow immediately on expanding 
$(n.1 \bt m),\ (n \bt m.1),\ (n^{-1}n \bt m)$ and $(n \bt m^{-1}m)$. 
\end{pf}

\begin{lem}
Pairs $(\kappa,\nu) : \calL \to \calM$ and  $(\lambda,\mu) = \calK \to \calN$ 
are morphisms of crossed modules, and so also are
$(\id_{L},\nu) : \calL \to \calD$,~ $(\id_{L},\mu) : \calK \to \calD$,~
$(\lambda,\id_{P}) : \calD \to \calN$  and  $(\kappa,1_{P}) : \calD \to \calM$.
\end{lem}
\begin{pf}
For $(\kappa,\nu)$ we note that (\ref{eq:Ssquare}) commutes,
that  $\kappa(\ell^m) = (\kappa l)^m$  by (a),
and that  $(\kappa \ell)^n = (\kappa \ell)^{\nu n}$  by (ii).
The arguments for the other five morphisms are similar.
\end{pf}

\medskip\noindent
Note in particular that
$$
\kappa(\ell^p) = (\kappa \ell)^p
\qquad \mbox{and} \qquad
\lambda(\ell^p) = (\lambda \ell)^p.
$$

\begin{lem} \label{lem:xsq}
In the crossed square  $\calS$  above:
\begin{enumerate}[{\em (a)}]
\item~
\vspace*{-5mm} 
$$
\ell^{(n \bt m)} \quad = \quad \ell^{[n,m]}\,,
$$
\item~
\vspace*{-5mm} 
$$
\ell^{mn}\,(n \bt m) ~=~ (n \bt m)\,\ell^{nm}
\qquad\mbox{and}\qquad
\ell^{(m^n)}\,(n \bt m) ~=~ (n \bt m)\,\ell^{m}\,, 
$$
\item~
\vspace*{-5mm} 
$$
m^{\lambda\ell} = m^{\kappa\ell}
\qquad\mbox{and}\qquad 
n^{\lambda\ell} = n^{\kappa\ell}\,,
$$
\item~
\vspace*{-5mm} 
$$
(n \bt m)^{m'n'} (n' \bt m') ~=~ (n' \bt m')(n \bt m)^{n'm'}\,. 
$$
\end{enumerate}
\end{lem}
\begin{pf}
\begin{enumerate}[(a)]
\item 
By the crossed module identity {\bf (X2)}, 
$$
\ell^{(n \bt m)}
  \;=\;  \ell^{\nu\lambda(n \bt m)}
  \;=\;  \ell^{\nu(n^{-1}n^m)}
  \;=\;  \ell^{\nu(n^{-1}n^{\mu m})}
  \;=\;  \ell^{[\nu n, \mu m]}
  \;=\;  \ell^{[n,m]}~.
$$
\item
The first identity is given by
$$
(n \bt m)^{-1}\,\ell^{mn}\,(n \bt m) \;=\;
\ell^{mn\lambda(n \bt m)} \;=\;
\ell^{mn({n^{-1}})n^m} \;=\;
\ell^{nm}~.
$$
Then replace $\ell$ by $\ell^{n^{-1}}$ to get the second.
\item
$$
m^{\lambda\ell} ~=~ m^{\nu\lambda\ell} ~=~ m^{\mu\kappa\ell} 
~=~  m^{\kappa\ell}~, \qquad\mbox{and similarly for}~ n. 
$$ 
\item
Expand $(nn' \bt mm')$ in two different ways. 
\end{enumerate}
\end{pf}


%%%%%%%%%%%%%%%%%%%%%%%%%%%%%%%%%%%%%%%%
\subsection{Examples of crossed squares}

\begin{example}
If $M, N$ are normal subgroups of the group $P$ 
then the diagram of inclusions 
$$
\xymatrix{
M \cap N \ar[rr]^(0.6){i_1} \ar[dd]_{i_2} \ar[rrdd]^{i_{12}} 
  && M \ar[dd]^{i_M} \\
       \\
N \ar[rr]_{i_N}  
  && P 
}
$$
together with the actions of $P$ on $M, N$ 
and $M \cap N$ given by conjugation and the function 
$$
\bt \;:\; N \times M \to M \cap N, \quad (n,m) \mapsto [n,m]\,=\,n^{-1}m^{-1}nm
$$
is a crossed square.
We may check the axioms as follows: 
\begin{enumerate}[{\rm (a)}]
\item
The identity maps preserve $P$-actions.
\item
The five crossed modules are all conjugation crossed modules.
\item 
We have already noted that (i) and (ii) are analogues of commutator identities. 
For (iii), 
$$
(n^p \bt m^p) \;=\; [n^p,m^p]
  \;=\;  \{p^{-1}np\}^{-1}\,\{p^{-1}mp\}^{-1}\,\{p^{-1}np\}\{p^{-1}mp\}  \\
 %\;=\;  p^{-1}m^{-1}p\,p^{-1}n^{-1}p\,p^{-1}mp\,p^{-1}np \\
  \;=\;  p^{-1}[n,m]p \;=\; [n,m]^p\,.
$$
\item\quad
$\ddiio (n \bt m) \;=\; n^{-1}\,(m^{-1}nm) \;=\; n^{-1}\,n^m$
\quad \emph{and} \quad
$\ddiit (n \bt m) \;=\; (n^{-1}m^{-1}n)m \;=\; {m^{-1}}^n\,m$~.
\item\quad
$(n \bt i_1 l) \;=\; (n^{-1}l^{-1}n)l \;=\; (l^{-1})^n\,l$
\quad \emph{and} \quad
$(i_2 l \bt m) \;=\; l^{-1}\,(m^{-1}lm) \;=\; l^{-1}\,l^m$~.
\end{enumerate}
\end{example}

%\medskip
\begin{example}
If $M, N$ are ordinary $P-$modules and $A$ is an arbitrary abelian group
on which $P$ is assumed to act trivially, 
then there is a crossed square 
$$
\xymatrix{
A \ar[rr]^{0} \ar[dd]_{0} \ar[rrdd]^{0}  
  && M \ar[dd]^{0} \\
  &&               \\
N \ar[rr]_{0} 
  && P 
}
$$
Note that $M$ acts trivially on $N$, and conversely, 
and that $n \bt m = 1_A$. 
\end{example}

%\medskip
\begin{example}
The diagram 
$$
\xymatrix{
M \ar[rr]^{\alpha} \ar[dd]_{\alpha}  && \Inn M \ar[dd]^{\iota} \\
       \\
\Inn M \ar[rr]_{\iota}  && \Aut M 
}
$$
is a crossed square, where $\alpha$ maps $m \in M$
to the inner automorphism  $\beta_m : M \to M,\; 
m^{\prime} \mapsto m^{-1}m^{\prime}m$~; 
where $\iota$ is the inclusion of $\Inn M$  in  $\Aut M$;
the actions are standard; and the crossed pairing is 
$$
\bt \;:\; \Inn M \times \Inn M \to M, \quad
(\beta_m, \beta_{m^{\prime}}) \;\mapsto\; [m, m^{\prime}]~.
$$
\end{example}

%\medskip
\begin{example}
If $U, V$ are subspaces of a space $X$ with a point $x_0$ in common, 
then the diagram of boundary maps 
$$
\xymatrix{
\pi_3(X;U,V, x_0) \ar[rr] \ar[dd]  && \pi_2(V, U \cap V, x_0) \ar[dd] \\
       \\
\pi_2(U, U \cap V, x_0) \ar[rr] && \pi_1(U \cap V, x_0) 
}
$$
in which $\pi_3(X;U,V,x_0)$ is the triad homotopy group, 
together with the standard actions and the triad Whitehead product
$$
\bt \;:\; 
 \pi_2(U, U \cap V, x_0) \times \pi_2(V, U \cap V, x_0) 
 \;\to\; \pi_3(X;U, V, x_0)
$$
is a crossed square. 
\end{example}

\begin{lem} \label{lem:Stranspose}
The \emph{transpose}
$$
\xymatrix{
  &  L \ar[rr]^{\lambda} \ar[dd]_{\kappa} 
     &&  N \ar[dd]^{\nu} \\
\tilde{\calS} \quad = 
  &  &&                             \\
  &  M \ar[rr]_{\mu} 
     && P 
}
\xymatrix{
  &&  L \ar[rr]^{\kappa} \ar[dd]_{\lambda} 
     &&  M \ar[dd]^{\mu} \\
  & \mbox{of} \qquad \calS \quad =  
   & &&                             \\
  &&  N \ar[rr]_{\nu} 
     &&  P 
}
$$
is a crossed square with crossed pairing
\begin{equation} \label{eq:btt}
\btt \;:\; M \times N \to L, \quad 
(m,n) \;\mapsto\; m \btt n := (n \bt m)^{-1}~.
\end{equation}
\end{lem}
\begin{pf}
\begin{eqnarray*}
m_1m_2 \btt n
  & = & (n \bt m_1m_2)^{-1}
  \;=\; ((n \bt m_2)\,(n \bt m_1)^{m_2})^{-1} \\
  & = & ((n \bt m_1)^{m_2})^{-1}\,(n \bt m_2)^{-1}
  \;=\; (m_1 \btt n)^{m_2}\,(m_2 \btt n)~;\\
m \btt n_1n_2
  & = & (n_1n_2 \bt m)^{-1}
  \;=\; ((n_1 \bt m)^{n_2}\,(n_2 \bt m))^{-1} \\
  & = & ((n_2 \bt m)^{-1}\,((n_1 \bt m)^{n_2})^{-1}
  \;=\; (m \btt n_2)\,(m \btt n_1)^{n_2}~;\\ 
(m \btt n)^p 
  & = & ((n \bt m)^{-1})^p 
  \;=\; ((n \bt m)^p)^{-1} 
  \;=\; (n^p \bt m^p)^{-1} 
  \;=\; (m^p \btt n^p)~. 
\end{eqnarray*}
\end{pf}

%\medskip
\begin{example} \label{ex:actor-square} 
The actor crossed module $\calA(\calX)$ of a crossed module $\calX$ 
(see subsection \ref{subs:AX}) 
\begin{equation} \label{eq:actor}
\vcenter{\xymatrix{ 
               &  S \ar[rr]^(0.45){\ddi} \ar[dd]_{\partial}
                  &&  W \ar[dd]^{\Delta}  \\
\calA \qquad =  &     &&  \\
               &  R \ar[rr]_(0.45){\di}
                  &&  A
}} 
\end{equation}
is a crossed square with crossed pairing 
$$
\bt \;:\;  R \times W \,\to\, S, \quad
(r,\chi) \,\mapsto\, \chi r~.
$$
We already know that the square  $\calA$  contains $5$ crossed modules,
but we still need to check the axioms (c), (d), and (e) 
which involve the crossed pairing: 
\begin{enumerate}[{\rm (a)}]
%\item
%(Find some way of passing by the first two items.)
%\item
\setcounter{enumi}{2}
\item
 \begin{enumerate}[{\rm (i)}]
 \item
  $(qr \bt \chi) 
  \;=\; \chi(qr)
  \;=\; (\chi q)^r\,(\chi r)
  \;=\; (q \bt \chi)^r\,(r \bt \chi)$~.
 \item
  $(r \bt \chi_1\star\chi_2)
  \;=\; (\chi_2 r)(\chi_1 r)(\chi_2\partial\chi_1 r)
  \;=\; (\chi_2 r)\,(\chi_1 r)^{\chi_2}
  \;=\; (r \bt \chi_2)\,(r \bt \chi_1)^{\chi_2}$~,
  \begin{flushright}
    using the action of $W$ on $S$ given in Lemma \ref{lem:deriv-act}. 
  \end{flushright}
 \item
  $(r^{\beta} \bt \chi^{\beta})
   \; = \; \chi^{\beta}(r^{\beta}) 
   \; = \; (\ddb \chi \db^{-1})(\db r) 
   \; = \; \ddb(\chi r) 
   \; = \; (r \bt \chi)^{\beta}$~.
 \end{enumerate}
\item
The first formula follows by\\
\hspace*{4mm}$\partial (r \bt \chi) 
  \;=\; \partial \chi(r) 
  \;=\; r^{-1} r (\partial \chi r) 
  \;=\; r^{-1} (\db_{\chi} r) 
  \;=\; r^{-1} r^{\chi}$~.\\ 
For the second formula, 
since $\ddi(r \bt \chi) = \ddi(\chi r) = \eta_{\chi r}$, we wish to prove that
$$
\eta_{\chi r} = (\chi^{-1})^r \star \chi
\quad\mbox{or, equivalently,}\quad
\chi^r = \chi \star (\eta_{\chi r})^{-1} = \chi \star \eta_{(\chi r)^{-1}}\,.
$$
Starting with the right-hand side, 
\begin{eqnarray*}
(\chi \star \eta_{(\chi r)^{-1}})q
 & = & (\eta_{(\chi r)^{-1}}q)(\ddb_{\eta_{(\chi r)^{-1}}} \chi q) 
        \hspace{62mm} \mbox{ by Lemma \ref{lem:gamma-beta-chi} (c)} \\
 & = & (\chi r)^q (\chi r)^{-1} (\chi q)^{(\chi r)^{-1}} 
        \hspace{61mm} \mbox{ by Lemma \ref{lem:princ-prop} (b)} \\
 & = & (\chi r)^q (\chi q) (\chi r)^{-1} 
 \;=\; (\chi r)^q (\chi q) (\chi r^{-1})^r 
        \hspace{34mm} \mbox{ by Lemma \ref{lem:invchir} (b)}    \\
 & = & \ddb_r((\chi r)^{qr^{-1}} (\chi q)^{r^{-1}} (\chi r^{-1})) 
 \;=\; \ddb_r \chi(rqr^{-1})
   ~=~ (\ddb_r\chi\db_r^{-1})q
        \hspace{2mm} \mbox{ by Lemma \ref{lem:gamma-beta-chi} (c)} \\
 & = & \chi^{\di r}q
   ~=~ \chi^r q\,.
\end{eqnarray*}
\item
\quad $(\partial s \bt \chi) 
  \;=\; \chi(\partial s) 
  \;=\; s^{-1} s (\chi\partial s) 
  \;=\; s^{-1} (\ddb_{\chi} s) 
  \;=\; s^{-1} s^{\chi}$~, \\
\medskip
\hspace*{4mm}$(r \bt \ddi(s)) \;=\; \eta_{s}(r) \;=\; (s^{-1})^r s$
  \hspace{74mm} by Lemma \ref{lem:gamma-eta_s}. \\
\end{enumerate}
\end{example}

\vspace*{12mm}
%% \newpage
%%%%%%%%%%%%%%%%%%%%%%%%%%%%%%%%%%%%%%%%%%%%%%%%%%%%%%%%%%%%%%%%%%%%%%%%%%%
\subsection{Morphisms of crossed squares} 
\index{morphism!of crossed squares} 

A morphism $\theta : \calS_1 \to \calS_2$ of crossed squares 
is a $4$-tuple of group homomorphisms which commute with the
morphisms in $\calS_1$ and $\calS_2$ and preserve all the actions
and the crossed pairings.

\begin{defn}
A morphism $\theta : \calS_1 \to \calS_2$ of crossed squares
consists of four group homomorphisms
$$
\theta_L : L_1 \to L_2, \quad
\theta_M : M_1 \to M_2, \quad
\theta_N : N_1 \to N_2, \quad
\theta_P : P_1 \to P_2,
$$
forming a commutative cube with the morphisms 
$\kappa_1, \lambda_1, \mu_1, \nu_1$ of $\calS_1$
and $\kappa_2, \lambda_2, \mu_2, \nu_2$ of $\calS_2$,
which pair off in appropriate ways to form crossed module morphisms
$$
(\theta_L,\theta_M) : \calK_1 \to \calK_2, \quad
(\theta_L,\theta_N) : \calL_1 \to \calL_2, \quad
(\theta_M,\theta_P) : \calM_1 \to \calM_2, \quad
(\theta_N,\theta_P) : \calN_1 \to \calN_2,
$$
and which preserve the crossed pairing: 
$$
\theta_L(n \bt_1 m) ~=~ 
(\theta_N n) \bt_2 (\theta_M m).
$$
\end{defn}

\begin{defn} \label{def:aut-xsq} 
\index{automorphism group!of a crossed square}
The group $\Aut(\calS)$ of automorphisms of the crossed square $\calS$ is 
$$
\Aut(\calS) ~=~ \{ \alpha = 
(\alpha_L, \alpha_M, \alpha_N, \alpha_P) ~:~ \calS \to \calS \}
$$ 
such that 
$(\alpha_L, \alpha_M) $  is an automorphism of  $\calK$,
$(\alpha_L, \alpha_N) $  is an automorphism of  $\calL$,
$(\alpha_M, \alpha_P) $  is an automorphism of  $\calM$, 
$(\alpha_N, \alpha_P) $  is an automorphism of  $\calN$, and 
$\alpha_L(n \bt m) ~=~ (\alpha_N n) \bt (\alpha_M m)$. 
\end{defn}

\vspace*{15mm}
\noindent
{\bf [Surely there is a crossed square version of Theorem \ref{thm:tp-top}~?]}

\begin{thm}
Every crossed square is a quotient of normal inclusion crossed squares.~ 
{\bf ?}
\end{thm}

\noindent
(Note: notes 10/7/03 only go one way.)


%%%%%%%%%%%%%%%%%%%%%%%%%%%%%%%%%%%%%%%%%%%%%%%%%%%%%%%%%%%%%%%%%%%%%%%%%%%
\newpage
\subsection{Cat$^2$-groups}  \label{subs:cattwo}
\index{cat$^2$-group} 

When defining a cat$^2$-group we may require all the homomorphisms to be 
endomorphisms, as in Definition \ref{defn:cat1-group1}, 
or we may take a more general view, as in Definition \ref{defn:cat1-group2}. 
For now, we consider only endomorphisms.  
When we come to define cat$^n$-groups we shall give a similar set of definitions. 
Firstly, we give the definition of a \emph{traditional} cat$^2$-group adapted from 
Section 5 of Brown and Loday \cite{brow:lod} and Ellis-Steiner \cite{ell:st}.

\begin{defn} \label{defn:cat2a} 
\emph{A cat$^2$-group  $\calC = (G;\tau_1,\theta_1,\tau_2,\theta_2)$  
comprises a group $G$ and $4$ endomorphisms, as shown in the following diagram, 
where $Q_1 = \im\tau_1 = \im\theta_1$, $Q_2 = \im\tau_2 = \im\theta_2$ 
and $Q_0 = Q_1 \cap Q_2$, }
\begin{equation} \label{eq:cat2-diag-a}
\vcenter{\xymatrix@R=30pt@C=30pt{
 & G \ar[dd] <+0.5ex>  \ar[dd] <-0.5ex>_{\tau_2,\theta_2}
     \ar[rr] <-0.5ex>  \ar[rr] <+0.5ex>^{\tau_1,\theta_1}
     \ar[ddrr] <-0.5ex>  \ar[ddrr] <+0.5ex>_(0.55){\tau_0,\theta_0}
    &&  Q_1 \ar[dd]<-0.5ex>  \ar[dd] <+0.5ex>^{\tau_2,\theta_2} \\ 
\calC \quad = \quad \\
 & Q_2 \ar[rr] <+0.5ex>  \ar[rr] <-0.5ex>_{\tau_1,\theta_1} 
    &&  Q_0 
 \\
}}
\end{equation}
\noindent
\emph{subject to the following axioms:}
\begin{enumerate}[(a)]
\item~
\emph{$\calC_1 = (G;\tau_1,\theta_1)$ and $\calC_2 = (G;\tau_2,\theta_2)$ 
are (traditional) cat$^1$-groups,}
\item~
$\tau_1 \circ \tau_2 = \tau_2 \circ \tau_1 = \tau_0, \quad  
 \theta_1 \circ \theta_2 = \theta_2 \circ \theta_1 = \theta_0,$  
\item~
$\tau_1 \circ \theta_2 = \theta_2 \circ \tau_1, \quad 
 \tau_2 \circ \theta_1 = \theta_1 \circ \tau_2.$ 
\end{enumerate}
\end{defn}

\noindent
If $r \in Q_0$ then $\tau_0r = \tau_1(\tau_2 r) = \tau_1r = r$, 
and similarly for $\theta_0r$, so $\im\tau_0 = \im\theta_0 = Q_0$. 
It is clear that $\calC_3 = (Q_1;\tau_2,\theta_2)$ 
is a sub-cat$^1$-group of $\calC_2$
and that $\calC_4 = (Q_2;\tau_1,\theta_1)$ 
is a sub-cat$^1$-group of $\calC_1$. 
It is easy to verify that $(G;\tau_0,\theta_0)$ is a cat$^1$-group since, 
for example, 
$$
\tau_0\circ\theta_0 ~=~ \tau_1\circ\tau_2\circ\theta_1\circ\theta_2 
                    ~=~ \tau_1\circ\theta_1\circ\tau_2\circ\theta_2 
                    ~=~ \theta_1\circ\theta_2 
                    ~=~ \theta_0.
$$
It also follows from these identities that, for example,  
$(\tau_1,\tau_1)$ and $(\theta_1,\theta_1) : \calC_2 \to \calC_3$ 
are morphisms of cat$^1$-groups. %% checked!

\medskip 
Secondly, we give a definition of a cat$^2$-group using homomorphisms, 
rather than endomorphisms. 

\begin{defn} \label{defn:cat2b} 
\emph{A cat$^2$-group  $\calC$ is generated by three cat$^1$-groups 
$\calC_i = (e_i;t_i,h_i : G \to R_i), 0 \leqslant i \leqslant 2,$ 
as shown in the following diagram,}
\begin{equation} \label{eq:cat2-diag-b}
\vcenter{\xymatrix@R=40pt@C=40pt{
 & G \ar[dd] <-1.2ex>  \ar[dd] <-2.0ex>_{t_2,h_2}
     \ar[rr] <+1.2ex>  \ar[rr] <+2.0ex>^{t_1,h_1}
     \ar[ddrr] <-0.2ex>  \ar[ddrr] <-1.0ex>_(0.55){t_0,h_0}
    &&  R_1  \ar[ll]^{e_1}
             \ar[dd]<+1.2ex>  \ar[dd] <+2.0ex>^{t_3,h_3} \\ 
\calC \quad = \quad \\
 & R_2 \ar[uu]_{e_2}
     \ar[rr] <-1.2ex>  \ar[rr] <-2.0ex>_{t_4,h_4} 
    &&  R_0 \ar[uu]^{e_3}  \ar[ll]_{e_4}  \ar[uull] <-1.0ex>_{e_0}
 \\
}}
\end{equation}
\noindent 
The remaining homomorphisms are defined by 
\begin{eqnarray*} 
t_3 = t_0 \circ e_1,  
  &  h_3 = h_0 \circ e_1, 
     &  e_3 = e_0 \circ t_1, \\ 
t_4 = t_0 \circ e_2,  
  &  h_3 = h_0 \circ e_2, 
     &  e_3 = e_0 \circ t_2. 
\end{eqnarray*} 
\noindent 
\emph{The following axioms must be satisfied:} 
\begin{enumerate}[(a)]
\item~ 
$
(e_1 \circ t_1) \circ (e_2 \circ t_2) 
= (e_2 \circ t_2) \circ (e_1 \circ t_1) 
= e_0 \circ t_0, \quad
(e_1 \circ h_1) \circ (e_2 \circ h_2) 
= (e_2 \circ h_2) \circ (e_1 \circ h_1) 
= e_0 \circ h_0, 
$
\item~ 
$
(e_1 \circ t_1) \circ (e_2 \circ h_2) = (e_2 \circ h_2) \circ (e_1 \circ t_1), \quad 
(e_2 \circ t_2) \circ (e_1 \circ h_1) = (e_1 \circ h_1) \circ (e_2 \circ t_2). 
$
\item~ 
$
(t_0 \circ e_1) \circ (t_1 \circ e_0) = 1, \quad 
(h_0 \circ e_1) \circ (t_1 \circ e_0) = 1, \quad  
(t_0 \circ e_2) \circ (t_2 \circ e_0) = 1, \quad 
(h_0 \circ e_2) \circ (t_2 \circ e_0) = 1. 
$
\end{enumerate}
\end{defn} 

We now show that Definition \ref{defn:cat2b} is equivalent to 
Definition \ref{defn:cat2a}. 
It is routine to check that 
$(e_3;t_3,h_3 : R_1 \to R_0)$ and $(e_4;t_4,h_4 : R_2 \to R_0)$
are cat$^1$-groups. 
In particular, $[\ker t_3, \ker h_3] = e_1^{-1}[\ker t_0, \ker h_0] = 1$. 

We may convert the cat$^2$-group in Definition \ref{defn:cat2b} 
into a traditional cat$^2$-group by defining: 
$$
\tau_1 = e_1 \circ t_1,~~ 
\theta_1 = e_1 \circ h_1,~~ 
\tau_2 = e_2 \circ t_2,~~ 
\theta_2 = e_2 \circ h_2,~~ 
\tau_0 = e_0 \circ t_0,~~ 
\theta_0 = e_0 \circ h_0, 
$$ 
and setting  $Q_1 = e_1R_1,~ Q_2 = e_2R_2,~ Q_0 = e_0R_0$. 

\bigskip 
\begin{prop} \label{prop:cat2c} 
A cat$^2$-group $\calC$ is a cat$^1$-group of cat$^1$-groups. 
\end{prop} 
\begin{pf} 
We have already seen that $(\tau_1,\tau_1) : \calC_2 \to \calC_3$ 
is a cat$^1$-group morphism -- the boundary in this case. 
We also have to consider the action of $\calC_3$ on $\calC_2$. 
[To be continued.]
\end{pf}

\vspace*{25mm}
Kernels of cat$^1$-mappings have been considered in 
subsection \ref{subs:precat1}. 
The diagram for the situation here is 
\begin{equation*} \label{eq:cat2cker}
\vcenter{\xymatrix{
   \ker\tau_1,\ \ker\theta_1 \ar[rr] \ar[dd]<1.0ex> \ar[dd]<0.2ex> 
     && G \ar[dd]<1.0ex>^{\tau_2,\theta_2} \ar[dd]<0.2ex> 
          \ar[rr]<1.0ex>^{\tau_1,\theta_1} \ar[rr]<0.2ex> 
        && Q_1 \ar[dd]<1.0ex>^{\tau_2,\theta_2} \ar[dd]<0.2ex>  \\
     &&  \\
   \ker\tau_1,\ \ker\theta_1 \ar[rr] 
     && Q_2  \ar[rr]<1.0ex>^{\tau_1,\theta_1} \ar[rr]<0.2ex>
        && Q_0
}}
\end{equation*}


\newpage
%%%%%%%%%%%%%%%%%%%%%%%%%%%%%%%%%%%%%%%%%%%%%%%%%%%%%%%%%%%%%
\subsection{The cat$^2$-group associated to a crossed square} 
\label{sect:cat2-xsq}


Given a crossed square
\begin{equation} \label{eq:Ssquare2cat2}
\vcenter{\xymatrix{
        &   &    L \ar[rr]^{\kappa} \ar[dd]_{\lambda}
             &&  M \ar[dd]^{\mu} \\
  \calS & = &&&  \\
        &   &    N \ar[rr]_{\nu}
             &&  P
}}
\end{equation}
with crossed pairing $\bt: N \times M \to L$, 
we wish to construct an associated cat$^2$-group.

\begin{prop} \label{prop:semidirect-actions}
For $\calS$ a crossed square (as in Definition \ref{def:xsq}) 
there are group actions of $P \ltimes M$  on $N \ltimes L$
and  $P \ltimes N$  on $M \ltimes L$ given by
\begin{eqnarray}
(n,\ell)^{(p,m)} & = &  (n^p, (n^p \bt m)\ell^{pm}) \label{eq:action1}\\ 
(m,\ell)^{(p,n)} & = &  (m^p, (n \bt m^p)^{-1}\,\ell^{pn})~, \label{eq:action2} 
\label{eq:semidirect-action}~.
\end{eqnarray}
\end{prop}
\begin{pf}
There are two axioms to be checked for the first identity: 

\medskip
\begin{eqnarray*}
(n_1,\ell_1)^{(p,m)} (n_2,\ell_2)^{(p,m)}  
& = &  (n_1^p, (n_1^p \bt m)\ell_1^{pm})(n_2^p, (n_2^p \bt m)\ell_2^{pm}) \\
& = &  (n_1^pn_2^p, (n_1^p \bt m)^{n_2^p} 
          [\ell_1^{pmn_2^p}(n_2^p \bt m)] \ell_2^{pm}) \\
& = &  (n_1^pn_2^p, (n_1^p \bt n)^{n_2^p} 
          [(n_2^p \bt m) \ell_1^{pn_2^pm}] \ell_2^{pm}) 
       \qquad\mbox{by Lemma \ref{lem:xsq}(b)} \\
& = &  ((n_1n_2)^p, ((n_1n_2)^p \bt m)(\ell_1^{n_2}\ell_2)^{pm}) \\
& = &  (n_1n_2, \ell_1^{n_2}\ell_2)^{(p,m)} \\
& = &  ((n_1,\ell_1)(n_2,\ell_2))^{(p,m)}~,  \\
&   &  \mbox{} \\
((n,\ell)^{(p_1,m_1)})^{(p_2,m_2)}
& = &  (n^{p_1}, (n^{p_1} \bt m_1)\ell^{p_1m_1})^{(p_2,m_2)} \\
& = &  ((n^{p_1})^{p_2}, (n^{p_1p_2} \bt m_2)
        ((n^{p_1} \bt m_1)\ell^{p_1m_1})^{p_2m_2}) \\
& = &  (n^{p_1p_2}, (n^{p_1p_2} \bt m_2)
        (n^{p_1p_2} \bt {m_1}^{p_2})^{m_2} \ell^{p_1m_1p_2m_2}) \\
& = &  (n^{p_1p_2}, (n^{p_1p_2} \bt {m_1}^{p_2}m_2) 
          \ell^{p_1p_2{m_1}^{p_2}m_2}) \\
& = &  (n,\ell)^{(p_1p_2,{m_1}^{p_2}m_2)} \\
& = &  (n,\ell)^{((p_1,m_1)(p_2,m_2))} ~.
\end{eqnarray*}
The second identity follows using the transpose crossed pairing (\ref{eq:btt}). 
%% These calculations _have_ been checked.
\end{pf}

%\bigskip\noindent
%{\bf [Is this the place to introduce a crossed module of cat$^1$-groups?]}

\medskip
We saw in (\ref{eq:xmodofcat1}) that the cat$^1$-group associated to a
crossed module $\calX$ has homomorphisms
$$
t,h : R \ltimes S \to S, \quad t(r,s) = r, \quad h(r,s) = r(\partial s)~.
$$
Applying this construction to $\calL$ and $\calM$
we obtain a \emph{crossed module of cat$^1$-groups}:

\begin{equation} \label{eq:xmod-of-cat1s}
\vcenter{\xymatrix{
   N \ltimes L 
     \ar[dd]<+0.5ex> \ar[dd]<-0.5ex>_{(t_2,h_2)} \ar[rr]^{(\nu,\kappa)}
   &&  P \ltimes M \ar[dd]<-0.5ex> \ar[dd]<+0.5ex>^{(t_3,h_3)} \\
   &&  \\
   N \ar[rr]_{\nu} 
   &&  P 
}} 
\end{equation}

\begin{lem}
The mapping 
$$
(\nu,\kappa) : N \ltimes L \to P \ltimes M 
~:~ (n,\ell) \mapsto (\nu n, \kappa l) 
$$
is a group homomorphism. 
\end{lem}
\begin{pf}
\vspace{-5mm}
\begin{eqnarray*}
(\nu,\kappa)((n_1,\ell_1)(n_2,\ell_2)) 
  & = & (\nu,\kappa)(n_1n_2,\ell_1^{n_2}\ell_2) \\ 
  & = & (\nu(n_1n_2), \kappa(\ell_1^{n_2}\ell_2)) \\ 
  & = & ((\nu n_1)(\nu n_2), 
         (\kappa \ell_1^{\nu n_2})(\kappa \ell_2)) \\ 
  & = & ((\nu n_1)(\nu n_2), 
         (\kappa \ell_1)^{\nu n_2}(\kappa \ell_2)) \\ 
  & = & (\nu n_1, \kappa \ell_1)(\nu n_2, \kappa \ell_2) \\ 
  & = & (\nu,\kappa)(n_1,\ell_1).(\nu,\kappa)(n_2,\ell_2). 
\end{eqnarray*}
\end{pf}

\begin{lem}
The action given in Proposition \ref{prop:semidirect-actions} makes 
$( (\nu,\kappa) : N \ltimes L \to P \ltimes M)$ a crossed module.
\end{lem}
\begin{pf}
{\bf X1:}
\vspace{-5mm}
\begin{eqnarray*}
(\nu,\kappa)((n, \ell)^{(p,m)}) 
  & = & (\nu,\kappa)(n^p, (n^p \bt m)\ell^{pm}) \\
  & = & (\nu(n^p), \kappa(n^p \bt m) \kappa(\ell^{pm})) \\
  & = & (\nu(n^p), (m^{-1})^{n^p} m \kappa(\ell^{pm})) \\
  & = & (\nu(n^p), (m^{-1})^{(\nu n^p)} m \kappa((\ell^p)^m)) \\
  & = & (\nu(n^p), (m^{-1})^{\nu (n^p)} (\kappa \ell)^p m) \\
  & = & (p^{-1}(\nu n)p, (m^{-1})^{p^{-1}(\nu n)p} (\kappa \ell)^p m) \\
  & = & (p^{-1}, (m^{-1})^{p^{-1}}) (\nu n, \kappa \ell) (p,m) \\
  & = & (p,m)^{-1}\;((\nu,\kappa)(n,\ell)) (p,m)~.
\end{eqnarray*}
{\bf X2:}
\begin{eqnarray*}
(n_1, \ell_1)^{(\nu,\kappa)(n_2, \ell_2)} 
  & = & (n_1, \ell_1)^{(\nu n_2, \kappa \ell_2)} \\
  & = & ({n_1}^{\nu n_2}, ({n_1}^{\nu n_2} \bt \kappa\ell_2) 
          {\ell_1}^{(\nu n_2)(\kappa \ell_2)}) \\
  & = & ({n_1}^{n_2}, (n_2^{-1}n_1n_2 \bt \kappa \ell_2)
          {\ell_1}^{(\nu n_2)(\kappa \ell_2)}) \\
  & = & ({n_1}^{n_2}, (n_2^{-1} \bt \kappa \ell_2)^{n_1n_2}
          (n_1 \bt \kappa \ell_2)^{n_2} (n_2 \bt \kappa \ell_2)
          {\ell_1}^{(\nu n_2)(\kappa \ell_2)}) \\
  & = & (n_2^{-1}n_1n_2, [(\ell_2^{-1})^{n_2^{-1}}\ell_2]^{n_1n_2})
          [(\ell_2^{-1})^{n_1}\ell_2]^{n_2})  
          [(\ell_2^{-1})^{n_2}\ell_2]) 
          \ell_2^{-1}{\ell_1}^{(\nu n_2)}\ell_2 \\
  & = & (n_2^{-1}n_1n_2, (\ell_2^{-1})^{n_2^{-1}n_1n_2} \ell_1^{n_2} \ell_2) \\
  & = & \left(n_2^{-1}, (\ell_2^{-1})^{n_2^{-1}}\right) 
          (n_1n_2, \ell_1^{n_2}\ell_2) \\ 
  & = & (n_2, \ell_2)^{-1}(n_1, \ell_1)(n_2, \ell_2) 
\end{eqnarray*} 
\end{pf}

\bigskip\noindent
We may then construct a \emph{cat$^1$-group of cat$^1$-groups}
where the required homomorphisms are:
\begin{eqnarray} 
\label{eq:cat2-th} 
t_1,h_1 \;:\; 
(P \ltimes M) \ltimes (N \ltimes L)
  & \to & P \ltimes M \\
\nonumber
t_1((p,m),(n,\ell))
  & \;=\; & (p,m) \\
\nonumber
h_1((p,m),(n,\ell))
  & \;=\; & (p,m)\,(\nu,\kappa)(n,\ell) \;=\; (p,m)(\nu n,\kappa\ell)
              \;=\;  (p(\nu n),\,m^{\nu n}(\kappa\ell)) \\
\label{eq:cat2-e}
\mbox{and} \quad\quad e_1 \;:\; P \ltimes M
  & \to &  (P \ltimes M) 
            \ltimes (N \ltimes L) \\
\nonumber
e_1(p,m)
  & \;=\; & ((p,m),(1,1))\ .
\end{eqnarray}

\bigskip
We now check that
$(e_1;t_1,h_1 : (P \ltimes M) \ltimes (N \ltimes L) \to P \ltimes M )$
is a cat$^1$-group.
Note that
\begin{eqnarray*}
\ker t_1  & = & 
\{~((1,1),(n,\ell))~\}   \\
\ker h_1  & = &
\{~((p,m),(n,\ell))~\}  \quad\mbox{where}\quad
p = (\nu n)^{-1}       \;\;\mbox{and}\;\;
m = (\kappa\ell^{-1})^p,
\end{eqnarray*}
so that
$$
\ell^{pn} ~=~ \ell^{(\nu n)^{-1}(\nu n)} ~=~ \ell
\qquad\mbox{in}\quad \ker h_1.
$$
The formula for the action is given by equation (\ref{eq:semidirect-action}): 
$$
(n,\ell)^{(p,m)} ~:=~ (~n^p,\ (n^p \bt m)\ell^{pm})\ .
$$

\medskip\noindent
We now check the cat$^1$-group axioms.

\medskip\noindent
\textbf{C1:}
\vspace{-8mm}
\begin{eqnarray*}
t_1 \circ e_1(p,m)
   & = & t_1((p,m),(1,1)) ~~ = ~~ (p,m) \\
h_1 \circ e_1(p,m) 
   & = & h_1((p,m),(1,1)) ~~ = ~~ (p,m) \\
\end{eqnarray*}

\medskip\noindent
\textbf{C2:} 
\vspace{-4mm}
\begin{eqnarray*} 
((1,1),(n_0, \ell_0))\,((p,m), (n, \ell)) 
  & = & ((p,m),(n_0,\ell_0)^{(p,m)}(n,\ell)) \\
  & = & ((p,m),({n_0}^p,({n_0}^p \bt m) {\ell_0}^{pm}) (n,\ell)) \\
  & = & ((p,m),({n_0}^{(\nu n)^{-1}} n, ({n_0}^p \bt m)^n
           {\ell_0}^{pmn} \ell)) \\
  & = & ((p,m),(nn_0n^{-1}n, (n_0 \bt \kappa\ell^{-1})^{pn}\,
           {\ell_0}^{p \kappa \ell^{-1} n} \ell)) \\
  & = & ((p,m),(nn_0, [\ell^{n_0}\ell^{-1}]^{pn} 
           {\ell_0}^{\kappa\ell^{-1}pn} \ell)) \\
  & = & ((p,m),(nn_0,\ell^{n_0}\ell^{-1}{\ell_0}^{\kappa\ell^{-1}pn} \ell)) \\
  & = & ((p,m),(nn_0,\ell^{n_0}\ell^{-1} 
           [\ell\ell_0\ell^{-1}]^{pn} \ell)) \\
  & = & ((p,m),(nn_0,\ell^{n_0} (\ell^{-1} \ell) 
           \ell_0^{p(\nu n)} (\ell^{-1}\ell))) \\
  & = & ((p,m),(n,\ell)(n_0,\ell_0)) \\
  & = & ((p,m),(n,\ell))\,((1,1),(n_0, \ell_0))  
\end{eqnarray*}
and so the two kernels commute. 

\medskip
\begin{thm}
The homomorphisms in (\ref{eq:cat2-th}) and (\ref{eq:cat2-e}) 
give a cat$^2$-group:
\begin{equation} \label{eq:cat2-sdp}
\vcenter{\xymatrix{
 & (P \ltimes N) \ltimes (M \ltimes L)
      \ar[ddd] <-0.4ex>  \ar[ddd] <-1.0ex>_{t_2,h_2}
      \ar[rrr] <+0.4ex>  \ar[rrr] <+1.0ex>^(0.6){t_1,h_1}
      \ar[dddrrr] <-0.4ex>  \ar[dddrrr] <-1.0ex>_{t_0,h_0}
    &&&  P \ltimes M
            \ar[lll]<+0.4ex>^(0.4){e_1}
            \ar[ddd]<+0.4ex>  \ar[ddd] <+1.0ex>^{t_3,h_3}  \\
\calC(\calR) \quad = 
 &   &&&   \\
 &   &&&   \\
 & P \ltimes N
    \ar[uuu] <-0.4ex>_{e_2}
    \ar[rrr] <-0.4ex>  \ar[rrr] <-1.0ex>_{t_4,h_4} 
    &&&  P \ar[uuu]<+0.4ex>^{e_3}   \ar[lll]<-0.4ex>_{e_4} 
           \ar[uuulll] <-0.4ex>_{e_0}
 \\
}} 
\end{equation}
\end{thm}
\begin{pf}
To be added.  (Is material required from Section 7.2~?)
\end{pf}


\vspace*{5mm} 
%%%%%%%%%%%%%%%%%%%%%%%%%%%%%%%%%%%%%%%%%%%%%%%%%%%%%%%%%%%%%%%%%%%%
\subsection{The other cat$^2$-structure}

Now the underlying diagram (\ref{eq:Ssquare}) of the crossed square $\calR$, 
together with the crossed pairing 
$$
\btt : M \times N \to L, \ (m \btt n) = (n \bt m)^{-1}
$$
forms a second crossed square $\tilde{\calS}$. 
% (see Lemma \ref{prop:princ}).
Thus we can form a second cat$^2$-group $\calC(\tilde{\calS})$ with 
$$
\tilde{t}_1,\tilde{h}_1 : (P \ltimes M) \ltimes (N \ltimes L) \to (P \ltimes M)~.
$$
Let
$$
G = (P \ltimes N) \ltimes (M \ltimes L), \qquad 
\tilde{G} = (P \ltimes M) \ltimes (N \ltimes L)~. 
$$

\begin{prop} \label{prop:tau}
There is an isomorphism between these two semidirect products: 
\begin{eqnarray}
\label{eq:cat1-iso}
\tau ~:~ G \to \tilde{G}, &&
((p,n),(m,\ell)) \mapsto ((p,m),(n,(n \bt m)\ell))~, \\
\nonumber
\tilde{\tau} ~:=~ \tau^{-1} ~:~ \tilde{G} \to G, &&
((p,m),(n,\ell)) \mapsto ((p,n),(m,(m \btt n) \ell))~.
\end{eqnarray}
\end{prop}
\begin{pf}
\vspace{-6mm}
\begin{eqnarray*}
 &   &  \tau((p_1,n_1),(m_1,\ell_1))((p_2,n_2),(m_2,\ell_2)) \\
 & = &  \tau((p_1,n_1)(p_2,n_2),
                  (m_1,\ell_1)^{(p_2,n_2)}(m_2,\ell_2)) \\
 & = &  \tau((p_1p_2,{n_1}^{p_2}n_2),
    ({m_1}^{p_2},(n_2 \bt {m_1}^{p_2})^{-1} {\ell_1}^{p_2n_2})(m_2,\ell_2)) \\
 & = &  \tau((p_1p_2,{n_1}^{p_2}n_2),
                  ({m_1}^{p_2}m_2,({n_2}^{m_2} \bt {m_1}^{p_2m_2})^{-1}  
                    {\ell_1}^{p_2n_2m_2} \ell_2)) \\
 & = &  ((p_1p_2,{m_1}^{p_2}m_2),
         ({n_1}^{p_2}n_2,({n_1}^{p_2}n_2 \bt {m_1}^{p_2}m_2)
         ({n_2}^{m_2} \bt {m_1}^{p_2m_2})^{-1} 
                    {\ell_1}^{p_2n_2m_2} \ell_2)) \\ 
 & = &  ((p_1p_2,{m_1}^{p_2}m_2), 
         ({n_1}^{p_2}n_2, ({n_1}^{p_2} \bt m_2)^{n_2} 
         ({n_1}^{p_2} \bt {m_1}^{p_2})^{m_2n_2} 
         (n_2 \bt m_2) {\ell_1}^{p_2n_2m_2} \ell_2)) \\ 
 & = &  ((p_1p_2,{m_1}^{p_2}m_2), 
         ({n_1}^{p_2}n_2, ({n_1}^{p_2} \bt {m_1}^{p_2}m_2)^{n_2} 
         (n_2 \bt m_2) \ell_1^{p_2n_2m_2} \ell_2)) \\  
 & = &  ((p_1p_2,{m_1}^{p_2}m_2), 
         ({n_1}^{p_2}n_2, ({n_1}^{p_2} \bt {m_1}^{p_2}m_2)^{n_2} 
         \ell_1^{p_2m_2n_2} (n_2 \bt m_2) \ell_2)) \\  
 & = &  ((p_1p_2,{m_1}^{p_2}m_2), 
         ({n_1}^{p_2}, ({n_1}^{p_2} \bt {m_1}^{p_2}m_2) \ell_1^{p_2m_2}) 
         (n_2, (n_2 \bt m_2) \ell_2)) \\  
 & = &  ((p_1p_2,{m_1}^{p_2}m_2), 
         ({n_1}^{p_2}, ({n_1}^{p_2} \bt {m_2})
         ({n_1}^{p_2} \bt {m_1}^{p_2})^{m_2} \ell_1^{p_2m_2}) 
         (n_2, (n_2 \bt m_2) \ell_2)) \\  
 & = &  ((p_1,m_1)(p_2,m_2), 
        (n_1,(n_1 \bt m_1)\ell_1))^{(p_2,m_2)} (n_2,(n_2 \bt m_2)\ell_2)) \\
 & = &  ((p_1,m_1),(n_1,(n_1 \bt m_1)\ell_1)) 
        \ ((p_2,m_2),(n_2,(n_2 \bt m_2)\ell_2)) \\ 
 & = &  \tau((p_1,n_1),(m_1,\ell_1))\ \tau((p_2,n_2),(m_2,\ell_2))
\end{eqnarray*}
\end{pf}

\noindent
Note that the subgroup 
$(1 \ltimes N) \ltimes (M \ltimes 1)$
does \emph{not} in general get mapped by $\tau$ to the subgroup 
$(1 \ltimes M) \ltimes (N \ltimes 1)$. 
