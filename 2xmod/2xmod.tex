%% 2xmod.tex,  version 02/05/17

%%%%%%%%%%%%%%%%%%%%%%%%%%%%%%%%%%%%%%%%%%%%%%%%
\section{$2$-crossed modules} \label{sect:2xmod}

The reader is refered to
Brown--Gilbert \cite{brow:gilb}, 
Conduch\'e \cite{conduche:jpaa},\cite{conduche:gmj}, 
and Mutlu--Porter \cite{mutlu:porter5} 
for background information on $2$-crossed modules.

\begin{defn} \index{$2$-crossed module}
A $2$-crossed module is comprised of the following: 
\begin{itemize}
\item
a $2$-complex of groups
$$
\xymatrix{
\calZ 
  & = &  (~T \ar[rr]^{\delta_2} 
         &&  U \ar[rr]^{\delta_1}
             && V~) \\
}
$$
(so that  $\delta_1\circ\delta_2 = 0 : T \to V$);
\item
an action of $V$ on $T$ and on $U$, and on itself by conjugation,
such that $\delta_1, \delta_2$ are morphisms which preserve the actions;
\item
a function, called the \emph{Peiffer lifting},
$$
\{~,~\} ~:~ U \times U \to T,
$$
making $(\delta_2 : T \to U)$ a crossed module with action
\begin{equation} \label{eq:2Xaction}
t^u ~:=~ t\,\{\delta_2t,u\}\;.
\end{equation}
\end{itemize}
The following axioms are also required: 
\begin{enumerate}[{\rm (2X1)}]
\item~
$\delta_2\{u_1,u_2\} = \langle u_1, u_2 \rangle
                     = u_1^{-1} u_2^{-1} u_1 {u_2}^{\delta_1 u_1}$ 
\quad\mbox{(a Peiffer commutator)},
\item~
$\{u, \delta_2 t\} = (t^{-1})^u\, t^{\delta_1 u}$,
\item~ 
$\{u_1u_2, u_3\} = \{u_1,u_3\}^{u_2} \{u_2,{u_3}^{\delta_1 u_1}\}$,
\item~ 
$\{u_1, u_2u_3\} = \{u_1,u_3\} \{u_1,u_2\}^{{u_3}^{\delta_1 u_1}}$,
\item~ 
$\{u_1,u_2\}^v = \{{u_1}^v, {u_2}^v\}$.
\end {enumerate}
\end{defn}

\noindent
An additional axiom, $\{\delta_2t,u\} = t^{-1}t^n$ id often specified, 
but we have used this identity in (\ref{eq:2Xaction}) to define the action. 
Note that $\delta_2$ maps {(2X3)}, {(2X4)} and {(2X5)} 
to identities (b), (c) and (d) in Lemma \ref{lem:Peiffer} 
for Peiffer commutators. 
Compare also (2X3), (2X4) with identities (a), (b) for crossed pairings 
in Definition \ref{def:xpair}. 

\medskip\noindent
We will check that the crossed module action  $t^u = t\,\{\delta_2t,u\}$ 
given by formula (\ref{eq:2Xaction}) is well defined:
\begin{eqnarray*}
(t_1t_2)^{u} 
  & = &  t_1 t_2 \{(\delta_2 t_1)(\delta_2 t_2), u \} \\
  & = &  t_1 t_2 \{\delta_2 t_1, u \}^{\delta_2 t_2} 
                 \{\delta_2 t_2, u^{\delta_1 \delta_2 t_1} \} \\
  & = &  t_1 t_2 t_2^{-1} \{\delta_2 t_1, u \}
                 t_2 \{\delta_2 t_2, u\} \\
  & = &  {t_1}^u\,{t_2}^u, \\
t^{(u_1u_2)} 
  & = &  t \{\delta_2 t, u_1u_2 \} \\
  & = &  t \{\delta_2 t,u_2\}\{\delta_2 t, u_1\}^{{u_2}^{\delta_1\delta_2 t}}\\
  & = &  t^{u_2} \{\delta_2 t, u_1\}^{u_2} \\
  & = &  (t^{u_1})^{u_2}. \\
\end{eqnarray*}

\begin{lem} \label{lem:Peiffer-props}
\begin{enumerate}[{\rm (a)}]
\item~
$\{\delta_2t_1,\delta_2t_2\} = [t_1,t_2]$.
\end{enumerate}
\end{lem}
\begin{pf}
\begin{enumerate}[(a)]
\item~
$\{\delta_2t_1,\delta_2t_2\} = t_1^{-1}\,{t_1}^{\delta_2t_2} = [t_1,t_2]$,
by definition of the crossed module action.
\end{enumerate}
\end{pf}


%%%%%%%%%%%%%%%%%%%%%%%%%%%%%%%%%%%%%%%%%%%%%%%%%%%%%%%%%%%%
\subsection{Morphisms and Homotopies of $2$-crossed modules} 
\label{subs:morph-homot}

\begin{defn} \index{morphism!of $2$-crossed modules} 
A morphism of $2$-crossed modules is a triple of group homomorphisms
$$
f_{\bullet} ~\equiv~ (f_2, f_1, f_0) ~:~ \cal Z \to \calZ'
$$
such that
$$
f_1\delta_2 = \delta_2'f_2,~~
f_0\delta_1 = \delta_1'f_1,~~
f_2(t^v) = (f_2t)^{f_0v},~~
f_1(u^v) = (f_1u)^{f_0v},~~
f_2\{u_1,u_2\} = \{f_1u_1,f_1u_2\}.
$$
\end{defn}
$$
\xymatrix{
\calZ ~~:
  &  T \ar[rr]^{\delta_2} \ar[dd]_{f_2}
     &&  U \ar[rr]^{\delta_1} \ar[dd]_{f_1}
         && V~ \ar[dd]_{f_0} \\
f_{\bullet} ~~:
  &  &&  &&  \\
\calZ' ~~:
  &  T' \ar[rr]^{\delta_2'} 
     &&  U' \ar[rr]^{\delta_1'}
         && V'~ \\
}
$$

\noindent
An automorphism of $\calZ$ is an endomorphism with inverse
${f_{\bullet}}^{-1} = (f_2^{-1},f_1^{-1},f_0^{-1})$.

Note that
$$
f_2(t^u) ~=~ (f_2t)\{f_1\delta_2t,f_1u\}
         ~=~ (f_2t)\{\delta_2'f_2t,f_1u\}
         ~=~ (f_2t)^{f_1u}
$$
so $(f_2,f_1) : (\delta_2 : T \to U) \to (\delta_2' : T' \to U')$ 
is a morphism of crossed modules.

\begin{defn} \index{homotopy!of $2$-crossed modules}
A homotopy of the $2$-crossed module $\calZ$
is a pair of homomorphisms $\phi_{\bullet} = (\phi_1,\phi_0)$
such that 
$$
\delta_2\phi_1 ~=~ \phi_0\delta_1 \qquad\mbox{{\bf [Is that all~???]}}
$$
\end{defn}
$$
\xymatrix{
    T \ar[rr]^{\delta_2} \ar[dd]_{f_2}
    &&  U \ar[rr]^{\delta_1} \ar[dd]_{f_1}
        && V~ \ar[dd]_{f_0} \\
    &&  &&  \\
    T \ar[rr]^{\delta_2} 
    &&  U \ar[rr]^{\delta_1} \ar[lluu]^{\phi_1}
        && V~ \ar[lluu]^{\phi_0}  \\
}
$$



%%%%%%%%%%%%%%%%%%%%%%%%%%%%%%%%%%%%%%%%%%%%%%%%%%%%%%%%%%%%%%%%%%%%%%%
\subsection{$2$-crossed modules of groupoids}  \label{subsec:2xmod-gpd}

[To be added.]





\newpage
\vspace*{5mm}
\begin{center}
{\bf [The rest of this section really belongs in a crossed squares chapter.]}
\end{center}

\vspace*{10mm}
%%%%%%%%%%%%%%%%%%%%%%%%%%%%%%%%%%%%%%%%%%%%%%%%%%%%%%%%%%%%%%%%%%
\subsection{The $2$-crossed module associated to a crossed square}

We follow Brown--Gilbert \cite{brow:gilb} in defining the appropriate 
Peiffer lifting to be 
$\{(m_1, n_1),(m_2,n_2) \} = {m_2}^{m_1} \boxtimes n_1$.

\medskip
\begin{prop} \label{prop:2xmod-xsq}
Given a crossed square $\calR$ there is an associated $2$-crossed module 
$\calZ$, as shown in the following diagram: 
$$
\xymatrix{
  & R_{[2]} \ar[rr]^{\ddbdyo} \ar[dd]_{\ddbdyt} 
    & & R_{\{2\}} \ar[dd]^{\dbdyt} 
        & & & & &  \\
  \calR~: 
  & & & & & \calZ~: 
            & (~R_{[2]} \ar[r]^(0.35){\delta_2} 
              &  R_{\{1\}} \ltimes R_{\{2\}} \ar[r]^(0.6){\delta_1}
                & R_{\emptyset}~) \\
  & R_{\{1\}} \ar[rr]_{\dbdyo}  
    & & R_{\emptyset} 
        & & & & &
}
$$
where
$$
\delta_2 \ell = (\ddbdyt \ell, \ddbdyo \ell^{-1})
\qquad\mbox{and}\qquad
\delta_1(m,n) = (\dbdyo m)(\dbdyt n)\;,
$$
$R_{\emptyset}$ acts diagonally on $R_{\{1\}} \ltimes R_{\{2\}}$,
and the Peiffer lifting is given by
$$
\{(m_1, n_1),(m_2,n_2) \} ~=~ {m_2}^{m_1} \boxtimes n_1\;.
$$
\end{prop}
\begin{pf}
We first check that $\delta_1$ and $\delta_2$ are homomorphisms 
preserving the $R_{\emptyset}$-actions,
and that $\delta_1\delta_2 = 0$\;:
\begin{eqnarray*}
% \delta_1 is a hom
\delta_1((m_1,n_1),(m_2,n_2))
& = & \delta_1(m_1m_2, {n_1}^{m_2}n_2)
      ~=~ \dbdyo(m_1m_2)\,\dbdyt({n_1}^{\dbdyo m_2} n_2) \\
& = & (\dbdyo m_1)(\dbdyo m_2)(\dbdyt n_1)^{\dbdyo m_2}(\dbdyt n_2)
      ~=~ \delta_1(m_1,n_1)\,\delta_1(m_2,n_2). \\
% \delta_1 preserves the action
\delta_1((m,n)^p)
& = & \delta_1(m^p,n^p)
      ~=~ \dbdyo(m^p)\,\dbdyt(n^p)
      ~=~ (\dbdyo m)^p\,(\dbdyt m)^p
      ~=~ (\delta_1(m,n))^p\;. \\
% \delta_2 is a hom
(\delta_2 \ell_1)(\delta_2 \ell_2)
& = & (\ddbdyt\ell_1,\ddbdyo\ell_1^{-1})(\ddbdyt\ell_2,\ddbdyo\ell_2^{-1})\\
& = & ((\ddbdyt \ell_1)(\ddbdyt \ell_2),
            (\ddbdyo \ell_1^{-1})^{\dbdyo\ddbdyt\ell_2}(\ddbdyo \ell_2^{-1}))\\
& = & (\ddbdyt(\ell_1\ell_2),
            (\ddbdyo \ell_1^{-1})^{\dbdyt\ddbdyo\ell_2}(\ddbdyo \ell_2^{-1}))\\
& = & (\ddbdyt(\ell_1\ell_2),(\ddbdyo \ell_2^{-1})(\ddbdyo \ell_1^{-1}))
      \hspace{35mm} \mbox{by {\bf X2:} for}~ \dcalRt\\
& = & \delta_2(\ell_1\ell_2).\\
% \delta_2 preserves the action
\delta_2(\ell^p)
& = & (\ddbdyt(\ell^p), \ddbdyo((\ell^p)^{-1}))
      ~=~ ((\ddbdyt \ell)^p, (\ddbdyo \ell^{-1})^p)
      ~=~ (\ddbdyt \ell, \ddbdyo \ell^{-1})^p
      ~=~ (\delta_2 \ell)^p\;. \\
% composite is zero
\delta_1\delta_2\ell
& = & \delta_1(\ddbdyt\ell, \ddbdyo\ell^{-1})
      ~=~ (\dbdyo\ddbdyt\ell)(\dbdyt\ddbdyo\ell^{-1})
      ~=~ 1\;.
\end{eqnarray*}

\noindent
Secondly we identify the crossed module action in $\calZ$ in this case
to be $\ell^{(m,n)} = \ell^m$. 
\begin{eqnarray*}
\ell^{(m,n)}
  & = &  \ell\,\{(\ddbdyt\ell,\ddbdyo\ell^{-1}),(m,n)\} \\
  & = &  \ell\,(m^{\ddbdyt\ell} \bt \ddbdyo\ell^{-1}) \\
  & = &  \ell\,\ell^{(\ddbdyt\ell^{-1})m(\ddbdyt\ell)}\,\ell^{-1} 
         \hspace{50mm} \mbox{by Definition \ref{def:xsq} (e)} \\
  & = &  \ell\,(\ell^m)^{(\ddbdyt\ell)}\,\ell^{-1}
   ~=~   \ell^m
         \hspace{54mm} \mbox{by {\bf X2:} (twice).} 
\end{eqnarray*}
It is clear that this \emph{is} an action,
so we verify the two crossed module axioms:
\begin{eqnarray*}
{\bf X1:} \qquad
(\delta_2\ell)^{(m,n)}
  & = &  (m^{-1},(n^{-1})^{m^{-1}})(\ddbdyt\ell,\ddbdyo\ell^{-1})(m,n) \\
  & = &  (m^{-1}(\ddbdyt\ell)m,\, 
          (n^{-1})^{\dbdyo\ddbdyt\ell^m} (\ddbdyo\ell^{-1})^m n) \\
  & = &  ((\ddbdyt\ell)^m,\, 
          (n^{-1})^{\dbdyt(\ddbdyo\ell)^m} (\ddbdyo\ell^m)^{-1} n) \\
  & = &  ((\ddbdyt\ell)^m,\, 
          (\ddbdyo\ell^m)^{-1} n^{-1} n) \\
  & = &  \delta_2(\ell^m) ~=~ \delta_2(\ell^{(m,n)})\,, \\
{\bf X2:} \hspace{16mm}
{\ell_0}^{\delta_2\ell}
  & = &  {\ell_0}^{(\ddbdyt\ell,\ddbdyo\ell^{-1})}
   ~=~   {\ell_0}^{\ddbdyt\ell}
   ~=~   {\ell_0}^{\ell}\,. \\
\end{eqnarray*}

\noindent
Thirdly, we verify the five axioms.
\begin{enumerate}[{\bf 2X1:}]
\item~

\vspace*{-12mm}
\begin{eqnarray*}
  &   &  (m_1,n_1)^{-1}\,(m_2,n_2)^{-1}\,
          (m_1,n_1)\,(m_2,n_2)^{\delta_1(m_1,n_1)} \\
  & = &  ({m_1}^{-1},({n_1}^{-1})^{{m_1}^{-1}})\,
          ({m_2}^{-1},({n_2}^{-1})^{{m_2}^{-1}})\,
           (m_1,n_1)\,
            (m_2,n_2)^{(\dbdyo m_1)(\dbdyt n_1)} \\
  & = &  (m_1^{-1}m_2^{-1}m_1{m_2}^{m_1n_1},
          (n_1^{-1})^{m_1^{-1}m_2^{-1}m_1{m_2}^{m_1n_1}}
           (n_2^{-1})^{m_2^{-1}m_1{m_2}^{m_1n_1}}
            {n_1}^{{m_2}^{m_1n_1}}
             {n_2}^{m_1n_1} ) 
\end{eqnarray*}
The left hand element is
$$
({m_2}^{m_1})^{-1}\, ({m_2}^{m_1})^{n_1}
  ~=~  \ddbdyt({m_2}^{m_1} \bt n_1)
$$
It follows that the right hand element is
\begin{eqnarray*}
  &   &  (n_1^{-1}(n_2^{-1})^{m_1})^{\ddbdyt({m_2}^{m_1} \bt n_1)}\,
          {n_1}^{{m_2}^{m_1n_1}}\,
           {n_2}^{m_1n_1} \\
  & = &  (n_1^{-1}({n_2}^{m_1})^{-1})^{(n_1^{-1})^{{m_2}^{m_1}}n_1}\,
          ({n_1}^{{m_2}^{m_1}})^{n_1}\,
           ({n_2}^{m_1})^{n_1} 
         \hspace{18mm} 
         \mbox{by Lemmas \ref{lem:gamma-eta_s}(c) and \ref{lem:invchir}(d)}, \\
  & = &  (n_1^{-1}) ({n_1}^{{m_2}^{m_1}}) (n_1^{-1}) ({n_2}^{m_1})^{-1} 
          ({n_1}^{{m_2}^{m_1}})^{-1} (n_1)\,
           (n_1^{-1}) ({n_1}^{{m_2}^{m_1}}) (n_1)\,
            (n_1^{-1}) ({n_2}^{m_1}) (n_1) \\
  & = &  \ddbdyo({m_2}^{m_1} \bt n_1)^{-1} 
\end{eqnarray*}
So the pair of elements is 
~$\delta_2({m_2}^{m_1} \bt n_1) ~=~ \delta_2\{(m_1,n_1),(m_2,n_2)\}$.
\item~
$$
\{(m,n),\delta_2\ell\}
  ~ = ~  \{(m,n),(\ddbdyt\ell,\ddbdyo\ell^{-1})\} 
  ~ = ~  \ddbdyt(\ell^m) \bt n
  ~ = ~  (\ell^m)^{-1}(\ell^m)^n
  ~ = ~  (\ell^{-1})^{(m,n)} \ell^{\delta_1(m,n)}\,.
$$
\item~

\vspace*{-12mm}
\begin{eqnarray*}
  &   &  \{(m_1,n_1),(m_3,n_3)\}^{(m_2,n_2)}\,
           \{(m_2,n_2),(m_3,n_3)^{m_1n_1}\} \\
  & = &  ({m_3}^{m_1} \bt n_1)^{m_2}\,
           ({m_3}^{m_1n_1m_2} \bt n_2) \\
  & = &  (m_0 \bt n_0)\,({m_0}^{n_0} \bt n_2)
         \hspace{50mm}\mbox{where}~ m_0 = {m_3}^{m_1m_2},~ n_0 = {n_1}^{m_2}\\
  & = &  (m_0 \bt n_2)\,(m_0 \bt n_0)^{n_2}
         \hspace{70mm} \mbox{by Proposition \ref{prop:xpair}(d)} \\
  & = &  m_0 \bt n_0n_2 \\
  & = &  {m_3}^{m_1m_2} \bt {n_1}^{m_2}n_2 \\
  & = &  \{(m_1m_2, {n_1}^{m_2}n_2), (m_3,n_3)\} \\
  & = &  \{(m_1,n_1)(m_2,n_2),(m_3,n_3)\}
\end{eqnarray*}
\item~
\begin{eqnarray*}
  &   &  \{(m_1,n_1),(m_2m_3, {n_2}^{m_3}n_3)\} \\
  & = &  (m_2m_3)^{m_1} \bt n_1 \\
  & = &  ({m_2}^{m_1} \bt n_1)^{{m_3}^{m_1}}\,({m_3}^{m_1} \bt n_1) \\
  & = &  (m_4 \bt n_1)^{n_1m_5}\,(m_5 \bt n_1)
         \hspace{47mm}
         \mbox{where}~ m_4 = {m_2}^{m_1n_1^{-1}},~ m_5 = {m_3}^{m_1}\\
  & = &  (m_5 \bt n_1)(m_4 \bt n_1)^{m_5n_1}
         \hspace{70mm} \mbox{by Proposition \ref{prop:xpair}(c)} \\
  & = &  ({m_3}^{m_1} \bt n_1)({m_2}^{m_1} \bt n_1)^{n_1^{-1}{m_3}^{m_1}n_1} \\
  & = &  \{(m_1,n_1),(m_3,n_3)\}\,
         \{(m_1,n_1),(m_2,n_2)\}^{(m_3,n_3)^{\delta_1(m_1,n_1)}}
\end{eqnarray*}
\item~
$$
\{({m_1}^p,{n_1}^p), ({m_2}^p,{n_2}^p)\}
  ~=~  ({m_2}^p)^{{m_1}^p} \bt {n_1}^p
  ~=~  ({m_2}^{m_1} \bt n_1)^p
  ~=~  \{(m_1,n_1),(m_2,n_2)\}^p
$$
\end{enumerate}
\end{pf}


%%%%%%%%%%%%%%%%%%%%%%%%%%%%%%%%%%%%%%%%%%%%%%%%%%%%%%%%%%%%%%%%%%
\subsection{The crossed square associated to a $2$-crossed module}

Maybe there is no exact construction?

\vspace*{10mm}


%%%%%%%%%%%%%%%%%%%%%%%%%%%%%%%%%%%%%%%%%%%%%%%%%%%%%%%
\subsection{Homotopies of the actor $2$-crossed module}

{\bf [This needs significant revision.]}

\begin{defn}
A homotopy of the $2$-crossed module $\calZ$
is a pair of homomorphisms $\phi_{\bullet} = (\phi_1,\phi_0)$
such that 
$\phi_0: R_{\emptyset} \to R_{\{1\}} \ltimes R_{\{2\}}, \ 
 p \mapsto(\phi_0^{N}p, \phi_0^{M}p)$ where
\begin{eqnarray*}
\phi_0(p_1p_2) 
  & = &  (\phi_0^{N}(p_1p_2), \phi_0^{M}(p_1p_2)) \\
  & = &  (\phi_0^{N}p_1, \phi_0^{M}p_1)(\phi_0^{N}p_2, \phi_0^{M}p_2) \\
  & = &  (\phi_0^{N}p_1 \phi_0^{N}p_2, 
          (\phi_0^{M}p_1)^{\phi_0^{N}p_2} \phi_0^{M}p_2)
\end{eqnarray*}
\end{defn}

So $\phi_0^{M}$ is a $\phi_0^{N}$-derivation.
$\phi_1:  R_{\{1\}} \ltimes R_{\{2\}} \to R_{[2]}, \ 
 (n, m) \mapsto (\phi_1^{N}n)( \phi_1^{M}m)$
\begin{eqnarray*}
\phi_1(n,m) & = & \phi_1((n,1)(1, m)) \\
                       & = & \phi_1(n,1) \phi_1(1,m) \\
                       & = & (\phi_1^{N}n)( \phi_1^{M}m)
\end{eqnarray*}
      
\bigskip\noindent
{\bf [Moved what was Lemma 7.6 to cat2-group section.]}

\begin{lem}
There is an action of $R_{\emptyset} \ltimes R_{\{2\}}$ 
on $R_{\{1\}} \ltimes R_{[2]}$
defined on the image of some derivation $\chi$ by
$$
\chi(p,m)^{(p_1,m_1)} = \chi(pp_1, m^{p_1}m_1) \chi(p_1, m_1)^{-1} 
$$
which is well defined {\bf (should use the usual action!)}
\end{lem}
\begin{pf}
\begin{eqnarray*}
\chi((p,m)^{(p_1,m_1})^{(p_2,m_2)} 
  & = &  (\chi(pp_1, m^{p_1}m_1) \chi(p_1,m_1)^{-1})^{(p_2, m_2)} \\
  & = &  \chi(pp_1, m^{p_1}m_1)^{(p_2,m_2)} (\chi(p_1,m_1)^{(p_2,m_2)})^{-1} \\
  & = &  \chi(pp_1p_2, m^{p_1p_2}m_1^{p_2}m_2)(\chi(p_2,m_2)^{-1} 
          (\chi(p_1p_2,m_1^{p_2}m_2) \chi(p_2,m_2)^{-1})^{-1} \\
  & = &  \chi(pp_1p_2, m^{p_1p_2}m_1^{p_2}m_2) 
          \chi(p_1p_2, m_1^{p_2}m_2)^{-1} \\
\chi(p,m)^{(p_1,m_1)(p_2,m_2)} 
  & = &  \chi(p,m)^{(p_1p_2, m_1^{p_2}m_2)} \\
  & = &  \chi(pp_1p_2, m^{p_1p_2}m_1^{p_2}m_2) \chi(p_1p_2, m_1^{p_2}m_2)^{-1}
\end{eqnarray*}
\end{pf}
\begin{lem}
Consider a derivation 
$\chi: R_{\emptyset} \ltimes R_{\{2\}} \to R_{\{1\}} \ltimes R_{[2]}$ 
such that $\chi(p,m) = (\chi^L(p,m), \chi^{N}(p,m))$.
Then the rules for $\chi^L$ and $\chi^N$ are as follows: ~?~?~?
\end{lem}
\begin{pf}
we will show that above derivation is satisfies the derivation rule
\begin{eqnarray*}
\chi((p_1,m_1)(p_2,m_2)) 
  & = &  \chi(p_1p_2, m_1^{p_2}m_2) \\
\chi((p_1,m_1)(p_2,m_2)) 
  & = &  \chi(p_1,m_1)^{(p_2,m_2)} \chi(p_2,m_2) \\
  & = &  \chi(p_1p_2, m_1^{p_2}m_2) \chi(p_2,m_2)^{-1} \chi(p_2, m_2) \\
  & = &  \chi(p_1p_2, m_1^{p_2}m_2)
\end{eqnarray*}
\end{pf}
\begin{defn} \index{derivation!of $2$-crossed module}
Now we can define a derivation which is depends on the derivation ${\chi^L}$
\begin{eqnarray*}
\theta_p & = & \chi^{L}(p,1) \\
\ddch m & = & \chi^{L}(1,m)
\end{eqnarray*}
\end{defn}
\begin{eqnarray*}
\theta(p_1p_2) 
  & = &  \chi^{L}(p_1p_2, 1) \\
  & = &  \chi^{L}((p_1,1),(p_1,1)) \\
  & = &  (\chi^{L}(p_1,1))^{(p_2,1)} (\chi^{L}(p_2,1)) \\
  & = &  (\theta p_1)^{p_2} (\theta p_2)
\end{eqnarray*}

\begin{lem}
$\chi^{L} : R_{\emptyset} \ltimes R_{\{2\}} \to R_{[2]}$ is a derivation.
\end{lem}
\begin{pf}
\begin{eqnarray*}
\chi^{L}(p,m) 
  & = &  \chi^{L}((p,1)(1,m)) \\
  & = &  (\chi^{L}(p,1))^{(1,m)} (\chi^{L}(1,m)) \\
  & = &  (\theta p)^{m} (\ddch m)
\end{eqnarray*}
\end{pf}
