%% oddments.tex,  version 13/06/08

%%%%%%%%%%%%%%%%%%
\section{Oddments}

This section is a place to type handwritten notes which may or may not 
end up in the final version.


%%%%%%%%%%%%%%%%%%%%%%%%%%%%%%%%%%%%%%%%%%%%%%%%%%%%%%%%%%%%%%%%%%%%%%%%
\subsection{Actor of a $2$-fold crossed module} \label{subs:actor-xxmod}

\begin{defn} \index{Whitehead group} \index{regular derivation} 
The Whitehead group $W_1(\calR)$ is defined to be the group of the monoid 
$\Der_1(\calR)$. 
Elements of $W_1(\calR)$ are called regular derivations.
\end{defn}

\noindent
The following result is a higher-dimensional version of
Lemma \ref{lem:chi-sigma-rho}.

\begin{lem} \label{lem:equiv-w1s}
The following statements are equivalent 
\begin{enumerate}[\rm (a)]
\item $\chi = (\ddch, \dch) \in W_1(\calR) $
\item $\sigma_{\chi} = (\dds_{\chi}, \ds_{\chi})  \in \Aut(\ddcalRt)$
\item $ \rho_{\chi} = (\ddr_{\chi}, \dr_{\chi})  \in \Aut(\dcalRt)$
\end{enumerate}
\end{lem}
\begin{pf}
Here is an incomplete argument.
Assume that $\sigma_{\chi}, \rho_{\chi}$ are invertible.
Then, by Lemma \ref{lem:chi-sigma-rho}, 
$$
\ddch^{-1}n ~=~ (\dds_{\chi}^{-1}\ddch n)^{-1} 
            ~=~ (\ddch\ddr_{\chi}^{-1} n)^{-1}
\qquad\mbox{and}\qquad
\dch^{-1}p ~=~ (\ds_{\chi}^{-1}\dch p)^{-1} 
           ~=~ (\dch\dr_{\chi}^{-1} p)^{-1},
$$
and hence
$$
\ddbdy_2 \circ \ddch^{-1} (n)
~=~ (\ddbdy_2 \dds_{\chi}^{-1} \ddch n)^{-1}
~=~ (\ds_{\chi}^{-1} \ddbdy_2 \ddch n)^{-1}
~=~ (\ds_{\chi}^{-1} \dch \dbdy_2 n)^{-1}
~=~ \dch^{-1} \circ \dbdy_2 (n).
$$
So $\chi^{-1} = (\ddch^{-1},\dch^{-1}) \in W_1(\calR)$.
\end{pf}

\begin{defn}
The group $\Aut(\calR)$ of automorphisms of the crossed square $\calR$ is 
$$
\Aut(\calR) ~=~ \{ \alpha = 
(\alpha_{[2]}, \alpha_{\{1\}}, \alpha_{\{2\}}, \alpha_{\emptyset}) \}
$$ 
such that 
$(\alpha_{[2]}, \alpha_{\{1\}}) $  is an automorphism of  $\ddcalRt$,
$(\alpha_{[2]}, \alpha_{\{2\}}) $  is an automorphism of  $\ddcalRo$,
$(\alpha_{\{2\}}, \alpha_{\emptyset}) $  is an automorphism of  $\dcalRt$, and 
$(\alpha_{\{1\}}, \alpha_{\emptyset}) $  is an automorphism of  $\dcalRo$.
\end{defn}

\begin{lem} \label{lem:alpha-chi}
If $\chi \in \Der_1(\calR)$ then
$\alpha_{\chi} = (\dds,\ds,\ddr,\dr) \in \Aut(\calR)$
where
$$
\dds \ell = \ell(\ddch \ddbdyo \ell), \quad
\ds m = m(\dch \dbdyo m), \quad
\ddr n = n(\ddbdyo \ddch  n), \quad
\dr p = p(\dbdyo \dch p). 
$$
\end{lem}
\begin{pf}
This follows immediately from Lemma \ref{lem:equiv-w1s}.
\end{pf}

\begin{lem} 
There is a crossed module
$$
\calA_1{\calR} ~=~ (\dD_1 : W_1(\calR) \to \Aut(\calR), ~
\chi \mapsto \alpha_{\chi})
$$
where the action of $\Aut(\calR)$ on $W_1(\calR)$ is given by
$\chi^{\alpha} = \psi = (\ddps, \dps)$
where 
$$ 
\ddps : R_{\{2\}} \to R_{[2]}  \;=\; \alpha_{\{2\}}^{-1} * \ddch * \alpha_{[2]},
\quad \mbox{and} \quad  
\dps : R_{\emptyset} \to R_{\{1\}}  \;=\; 
\alpha_{\emptyset}^{-1} * \dch * \alpha_{\{1\}}.
$$
\end{lem}
\begin{pf}
We first show that the boundary map $\dD_1$ is a homomorphism.
The proof is the same as the proof of Lemma 3.3 in the notes.

Secondly we must show that the action is well-defined.
\begin{eqnarray*}
(\chi^{\alpha})^{\beta} 
  & = & ((\alpha_{\{2\}}^{-1} * \ddch * \alpha_{[2]})^{\psi_2}, \,  
        (\alpha_{\emptyset}^{-1} * \dch * \alpha_{\{1\}})^{\psi_2})  \\
  & = & (\beta_{\{2\}}^{-1} * (\alpha_{\{2\}}^{-1} * \ddch 
        * \alpha_{[2]}) * \beta_{[2]},\,
         \beta_{\emptyset}^{-1} * (\alpha_{\emptyset}^{-1} * \dch 
        * \alpha_{\{1\}}) * \beta_{\{1\}} )\\
  & = & ((\alpha_{\{2\}} \beta_{\{2\}})^{-1} * \ddch 
        * (\alpha_{[2]} \beta_{[2]}), \,  
         (\alpha_{\emptyset} \beta_{\emptyset})^{-1} * \dch 
        * (\alpha_{\{1\}} \beta_{\{1\}})) \\
  & = & \chi^{\alpha*\beta}
\end{eqnarray*}

Thirdly we show that $\psi = \chi^{\alpha}$ commutes with 
$\partial_1 = (\ddbdyo,\dbdyo)$, so that  $\dps \dbdyo = \ddbdyo \ddps$.
\begin{eqnarray*}
(\dps \dbdyo)(n) 
  & = & (\alpha_{\emptyset}^{-1} * \dch * \alpha_{\{1\}})(\dbdyo n) \\
  & = & (\alpha_{\{1\}} \dch \alpha_{\emptyset}^{-1})(\dbdyo n) \\
  & = & \alpha_{\{1\}} \dch \dbdyo \alpha_{\{2\}}^{-1} n \\
  & = & \alpha_{\{1\}} \ddbdyo \ddch \alpha_{\{2\}}^{-1} n \\
  & = & \ddbdyo \alpha_{[2]} \ddch \alpha_{\{2\}}^{-1} n \\
  & = & \ddbdyo(\alpha_{\{2\}}^{-1} \star \ddch \star \alpha_{[2]})(n) \\
  & = & \ddbdyo \ddps (n)~. \\
\end{eqnarray*}

Now we verify the first crossed module axiom,~
$\dD_1(\chi^{\alpha}) = \alpha^{-1} * \dD_1 \chi * \alpha$~
where
\begin{eqnarray*}
\alpha^{-1} * \dD_1\chi * \alpha 
  & = & (\alpha_{[2]}, \alpha_{\{1\}}, \alpha_{\{2\}}, 
         \alpha_{\emptyset})^{-1} * (\dds_{\chi}, 
        \ds_{\chi}, \ddr_{\chi}, \dr_{\chi}) 
        * (\alpha_{[2]}, \alpha_{\{1\}}, \alpha_{\{2\}}, \alpha_{\emptyset}) \\
  & = & (\alpha_{[2]}^{-1} * \dds_{\chi} * \alpha_{[2]}, \ 
         \alpha_{\{1\}}^{-1} * \ds_{\chi} * \alpha_{\{1\}}, \ 
         \alpha_{\{2\}}^{-1} * \ddr_{\chi} 
        * \alpha_{\{1\}}, \ \alpha_{\emptyset}^{-1} * \dr_{\chi} 
        * \alpha_{\emptyset})
\end{eqnarray*}
We verify just the first and the last coordinate, since the other two
are similar.
\begin{eqnarray*}
\dds_{\psi} \ell 
  & = & \ell(\ddps \ddbdyo \ell) \\
  & = & \ell(\alpha_{\{2\}}^{-1} * \ddch * \alpha_{[2]})(\ddbdyo \ell) \\
  & = & \ell (\alpha_{[2]}  \ddch  \alpha_{\{2\}}^{-1} \ddbdyo \ell) \\
  & = & \ell (\alpha_{[2]}  \ddch \ddbdyo  \alpha_{[2]}^{-1} \ell) \\
(\alpha_{[2]}^{-1} *   \dds_{\chi} *  \alpha_{[2]}) \ell 
  & = &  \alpha_{[2]}  \dds_{\chi}( \alpha_{[2]}^{-1} \ell) \\
  & = & \alpha_{[2]} \{ (\alpha_{[2]}^{-1} \ell)(\ddch \ddbdyo \alpha_{[2]}^{-1} 
         \ell) \} \\
  & = & \ell (\alpha_{[2]}  \ddch \ddbdyo  \alpha_{[2]}^{-1} \ell)
\end{eqnarray*}
\begin{eqnarray*}
\dr_{\dps}(p) 
  & = & p(\dbdyo \dps p) \\
  & = & p (\dbdyo(\alpha_{\emptyset}^{-1} * \dch * \alpha_{\{1\}}))(p)) \\
  & = & p (\dbdyo (\alpha_{\{1\}} \dch \alpha_{\emptyset}^{-1})(p)) \\
  & = & p (\alpha_{\{1\}} \dbdyo \dch \alpha_{\{1\}}^{-1} p) \\
(\alpha_{\emptyset}^{-1} *   \dr_{\chi} *  \alpha_{\emptyset}) 
  & = & \alpha_{\emptyset} \dr_{\chi}(\alpha_{\emptyset}^{-1} p) \\
  & = & \alpha_{\emptyset} \{ (\alpha_{\emptyset}^{-1} p) 
          (\dbdyo \dch  \alpha_{\emptyset}^{-1} p) \} \\
  & = & p (\alpha_{\{1\}} \dbdyo \dch \alpha_{\{1\}}^{-1} p)
\end{eqnarray*}  

Finally, we show that the second crossed module axiom holds:
$$ 
\chi_1 \star \chi_2 ~=~ \chi_2 \star {\chi_1}^{\dD_1 \chi_2} 
                    ~=~ \chi_2 \star \psi.
$$
We have $\dD_1 \chi_2 = (\dds_2, \ds_2, \ddr_2, \dr_2)$ and  
$\ddps = \ddr_2^{-1} * \ddch_1 * \dds_2$ and 
$\dps = \dr_2^{-1} * \dch_1 * \ds_2$,
so
$$
(\ddch_1 \star \ddch_2)(n) 
  ~=~ (\ddch_2 n)(\dds_2 \ddch_1 n) \\
  ~=~ (\ddch_2 n) (\ddps n) \\
  ~=~ (\ddch_2 \star \ddps)(n) \\
$$
$$
(\dch_1 \star \dch_2)(p) 
  ~=~ (\dch_2 p)(\ds_2 \dch_1 p) \\
  ~=~ (\dch_2 p) (\dps p) \\
  ~=~ (\dch_2 \star \dps)(p)~.
$$
(where we have used Lemma \ref{lem:gamma-beta-chi} (c)  
 -- need an $\calR$-version?).
\end{pf}


%%%%%%%%%%%%%%%%%%%%%%%%%%%%%%%%%%%%%%%%%%%%%%%%%%%%%%%%%%%%%%%%%%%%%%%%%
\subsection{Searching for a map from $G_{\emptyset}$ to $G_{[2]}$}
\label{subs:search}

We are hoping to find a function
$$
\xi ~:~ R_{\emptyset} \to (R_{\emptyset} \ltimes R_{\{1\}})
                 \ltimes (R_{\{2\}} \ltimes R_{[2]}),
\quad
p \mapsto ((p, \chi p), (\phi p, \theta p)),
$$
such that $\xi(pq) = (\xi p)(\xi q)$.
So we require
\begin{eqnarray*}
((pq, \chi(pq)), (\phi(pq), \theta(pq)))
  & = & ((p, \chi p),(\phi p, \theta p))((q, \chi q),(\phi q, \theta q)) \\
  & = & ((p, \chi p)(q, \chi q),
         (\phi p, \theta p)^{(q, \chi q)}(\phi q,\theta q)) \\
  & = & ((pq, (\chi p)^q \chi q),
         ((\phi p)^q, (\chi q \boxtimes (\phi p)^q)^{-1} 
                      (\theta p)^{q (\chi q)}) (\phi q, \theta q)) \\
  & = & ((pq, (\chi p)^q \chi q),
         ((\phi p)^q, ((\phi p)^q \boxtimes \chi q) 
                      (\theta p)^{q (\chi q)}) (\phi q, \theta q)) \\
  & = & ((pq, (\chi p)^q \chi q),
         ((\phi p)^q \phi q, ((\phi p)^q \boxtimes \chi q)^{\phi q} 
                      (\theta p)^{[q (\chi q)(\phi q)]} (\theta q))) \\
\end{eqnarray*}

\noindent
Thus $\chi$ has to be a derivation $R_{\emptyset} \to R_{\{1\}}$, 
and $\phi$ has to be a derivation $R_{\emptyset} \to R_{\{2\}}$, 
while $\theta : R_{\emptyset} \to R_{[2]}$  must be such that $\theta(pq)$
depends upon $p,q,\chi q,\phi p, \theta p, \theta q$, 
and satisfies 
$$
\theta(pq) ~=~ ((\phi p)^q \boxtimes \chi q)^{\phi q} 
                (\theta p)^{[q (\chi q)(\phi q)]} (\theta q).
$$

\bigskip\noindent
{\bf [This needs checking/expanding.]}



\vspace*{15mm}
%%%%%%%%%%%%%%%%%%%%%%%%%%%%%%%%%%%%%%%%%%%%%%%%%%%%%%%%%%%%%%%%%%%%%%%%%%%
\subsection{Further Oddments}

No entries here so far.


