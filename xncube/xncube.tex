%% xncube.tex, version 22/04/17


%%%%%%%%%%%%%%%%%%%%%%%%%%%%%%%%%%%%%
\section{Crossed $n$-cubes of groups}

\noindent
Here we include the basic ideas about crossed cubes (of groups)
taken from Chapter 1 of Ellis' thesis \cite{ellis-thesis},
Ellis-Steiner \cite{ell:st}, and Brown-Loday \cite{brow:lod}, 
and the associated cat$^3$-groups.

%%%%%%%%%%%%%%%%%%%%%%%%%%%%%%%%%%%%%%%%%%%%
\subsection{Crossed cubes}\label{sect:xcube}

\noindent \index{crossed cube} 
Here we include the basic ideas about crossed cubes (of groups)
taken from Chapter 1 of Ellis' thesis \cite{ellis-thesis},
Ellis-Steiner \cite{ell:st}, and Brown-Loday \cite{brow:lod}, 
and the associated cat$^3$-groups.

\noindent\bigskip
Let $[3] = \{1,2,3\}$, and let $A,B,C,\ldots$ be subsets of $[3]$.

\begin{defn}
A crossed cube consists of the following.
\begin{enumerate}[{\rm (i)}]
\item
Groups  $R_A$  for each subset  $A$  of  $[3]$,
where we write $R$ for $R_{\emptyset}$.

\item
Group homomorphisms
$$
\partial_i : R_A \to R_{A \setminus \{i\}}  \quad\mbox{for all} \quad 
A \subseteq [3],~ i \in [3]
$$
such that  $\partial_i = \id_{R_A}$  when  $i \notin A$, and  
$\partial_i\partial_j = \partial_j\partial_i ~\mbox{for all}~ i,j \in [3]$.

\noindent
Since the $\partial_i$ commute, composite homomorphisms
$\partial_B = \bigcirc_{i \in B}\,\partial_i : R_A \to R_{A \setminus B}$ 
are well defined and 
$\partial_A = \partial_{A \setminus B} \circ \partial_B$. 

\item
For all $B \subseteq A$ an action of $R_{A \setminus B}$ on $R_A$
making $\calR_{A,A \setminus B} = (\partial_B : R_A \to R_{A \setminus B})$
a crossed module.

For each $j \in [3]$ the maps 
$$
(1,\partial_j) : \calR_{A,A \setminus \{i\}} \to 
\calR_{A,A \setminus \{i,j\}}
\quad\mbox{and}\quad
(\partial_j,1) : \calR_{A,A \setminus \{i,j\}} \to 
\calR_{A \setminus \{j\},A \setminus \{i,j\}}
$$
are crossed module homomorphisms.

It follows that all the actions act via $R$~:
$$
a^b ~=~ a^{\partial_B b} \quad\mbox{for}\quad
a \in R_A,~ b \in R_B, ~~\mbox{and}~~ B \subseteq A.
$$

\item
For all  $A,B \subseteq [3]$  a crossed pairing 
$\bt_{A,B} : R_A \times R_B \to R_{A \cup B}$,
such that
$(b\,\bt_{B,A}\,a) = (a\,\bt_{A,B}\,b)^{-1}$  and, when $B \subseteq A$,
$\bt_{A,B}$  is the principal crossed pairing for $\calR_{A,B}$ 
given by ~$a \bt b = a^{-1}a^b$ and $b \bt a = (a^{-1})^ba$.

\item
Various axioms relating the homomorphisms, actions, and crossed pairings, 
for example
\begin{itemize}
\item~ $\partial_i(a \bt_{A,B} b) ~=~ 
        (\partial_i a) \bt_{A\setminus\{i\},B\setminus\{i\}} (\partial_i b)$.
\item~ {\bf [More to follow?]}
\end{itemize}
\end{enumerate}
\end{defn}

\noindent
Note that the $\bt_{A,B}$ define actions of $B$ on $A$ for all 
$A,B \subseteq [3]$ by
$$
a^b ~:=~ a\ \partial_{B \setminus A}(a \bt b).
$$



%%%%%%%%%%%%%%%%%%%%%%%%%%
\subsection{Cat$^3$-groups} \index{cat$^3$-group} 

The corresponding notion of cat$^3$-group may be defined in a similar way.

\begin{defn} 
A cat$^3$-group consists of the following.
\begin{enumerate}[{\rm (i)}]
\item
Groups  $G_A$  for each subset  $A$  of  $[3]$. 

\item
Group homomorphisms
$$
t_i, h_i : G_A \to G_{A \setminus \{i\}}, ~~
e_i : G_{A \setminus \{i\}} \to G_A,
\quad\mbox{for all} \quad 
A \subseteq [3],~ i \in [3]
$$
such that
\begin{itemize}
\item~  
$t_i = h_i = e_i = \id_{R_A}$  when  $i \notin A$,
\item~ 
$t_i t_j = t_j t_i,~
 h_i h_j = h_j h_i ~\mbox{and}~
 e_i e_j = e_j e_i ~\mbox{for}~ i,j \in [3]$,
\item~
$t_i h_j = h_j t_i ~\mbox{for}~ i \neq j$.
\end{itemize}

\noindent
Since the $t_i, h_i$ and $e_i$ commute, composite homomorphisms
$t_B, h_B : R_A \to R_{A \setminus B}$ and 
$e_A : R_B \to R_{A \cup B}$ 
are well defined for all $A,B \subseteq [3]$.
\end{enumerate}
\end{defn}



\vspace{12mm}
%%%%%%%%%%%%%%%%%%%%%%%%%%%%%%%%%%%%%%%%%%%%%%%%%%%
\subsection{Crossed $n$-cubes with $n \geqslant 4$} 
\index{crossed $n$-cube}

Let $[n] = \{1,2,...,n\}$, and let $A,B,C,\ldots$ be subsets of $[n]$.

\bigskip
A crossed $n$-cube consists of the following.
\begin{enumerate}[(i)]
\item
Groups  $R_A$  for each subset  $A$  of  $[n]$,
where we write $R$ for $R_{\emptyset}$.

\item
Group homomorphisms
$$
\partial_i : R_A \to R_{A \setminus \{i\}}  \quad\mbox{for all} \quad 
A \subseteq [n],~ i \in [n]
$$
such that  $\partial_i = \id_{R_A}$  when  $i \notin A$,
and  $\partial_i\partial_j = \partial_j\partial_i ~\mbox{for}~ i,j \in [n]$.

\noindent
Since the $\partial_i$ commute, composite homomorphisms
$\partial_B = \bigcirc_{i \in B}\,\partial_i : R_A \to R_{A \setminus B}$ 
are well defined and 
$\partial_A = \partial_{A \setminus B} \circ \partial_B$. 

\item
For all $B \subseteq A$ an action of $R_{A \setminus B}$ on $R_A$
making $\calR_{A,A \setminus B} = (\partial_B : R_A \to R_{A \setminus B})$
a crossed module.

For each $j \in [n]$ the maps 
$$
(1,\partial_j) : \calR_{A,A \setminus \{i\}} \to 
\calR_{A,A \setminus \{i,j\}}
\quad\mbox{and}\quad
(\partial_j,1) : \calR_{A,A \setminus \{i,j\}} \to 
\calR_{A \setminus \{j\},A \setminus \{i,j\}}
$$
are crossed module homomorphisms.

It follows that all the actions act via $R$~:
$$
a^b ~=~ a^{\partial_B b} \quad\mbox{for}\quad
a \in R_A,~ b \in R_B, ~~\mbox{and}~~ B \subseteq A.
$$

\item
For all  $A,B \subseteq [n]$  a crossed pairing  ~
$\bt_{A,B} : R_A \times R_B \to R_{A \cup B}$~,
such that
$(b\,\bt_{B,A}\,a) = (a\,\bt_{A,B}\,b)^{-1}$  and, when $B \subseteq A$,
$\bt_{A,B}$  is the principal crossed pairing for $\calR_{A,B}$ 
given by ~$a \bt b = a^{-1}a^b$ and $b \bt a = (a^{-1})^ba$.

\item
Various axioms relating the homomorphisms, actions, and crossed pairings, 
for example
\begin{itemize}
\item~ $\partial_i(a \bt_{A,B} b) = 
        (\partial_i a) \bt_{A\setminus\{i\},B\setminus\{i\}} (\partial_i b)$.
\item~ when $i \in A \cap B$, so that
$A \cup B = (A \setminus \{i\}) \cup B = A \cup (B \setminus \{i\})$,
$$
a \bt_{A,B} b
~=~ \partial_i a \bt_{A\setminus\{i\},B} b
~=~ a \bt_{A,B\setminus\{i\}} \partial_i b.
$$
(This means that we need only define $\bt_{A,B}$ when $A \cup B = \emptyset$.)
\item~ $(a \bt b)^c ~=~ a^c \bt b^c$ when $C \subseteq A$ and $C \subseteq B$.
\qquad
\mbox{{\bf [Is this correct?]}}
\end{itemize}
\end{enumerate}

\bigskip
What wish to define an \emph{$n$-derivation}, 
which would seem to be a set of maps
$$
\chi_{B,A} : R_B \to R_A  \quad\mbox{for all}\quad  B \subseteq A
$$
satifying suitable axioms:
\begin{enumerate}[(i)]
\item
perhaps  \qquad 
$\chi_{B,A}(bb') = (\chi_{B,A} b)^{b'} (\chi_{B,A} b')$ ~?
\item
closure: \qquad
$\chi_{B,A} \circ \chi_{C,B} = \chi_{C,A}$ ~?
\item ???
\end{enumerate}

\begin{exercise}
\emph{Derive the crossed square axioms 
from those of a crossed $2$-cube.}
\end{exercise}


\bigskip\noindent
{\bf [These are just some thoughts to be worked on!]}


%%%%%%%%%%%%%%%%%%%%%%%%%%%
\subsection{Cat$^n$-groups}  \index{cat$^n$-group} 

A cat$^n$-group consists of the following.
\begin{enumerate}[(i)]
\item
Groups  $G_A$  for each subset  $A$  of  $[n]$. 

\item
Group homomorphisms
$$
t_i, h_i : G_A \to G_{A \setminus \{i\}}, ~~
e_i : G_{A \setminus \{i\}} \to G_A,
\quad\mbox{for all} \quad 
A \subseteq [n],~ i \in [n]
$$
such that
\begin{itemize}
\item~  
$t_i = h_i = e_i = \id_{R_A}$  when  $i \notin A$,
\item~ 
$t_i t_j = t_j t_i,~
 h_i h_j = h_j h_i ~\mbox{and}~
 e_i e_j = e_j e_i ~\mbox{for}~ i,j \in [n]$,
\item~
$t_i h_j = h_j t_i ~\mbox{for}~ i \neq j$.
\end{itemize}

\noindent
Since the $t_i, h_i$ and $e_i$ commute, composite homomorphisms
$t_B, h_B : R_A \to R_{A \setminus B}$ and 
$e_A : R_B \to R_{A \cup B}$ 
are well defined.
\end{enumerate}



