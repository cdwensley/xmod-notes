%% xmod-gpd.tex,  version 26/06/07


\newpage
%%%%%%%%%%%%%%%%%%%%%%%%%%%%%%%%%%%%%%%%%%%%%%%%%%%%%%%%%%%
\section{Crossed modules of groupoids} \label{sec:xmod-gpd}


%%%%%%%%%%%%%%%%%%%%%%%%%%%%%%%%%%%%%%%%%%%%%%%%%%%%%%%%%
\subsection{Basic definitions}  \label{sec:xmod-gpd-defs}

Let $\bbR=(R_1,R_0)$ be a groupoid and $\bbQ=(Q_1,Q_0)$ 
a union of groups with the same object set $Q_0=R_0$, 
and let $\bbQ,\bbR$ act upon themselves by conjugation 
(as in (\ref{eq:gpd-conj})):
$$
{q_1}^q = q^{-1}q_1q, 
\qquad {r_1}^r = r^{-1} \diamond r_1 \diamond r, 
$$
defined when $q_1,q$ are loops at the same vertex and when $sr = tr_1$. 
A \emph{pre-crossed module of groupoids}  
\index{pre-crossed module!of groupoids}
$\calQ = (\delta : \bbQ \to \bbR)$ 
consists of a morphism of groupoids $(\delta,\id)$ 
(which we usually call $\delta$), 
the \emph{boundary} of $\calQ$, \index{boundary}
$$
\bbQ ~=~ 
\vcenter{
\xymatrix{ 
  Q_1  \ar[r]^{\delta}  \ar[d]<+0.5ex>^t \ar[d]<-0.5ex>_s 
     &  R_1 \ar[d]<+0.5ex>^t \ar[d]<-0.5ex>_s \\ 
  Q_0  \ar[r]_{\id}
     &  R_0  \\
}}
\qquad\text{or, more simply,}\qquad
\xymatrix{ 
  Q_1  \ar[r]^{\delta}  
     &  R_1 \ar[r]<+0.5ex>^s \ar[r]<-0.5ex>_t 
        &  R_0 ~, \\
}
$$
together with an action $\alpha$ of $\bbR$ on $\bbQ$ 
such that $\delta$  is an $\bbR$-morphism.
So, for all $q \in \Arr(\bbQ)$  and  $r \in \Arr(\bbR)$,
\begin{center}
\begin{tabular}{c r c l }
\textbf{X1:} 
  &  $\delta(q^r)$   &  =  &  $r^{-1} \diamond (\delta q) \diamond r$~.
\end{tabular}
\end{center}

\noindent
The pre-crossed module  $\calQ$  is a \emph{crossed module of groupoids}
\index{crossed module!of groupoids}
if it also satisfies, for all  $q_1,q \in \Arr(\bbQ)$,
\begin{center}
\begin{tabular}{c r c l }
\textbf{X2:} 
  &  ${q_1}^{\delta q}$  &  =  &  $q^{-1}q_1q$~. 
\end{tabular}
\end{center}
Note that, for all $u \in Q_0$, $(\delta : \bbQ(u) \to \bbR(u))$ 
is a crossed module of groups. 

\medskip 
\index{morphism!of crossed modules}
A \emph{morphism of pre-crossed modules} $\alpha : \calQ_1 \to \calQ_2$ 
is a pair  $\alpha = (\dda, \da)$, 
where $\dda : Q_1 \to Q_2$ and $\da : R_1 \to R_2$ 
are morphisms of groupoids satisfying
$$
\delta_2 \dda = \da \delta_1, \; \;
\dda(q^r) = (\dda  q)^{\da r},
$$
making the following diagram commute:
$$
\xymatrix{ 
  Q_1 \ar[r]^{\dda} \ar[d]_{\delta_1}
     & Q_2 \ar[d]^{\delta_2} \\
  R_1 \ar[r]_{\da}
     & R_2
}
$$
When $\bbQ_1,\bbQ_2$ are crossed modules, 
$\alpha$ is a morphism of crossed modules.


\begin{example}
Let $N \unlhd G$ and $\bbG = \Gpd(G)$. 
Take $\bbC = \bbG\times\bbI_n$ and let $\bbN$ be the totally disconnected 
subgroupoid, consisting of $n$ copies of $\Gpd(N)$. 
Then $\bbC$ acts on $\bbN$ by conjugation, 
and $\calX = (\iota : \bbN \to \bbC)$ is a \emph{conjugation crossed module}, 
where $\iota$ is the inclusion map. 
\end{example}



\newpage
%%%%%%%%%%%%%%%%%%%%%%%%%%%%%%%%%%%%%%%%%%%%%%%%%%%%%%%%%
\subsection{Derivations of a crossed module of groupoids}

Recall that for $\calX = ((\partial,\id) : \bbC_2 \to \bbC_1$ 
a $1$-homotopy $H : f \simeq g$, where $f,g$ are automorphisms of $\calX$, 
is a pair of functions $(H_1:C_1 \to C_2,~ H_0:C_0 \to C_1)$ 
satisfying various conditions including, in particular, 
\begin{equation} \label{eq:deriv-cond}
H_1(cc') = (H_1c)^{gc'}(H_1c').
\end{equation}

In the special case that $g=\id_{\calX}$ we call $H$ a \emph{free derivation}. 
In another special case, when $H_0u = 1_u$ for all $u \in C_0$ 
we call $H$ a \emph{derivation over the identity}. 
A free derivation over the identity is simply called a \emph{derivation}. 

\noindent
[It might be better to call $H_1$ the free derivation or derivation?]

\medskip
We show first that when $\bbC_1 = \bbG \times \bbI_n$ that a derivation 
is determined by 
\begin{itemize}
\item
a derivation $\chi : C_1(1) \to C_2(1)$ 
for the crossed module of groups $(\partial : C_2(1) \to C_1(1))$, 
so that $H_1(1,c,1) = (1,\chi c,1)$, 
\item
a choice of images $H_1(1,e,j) = (j,a_j,j),~ 2 \leqslant j \leqslant n$. 
\end{itemize}
Applying condition (\ref{eq:deriv-cond}) we find 
\begin{eqnarray*}
H_1(j,e,1) &=& (H_1(1,e,j)^{-1})^{(j,e,1)} 
               ~=~ (1,a_j^{-1},1), \\
H_1(1,c,j) &=& (1,\chi c,1)^{(1,e,j)}(j,a_j,j) 
               ~=~ (j,(\chi c)a_j,j), \\
H_1(j,c,j) &=& (1,a_j^{-1},1)^{(1,c,j)}(1,\chi c,1)^{(1,e,j)}(j,a_j,j) 
               ~=~ (j,(a_j^{-1})^c(\chi c)a_j,j), \\
H_1(j,c,1) &=& (j,(a_j^{-1})^c(\chi c)a_j,j)^{(j,e,1)}(1,a^{-1},1) 
               ~=~ (1,(a_j^{-1})^c(\chi c),1), \\
\text{and, in general,}\qquad
H_1(j,c,k) &=& (k,(a_j^{-1})^c(\chi c)a_k,k). 
\end{eqnarray*}

\noindent 
When $H,K$ are two derivations, 
determined by $\chi,\psi : C_1(1) \to C_2(1)$ respectively, 
and by $H_1(1,e,j) = (j,a_j,j),~ K_1(1,e,j) = (j,b_j,j)$, 
the Whitehead product $H_1 \star K_1$ is given, as usual, by the formula 
$$
(H_1 \star K_1)x ~=~ (K_1 x)(H_1 x)(K_1 \partial H_1 x). 
$$
For the loops at $1$ this gives 
$$
(H_1 \star K_1)(1,c,1) ~=~ (1, (\chi\star\psi)c, 1),  
$$
while the image of $(1,e,j)$ is given by 
\begin{eqnarray*}
(H_1 \star K_1)(1,e,j) 
  &=&  (j,b_j,j)(j,a_j,j)(K_1\partial(j,a_j,j)) \\
  &=&  (j,b_ja_j,j)(K_1(j,\partial a_j,j)) \\
  &=&  (j,b_ja_j,j)(j,(b_j^{-1})^{\partial a_j}(\psi\partial a_j)b_j,j) \\
  &=&  (j,a_j(\psi\partial a_j)b_j,j).
\end{eqnarray*}

\medskip
Repeating the above calculations for homotopies 
$H : f \simeq g$ over the identity, where $g \neq 1$, 
we show that when $\bbC_1 = \bbG \times \bbI_n$ that a derivation 
over the identity is determined by 
\begin{itemize}
\item
a $g$-derivation $\chi : C_1(g1) \to C_2(g1)$ 
for the crossed module of groups $(\partial : C_2(1) \to C_1(1))$, 
so that $H_1(1,c,1) = (g1,\chi c,g1)$, 
\item
a choice of images $H_1(1,e,j) = (gj,a_j,gj),~ 2 \leqslant j \leqslant n$, 
\item 
the elements $g_j \in G$ where $g(1,e,j) = (g1,g_j,gj)$. 
\end{itemize}
Again, applying condition (\ref{eq:deriv-cond}), we find 
\begin{eqnarray*}
H_1(j,e,1) &=& (H_1(1,e,j)^{-1})^{(gj,g_j^{-1},g1)} 
               ~=~ (g1,(a_j^{-1})^{g_j^{-1}},g1), \\
H_1(1,c,j) &=& (g1,\chi c,g1)^{(g1,g_j,gj)}(gj,a_j,gj) 
               ~=~ (gj,(\chi c)^{g_j}a_j,gj), \\
H_1(j,c,j) &=& (g1,a_j^{-1},g1)^{(g1,gc,gj)}(g1,\chi c,g1)^{(g1,e,gj)}(gj,a_j,gj) 
               ~=~ (gj,(a_j^{-1})^{gc}(\chi c)a_j,gj), \\
H_1(j,c,1) &=& (gj,(a_j^{-1})^{gc}(\chi c)a_j,gj)^{(gj,e,g1)}(g1,a^{-1},g1) 
               ~=~ (g1,(a_j^{-1})^{gc}(\chi c),g1), \\
H_1(j,c,k) &=& (gk,(a_j^{-1})^{gc}(\chi c)a_k,gk). 
\end{eqnarray*}

\noindent 
When $H,K$ are two $g$-derivations, 
determined by $\chi,\psi : C_1(g1) \to C_2(g1)$ respectively, 
and by $H_1(1,e,j) = (gj,a_j,gj),~ K_1(1,e,j) = (gj,b_j,gj)$, 
the Whitehead product $H_1 \star K_1$ is given, as usual, by the formula 
$$
(H_1 \star K_1)x ~=~ (K_1 x)(H_1 x)(K_1 g^{-1} \partial H_1 x). 
$$
For the loops at $1$ this gives 
$$
(H_1 \star K_1)(1,c,1) ~=~ (g1, (\chi\star\psi)c, g1),  
$$
while the image of $(1,e,j)$ is given by 
\begin{eqnarray*}
(H_1 \star K_1)(1,e,j) 
  &=&  (gj,b_j,gj)(gj,a_j,gj)(K_1 g^{-1}\partial(gj,a_j,gj)) \\
  &=&  (gj,b_ja_j,gj)(K_1(j,g^{-1}\partial a_j,j)) \\
  &=&  (gj,b_ja_j,gj)(gj,(b_j^{-1})^{\partial a_j}(\psi g^{-1}\partial a_j)b_j,gj) \\
  &=&  (gj,a_j(\psi g^{-1} \partial a_j)b_j,gj).
\end{eqnarray*}









\subsection{Homotopies of crossed modules of groupoids}

This subsection is intended to cover section 2 of
Brown and \.{I}\c{c}en \cite{brow:icen}.


