%% xmod-gpd.tex,  version 22/04/17


\newpage
%%%%%%%%%%%%%%%%%%%%%%%%%%%%%%%%%%%%%%%%%%%%%%%%%%%%%%%%%%%
\section{Crossed modules of groupoids} \label{sec:xmod-gpd}


%%%%%%%%%%%%%%%%%%%%%%%%%%%%%%%%%%%%%%%%%%%%%%%%%%%%%%%%%
\subsection{Basic definitions}  \label{sec:xmod-gpd-defs}

Let $\bbC_1=(C_1,C_0)$ be a groupoid and $\bbC_2=(C_2,C_0)$ 
a union of groups with the same object set, 
and let $\bbC_2,\bbC_1$ act upon themselves by conjugation:
$$
{a_1}^a = a^{-1}a_1a, 
\qquad {c_1}^c = c^{-1} \diamond c_1 \diamond c, 
$$
defined when $a_1,a$ are loops in $\bbC_2$ at the same object 
and when $sc = sc_1 = tc_1$ in $\bbC_1$. \\
{\bf [But is this the correct notion of conjugacy?]}

\medskip
A \emph{pre-crossed module of groupoids}  
\index{pre-crossed module!of groupoids}
$\calC = (\gamma : \bbC_2 \to \bbC_1)$ 
consists of a morphism of groupoids $\gamma = (\gamma,\id)$ 
(abusing notation), 
the \emph{boundary} of $\calC$, \index{boundary} 
pictured as: 
$$
\bbC_2 ~=~ 
\vcenter{
\xymatrix{ 
  C_2  \ar[r]^{\gamma}  \ar[d]<+0.5ex>^t \ar[d]<-0.5ex>_s 
     &  C_1 \ar[d]<+0.5ex>^t \ar[d]<-0.5ex>_s \\ 
  C_0  \ar[r]_{\id}
     &  C_0  \\
}}
\qquad\text{or, more simply,}\qquad
\xymatrix{ 
  C_2  \ar[r]^{\gamma}  
     &  C_1 \ar[r]<+0.5ex>^s \ar[r]<-0.5ex>_t 
        &  C_0 ~, \\
}
$$
together with an action %%% $\alpha$ 
of $\bbC_1$ on $\bbC_2$ 
such that $\gamma$  is a $\bbC_1$-morphism.
So $\gamma*s=s,~ \gamma*t=t$ and, 
for all $a \in \Arr(\bbC_2)$  and  $c \in \Arr(\bbC_1)$,
\begin{center}
\textbf{X1:} \qquad
  $\gamma(a^c) ~=~ c^{-1} (\gamma a) c$ 
  \quad when \quad $sa = ta = sc$.\qquad
\end{center}

\noindent
The pre-crossed module  $\calC$  is a \emph{crossed module of groupoids}
\index{crossed module!of groupoids}
if it also satisfies 
\begin{center}
\textbf{X2:} \qquad
  ${a_1}^{\gamma a} ~=~ {a_1}^a$ 
  \quad for all \quad  
  $a_1,a \in \bbC_2(u),~ u \in C_0$.  
\end{center}
Note that, when both axioms are satisfied, 
the restriction $(\gamma_u : \bbC_2(u) \to \bbC_1(u))$ 
is a crossed module of groups for all $u \in C_0$. 

\medskip 
\index{morphism!of crossed modules}
A \emph{morphism of pre-crossed modules} $\alpha : \calC \to \calD$, 
where $\calD = (\delta : \bbD_2 \to \bbD_1)$, 
is a triple  $(\alpha_2, \alpha_1, \alpha_0)$, 
where $(\alpha_2,\alpha_0) : \bbC_2 \to \bbD_2$ 
and $(\alpha_1,\alpha_0) : \bbC_1 \to \bbD_1$ 
are morphisms of groupoids satisfying 
$$
\alpha_2*\delta ~=~ \gamma*\alpha_1, \qquad
\alpha_2(a^c) ~=~ (\alpha_2 a)^{\alpha_1 c},
$$
making the following diagram commute:
$$
\xymatrix{ 
  C_2 \ar[r]^{\gamma} \ar[d]_{\alpha_2}
    &  C_1 \ar[d]_{\alpha_1} \ar[r]<+0.5ex>^s \ar[r]<-0.5ex>_t 
       &  C_0  \ar[d]^{\alpha_0}  \\
  D_2 \ar[r]_{\delta} 
    &  D_1 \ar[r]<+0.5ex>^s \ar[r]<-0.5ex>_t 
       &  D_0
}
$$
When $\calC,\calD$ are crossed modules, 
$\alpha$ is a morphism of crossed modules.


\begin{example}
Let $N \unlhd G$ and $\bbG = \Gpd(G)$. 
Take $\bbC_1 = \bbG\times\bbI_n$ and let $\bbC_2 = \bbN \times \bbO_n$ 
be the totally disconnected subgroupoid 
consisting of $n$ copies of $\bbN=\Gpd(N)$. 
Then $\bbC_1$ acts on $\bbC_2$ by conjugation, 
and $\calC = (\iota : \bbC_2 \to \bbC_1)$ is a 
\emph{conjugation crossed module}, where $\iota$ is the inclusion map. 
\end{example}



\newpage
%%%%%%%%%%%%%%%%%%%%%%%%%%%%%%%%%%%%%%%%%%%%%%%%%%%%%%%%%
\subsection{Homotopies of a crossed module of groupoids}

This subsection is intended to cover section 2 of
Brown and \.{I}\c{c}en \cite{brow:icen}.

\bigskip
Let $g,h : \calC=(\gamma:\bbC_2\to\bbC_1) \to \calD=(\delta:\bbD_2\to\bbD_1)$ 
be morphisms of crossed modules. 
A \emph{$(g,h)$-homotopy} $H : g \simeq h$ 
is a pair of functions $(H_0,H_1)$ such that 
\begin{itemize} 
\item
$H_0 : C_0 \to C_1$ is a $(g,h)$-section:~ $H_0 * s = g_0,~~ H_0 * t = h_0$, 
\item
$H_1 : C_1 \to C_2$ satisfies:~ 
$H_1*t = h_1*t,~~ H_1(cc') ~=~ (H_1c)^{h_1c'}(H_1c')$, 
\item 
for all $(c : u \to v) \in \bbC_1$ and $(a : v \to v) \in \bbC_2$, 
$\delta H_1 c$ and $H_1 \gamma a$ measure the divergence from 
commutativity of the following squares 
(in the second square dashed lines denote arrows in $\bbC_2$), 
$$
\xymatrix{
g_0u \ar[rr]^{g_1c} \ar[dd]_{H_0u} 
  &&  g_0v \ar[dd]^{H_0v}
      &&  &&  g_0v \ar[rr]^{g_2a} \ar@{-->}[dd]_{H_0v} 
              &&  g_0v \ar@{-->}[dd]^{H_0v} \\
  &&  &&  &&  &&                         \\ 
h_0u \ar[rr]_{h_1c} 
  &&  h_0v 
      &&  &&  h_0v \ar[rr]_{h_2a} 
              &&  h_0v
}$$
\begin{equation}  \label{eq:comm-rules}
\delta H_1 c ~=~ (h_1c)^{-1}(H_0u)^{-1}(g_1c)(H_0v), \qquad 
H_1 \gamma a ~=~ (h_2a)^{-1} (g_2a)^{H_0v}. 
\end{equation}
\end{itemize}

When $H$ is a $(g,h)$-homotopy, we call $H_1$ a $(g,h)$-derivation. 
We shall usually be concerned with the case $\calC=\calD$, 
so that $g,h$ are automorphisms. 
In the special case that $g=\id_{\calC}$ we call $H$ a \emph{free homotopy} 
and $H_1$ a \emph{free derivation}. 
In another special case, when $H_0u = 1_u$ for all $u \in C_0$ 
we call $H$ a \emph{homotopy over the identity} 
and $H_1$ a \emph{derivation over the identity}. 
A free derivation over the identity is simply called a \emph{derivation}. 

\medskip
Consider the case when $\calC = (\gamma : \bbC_2 \to \bbC_1)$ is connected, 
so $\bbC_1 = \bbG_1 \ltimes \bbI_n,~ \bbC_2 = \bbG_2 \times \bbO_n,~ 
\bbG_1 = \Gpd(G_1),~ \bbG_2 = \Gpd(G_2)$, and 
$\calX = (\overline{\gamma} : G_2 \to G_1)$ is a crossed module of groups, 
where $\gamma(j,a,j) = (j,\overline{\gamma}a,j)$. 
Because of the multiplication rule for $H_1$, we may define an $h$-derivation 
by specifying the images of a generating set 
(just as we did for automorphisms).  
Thus an $h$-derivation of $\calC$ is determined by 
\begin{itemize}
\item
a derivation $\chi : G_1 \to G_2$ for the crossed module $\calX$, 
so $\chi(cc') = (\chi c)^{c'}(\chi c')$, 
\item
a choice of images $H_1(1,e,q) = (h_0q,a_q,h_0q),~ 2 \leqslant q \leqslant n$ 
for arrows in the tree $T_1$. 
\end{itemize}
The $h$-derivation for $\calX$ associated to $\chi$ 
is given by $\psi c = \chi h_1 c$, 
so that $\psi(cc') = (\psi c)^{h_1 c'}(\psi c')$, 
and we define $H_1(1,c,1) = (h_01,\chi h_1 c, h_01)$. 
Applying the multiplication rule, we find 
\begin{eqnarray*}
H_1(q,e,1) &=& (H_1(1,e,q)^{-1})^{h(q,e,1)} 
               ~=~ (h_01,a_q^{-1},h_01), \\
H_1(1,c,q) &=& (h_01,\chi h_1c,h_01)^{h(1,e,q)}(h_0q,a_q,h_0q) 
               ~=~ (h_0q,(\chi h_1 c)a_q,h_0q), \\
H_1(q,c,q) &=& (h_01,a_q^{-1},h_01)^{h(1,c,q)}
                (h_01,\chi h_1c,h_01)^{h(1,e,q)}(h_0q,a_q,h_0q) 
               ~=~ (h_0q,(a_q^{-1})^{h_1c}(\chi h_1c)a_q,h_0q), \\
H_1(q,c,1) &=& (h_0q,(a_q^{-1})^{h_1c}(\chi h_1 c)a_q,h_0q)^{h(q,e,1)} 
                (h_01,a_q^{-1},h_01) 
               ~=~ (h_01,(a_q^{-1})^{h_1c}(\chi h_1c),h_01), \\
H_1(q,c,r) &=& (h_0r,(a_q^{-1})^{h_1c}(\chi h_1c)a_r,h_0r). 
\end{eqnarray*}

\medskip\noindent
We may check the multiplication rule as follows: 
\begin{eqnarray*}
 & &  H_1 \left( (p,c,q)(q,c',r) \right) \\
 &=&  (h_0q,(a_p^{-1})^{h_1c}(\chi h_1c)a_q,h_0q)^{h(q,c',r)}~
       (h_0r,(a_q^{-1})^{h_1c'}(\chi h_1c')a_r,h_0r) \\ 
 &=&  (h_0r,((a_p^{-1})^{h_1c}(\chi h_1c)a_q)^{h_1c'},h_0r) 
       (h_0r,(a_q^{-1})^{h_1c'}(\chi h_1c')a_r,h_0r) \\ 
 &=&  (h_0r,(a_p^{-1})^{(h_1c)(h_1c')}(\chi h_1c)^{h_1c'}{a_q}^{h_1c'} 
             (a_q^{-1})^{h_1c'}(\chi h_1c')a_r,h_0r) \\ 
 &=&  (h_0r,(a_p^{-1})^{h_1(cc')}(\chi h_1c)^{h_1c'}(\chi h_1c')a_r,h_0r) \\ 
 &=&  (h_0r,(a_p^{-1})^{h_1(cc')}(\chi h_1(cc')a_r,h_0r) \\ 
 &=&  H_1(p,cc',r).
\end{eqnarray*}


\bigskip\noindent
{\bf [It is the case that the multiplication rule holds for this $H_1$, 
but we probably also need to check the axioms in (\ref{eq:comm-rules}).]}

\bigskip\noindent 
When $H,K$ are two $(g,h)$-homotopies determined by derivations 
$\chi,\zeta$ of $\calX$, 
by $H_1(1,e,q) = (h_0q,a_q,h_0q)$, and by $K_1(1,e,q) = (h_0q,b_q,h_0q)$, 
the Whitehead product $H_1 \star K_1$ is given, as usual, by the formula 
$$
(H_1 \star K_1)x ~=~ (K_1 x)(H_1 x)(K_1 \gamma h_1^{-1} H_1 x). 
\qquad\text{{\bf ???}} 
$$

\medskip\noindent
{\bf [The following calculation is probably wrong, and needs to be checked!]}

\medskip\noindent
For the loops at $1$ this gives 
$$
(H_1 \star K_1)(1,c,1) ~=~ (h_01, (\chi\star\zeta)h_1c, h_01),  
$$
while the image of $(1,e,q)$ is given by 
\begin{eqnarray*}
(H_1 \star K_1)(1,e,q) 
  &=&  (h_0q,b_q,h_0q)(h_0q,a_q,h_0q)(K_1\gamma(h_0q,a_q,h_0q)) \\
  &=&  (h_0q,b_qa_q,h_0q)(K_1(h_0q,\gamma a_q,h_0q)) \\
  &=&  (h_0q,b_qa_q,h_0q)(h_0q,(b_q^{-1})^{\gamma a_q}
         (\zeta\gamma a_q)b_q,h_0q) \\
  &=&  (h_0q,a_q(\zeta\gamma a_q)b_q,h_0q).
\end{eqnarray*}

