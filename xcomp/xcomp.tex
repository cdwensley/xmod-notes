%% xcomp.tex  (version 28/03/18)

\section{Crossed complexes}

The main references for this section are the book \cite{brow:siv} 
which covers much of the material in the papers 
\cite{brow:gilb,brow:higg:1987,brow:higg:1989,brow:higg:1990,brow:higg:1991}. 

\begin{defn}
A (many object) crossed complex $(\calC,\chi)$ is a sequence of morphisms 
of groupoids with common object set $C_0$,
\begin{equation} \label{eq:xcomp} 
\xymatrix{ 
  \ldots  \ar[r]^{\chi_{n+1}} 
    &  C_n  \ar[r]^{\chi_n} \ar[d]_{t=h} 
       &  C_{n-1}  \ar[r]^{\chi_{n-1}} \ar[d]_{t=h} 
          & \ldots \ar[r]^{\chi_3} 
             &  C_2  \ar[r]^{\chi_2} \ar[d]_{t=h} 
                &  C_1 \ar[d]<+0.7ex>^h \ar[d]<-0.7ex>_t  \\
    &  C_0 
       &  C_0 
          &  &  C_0 
                &  C_0 \\
}
\end{equation}
where: 
\begin{itemize}
\item
$\bbC_n = (C_n,C_0)$ is a groupoid for $n \geqslant 1$ and, 
for $n \geqslant 2$, $\bbC_n$ is totally disconnected, 
\item
for $n \geqslant 3$, $\bbC_n$ is abelian, and is a $\bbC_1$-module 
such that the image of $\chi_2$ acts trivially,   
\item
for $n \geqslant 2$, $((\chi_n,\id) : \bbC_n \to \bbC_{n-1})$  
is a groupoid morphism which preserves the action of $\bbC_1$ 
(in the case $n=2$ the action of $\bbC_1$ on itself is conjugation) so, 
for $a \in C_1, c_n \in C_n$, we have $\chi_n({c_n}^a) = (\chi c_n)^a$, 
\item
$((\chi_2,\id) : \bbC_2 \to \bbC_1)$ is a crossed module of groupoids, 
so $\{a_2\}^{\chi_2b_2} ~=~ b_2^{-1}a_2b_2$, 
\item
for $n \geqslant 3$ the composite $\chi_n*\chi_{n-1}$ is the zero map. 
\end{itemize}
\end{defn}

The repeated $C_0$ in equation (\ref{eq:xcomp}) is rather untidy, 
so the diagram of a crossed complex is usually simplified, 
as shown for $\bbC$ and $\bbD$ in (\ref{eq:xcomp-mor}). 

For convenience in the general formulae to be considered later, 
we specify source and target maps on objects by defining 
$s,t : C_0 \to C_0$ to be the identity map on the set of objects. 

\medskip
A \emph{morphism of crossed complexes} $\phi:(\calC,\chi) \to (\calD,\delta)$ 
is a family of groupoid morphisms 
$\{(\phi_n,\phi_0) : \bbC_n \to \bbD_n\}_{n \geqslant 1}$ 
\begin{equation} \label{eq:xcomp-mor} 
\xymatrix{ 
  \ldots  \ar[r]^{\chi_{n+1}} 
    &  C_n  \ar[r]^{\chi_n}  \ar[d]_{\phi_n} 
       &  C_{n-1}  \ar[r]^{\chi_{n-1}}  \ar[d]_{\phi_{n-1}}  
          & \ldots \ar[r]^{\chi_3}  
             &  C_2  \ar[r]^{\chi_2} \ar[d]_{\phi_2}
                &  C_1 \ar[r]<+0.5ex>^t \ar[r]<-0.5ex>_h \ar[d]_{\phi_1} 
                   &  C_0  \ar[d]^{\phi_0} \\
  \ldots  \ar[r]_{\delta_{n+1}} 
    &  D_n  \ar[r]_{\delta_n} 
       &  D_{n-1}  \ar[r]_{\delta_{n-1}} 
          & \ldots  \ar[r]_{\delta_3} 
             &  D_2  \ar[r]_{\delta_2} 
                &  D_1 \ar[r]<+0.5ex>^{t'} \ar[r]<-0.5ex>_{h'}  
                   &  D_0   \\
}
\end{equation}
compatible with the morphisms and actions on $\calC,\calD$, so that 
$$
\phi_n*\delta_n ~=~ \chi_n*\phi_{n-1} 
\qquad\text{and}\qquad 
\phi_n({c_n}^a) ~=~ (\phi_n c_n)^{\phi_1 a}
\qquad \forall~ c_n \in C_n,~ n \geqslant 2,~ a \in C_1.
$$
The category \catXComp\; has crossed complexes as objects 
and their morphisms as arrows. 

\begin{example}
\emph{The \emph{unit interval crossed complex} $(\calI,1)$ 
has for $\bbI_1$ the groupoid $\bbI$ of Example \ref{ex:unit-gpd} and, 
for $n \geqslant 2$, $\bbI_n$ is the trivial groupoid $\ids(\calI)$. 
The morphisms are all inclusion morphisms. 
}\end{example} 

When, for $n>m$, $\bbC_n$ is the trivial groupoid $\ids(\calC)$ 
of Example \ref{ex:triv-gpd}, and $\chi_n$ is the inclusion morphism, 
we say that $\calC$ is \emph{$m$-truncated}. 
This means that the structures above level $m$ may be ignored, 
so that a $1$-truncated crossed complex is effectively a groupoid 
(for example $(\bbI,\iota)$), 
and a $2$-truncated crossed complex is a crossed module of groupoids. 
We denote the $m$-truncated subcrossed complex of $\calC$ by $\calC^m$. 
(We could refer here to the $m$-th \emph{skeleton functor} $\sk^m$.) 



%%%%%%%%%%%%%%%%%%%%%%%%%%%%%%%%%%%%%%%%%%%%%%%%
\subsection{Tensor Product of Crossed Complexes} 
\label{subsect:xcomp-tensor}

In \cite{brow:siv} the tensor product $\calC\otimes\calD$ of 
crossed complexes $\calC,\calD$ is defined using a universal bimorphism, 
but we will not discuss bimorphisms of crossed complexes here 
(not yet, anyhow). 

\begin{defn} \label{defn:xcomp-tensor}
Let $(\calC,\chi),(\calD,\delta)$ be crossed complexes. 
Then $(\calC\otimes\calD,\partial)$ is the crossed complex generated 
by elements $c \otimes d$ in dimension $m+n$, where $c \in C_m, d \in D_n$, 
with the following defining relations (plus the laws for crossed complexes). 

\medskip\noindent
{\bf Source and target:}
$$
s(c \otimes d) = sc \otimes sd, \qquad
t(c \otimes d) = tc \otimes td.  
$$

\medskip\noindent
{\bf Action axioms:}
$$
c \otimes d^{d_1} = (c \otimes d)^{(tc \otimes d_1)}~~ 
\text{when}~ n \geqslant 2,~ d_1 \in D_1, \qquad
c^{c_1} \otimes d = (c \otimes d)^{(c_1 \otimes td)}~~ 
\text{when}~ m \geqslant 2,~ c_1 \in C_1.
$$

\medskip\noindent
{\bf Product axioms:}
\begin{eqnarray*}
c \otimes (dd_1) &=& \left\{ \begin{array}{ll} 
                    (c \otimes d)^{(tc \otimes d_1)}(c \otimes d_1) 
                       & \text{if}~ n=1,~ m \geqslant 1, \\
                    (c \otimes d)(c \otimes d_1) 
                       & \text{otherwise}, \\
                    \end{array} \right.                            \\
(cc_1) \otimes d &=& \left\{ \begin{array}{ll} 
                    (c_1 \otimes d)(c \otimes d)^{(c_1 \otimes td)} 
                       & \text{if}~ m=1,~ n \geqslant 1, \\
                    (c \otimes d)(c_1 \otimes d) 
                       & \text{otherwise}. \\
                    \end{array} \right. 
\end{eqnarray*}

\medskip\noindent
{\bf Boundary map:}
$$
\partial_{m+n}(c \otimes d) ~=~ 
  \left\{ \begin{array}{ll}
  (tc \otimes d)^{-1}(c \otimes sd)^{-1}(sc \otimes d)(c \otimes td) 
    & \text{if}~ m=n=1, \\
  c \otimes \delta_n d 
    & \text{if}~ m=0,~ n \geqslant 2, \\
  \chi_m c \otimes d 
    & \text{if}~ m \geqslant 2,~ n=0, \\
  (c \otimes \delta_n d)^{-1}(tc \otimes d)^{-1}(sc \otimes d)^{(c \otimes td)}
    & \text{if}~ m=1,~ n \geqslant 2, \\
  (c \otimes td)^{(-1)^{m+1}}
     \left((c \otimes sd)^{(tc \otimes d)}\right)^{(-1)^m}(\chi_m c \otimes d) 
    & \text{if}~ n=1,~ m \geqslant 2, \\
  (\chi_m c \otimes d)(c \otimes \delta_n d)^{(-1)^m} 
    & \text{if}~ n \geqslant 2,~ m \geqslant 2. 
  \end{array} \right.
$$
\end{defn}

\noindent
Note that $(\calC \otimes \calD)_0 = C_0 \times D_0$, 
so we may write $u \otimes x$ as $(u,x)$ when $u \in C_0, x \in D_0$. 

\medskip 
The groupoid $((\calC \otimes \calD)_1, (\calC \otimes \calD)_0)$ 
is isomorphic to $C_1\ \#\ D_1$, the groupoid coproduct of 
$\calC^1 \times \ids(\calD^1)$ and $\ids(\calC^1) \times \calD^1$. 
Every element of $C_1\ \#\ D_1$ is uniquely expressible in one 
of the following normal forms. 
\begin{enumerate}[(i)] 
\item
An identity arrow $(1_u,1_x)$. 
\item
A \emph{generating arrow} $(c,1_x)$ or $(1_u,d)$, 
where $c \in C_1,~ x \in D_0,~ u \in C_0,~ d \in D_1$ 
and $c,d$ are not identities. 
We write $\arr(c,1_x)=c,~ \obj(c,1_x)=x$ and $\arr(1_u,d)=d,~\obj(1_u,d)=u$. 
\item
A composite $k=k_1k_2\ldots k_n~ (n \geqslant 2)$ of generating arrows 
in which the $\arr(k_i)$ lie alternately in $C_1$ and $D_1$ 
(or conversely), and the odd and even products 
~$\odd(k) = \arr(k_1)\arr(k_3)\cdots$~ 
and ~$\even(k) = \arr(k_2)\arr(k_4)\cdots$~ 
are defined in $C_1$ or $D_1$. 
We define $k_{C_1}$ to be $\odd(k)$ or $\even(k)$, 
whichever composite arrow is in $C_1$, 
and then $k_{D_1}$ is the other composite arrow. 
\end{enumerate} 



%%%%%%%%%%%%%%%%%%%%%%%%%%%%%%%%%%%%%%%%%%%%%%%%%%%%%%%%%%%%%%%%%%%%
\subsection{Tensor product of groupoids} \label{subsect:tensor-gpds}

When $\calC,\calD$ are groupoids, considered as $1$-truncated crossed 
complexes, the tensor product $\calC\otimes\calD$ is $2$-truncated. 
We have already described the groupoid at level $1$, 
so it remains to specify $(\calC\otimes\calD)_2$. 

There is a canonical morphism $\sigma : C_1\ \#\ D_1 \to C_1 \times D_1$ 
mapping $k$ to $(k_{C_1},k_{C_2})$. 
The kernel of $\sigma$ is the \emph{Cartesian subgroup} 
$C_1\ \Box\ D_1$ of $C_1\ \#\ D_1$ consisting of all the identity arrows 
and all the words $k$ such that $k_{C_1}$ and $k_{D_1}$ are identities. 
It is generated by ``commutators'' $[c,d]$ where, 
when $(c : u \to v) \in C_1$ and $(d : x \to y) \in D_1$ 
are not identities, $[c,d]$ is the loop at $(v,y)=(tc,td)$ given by 
$$
[c,d] ~:=~ (c^{-1},1_y)(1_u,d^{-1})(c,1_x)(1_v,d), 
$$
as shown in the following diagram:
$$
\xymatrix{ 
  (v,y) \ar[dd]_{(c^{-1},1_y)}
    &&  (v,x) \ar[ll]_{(1_v,d)} \\
    &&  \\
  (u,y) \ar[rr]_{(1_u,d^{-1})} 
    &&  (u,x) \ar[uu]_{(c,1_x)}  \\
}
$$
These commutators satisfy the usual commutator identities: 
\begin{eqnarray*}
\,    [d,c]  &=&  [c,d]^{-1}, \\ 
\, [dd_1,c]  &=&  [d,c]^{d_1}[d_1,c], \\
\, [d,cc_1]  &=&  [d,c_1][d,c]^{c_1},  
\end{eqnarray*}
whenever $cc_1,dd_1$ are defined. 
Note that the action is conjugation: for example, 
when $(d_1 : y \to z) \in D_1$ the loop $[d,c]^{d_1}$ at $(v,z)$ is given by 
\begin{eqnarray*}
[d,c]^{d_1} 
  &=& (1_v,d_1^{-1})(1_v,d^{-1})(c^{-1},1_x)(1_u,d)(c,1_y)(1_v,d_1) \\
  &=& (1_v,(dd_1)^{-1})(c^{-1},1_x)(1_u,dd_1)(c,1_z)
         .(c^{-1},1_z)(1_u,d_1^{-1})(c,1_y)(1_v,d_1) \\
  &=& [dd_1,c][c,d_1], 
\end{eqnarray*}
and similarly $[d,c]^{c_1} = [c_1,d][d,cc_1]$. 

It turns out that $(\calC\otimes\calD)_2 = C_1\ \Box\ D_1$ 
with $c \otimes d := [d,c]$. 
We may check the product axioms as follows: 
\begin{eqnarray*}
cc_1 \otimes d &=& [d,cc_1] ~=~ [d,c_1][d,c]^{c_1} 
                   ~=~ (c_1 \otimes d)(c \otimes d)^{(c_1 \otimes td)}, \\
c \otimes dd_1 &=& [dd_1,c] ~=~ [d,c]^{d_1}[d_1,c] 
                   ~=~ (c \otimes d)^{(tc \otimes d_1)}(c \otimes d_1).
\end{eqnarray*}

\noindent
We have constructed a normal inclusion crossed module of groupoids 
$$
\xymatrix{
(C_1\ \Box\ D_1 \ar[r] 
  &  C_1\ \#\ D_1 \ar[r]<+0.5ex>\ar[r]<-0.5ex>
     &  C_0 \times D_0).  
}$$
The action on loops in dimension two is given by: 
\begin{eqnarray*}
(c \otimes d)^{(tc \otimes d_1)} &=& (c \otimes dd_1)(c \otimes d_1)^{-1}\ ,\\
(c \otimes d)^{(c_1 \otimes td)} &=& (c_1 \otimes d)^{-1}(cc_1 \otimes d)\ .\\
\end{eqnarray*}

\bigskip
\begin{example} \label{ex:itensori}
\emph{The tensor product $\calI^{\otimes 2} = \calI\otimes\calI$ 
has four objects and eight generating arrows in dimension one, 
as shown in the following diagram. 
We relabel $\iota$ in the second factor as $\kappa$ 
so that the ``commutator'' notation is not ambiguous.} 

\begin{figure}[htbp]
\begin{center}
\input{xcomp/itensori.pstex_t}
\label{figure:itensori}
\end{center}
\end{figure}

\emph{The generating commutators for $I^{\otimes 2}_2(0,0)$ as:} 
\begin{eqnarray*}
k_{(0,0)} 
  &=& (\iota,1_0)(1_1,\kappa)(\iota^{-1},1_1)(1_0,\kappa^{-1}) 
  ~=~ [\iota^{-1}, \kappa^{-1}], \\ 
{k_{(0,0)}}^{-1} 
  &=& (1_0,\kappa)(\iota,1_1)(1_1,\kappa^{-1})(\iota^{-1},1_0) 
  ~=~ [\kappa^{-1},\iota^{-1}]. 
\end{eqnarray*} 
\emph{So the vertex group $I^{\otimes 2}_2(0,0)$ is free on one generator. 
It is easy to see that the vertex groups at 
$(1,0), (1,1), (0,1)$ are respectively generated by} 
$$
k_{(1,0)} = [\iota, \kappa^{-1}], \qquad
k_{(1,1)} = [\iota, \kappa], \qquad
k_{(0,1)} = [\iota^{-1}, \kappa]. 
$$
\end{example}

\medskip
\begin{example}
\emph{Consider $\calD = \calI\otimes\calC$, where $\calC$ is a crossed module 
of groups with $C_0=\{\bullet\}$. 
The objects in $D_0$ are $\{(0,\bullet),(1,\bullet)\}$, 
but it is convenient to replace these by $\{0,1\}$. 
The generating arrows in $D_1$ are}
$$
\{ ((\iota,1_{\bullet}) : 0 \to 1),~ ((\iota^{-1},1_{\bullet}) : 1 \to 0)\} 
~\cup~
\{((1_0,g) : 0 \to 0),~ ((1_1,g) : 1 \to 1) ~|~ g \in C_1 \}, 
$$
\emph{where we may restrict $g$ to be a member of a generating set for $C_1$ 
since $(1_0,g_1)(1_0,g_2) = (1_0,g_1g_2)$.} 

\begin{figure}[htbp]
\begin{center}
\input{xcomp/itensorc.pstex_t}
\label{figure:itensorc}
\end{center}
\end{figure}

\emph{There should be no confusion if we simply write $\iota$ for 
$(\iota,1_{\bullet})$ and $g$ for $(1_0,g)$, etc. 
A typical element of $D_1$ with source $0$ is therefore 
$k = g_1 \iota g_2 \iota^{-1} g_3 \iota \ldots$~.}
\emph{As before, we define} 
$$
\iota \otimes g ~=~ [g,\iota] ~=~ g^{-1}\iota^{-1}g\iota, 
\qquad\text{\emph{a loop at}}~ 1. 
$$
\emph{In dimension $2$ we also have a contribution from elements of $C_2$. 
For $c \in C_2$ we write $c$ for $c \otimes 0$ and $c \otimes 1$ 
when the object is clear. 
Then a typical loop at $0$ in $D_2$ is} 
$$
c_1 w_1 c_2 w_2 c_3 \ldots 
\quad\text{\emph{where}}~ w_i ~\text{\emph{is a word in the}}~ [g,\iota]. 
$$
\end{example}

\bigskip
\begin{example}
\emph{The tensor product $\calI^{\otimes 3} = \calI\otimes\calI\otimes\calI$ 
should, strictly speaking, be calculated as one of the isomorphic products 
$\calI^{\otimes 2}\otimes\calI$ or $\calI\otimes\calI^{\otimes 2}$. 
Thus we might consider, for example, the object $((0,0),0)$ 
and the generating arrows $(1_{(0,0)},\iota)$ and $((1_0,\iota),1_0)$. 
It is simpler to consider the $8$ objects and $24$ generating arrows
in dimension one as triples, as shown in the following cubical diagram where, 
to avoid confusion, we relabel the three copies of $\iota$ 
as $\iota,\kappa,\lambda$.} 

\begin{figure}[htbp]
\begin{center}
\input{xcomp/ioioi.pstex_t}
\label{figure:ioioi}
\end{center}
\end{figure}

\noindent
\emph{For dimension $2$ loops at $(1,1,1)$ we have commutators 
from the back, top, and right faces:} 
\begin{equation} \label{eq:ioioi-k2comms}
[\iota,\kappa], \qquad
[\iota,\lambda], \qquad 
[\kappa,\lambda].
\end{equation}
\emph{The other three faces also contribute loops at $(1,1,1)$. 
For example. the front face is traversed (in a clockwise direction) by} 
$$
[\kappa,\iota]^{\lambda} ~=~ 
(1_1,1_1,\lambda^{-1})(1_1,\kappa^{-1},1_0)(\iota^{-1},1_0,1_0)
(1_0,\kappa,1_0)(\iota,1_1,1_0)(1_1,1_1,\lambda).
$$
\emph{A typical element in the vertex group $K_2((1,1,1))$ is a word 
in the three commutators (\ref{eq:ioioi-k2comms}) and their conjugates.} 

\medskip\noindent
\emph{Finally, there are non-trivial vertex groups in $K_3$, 
associated to the whole cube. 
Each of the eight elements 
$(\iota^{\pm 1}\otimes\kappa^{\pm 1}\otimes\lambda^{\pm 1}) \in K_3$ 
generates an infinite cyclic group at 
$(h\iota^{\pm 1},h\kappa^{\pm 1},h\lambda^{\pm 1})$. 
The boundary map is given by a version of the Jacobi-Hall-Witt 
identity for commutators,} 
$$
[x^y,[y,z]]\ [y^z,[z,x]]\ [z^x,[x,y]] ~=~ 1, 
$$
\emph{which is used by Ellis in \cite{ellis-jsc2004} to give an identity 
among the relators for the free abelian group on three generators. 
(Another relevant reference is \cite{brow:higg:1981}.) 
Thus, at $K_3((1,1,1))$, we define} 
$$
\partial_3(\iota\otimes\kappa\otimes\lambda) ~=~ 
[\lambda,\kappa]^{\iota^{\kappa}}\ [\kappa,\lambda]\ 
[\iota,\lambda]^{\kappa^{\lambda}}\ [\lambda,\iota]\ 
[\kappa,\iota]^{\lambda^{\iota}}\ [\iota,\kappa]\ ,
$$
\emph{where} 
\small{
\begin{eqnarray*}
\,\partial_2[\lambda,\kappa]^{\iota^{\kappa}} 
  &=&  (1_1,\kappa^{-1},1_1)(\iota^{-1},1_0,1_1)(1_0,\kappa,1_1) 
         (1_0,1_1,\lambda^{-1})(1_0,\kappa^{-1},1_0) 
           (1_0,1_0,\lambda)(\iota,1_0,1_1)(1_1,\kappa,1_1), \\
\,\partial_2[\kappa,\lambda]~~ 
  &=&  (1_1,\kappa^{-1},1_1)(1_1,1_0,\lambda^{-1}) 
         (1_1,\kappa,1_0)(1_1,1_1,\lambda), \\
\,\partial_2[\iota,\lambda]^{\kappa^{\lambda}} 
  &=&  (1_1,1_1,\lambda^{-1})(1_1,\kappa^{-1},1_0)(1_1,1_0,\lambda) 
         (\iota^{-1},1_0,1_1)(1_0,1_0,\lambda^{-1}) 
           (\iota,1_0,1_0)(1_1,\kappa,1_0)(1_1,1_1,\lambda), \\
\,\partial_2[\lambda,\iota]~~ 
  &=&  (1_1,1_1,\lambda^{-1})(\iota^{-1},1_1,1_0) 
            (1_0,1_1,\lambda)(\iota,1_1,1_1), \\
\,\partial_2[\kappa,\iota]^{\lambda^{\iota}} 
  &=&  (\iota^{-1},1_1,1_1)(1_0,1_1,\lambda^{-1})(\iota,1_1,1_0) 
         (1_1,\kappa^{-1},1_0)(\iota^{-1},1_0,1_0)
           (1_0,\kappa,1_0)(1_0,1_1,\lambda)(\iota,1_1,1_1), \\
\,\partial_2[\iota,\kappa]~~ 
  &=&  (\iota^{-1},1_1,1_1)(1_0,\kappa^{-1},1_1)
            (\iota,1_0,1_1)(1_1,\kappa,1_1). 
\end{eqnarray*}
} 
\emph{The Jacobi-Hall-Witt identity ensures that 
$\partial^2(\iota\otimes\kappa\otimes\lambda) = 1_{(1,1,1)}$.} 
\end{example}

\bigskip
Here are the formulae for the boundary maps of the tensor product 
$\calC\otimes\calD$, 
where $(c_1 : u \to v) \in C_1$ $(d_1 : x \to y) \in D_1$, 
$(c : w \to w) \in C_m,~ m \geqslant 2$, 
and $(d : z \to z) \in D_n,~ n \geqslant 2$.   
\begin{eqnarray*}
\partial_{0+n}(u \otimes d) 
  &=&  u \otimes \partial_n d, \\
\partial_{m+0}(c \otimes x) 
  &=&  \partial_m c \otimes x, \\
\partial_2(c_1 \otimes d_1) 
  &=&  (v \otimes d_1)^{-1} (c_1 \otimes x)^{-1} 
          (u \otimes d_1) (c_1 \otimes y), \\
\partial_{1+n}(c_1 \otimes d) 
  &=&  (c_1 \otimes \partial_n d)^{-1} (v \otimes d)^{-1} 
         (u \otimes d)^{(c_1 \otimes z)}, \\
\partial_{m+1}(c \otimes d_1) 
  &=&  (c \otimes y)^{(-1)^{m+1}} 
         \left((c \otimes x)^{(w \otimes d_1)}\right)^{(-1)^m} 
         (\partial_mc \otimes d_1),\\
\partial_{m+n}(c \otimes d) 
  &=&  (\partial_m c \otimes d) + (-1)^m(c \otimes \partial_n d), 
\quad m,n \geqslant 2. \\
\end{eqnarray*}


\newpage
%%%%%%%%%%%%%%%%%%%%%%%%%%%%%%%%%%%%%%%%%%%%%%%%%%%%%%%%%%%%%%
\subsection{Homotopies between morphisms of crossed complexes}
\label{subsect:homotopy-xcomp}

For $f,g$ two automorphisms of $\calC$, a \emph{$1$-homotopy} 
$H : f \simeq g$ is a set of maps 
$H_n : C_n \to C_{n+1},~ n \geqslant 0$, 
satisfying various axioms. 
These axioms are most easily obtained by viewing the homotopy 
as a morphism $H : \calI\otimes\calC \to \calC$, 
making the following diagram commute where $i_0c = 0 \otimes c$ 
and $i_1c = 1 \otimes c$. 
$$
\xymatrix{ 
   & \calC \ar[ld]_{i_0} \ar[drr]^f
     & & \\
  \calI\otimes\calC \ar[rrr]^{H}  
   & & & \calC \\
   & \calC \ar[lu]^{i_1} \ar[rru]_g 
     & & \\
}
$$
Such an $H$ comprises maps 
$H_{m,n} : I_m \times C_n \to C_{m+n},~ m,n \geqslant 0$.  
Part of such a morphism is shown in the following diagram. 

\begin{figure}[htbp]
\begin{center}
\input{xcomp/xcomp-1hom.pstex_t}
\label{figure:xcomp-1hom}
\end{center}
\end{figure}

%%\newpage
We now consider the maps $H_{m,n}$ for small $m,n$. 
\begin{eqnarray*}
H_{0,0} : \{0,1\}\times C_0 \to C_0,  
  &&  (0,u) \mapsto fu,~ (1,u) \mapsto gu, \\
H_{0,n} : \{0,1\}\times C_n \to C_n,  
  &&  0 \otimes c \mapsto fc,~ 1 \otimes c \mapsto gc, \\
H_{1,0} : \{1_0,1_1,\iota,\iota^{-1}\}\times C_0 \to C_1,  
  &&  1_0 \otimes u \mapsto 1_{fu},~ 1_1 \otimes u \mapsto 1_{gu}, \\
  &&  \iota \otimes u \mapsto (H_0 u : fu \to gu), \\
  &&  \iota^{-1} \otimes u \mapsto (H_0 u)^{-1} : gu \to fu, \\
  &&  (\text{which defines the map}~ H_0 : C_0 \to C_1), \\
H_{m,0} : \{1_0,1_1\}\times C_0 \to C_m,  
  &&  1_0 \otimes u \mapsto 1_{fu},~ 1_1 \otimes u \mapsto 1_{gu},~ 
        m \geqslant 2, \\
H_{1,1} : \{1_0,1_1,\iota,\iota^{-1}\}\times C_1 \to C_2,  
  &&  1_0 \otimes c \mapsto 1_{fv},~ 1_1 \otimes c \mapsto 1_{gv}, \\
  &&  \iota \otimes c \mapsto H_1 c = [c,\iota] 
        \in C_2(t\iota, tc) = C_2(1,v), \\
  &&  (\text{which defines the map}~ H_1 : C_1 \to C_2).
\end{eqnarray*}

We now derive some of the axioms. 
In $\calI\otimes\calC$ the arrow $\iota \otimes c$ decomposes as
$$
(1_0 \otimes c)(\iota \otimes 1_v) 
~=~ (\iota \otimes c) ~=~ (\iota \otimes 1_u)(1_1 \otimes c), 
$$
so this commuting square is mapped to a commuting square in $\calC$, defining 
$$
H_0 : C_1 \to C_1,\quad  c \mapsto (fc)(H_0 v) = (H_0 u)(gc). 
$$

\noindent
The image under $H_1$ of a composite arrow is given by 
$$
H_1(cc') ~=~ \iota \otimes cc' 
           ~=~ (\iota \otimes c)^{c'}(\iota \otimes c') 
           ~=~ (H_1 c)^{(1_1 \otimes c')}(H_1 c') 
           ~=~ (H_1 c)^{gc'}(H_1 c').
$$
Similarly, applying the boundary maps, 
\begin{eqnarray*}
\partial(H_1 c) 
  &=&  H\left((1_1,c^{-1})(\iota^{-1},1_u)(1_0,c)(\iota,1_v)\right) \\
  &=&  H_{0,1}(1,c^{-1})\ H_{1,0}(\iota^{-1},u)\ 
       H_{0,1}(0,c)\      H_{1,0}(\iota,v) \\ 
  &=&  (gc)^{-1}\ (H_0 u)^{-1}\ (fc)\ (H_0 v).
\end{eqnarray*}

\noindent
In the formulae for $H_{1,1}$ above the image of $\iota^{-1} \otimes c$ 
has not been defined.  
It may be determined on expanding $\iota\iota^{-1} \otimes c$ as follows: 
\begin{eqnarray*}
1_0 \otimes c 
  &=&  \iota\iota^{-1} \otimes c 
       ~=~ (\iota^{-1} \otimes c)(\iota \otimes c)^{(\iota \otimes v)^{-1}} \\
\Rightarrow \qquad\qquad\qquad~ 1_{fv} 
  &=&  H(\iota^{-1} \otimes c)\,(H_1c)^{(H_0v)^{-1}} \\
\Rightarrow \qquad H_{1,1}(\iota^{-1} \otimes c)
  &=&  ((H_1 c)^{-1})^{(H_0v)^{-1}} .
\end{eqnarray*}


\bigskip
For $f$ an automorphism of $\calC$, a \emph{$2$-homotopy $K$ over $f$} 
is a set of maps $K_n : C_n \to C_{n+2},~ n \geqslant 0$, 
satisfying various axioms. 
These axioms are most easily obtained by viewing the homotopy 
as a morphism $K : \calF_2\otimes\calC \to \calC$, 
making the following diagram commute 
where $\calF_2$ is the free crossed complex 
$$
\xymatrix{ 
  \cdots \ar[r] 
    & 1 \ar[r] 
      & F_2 = \langle y \rangle \ar[r]^{\partial}  
        & F_1 = \langle z \rangle \ar[r] 
          & \{\bullet\}~, 
}
$$
the boundary map is given by $\partial y = z$, 
the action is trivial, $y^z=y$, and $ic = \bullet \otimes c$. 
$$
\xymatrix{ 
   & \calC \ar[ld]_{i} \ar[drr]^f
     & & \\
  \calF_2\otimes\calC \ar[rrr]_{K}  
   & & & \calC 
}
$$

\newpage
Such a $K$ comprises maps 
$K_{m,n} : (F_2)_m \times C_n \to C_{m+n},~ m,n \geqslant 0$.  
Part of such a morphism is shown in the following diagram where, 
since there is only one object in $\calF_2$, 
we write $u$ for $\bullet \otimes u$. 

\bigskip \bigskip
\begin{figure}[htbp]
\begin{center}
\input{xcomp/xcomp-2hom.pstex_t}
\label{figure:xcom-2hom}
\end{center}
\end{figure}

\newpage
We now consider the maps $K_{m,n}$ for small $m,n$. 
\begin{eqnarray*}
K_{0,0} : \{\bullet\}\times C_0 \to C_0,  
  &&  u \mapsto fu, \\
K_{0,n} : \{\bullet\}\times C_n \to C_n,  
  &&  c \mapsto fc, \\
K_{2,0} : \langle y \rangle \times C_0 \to C_2,  
  &&  y \otimes u \mapsto K_0 u, 
      \quad\text{where}~ \partial K_0 u = z \otimes u, \\
  &&  (\text{which defines the map}~ K_0 : C_0 \to C_2), \\
K_{1,0} : \langle z \rangle \times C_0 \to C_1,  
  &&  z \otimes u \mapsto \partial(K_0u), \\
K_{1,1} : \langle z \rangle \times C_1 \to C_2,  
  &&  z \otimes c \mapsto [c,z] \in C_2(v) \quad \text{where} \\
  &&  \partial[c,z] = K\left((c^{-1},1_v)(1_u,z^{-1})(c,1_u)(1_v,z)\right) \\
  &&  \quad\quad = (fc)^{-1}(\partial K_0u)^{-1}(fc)(\partial K_0 v), \\
K_{2,1} : \langle y \rangle \times C_1 \to C_2,  
  &&  y \otimes c \mapsto K_1c \in C_3(v) 
      \qquad\text{where}~ \partial K_1 c = ??? \\
  &&  (\text{which defines the map}~ K_1 : C_1 \to C_3). \\
\end{eqnarray*}

\noindent
We now derive some of the axioms or properties of these maps. 

\noindent
\begin{enumerate}[(a)] 
\item
\begin{eqnarray*}
\partial_3(y \otimes c) 
  &=&  (y \otimes v)^{-1}(y \otimes u)^c(z \otimes c) \\
\Rightarrow \qquad \partial_3 K_1c 
  &=& (K_0 v)^{-1} (K_0 u)^c [c,z].
\end{eqnarray*}
\end{enumerate}


\bigskip
%%%%%%%%%%%%%%%%%%%%%%%%%%%%%%%%%%%%%%%%%%%%%%%%
\subsection{Whitehead product of 1-homotopies}
\label{subsect:wprod-xcomp}

We wish to define a monoid structure on the set of homotopies to $g$, 
so let $H^0 : f^0 \simeq g$ and $H^1 : f^1 \simeq g$ be two homotopies 
between automorphisms of $\calC$. 
Then we define the \emph{Whitehead product} of $H^0,H^1$ to be 
the homotopy $H^0 \star H^1 : f^0*g^{-1}*f^1 \simeq g$ where 
\begin{eqnarray*}
(H^0 \star H^1)_0(u) &=& (H^1_0g^{-1}_0f^0_0u)(H^0_0u), \\ 
(H^0 \star H^1)_1(c) &=& (H^1_1c)(H^0_1c)(H^1_1g^{-1}_1\partial H^0_1c). 
\end{eqnarray*}
The arrows in these composite formulae are shown in the following diagram.

\begin{figure}[htbp]
\begin{center}
\input{xcomp/xcomp-wprod.pstex_t}
\label{figure:xcomp-wprod}
\end{center}
\end{figure}

\medskip\noindent
This product is associative: 
the homotopy is $H^0 \star H^1 : f^0*g^{-1}*f^1 \simeq g$ where 
\begin{eqnarray*}
(H^0 \star H^1 \star H^2)_0(u) 
  &=&  (H^2_0g^{-1}_0f^1_0g^{-1}_0f^0_0u)(H^1_0g^{-1}_0f^0_0u)(H^0_0u), \\ 
(H^0 \star H^1 \star H^2)_1(c) 
  &=&  (H^2_1c)(H^1_1c)(H^2_1g^{-1}_1\partial H^1_1c)
       (H^0_1c)(H^2_1g^{-1}_1\partial H^0_1c)(H^1_1g^{-1}_1\partial H^0_1c) 
       (H^2_1g^{-1}_1\partial H^1_1g^{-1}_1\partial H^0_1c). 
\end{eqnarray*}

