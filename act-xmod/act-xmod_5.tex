%% act-xmod_5.tex  (version 22/04/17)


%%%%%%%%%%%%%%%%%%%%%%%%%%%%%%%%%%%%%%%%
\subsection{Actions of a Crossed Module}
\label{subs:xmod-action}

The material in the rest of this section is taken, in the main, 
from Norrie's thesis \cite{nor2}.
Recall that an action of a group $H$ on a group $G$
is a group homomorphism from $H$ to the actor of $G$.
The following definition is a straightforward generalisation.

\begin{defn} \index{action!of a crossed module}
An action of a crossed module  $\calY = (\delta : Q \to P)$
on a crossed module  $\calX = (\partial : S \to R)$ 
is a morphism of crossed modules
$$
\alpha = (\dda,\da) \quad : \quad \calY \to \calA(\calX) = \Act(\calX)~,
$$
from $\calY$ to the \emph{actor} of $\calX$, as in the following diagram.

\begin{equation} \label{eq:sq2}
\vcenter{\xymatrix{ 
  S \ar[rr]^(0.45){\ddi} \ar[dd]_{\partial}
     && W \ar[dd]^{\Delta}
     && Q \ar[ll]_(0.45){\dda} \ar[dd]^{\delta}  \\
     &&  &&  \\
  R \ar[rr]_(0.45){\di}
     && A
     && P \ar[ll]^(0.45){\da}  \\
  \calX
     && \calA(\calX)
         && \calY  \\
}}
\end{equation}
\end{defn}

\noindent
Here 
\begin{itemize}
\item
$\calA(\calX)$  is the crossed module  $(\Delta : W \to A)$ 
of Subsection \ref{subs:AX},
\item
$\dda q \,=\, \chi_q : R \to S$, a derivation of $\calX$;
\item
$\da p \,=\, \beta_p = (\ddb_p,\db_p)$,
an automorphism of  $\calX$  giving actions of  $P$  on  $S$  and  $R$~:
$$
s^p = \ddb_p s
\quad \quad \mbox{and} \quad \quad  
r^p = \db_p r ~.
$$
\end{itemize}

\noindent
We have seen in Theorem \ref{thm:iota} that $\iota = (\ddi,\di)$ 
is the \emph{inner action} of $\calX$ on itself.

\bigskip
Here are five useful identities.
\begin{lem} \label{lem:ids5}
\mbox{}\\
\vspace{-5mm}
\begin{enumerate}[{\rm (a)}]
\item
$\partial(s^p) \;=\;
 \partial \ddb_p s \;=\;
 \db_p \partial s \;=\;
 (\partial s)^p$~;
\item
$s^q \;=\;
 s^{\delta q} \;=\;
 \ddb_{\delta q}s \;=\;
 \ddb_{\chi_q}s \;=\;
 s(\chi_q \partial s)$~; 
\item
$\db_{\delta q}r \;=\;
 r^{\delta q} \;=\;
 r^{\da \delta q} \;=\;
 r^{\Delta \chi_q} \;=\;
 \db_{\chi_q} r \;=\;
 r(\partial\chi_q r)$~;
\item
$\chi_{q_1q_2} r \;=\;
 (\chi_{q_2} r)(\chi_{q_1} r)^{q_2}$~,
\item
$\chi_{q^p} \;=\; \ddb_p \circ \chi_q \circ {\db_p}^{-1}$~.
\end{enumerate}
\end{lem}
\begin{pf}
\mbox{}\\
\vspace{-5mm}
\begin{enumerate}[{\rm (a)}]
\item
$\partial(s^p) \;=\;
 \partial \ddb_p s \;=\;
 \db_p \partial s \;=\;
 (\partial s)^p$~.
\item
The action of $Q$ on $S$ is via $P$: 
$$
s^q \;=\; s^{\delta q} \;=\; \ddb_{\delta q}s 
    \;=\; \ddb_{\chi_q}s \;=\; s(\chi_q \partial s)~,
$$
and we may check that
\begin{eqnarray*}
(s^{q_1})^{q_2}
  & = &  (s(\chi_{q_1}\partial s))^{q_2} \\
  & = &  s(\chi_{q_1}\partial s)\; 
          \chi_{q_2}((\partial s)(\partial(\chi_{q_1}\partial s))) \\
  & = &  s(\chi_{q_1}\partial s) 
          (\chi_{q_2}\partial s)^{\partial\chi_{q_1}\partial s}
          (\chi_{q_2}\partial\chi_{q_1}\partial s) \\
  & = &  s(\chi_{q_2}\partial s)(\chi_{q_1}\partial s)
          (\chi_{q_2}\partial\chi_{q_1}\partial s) \\
  & = &  s(\chi_{q_1} \ast \chi_{q_2})(\partial s) \\
  & = &  s^{q_1q_2}~. 
\end{eqnarray*}
\item
$\db_{\delta q}r \;=\;
 r^{\delta q} \;=\;
 r^{\da \delta q} \;=\;
 r^{\Delta \chi_q} \;=\;
 \db_{\chi_q} r \;=\;
 r(\partial\chi_q r)$~. 
\item
$\chi_{q_1q_2}r \;=\; 
 (\chi_{q_1} \star \chi_{q_2})r \;=\; 
 (\chi_{q_2}r)(\ddb_{\chi_{q_2}}\chi_{q_1}r) \;=\; 
 (\chi_{q_2}r)(\chi_{q_1}r)^{q_2}$~. 
\item
$\chi_{q^p} \;=\;
 \dda(q^p) \;=\;
 (\dda q)^{\da p} \;=\;
 (\chi_q)^{\beta_p} \;=\;
 \ddb_p \circ \chi_q \circ {\db_p}^{-1}
 \quad\mbox{by (\ref{lem:AonW})}$.
\end{enumerate}
\end{pf}

\vspace{3mm}
\noindent
{\bf [Could do with some more examples of crossed module actions!]}
\vspace{3mm}