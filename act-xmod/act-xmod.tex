%% act-xmod.tex,  version 07/11/19

\section{The Actor of a Crossed Module} \label{sect:actor}

This section is based on the material covered in pages 25-28 
of Norrie's thesis \cite{norrie-thesis}. 
We will, however, be extending her actor crossed module $\Act(\calX)$ 
to the more general $\Act_{\gamma}(\calX)$ where $\gamma$ 
is an automorphism of $\calX$. 
Here is a table giving Norrie's symbols and the ones used here.

\vspace{5mm}
\begin{center}
\begin{tabular}{|c|c|c|c|}
\hline
  section  &  type  &  old symbol  &  new symbol  \\
\hline
\ref{sect:actor}
 &  xmod &   $\partial : T \to G$  & $\calX = (\partial : S \to R)$   \\
 &  Whitehead group  &  $\Der(G,T)$  &  $W(\calX)$  \\
 &  xmod morphism    &  $(\sigma,\theta)$  & 
      $\beta_{\chi} \;=\; (\ddb_{\chi},\db_{\chi}) \;=\; (\sigma,\rho)$  \\
 & principal derivations  &  $E(G,T)$  &  $E(\calX)$  \\
\hline
\ref{subs:xmod-action}
 &  xmod &   $\mu : M \to P$    &   $\calX = (\partial : S \to R)$    \\
 &  xmod &   $\nu : N \to V$    &   $\calY = (\delta : Q \to P)$    \\
 &  xmod morphism & $<\epsilon,\rho>$  &
                  $\beta\;(\,=\;(\ddb,\db)\,)$ \\
 &  derivation group &   $D(P,M)$           &   $W = W(\calX)$  \\
 &  automorphism group &   $\Aut(M,P)$        &   $A = A(\calX)$  \\
 &  semidirect product &   $(M,P)\,\sqsupset_{<\epsilon,\rho>}\,(N,V)$  &  
                  $\calY \ltimes \calX$  \\
 &  xmod &   $\pi \;:\; M \sqsupset N \;\to\; \;P \sqsupset V$  &
                  $(\pi \;:\; Q \ltimes S \to P \ltimes R)$      \\
 &  automorphism &   $\rho(v) = (\rho_1(v),\rho_2(v))$  &  
                  $\beta(p) = \beta_p : \calX \to \calX$ \\
 &  elements &   $m,\;n,\;p,\;v$    &    $s,\;q,\;r,\;p$  \\
 &  Whitehead boundary &   $\Delta(\chi) \;=\; <\theta_{\chi},\ddb_{\chi}>$ &
         $\Delta(\chi) \;=\; \beta_{\chi} \;=\; (\ddb_{\chi},\db_{\chi})$  \\
\hline
\end{tabular}
\end{center}

\bigskip
When $\gamma$ is an automorphism of $\calX$, 
the group of automorphisms $\Aut_{\gamma}\calX$, 
has composition (as in (\ref{eq:star-gamma-calX})) given by 
$$
\alpha_1 \ast_{\gamma} \alpha_2 
\;:=\; 
(\dda_1 \ast_{\ddg} \dda_2,\, \da_1 \ast_{\dg} \da_2) 
\;=\; 
(\dda_1 * \ddg^{-1} * \dda_2,\, \da_1 * \dg^{-1} * \da_2)\,. 
$$

%%%%%%%%%%%%%%%%%%%%%%%%%%%%%%%%%%%%%%%%%%%%%%%%%%%%%%%%%%%%%%%%%%%%%%%%%
\subsection{Lue and Norrie crossed modules} 
\label{subs:SR-to-autX} 

We generalise the automorphism crossed module 
$(\iota : R \to \Aut\,R)$, where $\iota r$ is conjugation by $r$, 
to the \emph{Norrie crossed module} 
$\calN_{\gamma} = (\di_{\gamma} : R \to \Aut_{\gamma}\calX)$ where: 
\begin{itemize} 
\item 
the $\gamma$-conjugation map is given by~ 
$\di_{\gamma}r := \beta_r$ where 
$\db_r q := (\dg q)^r,~ 
 \ddb_r s := (\ddg s)^r$, and 
\item 
$\Aut_{\gamma}\calX$ has right actions on $R$ and $S$ given by~ 
$r^{\alpha} \;:=\; \da \dg^{-1} r, \quad 
 s^{\alpha} \;:=\; \dda \ddg^{-1} s$\,. 
\end{itemize} 
(An alternative set of definitions is given by 
$\db_r q = (\dg q)^r,\; 
 \ddb_r s = (\ddg s)^r,\; 
 r^{\alpha} = \da\dg^{-1}r,\; 
 s^{\alpha} = \dda\ddg^{-1}s$, 
but these do not combine with the principal derivation map 
to give a morphism of crossed modules.) 

\noindent 
Note that $\beta_r^{-1}$ is given by 
$\db_r^{-1}q = \dg^{-1}(q^{r^{-1}}),~ \ddb_r^{-1}s = \ddg^{-1}(s^{r^{-1}})$. 

\medskip\noindent
We now check the various axioms for $\calN_{\gamma}$. 

\noindent 
The map $\di_{\gamma}$ is a homomorphism: 
\begin{eqnarray*} 
\left( \beta_{r_1} \ast_{\gamma} \beta_{r_2} \right)q 
 & = & \db_{r_2} \dg^{-1} ((\dg q)^{r_1}) 
 \;=\; \db_{r_2} (q^{\dg^{-1}r_1}) 
 \;=\; (\dg (q^{\dg^{-1}r_1}))^{r_2} 
 \;=\; (\dg q)^{r_1r_2} 
 \;=\; \db_{r_1r_2} q\,, \\ 
\left( \beta_{r_1} \ast_{\gamma} \beta_{r_2} \right)s 
 & = & \ddb_{r_2} \ddg^{-1} ((\ddg s)^{r_1}) 
 \;=\; \ddb_{r_2} (s^{\dg^{-1}r_1}) 
 \;=\; (\ddg (s^{\dg^{-1}r_1}))^{r_2} 
 \;=\; (\ddg s)^{r_1r_2}  
 \;=\; \ddb_{r_1r_2} s\,. 
\end{eqnarray*}

\noindent 
The given formulae do specify an action of $\Aut_{\gamma}\calX$ on $\calX$\,: 
%% \begin{equation} \label{eq:autR-action-on-calX}
\begin{eqnarray*}
(r^{\alpha_1})^{\alpha_2} 
& = & \da_2 \dg^{-1} (\da_1 \dg^{-1} r)
\;=\; (\da_1 \ast_{\dg} \da_2) \dg^{-1} r
\;=\; r^{\alpha_1 \ast_{\gamma} \alpha_2}\,, \\ 
(s^{\alpha_1})^{\alpha_2} 
& = & \dda_2 \ddg^{-1} (\dda_1 \ddg^{-1} s)
\;=\; (\dda_1 \ast_{\ddg} \dda_2) \ddg^{-1} s
\;=\; s^{\alpha_1 \ast_{\gamma} \da_2}\,.  
\end{eqnarray*} 

\noindent 
First crossed module axiom 
(using the $\gamma$-conjugation of (\ref{eq:g-conj})): 
\begin{eqnarray*} 
(\wedge_{\gamma} \alpha)\db_r q 
 &=& \da\dg^{-1}\db_r(\da^{-1}\dg q) 
  =  \da\dg^{-1} ((\dg\da^{-1}\dg q)^r) 
  =  \da ((\da^{-1}\dg q)^{\dg^{-1}r}) 
  =  (\dg q)^{r^{\alpha}}  
  =  \db_{r^{\alpha}} q\,, \\
(\wedge_{\gamma} \alpha)\ddb_r s 
 &=& \dda\ddg^{-1}\ddb_r(\dda^{-1}\ddg s) 
  =  \dda\ddg^{-1} ((\ddg\dda^{-1}\ddg s)^r) 
  =  \dda ((\dda^{-1}\ddg s)^{\dg^{-1}r}) 
  =  (\ddg s)^{r^{\alpha}}  
  =  \ddb_{r^{\alpha}} s\,. 
\end{eqnarray*}

\noindent 
Second crossed module axiom: 
$$
r^{\di_{\gamma} r'} 
 = \db_{r'} \dg^{-1}r 
 = r^{r'}\,.
$$

\medskip\noindent 
Similarly, for the \emph{Lue crossed module} 
$\calL_{\gamma}(\calX) = (\partial*\di_{\gamma} : S \to \Aut_{\gamma}\calX)$, 
\begin{itemize} 
\item
the boundary maps $s \in S$ to $\beta_{\partial s}$ where 
$\db_{\partial s}q = (\dg q)^{\partial s},\, 
\ddb_{\partial s}s' = (\ddg s')^{\partial s}$,  and 
\item
the action of $\Aut_{\gamma}\calX$ on $R$ and $S$ are as above. 
\end{itemize} 
The verification of the crossed module axioms for $\calL_{\gamma}\calX$ 
are similar to those for $\calN_{\gamma}\calX$. 



%%%%%%%%%%%%%%%%%%%%%%%%%%%%%%%%%%%%%%%%%%%%%%%%%%%%%%%%%%%%%%%%%%%%%%%% 
\subsection{The actor crossed module}  
\label{subs:AX}

The missing part of the structure of the actor crossed module 
$\Act_{\gamma}(\calX)$ is a $\gamma$-action of the automorphisms 
on the derivations. 

\begin{lem} \label{lem:AonW}
There is an action of  $\Aut_{\gamma}(\calX)$  on  $W_{\gamma}(\calX)$  
given by
$$
\chi^{\alpha} \;=\; \gamma * \alpha^{-1} * \chi * \gamma^{-1} * \alpha 
             \;:\; R \to S\,,
    \quad r \mapsto \dda \ddg^{-1} \chi \da^{-1} \dg \,r\,, 
$$
such that $\beta_{\chi^{\alpha}} = (\wedge_{\gamma}\alpha)(\beta_{\chi})$ 
where $\wedge_{\gamma}\alpha$ is the $\gamma$-conjugation automorphism 
of (\ref{eq:g-conj}). 
\end{lem}
\begin{pf} 
We first check the axiom for an action: 
$$
(\chi^{\alpha_1})^{\alpha_2}
\,=\, \gamma*\alpha_2^{-1}*
         (\gamma*\alpha_1^{-1}*\chi*\gamma^{-1}*\alpha_1)*
             \gamma^{-1}*\alpha_2
\,=\, \gamma*(\alpha_1\ast_{\gamma}\alpha_2)^{-1}*
         \chi*\gamma^{-1}*(\alpha_1\ast_{\gamma}\alpha_2)
\,=\, \chi^{(\alpha_1\ast_{\gamma}\alpha_2)}.
$$
Secondly, we observe that 
$(\wedge_{\gamma}\alpha)(\beta_{\chi}) 
 = \gamma*\alpha^{-1}*\beta_{\chi}*\gamma^{-1}*\alpha$. 
\end{pf}

\begin{defn} \index{actor!of a crossed  module}
For $\gamma$ an automorphism of $\calX = (\partial : S \to R)$, 
the \emph{actor crossed module over $\gamma$ of $\calX$} is 
$\calA_{\gamma}(\calX) = (\Delta_{\gamma} : W_{\gamma} \to A_{\gamma})$  
where  
\begin{itemize}
\item
$W_{\gamma} = W_{\gamma}(\calX)$  
is the Whitehead group of invertible derivations
\begin{equation} \label{eq:der-rule}
\chi : R \to S, \quad \mbox{such that} \quad
\chi(qr) = (\chi q)^{\dg r}\,(\chi r)
\quad\mbox{for all}\quad
q,r \in R~,
\end{equation}
and with Whitehead multiplication (on the right)
\begin{equation} \label{eq:W-mult}
\chi_1 \,\star_{\gamma}\, \chi_2 \;:\; R \to S, \;\;
r \mapsto (\chi_2 r)(\chi_1 r)(\chi_2\dg^{-1}\partial\chi_1 r)~;
\end{equation}
\item
$A_{\gamma} = \Aut_{\gamma}(\calX)$  is the group of automorphisms of  $\calX$,
namely those invertible  $\alpha = (\dda,\da) : \calX \to \calX$  such that
$$
\da \partial = \partial \dda~, \quad
\dda(s^r) = (\dda s)^{\da r}
\quad \mbox{and} \quad
\da(q^r) = (\da q)^{\da r}
\quad \mbox{for all} \;\;
s \in S \;\; \mbox{and} \;\; q,r \in R\,, 
$$ 
with composition \quad 
$\alpha_1 \ast_{\gamma} \alpha_2 
:= (\dda_1 \ast_{\ddg} \dda_2,\, \da_1 \ast_{\dg} \da_2)$, 
and action given by Lemma \ref{lem:AonW}. 
\item 
The boundary map is obtained by restricting the monoid homomorphism 
$\Delta_{\gamma} : \Der_{\gamma}(\calX) \to \End_{\gamma}(\calX)$  
of Theorem \ref{thm:Delta} to the regular derivations: 
$$
\Delta_{\gamma} \,:\, W_{\gamma} \to A_{\gamma}, \;\; \chi \mapsto 
   \beta_{\chi} = (\ddb_{\chi}, \db_{\chi})~,
$$
\begin{equation} \label{eq:actor-bdy}
\mbox{where} \quad
\ddb_{\chi} : S \to S, \;\; s \mapsto (\ddg s)(\chi \partial s), \quad
  \db_{\chi} : R \to R, \;\; r \mapsto (\dg r)(\partial \chi r)\,.
\end{equation}
\end{itemize}
\end{defn}

When it is convenient not to distinguish the two group homomorphisms
in the crossed module morphism,
we write  $\alpha$  for both  $\dda$  and  $\da$.

These groups and morphisms are exhibited in the following diagram 
(the inner morphism $\iota_{\gamma} = (\ddi_{\gamma},\di_{\gamma})$ 
is defined in Subsection \ref{subs:inner-morphism} below):
\begin{equation} \label{eq:sqX}
\vcenter{\xymatrix{
  S \ar[dd]_{\partial}
    &&&  S \ar[lll]_{\dda,\;\ddb_r,\;\ddb_{\chi},\;\ddg} 
            \ar[dd]_{\partial} \ar[rr]^{\ddi_{\gamma}}
         &&  W_{\gamma}  \ar[dd]^{\Delta_{\gamma}}   \\
    &&& &&   \\
  R
    &&&  R \ar[lll]^{\da,\;\db_r,\;\db_{\chi},\;\dg} 
            \ar[rr]_{\di_{\gamma}}
         &&  A_{\gamma} \\
}} 
\end{equation}

\begin{thm}
With this action, 
$\calA_{\gamma}(\calX) = (\Delta_{\gamma} : W_{\gamma} \to A_{\gamma})$ 
is a crossed module.
\end{thm}
\begin{pf}
We have already shown that  $\Delta_{\gamma}$  is a group homomorphism.

\medskip\noindent
We verify the first crossed module axiom for  
$\calA_{\gamma}(\calX)$  as follows. 
$$
\mbox{\textbf{X1:}} \quad\quad
\Delta_{\gamma}(\chi^{\alpha}) 
 \;=\;  (\wedge_{\gamma} \alpha)(\Delta_{\gamma}\chi) 
 \;=\;  \gamma*\alpha^{-1}*\beta_{\chi}*\gamma^{-1}*\alpha\,.
$$

\noindent
Now $\Delta_{\gamma}(\chi^{\alpha}) \;=\; \beta_{\chi^{\alpha}}\,$ where 
\begin{eqnarray*} 
\ddb_{\chi^{\alpha}} s \;=\; (\ddg s)(\chi^{\alpha} \partial s)
  & = & 
    (\ddg s)(\dda\ddg^{-1}\chi\da^{-1}\dg\partial s)
  \;=\;
    (\ddg s)(\dda\ddg^{-1}\chi\partial\dda^{-1}\ddg s)
  \;=\;
    \dda\ddg^{-1}((\ddg\dda^{-1}\ddg s)(\chi\partial\dda^{-1}\ddg s)) \\
  & = &
    \dda\ddg^{-1}\ddb_{\chi}(\dda^{-1}\ddg s)
  \;=\;
    (\gamma * \alpha^{-1} * \beta_{\chi} * \gamma^{-1} * \alpha) \, s ~, \\
\db_{\chi^{\alpha}} r \;=\; (\dg r)(\partial \chi^{\alpha} r)
  & = &
    (\dg r)(\partial \dda \ddg^{-1} \chi \da^{-1} \dg r)
  \;=\;
    (\dg r)(\da \dg^{-1} \partial \chi \da^{-1} \dg r)
  \;=\;
    \da \dg^{-1} (\dg \da^{-1} \dg r) (\partial \chi \da^{-1} \dg r) ) \\
  & = &
    \da \dg^{-1} \db_{\chi}(\da^{-1} \dg r)
  \;=\;
    (\gamma * \alpha^{-1} * \beta_{\chi} * \gamma^{-1} * \alpha) \, r ~. 
\end{eqnarray*}

\noindent
The second crossed module axiom for  $\calA_{\gamma}(\calX)$, 
$$
\mbox{\textbf{X2:}} \quad\quad
{\chi_1}^{\Delta_{\gamma} \chi_2} \;=\;
  \overline{\chi_2} \,\star_{\gamma}\, \chi_1 \,\star_{\gamma}\, \chi_2\,,
$$
is verified by showing that 
$\chi_2 \star_{\gamma} {\chi_1}^{\Delta_{\gamma}\chi_2} 
 = \chi_1 \star_{\gamma} \chi_2$\,, 
using Lemma \ref{lem:gamma-beta-chi} (c), 
$$
(\chi_2 \,\star_{\gamma}\,{\chi_1}^{\Delta_{\gamma} \chi_2})\,r 
  \;=\; (\chi_2 r)({\chi_1}^{\Delta_{\gamma}\chi_2}\dg^{-1}\db_{\chi_2}r)
  \;=\; (\chi_2 r)(\ddb_{\chi_2} \ddg^{-1} \chi_1 r)
  \;=\; (\chi_1 \star_{\gamma} \chi_2)r\,. 
$$ 
\end{pf}



%%%%%%%%%%%%%%%%%%%%%%%%%%%%%%%%%%%%%%%%%%%%%%%%%%%%%%%%%%%%%%%%
\subsection{The inner morphism}
\label{subs:inner-morphism}

We next describe the morphism of crossed modules
$\iota_{\gamma} = (\ddi_{\gamma},\di_{\gamma}) 
 : \calX \to \calA_{\gamma}(\calX)$. 
The conditions in (\ref{eq:prexmod-morph}) 
for $\iota_{\gamma}$ to be a morphism are: 
\begin{equation} \label{eq:6}
\ddi_{\gamma}(s_1s_2) 
\;=\; \ddi_{\gamma} s_1 \star_{\gamma} \ddi_{\gamma} s_2\,,
\quad\quad 
\di_{\gamma}(r_1r_2) \;=\; \di_{\gamma} r_1 \ast_{\gamma} \di_{\gamma} r_2\,.
\end{equation}

\noindent
The range part  $\di_{\gamma}$  of  $\iota_{\gamma}$  
is given in Subsection \ref{subs:SR-to-autX} by: 
$$
\di_{\gamma} \,:\, R \to A_{\gamma}, \quad
r \,\mapsto\, \beta_r = (\ddb_r,\db_r) : \calX \to \calX,
\quad \ddb_r s_0 = (\ddg s_0)^r,
\quad  \db_r r_0 = (\dg r_0)^r\,.
$$

\noindent
The source part  $\ddi_{\gamma}$  of $\iota_{\gamma}$  
maps $s$ to its principal derivation 
(see Lemmas \ref{lem:gamma-eta_s}, \ref{lem:princ-prop}): 
$$
\ddi_{\gamma} \,:\, S \to W_{\gamma}, \quad 
s \,\mapsto\, \eta_s : R \to S,\; r \mapsto (s^{-1})^{\dg r}\,s\,.
$$

\begin{thm} \label{thm:iota}
The pair of group homomorphisms  
$\iota_{\gamma} = (\ddi_{\gamma},\di_{\gamma}) 
 \,:\,  \calX \to \calA_{\gamma}(\calX)$
is a morphism of crossed modules.
\end{thm}
\begin{pf}
The square commutes if 
$\Delta_{\gamma}\,\ddi_{\gamma} \,=\, \di_{\gamma}\,\partial$. 
To verify this we show that
$(\Delta_{\gamma}\ddi_{\gamma})s \,=\, \Delta_{\gamma}\eta_s 
                                 \,=\, \beta_{\eta_s}$
is the same automorphism of  $\calX$  as
$(\di\partial)s \,=\, \beta_{\partial s}$~. 
By definition of  $\beta_{\eta_s} = (\ddb_{\eta_s},\db_{\eta_s})$  we have: 
\begin{eqnarray*}
\ddb_{\eta_s}(s_0) 
  & = & (\ddg s_0)(\eta_s\partial s_0) 
  \;=\; (\ddg s_0)(s^{-1})^{\partial \ddg s_0}s
  \;=\; s^{-1}(\ddg s_0)s 
  \;=\; (\ddg s_0)^{\partial s} 
  \;=\; \ddb_{\partial s}(s_0) \,, \\
\db_{\eta_s}(r_0) 
  & = & (\dg r_0)(\partial\eta_s r_0) 
  \;=\; (\dg r_0) \partial((s^{-1})^{\dg r_0}s)
  \;=\; (\partial s)^{-1}(\dg r_0)(\partial s) 
  \;=\; (\dg r_0)^{\partial s} 
  \;=\; \db_{\partial s}(r_0) \,. 
\end{eqnarray*}

\noindent
Then we check that the action is preserved:
\begin{eqnarray*} 
(\ddi s)^{\di r}(q) 
  & = &  (\eta_s)^{\beta_r}q 
  \;=\;  \ddb_r \ddg^{-1} \eta_s \db_r^{-1} \dg q 
  \;=\;  \ddb_r \ddg^{-1} \eta_s \dg^{-1} \left( (\dg q)^{r^{-1}} \right) 
  \;=\;  \left( \ddg \ddg^{-1} \eta_s ( q^{\dg^{-1}r^{-1}} ) \right)^r \\ 
  & = &  \left( (s^{-1})^{r(\dg q)r^{-1}} s \right)^r 
  \;=\;  \left( (s^r)^{-1} \right)^{\dg q} s^r 
  \;=\;  \eta_{(s^r)}q 
  \;=\;  \ddi(s^r)(q) \,.
\end{eqnarray*} 
\end{pf}

The $\gamma$-version of the \emph{inner actor crossed module} of $\calX$ 
is the image $\iota_{\gamma}\calX$. 
The source group consists of the principal $\gamma$-derivations, 
and the range group consists of the $\gamma$-conjugation automorphisms. 
For further details see Norrie's thesis \cite{norrie-thesis}. 



%%%%%%%%%%%%%%%%%%%%%%%%%%%%%%%%%%%%%%%%%%%%%%%%%%%%%%%%%%%%%%%%%%%%%%%%%%%%
\subsection{The Whitehead crossed module} 
\label{subs:Whitehead-xmod}

\bigskip
\begin{lem} \label{lem:deriv-act}
There is an action of the Whitehead group $W_{\gamma}$ on $S$ given by
$$
s^{\chi} \;:=\; s^{\beta_{\chi}} 
          \;=\; \ddb_{\chi} \ddg^{-1} s
          \;=\; s(\chi\dg^{-1}\partial s)
$$
which makes  $\calW_{\gamma}(\calX) = (\ddi_{\gamma} : S \to W_{\gamma})$  
a crossed module.
\end{lem}
\begin{pf}
We first verify that this is an action:
\begin{eqnarray*} 
\left( s^{\chi_1} \right)^{\chi_2} 
  & = &  \ddb_{\chi_2} \ddg^{-1} \left( s(\chi_1\dg^{-1}\partial s) \right) 
  \;=\;  s(\chi_1\dg^{-1}\partial s)\, 
         \chi_2\partial\ddg^{-1} \left( s(\chi_1\dg^{-1}\partial s) \right) \\ 
  & = &  s(\chi_1\dg^{-1}\partial s)\, 
         \chi_2\left( (\dg^{-1}\partial s)
                      (\dg^{-1}\partial\chi_1\dg^{-1}\partial s) \right) \\  
  & = &  s(\chi_1\dg^{-1}\partial s)\, 
         (\chi_2\dg^{-1}\partial s)^{\partial\chi_1\dg^{-1}\partial s} 
             (\chi_2\dg^{-1}\partial\chi_1\dg^{-1}\partial s)  \\
  & = &  s(\chi_2\dg^{-1}\partial s)(\chi_1\dg^{-1}\partial s)
             (\chi_2\dg^{-1}\partial\chi_1\dg^{-1}\partial s)  \\
  & = &  s \left( (\chi_1 \star_{\gamma} \chi_2)(\dg^{-1}\partial s) \right) 
  \;=\;  s^{(\chi_1 \star_{\gamma} \chi_2)} \,.
\end{eqnarray*} 

\noindent
The first crossed module axiom 
$\eta_{(s^{\chi})} \,=\, 
 \overline{\chi} \star_{\gamma} \eta_s \star_{\gamma} \chi$
is verified by checking 
\begin{eqnarray*} 
(\eta_s \star_{\gamma} \chi)\,r
  & = &  (\chi r)(\ddb_{\chi} \ddg^{-1} \eta_s r) 
  \;=\;  (\chi r)\left(\ddb_{\chi}\ddg^{-1}((s^{-1})^{\dg r}s)\right)  
  \;=\;  (\chi r)(\ddb_{\chi}\ddg^{-1}s^{-1})^{\db_{\chi}r} 
           (\ddb_{\chi}\ddg^{-1}s) \\
  & = &  \left((s^{\chi})^{-1}\right)^{\dg r}(\chi r)(s^{\chi}) 
  \;=\;  \left((s^{\chi})^{-1}\right)^{\dg r}
            (s^{\chi})(\chi r)
            \left((s^{\chi})^{-1}\right)^{\dg(\dg^{-1}\partial\chi r)} 
            (s^{\chi}) \\  
  & = &  \left(\eta_{(s^{\chi})}r\right)(\chi r)
           \left(\eta_{(s^{\chi})}\dg^{-1}\partial\chi r\right) 
  \;=\;  \left(\chi \star_{\gamma} \eta_{(s^{\chi})}\right)\,r\,.
\end{eqnarray*}

\noindent
The second crossed module axiom is verified by:
$$ 
s^{\ddi s^{\prime}} 
  \;=\; s^{\eta_{s'}} 
  \;=\; s^{\beta_{\eta_{s'}}} 
  \;=\; \ddb_{\eta_{s'}} \ddg^{-1} s 
  \;=\; s(\eta_{s'} \partial \ddg^{-1} s) 
  \;=\; s[s,s'] 
  \;=\; s^{s'}\,. 
$$ 
\end{pf}

\vspace{3mm}
\noindent
{\bf [We might show here that $(\partial,\Delta)$ is a morphism.]}
\vspace{3mm}

\begin{lem}
$$
(\eta_s)^{\alpha} \;\;=\;\; \eta_{(s^{\alpha})}~.
$$
\end{lem}
\begin{pf} 
\begin{eqnarray*}
(\eta_s)^{\alpha}q 
  & = &  \dda \ddg^{-1} \eta_s \da^{-1} \dg q  
  \;=\;  \dda \ddg^{-1} 
           \left( (s^{-1})^{\dg\da^{-1}\dg q}\,s \right) 
  \;=\;  \dda\left( (\ddg^{-1}s^{-1})^{\da^{-1}\dg q}\,(\ddg^{-1}s)\right) \\ 
  & = &  (\dda \ddg^{-1} s^{-1})^{\dg q}\,(\dda \ddg^{-1}s)  
  \;=\;  ((s^{\alpha})^{-1})^{\dg q}\,(s^{\alpha}) 
  \;=\;  \eta_{(s^{\alpha})}q \,. 
\end{eqnarray*}
\end{pf}

\bigskip 
The right-hand square of morphisms of crossed modules in (\ref{eq:sqX})
becomes a \emph{crossed square} $\calS_{\gamma}(\calX)$ 
(see Example \ref{ex:actor-square} for the identity case) 
when the \emph{crossed pairing} (see Section \ref{sect:natp}) 
$\bt : R \times W_{\gamma} \to S,\; (r,\chi) \mapsto \chi \dg^{-1} r$, 
is added to the structure. 



%%%%%%%%%%%%%%%%%%%%%%%%%%%%%%%%%%%%%%%%%%%%%%%%%%%%%
\subsection{The actor of a cat$^1$-group} \label{subs:AC}

(This Subsection (for now) covers only identity derivations and sections.) 

\medskip\noindent
The diagram corresponding to equation (\ref{eq:sqX}) is
\begin{equation} \label{eq:sqC}
\vcenter{\xymatrix{
  G \ar[dd]  \ar[dd] <-1.0ex>_{t,h}
    &&  G \ar[ll]_{\barb}  \ar[rr]^(0.4){\ddk}
          \ar[dd]  \ar[dd] <-1.0ex>_{t,h}
       &&  A(\calC) \ltimes W(\calC)  
            \ar[dd]  \ar[dd] <-1.0ex>_{T,H}  \\
    && &&   \\
  R \ar[uu] <-1.0ex>_e 
    &&  R \ar[uu] <-1.0ex>_e \ar[ll]^{\db} \ar[rr]_(0.4){\dk}
       &&  A(\calC) \ar[uu] <-1.0ex>_{E} \\
}} 
\end{equation}
where $W = W(\calC)$ and $A = A(\calC)$ are defined as follows:
\begin{itemize}
\item~
$W$ is the group of sections of $\calC$
with composition given by equation (\ref{eq:section-comp}),
\item~
$A=\Aut(\calC)$ is the group of automorphisms of $\calC$,
\item~
$\Delta_{\calC} : W \to A,~
\xi \mapsto (\barb_{\xi}, \db_{\xi})$~
(see equations (\ref{eq:five-endos})).
\end{itemize}

\noindent
{\bf Note:}~~ $T,H,E$ were previously written $\Delta_t,\Delta_h,\Delta_e$. 

\bigskip\noindent
The homomorphisms 
$\ddk, \dk, T, H, E$ are given as follows.
\begin{itemize}
\item~
$\dk \,:\, R \to A, \quad
r \,\mapsto\, \beta'_r = (\barb_r,\db_r) : \calC \to \calC,
\quad  \barb_r g_0 = g_0^r\,,
\quad  \db_r r_0 = r_0^r = r^{-1}r_0r$\,,\\
using the action of $R$ on $G$ in equation (\ref{eq:RactsonG}).
\item~
$\ddk : G \to A \ltimes W,\;
g \mapsto (\dk tg, \kappa_g : R \to G)
\quad\mbox{where}\;\;  \kappa_g(r) = (er)^{(etg^{-1})g}$~.
\item~
$T(\beta,\xi) = \beta, \quad
H(\beta,\xi) = \beta \ast \Delta_{\calC}(\xi), \quad
E(\beta) = (\beta,\id)$.
\end{itemize}

\bigskip\noindent
{\bf [Add in here the associated cat2-group with groups 
$(A \ltimes W) \ltimes (R \ltimes S),~
A \ltimes W,~ A \ltimes R,~ A$.]}
