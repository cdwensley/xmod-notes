%% gpd.tex,  version 28/03/18

%%%%%%%%%%%%%%%%%%%%%%%%%%%%%%%%%%%%%%%%%%%%%%%%%%%%%%%%%%%%%%%%%%%%%%%%%%%%
\section{Groupoids} \label{sec:gpds}


%%%%%%%%%%%%%%%%%%%%%%%%%%%%%%%%%%%%%%%%%%%%%%%%%%%%%%%
\subsection{Basic definitions} \label{subsect:gpd-defs}
\index{groupoid} \index{object!of a groupoid} \index{arrow!of a groupoid}

A \emph{groupoid} is a category in which every arrow is invertible. 
Thus a groupoid $\bbC = (C_1,C_0)$ consists of the following: 
\begin{itemize}
\item
a set $\Ob(\bbC) = C_0$ of \emph{objects}, 
\item 
a set $\Arr(\bbC) = C_1$ of \emph{arrows}, 
\item
source and target maps $s,t : C_1 \to C_0$, so that we write 
$(a:u \to v)$ whenever $sa=u$ and $ta=v$, 
and we denote by $\bbC(u,v)$ the set of arrows with source $u$ and target $v$, 
\item 
an \emph{identity arrow} $1_u$ at each object $u$, with $s1_u=t1_u=u$, 
\item 
an associative  partial composition $\diamond: C_1 \times_0 C_1 \to C_1$, 
with $a \diamond b$ defined whenever $ta=sb$, such that 
$s(a \diamond b) = sa$ and $t(a \diamond b) = tb$, 
so that $\bbC(u):=\bbC(u,u)$ is a group, 
called the \emph{object group} at $u$,   
\item 
for each arrow $(a : u \to v)$ an inverse arrow $(a^{-1} : v \to u)$ 
such that $a \diamond a^{-1} = 1_u$ and $a^{-1} \diamond a = 1_v$. 
\end{itemize}
It will often be convenient to omit the symbol $\diamond$ and use simple 
juxtaposition to indicate composition. 
(In our {\GAP} implementation source and target are called 
\emph{tail} and \emph{head}.) 

A morphism of groupoids, as for general categories, is called a functor. 
\index{functor} \index{morphism!of groupoids} 
Thus a \emph{functor} $\phi = (\phi_1,\phi_0) : \bbC \to \bbD$
is a pair of maps $\phi_1 : C_1 \to D_1$ and $\phi_0 : C_0 \to D_0$ 
such that $\phi_1 1_u = 1_{\phi_0 u}$ and 
$\phi_1(a \diamond b) = (\phi_1a)\diamond(\phi_1b)$ 
whenever the composite arrow is defined. 
It is often convenient to omit the subscripts $0,1$ since it should be clear 
from the context whether an object or an arrow is being mapped. 
A morphism $\phi$ is \emph{injective} and/or \emph{surjective} 
if both $\phi_0,\phi_1$ are. 

\begin{example} 
\emph{A group is a groupoid with a single object (usually written $*$). 
This gives a functor \Gpd\; from \catGp\; to \catGpd.} 
\end{example}

\begin{example} \label{ex:triv-gpd} \index{trivial groupoid}
\emph{For $X$ a set, the \emph{trivial groupoid $\bbO(X)$ on $X$} 
has $\Ob(\bbO)=X$ and $\Arr(\bbO)=\{1_u ~|~ u \in \Ob(\bbO)\}$. 
We denote $\bbO(\{1,\ldots,n\})$ by $\bbO_n$.} 
\end{example}

\begin{example} \label{ex:unit-gpd} \index{unit groupoid} 
\emph{The \emph{unit groupoid} $\bbI$ has two objects $0,1$ and four arrows. 
The two non-identity arrows are $(\iota : 0 \to 1)$ 
and its inverse $(\iota^{-1} : 1 \to 0)$.} 
\end{example}

The \emph{underlying digraph} of a groupoid 
\index{underlying digraph!of a groupoid}
is obtained by forgetting the composition, 
so the objects become vertices, the arrows become arcs, 
while the source and target maps keep their usual digraph meaning. 
A groupoid is \emph{connected} if its underlying digraph is connected. 

\begin{example} \label{ex:tree-groupoid} \index{tree groupoid} 
\emph{The \emph{tree groupoid} $\bbI_n$ has $n$ objects $\{1,2,\ldots,n\}$ 
and $n^2$ arrows $\{(p,q) ~|~ 0 \leqslant p,q \leqslant n\}$ where 
$s(p,q)=p$, $t(p,q)=q$, $(p,q)\diamond(q,r) = (p,r)$, 
and $(p,q)^{-1} = (q,p)$. 
Note that $\bbI_2 \cong \bbI$. 
We also write $\bbI(X)$ for the tree groupoid on a set of objects $X$. 
The underlying digraph of $\bbI_n$ is complete.} 

\emph{The name tree groupoid comes from the fact that a subset of arrows 
which form a spanning tree in the underlying digraph generate the 
whole groupoid using composition and inversion. 
For example, 
taking the subset $X_n = \{(1,p) ~|~ 2 \leqslant p \leqslant n\}$, 
we have $(q,r) = (q,1)^{-1}\diamond(1,r)$.} 
\end{example}

The \emph{product} $\bbC\times\bbD$ of groupoids $\bbC,\bbD$ 
\index{product!of groupoids}
has objects $C_0 \times D_0$, arrows $C_1 \times D_1$, 
and composition 
$(a_1,b_1)\diamond(a_2,b_2) = (a_1 \diamond a_2, b_1 \diamond b_2)$, 
so that $(a,b)^{-1} = (a^{-1},b^{-1})$. 

\begin{example} \label{ex:gp-tree-gpd}
\emph{If $\bbG$ is a group, considered as a one-object groupoid, 
and $\bbI_n$ is a tree groupoid, then $\bbC = \bbG \times \bbI_n$ 
may be thought of as the groupoid with $n$ objects $\{1,2,\ldots,n\}$ 
and $n^2|G|$ arrows $\{(p,g,q) ~|~ g \in G, 1 \leqslant p,q \leqslant n\}$, 
with $t(p,g,q)=p$, $h(p,g,q)=q$, 
composition $(p,g,q)\diamond(q,h,r) = (p,gh,r)$, 
and inverses $(p,g,q)^{-1} = (q,g^{-1},p)$. 
A generating set for $\bbG$ is given by 
$\{(1,g,1) ~|~ g \in X_G\} \cup X_n$ 
where $X_G$ is any generating set for $\bbG$. 
Every finite, connected groupoid is isomorphic to a direct product 
of a group and a tree groupoid in this way, and we call such a 
representation a \emph{standard connected groupoid}.}
\end{example}

A groupoid $\bbA$ is \emph{abelian} \index{abelian groupoid}  
if and only if all its object groups are abelian. 

\medskip
We now describe the construction of a free groupoid on a graph. 
Let $D$ be a digraph with vertices $V = V(D)$, arcs $A^+=A(D)$, 
and source and target maps $s,t : A^+ \to V$. 
Let $A^- = \{a^- ~|~ a^+ \in A^+\}$ be a copy of $A^+$, 
and let $A = A^+ \cup A^-$.  
Extend $s,t$ to $A$ by defining $sa^-=ta^+,~ ta^-=sa^+$. 
Consider $A$ as an alphabet with $A^*$ the monoid of words in $A$ 
under concatenation. 
A word $w=a_1^{\epsilon_1}a_2^{\epsilon_2} \ldots a_k^{\epsilon_k} \in A^*$, 
where $\epsilon_i \in \{+,-\}$, is \emph{composable} 
if $ha_i^{\epsilon_i} = ta_{i+1}^{\epsilon_{i+1}}$ 
for all $1 \leqslant i < k$. 
The \emph{free groupoid} $\bbD$ on $D$ is defined by: 
\begin{itemize} \index{free groupoid}
\item 
the object set is $\Ob(\bbD) = V(D)$,  
\item
$\Arr(\bbD)$ is the set of all composable words in $A^*$, 
\item
$s(a_1^{\epsilon_1}a_2^{\epsilon_2} \ldots a_k^{\epsilon_k}) 
   = sa_1^{\epsilon_1}$, 
and $t(a_1^{\epsilon_1}a_2^{\epsilon_2} \ldots a_k^{\epsilon_k}) 
   = ta_k^{\epsilon_k}$, 
\item
$w_1 \diamond w_2$ is the concatenation of $w_1$ and $w_2$, 
defined if the result is a composable word.
\end{itemize} 
A groupoid $\bbD$ is \emph{free} if it is isomorphic to the free groupoid 
on some digraph $D$. 

\medskip \index{subgroupoid} 
A \emph{subgroupoid} $\bbS=(S_1,S_0)$ of $\bbC=(C_1,C_0)$ 
is a groupoid with $S_0 \subseteq C_0$, $S_1 \subseteq C_0$, 
having the same source, target and composition. 
A subgroupoid $\bbS$ is \emph{full} \index{full subgroupoid} 
if $\bbS(u,v) = \bbC(u,v)$ for all $u,v \in S_0$ 
and \emph{wide} \index{wide subgroupoid} 
if $\Ob(\bbS) = \Ob(\bbC)$. 
The \emph{(connected) components} of $\bbC$ are its maximal 
connected subgroupoids, \index{component!of a groupoid} 
with one component $\bbC_i$ for each of the $k$ connected components 
$\Gamma_i$ of the underlying digraph. 
We write $\bbC = \bbC_1 \cup \cdots \cup \bbC_k$.  
A groupoid whose components all have a single object is a 
\emph{union of groups}, \index{union of groups} 
and is said to be \emph{totally disconnected}. 
\index{totally disconnected groupoid} 
We denote by $\ids(\bbC)$ the wide trivial subgroupoid $\bbO(\Ob(\bbC))$. 

\medskip
Given a wide subgroupoid $\bbS \subseteq \bbC$, 
we may define a relation $\equiv_R$ on $\Arr(\bbC)$ by 
$c \equiv_R c' ~\Leftrightarrow~ c = a \diamond c'$ 
for some $a \in \Arr(\bbS)$. 
This is an equivalence relation since: 
\begin{itemize}
\item
$c = 1_{sc} \diamond c$, since $\bbS$ contains 
all the identity loops in $\bbC$,  
\item
$c \equiv_R c' \Rightarrow c=a \diamond c' \Rightarrow a^{-1}\diamond c=c' 
               \Rightarrow c' \equiv_R c$, so $\equiv_R$ is symmetric, 
\item
$(c_1 \equiv_R c_2,~ c_2 \equiv_R c_3) 
 ~\Rightarrow~ (c_1=a_1 \diamond c_2,~ c_2=a_2 \diamond c_3) 
 ~\Rightarrow~ c_1=a_1 \diamond a_2 \diamond c_3 
 ~\Rightarrow~ c_1 \equiv_R c_3$. 
\end{itemize} 
The equivalence classes $\bbS c$ for this relation are called the 
\emph{right cosets of $\bbS$ in $\bbC$}. 
\index{right coset!of a groupoid} 

\medskip
For $u \in \Ob(\bbC)$ the \emph{star} $\Star(u)$ of $u$ is the set 
\index{star!of a groupoid}
$\{a \in \Arr(\bbC) ~|~ sa=u\}$, the set of all arrows with source $u$. 
Similarly the \emph{costar} $\Costar(u)$ of $u$ is the set 
\index{costar!of a groupoid}
$\{a \in \Arr(\bbC) ~|~ ta=u\}$, the set of all arrows with target $u$.
Note that each right coset of $\bbS$ in $\bbC$ is a subset of a costar. 
We may define a second equivalence relation $\equiv_L$ on $\Arr(\bbC)$ by 
$c \equiv_L c' \Rightarrow c = c'a$ for some $a \in \Arr(\bbS)$. 
The equivalence classes $c\bbS$ for this relation are the 
\emph{left cosets of $\bbS$ in $\bbC$}, 
\index{left coset!of a groupoid} 
and this time each class is a subset of some star. 

\newpage
\begin{example}
\emph{Let $\bbG = \Gpd(G = \langle a,b ~|~a^3,b^2,(ab)^2 \rangle)$, 
where we write $e$ for the identity in $G$, 
and let $\bbC = \bbG \times \bbI_3$. 
Let $\bbS$ be the union of $\Gpd(C_2) \times \bbI(\{1,2\})$ 
with $\Gpd(C_3)$ at object $3$, where $C_2=\{e,b\}$ and $C_3=\{e,a,a^2\}$ 
are subgroups of the symmetric group $G$. 
The $54$ arrows in $\bbC$ form $15$ right cosets of $\bbS$ in $\bbC$, 
as shown in the following table. 
Note that $|\Costar(1)|=4=|\Costar(2)|$ while $|\Costar(3)|=3$, 
so some cosets contain $4$ arrows and some $3$. 
Note also that the $11$ arrows in $\bbS$ are partitioned into 
the first, second and fourteenth coset.} 
{\small 
\begin{center} 
\begin{tabular}{ll}
$\bbS(1,e,1) = \{(1,e,1),(1,b,1),(2,e,1),(2,b,1)\}$ 
  & $\bbS(1,e,2) = \{(1,e,2),(1,b,2),(2,e,2),(2,b,2)\}$ \\  
$\bbS(1,a,1) = \{(1,a,1),(1,a^2b,1),(2,a,1),(2,a^2b,1)\}$ 
  & $\bbS(1,a,2) = \{(1,a,2),(1,a^2b,2),(2,a,2),(2,a^2b,2)\}$ \\  
$\bbS(1,a^2,1) = \{(1,a^2,1),(1,ab,1),(2,a^2,1),(2,ab,1)\}$ 
  & $\bbS(1,a^2,2) = \{(1,a^2,2),(1,ab,2),(2,a^2,2),(2,ab,2)\}$ \\  
$\bbS(1,e,3) = \{(1,e,3),(1,b,3),(2,e,3),(2,b,3)\}$ 
  & $\bbS(1,a,3) = \{(1,a,3),(1,a^2b,3),(2,a,3),(2,a^2b,3)\}$ \\  
$\bbS(1,a^2,3) = \{(1,a^2,3),(1,ab,3),(2,a^2,3),(2,ab,3)\}$ 
  &  \\  
$\bbS(3,e,1) = \{(3,e,1),(3,a,1),(3,a^2,1)\}$ 
  & $\bbS(3,b,1) = \{(3,b,1),(3,ab,1),(3,a^2b,1)\}$ \\  
$\bbS(3,e,2) = \{(3,e,2),(3,a,2),(3,a^2,2)\}$ 
  & $\bbS(3,b,2) = \{(3,b,2),(3,ab,2),(3,a^2b,2)\}$ \\  
$\bbS(3,e,3) = \{(3,e,3),(3,a,3),(3,a^2,3)\}$ 
  & $\bbS(3,b,3) = \{(3,b,3),(3,ab,3),(3,a^2b,3)\}$ \\  
\end{tabular}
\end{center}}
\end{example}



A subgroupoid $\bbN$ of $\bbC$ is \emph{normal in $\bbC$}, 
written $\bbN \unlhd \bbC$, 
if $\Ob(\bbN) = \Ob(\bbC)$ and $a^{-1}\bbN(u)a \unlhd \bbC(v)$ 
for all $(a : u \to v) \in \bbC(u,v)$. 

\begin{example}
\emph{Let $\bbC = \bbG \times \bbI_n$, as in Example \ref{ex:gp-tree-gpd} 
with $\bbG = \Gpd(G)$. \\
If $N \unlhd G$, then we may construct:} 
\begin{itemize}  \index{normal subgroupoid} 
\item
\emph{a normal subgroupoid $\Gpd(N) = \bbN \unlhd \bbG$,}
\item
\emph{for each partition $\pi = \pi_1 \cup \cdots \cup \pi_k$ 
of $\{1,2,\ldots,n\}$ into $k$ parts, a normal subgroupoid 
$(\bbN \times \bbI(\pi_1)) \cup \cdots \cup (\bbN \times \bbI(\pi_k))$.} 
\end{itemize}  
\emph{The two extreme cases of the second construction 
are when $\pi$ has $n$ singleton parts, 
giving the totally disconnected normal subgroupoid 
$(\bbN \times \bbO_n) \unlhd \bbC$, 
and when $\pi$ has a single part, 
giving a connected normal subgroupoid $(\bbN \times \bbI_n) \unlhd \bbC$.} 
\end{example}

\medskip
When $\bbN$ is a normal subgroupoid of $\bbC$ it is \emph{not} in general 
the case that left cosets coincide with right cosets. 
A different equivalence relation is therefore required on $\Arr(\bbC)$ 
in order to be able to define a \emph{quotient groupoid}. 
The following material is taken from Higgins \cite{higgins-gpds}.
Define a relation $\equiv$ on $\Arr(\bbC)$ by 
$c \equiv c' \Leftrightarrow c = m \diamond c' \diamond n$ 
for some $m,n \in \Arr(\bbN)$. 
This is an equivalence relation since 
\begin{itemize}
\item
$c = 1_{sc}\diamond c \diamond 1_{tc}$ since $\bbN$ contains 
all the identity loops in $\bbC$,  
\item
$c \equiv c' \Rightarrow c = m \diamond c' \diamond n 
             \Rightarrow m^{-1} \diamond c \diamond n^{-1} = c' 
             \Rightarrow c' \equiv c$, so $\equiv$ is symmetric, 
\item
$c_1 \equiv c_2, c_2 \equiv c_3 
 ~\Rightarrow~ c_1 = m_1 \diamond c_2 \diamond n_1, 
               c_2 = m_2 \diamond c_3 \diamond n_2 
 ~\Rightarrow~ c_1 = m_1 \diamond m_2 \diamond c_3 \diamond n_2 \diamond n_1 
 ~\Rightarrow~ c_1 \equiv c_3$. 
\end{itemize} 
Note that equivalent arrows have sources in the same component of $\bbN$ 
and similarly for their targets, so we define an equivalence relation 
$\equiv_0$ on $\Ob(\bbC)$ by $u \equiv_0 v$ if $u,v$ 
are in the same component of $\bbN$. 
We denote the $\equiv_0$-class of $u \in \Ob(\bbC)$ by $\overline{u}$ 
and the $\equiv$-class of $c \in \Arr(\bbC)$ by $\overline{c}$. 
The quotient groupoid $\bbQ = \bbC/\bbN$ 
has $\Ob(\bbQ) = \Ob(\bbC)/\!\equiv_0$ and $\Arr(\bbQ) = \Arr(\bbC)/\!\equiv$. 
Source and target are given by $s\overline{c} = \overline{sc}$, 
$t\overline{c} = \overline{sc}$ (it is clear that $s,t$ are well-defined). 
Arrows $\overline{c}_1, \overline{c}_2$ are defined to be composable 
if there exist $a_1 \equiv c_1,~ a_2 \equiv c_2$ 
with $a_1 \diamond a_2$ defined in $\bbC$, 
in which case $\overline{c_1}\diamond\overline{c_2}$ 
is defined to be $\overline{a_1 \diamond a_2}$. 
This composition is well-defined since, 
if there exist $b_1 \equiv c_1,~ b_2 \equiv c_2$ 
with $b_1 \diamond b_2$ defined in $\bbC$, then 
$b_1b_2 = (p_1^{-1}m_1)a_1(n_1q_1^{-1}p_2^{-1}m_2)a_2(n_2q_2^{-1})$ 
where $m_i,n_i,p_i,q_i \in \Arr(\bbN)$, as shown in the diagram below.  
Since $\ell=n_1q_1^{-1}p_2^{-1}m_2 \in \bbN(sa_2)$, 
normality of $\bbN$ implies that $a_2^{-1} \ell a_2 \in \bbN(ta_2)$ 
so that $b_1b_2 = (p_1^{-1}m_1)a_1a_2(a_2^{-1} \ell a_2n_2q_2^{-1})$, 
and $b_1b_2 \equiv a_1a_2$.  

\begin{figure}[htbp]
\begin{center}
\input{gpd/gpd-equiv.pstex_t}
\label{figure:gpd-equiv}
\end{center}
\end{figure}

\medskip
\begin{example}
\begin{enumerate}[(a)]
\item
\emph{When $N \unlhd G$, $\bbG = \Gpd(G)\times\bbI_n$ and 
$\bbN = \Gpd(N)\times\bbI_n$, the quotient groupoid is 
the single-object $\Gpd(G/N)$.} 
\item
\emph{When $\bbG$ is as in (a), but $\bbN$ is totally disconnected, 
with $n$ copies of $\Gpd(N)$, the quotient groupoid is the 
connected $\Gpd(G/N)\times\bbI_n$.} 
\item
\emph{The general case is when $\bbG$ is as above, 
$\pi = \pi_1 \cup \cdots \cup \pi_k$ is a partition of $\{1,2,\ldots,n\}$ 
into $k$ parts, and 
$\bbN = (\Gpd(N)\times\bbI(\pi_1))\cup\cdots\cup(\Gpd(N)\times\bbI(\pi_k))$.  
Then the quotient groupoid is $\Gpd(G/N)\times\bbI_k$.} 
\end{enumerate}
\end{example}


\medskip \index{kernel!of groupoid morphism} 
The \emph{kernel} $\ker\phi$ of a groupoid morphism $\phi : \bbC \to \bbD$ 
is the set of arrows in $C_1$ which are mapped to 
one of the identity arrows in $D_1$. 
Since identities are always mapped to identities, 
this kernel is normal in $\bbC$. 

\begin{example}
\emph{Let $S_3 = \langle a,b ~|~ a^3=b^2=(ab)^2=e \rangle$ 
be the symmetric group with normal subgroup $C_3 = \langle a \rangle$, 
and let $C_2 = \langle c ~|~ c^2=e \rangle$. 
There is a groupoid morphism from $\bbC = \Gpd(S_3)\times\bbI\{u,v,w\}$ 
to $\bbD = \Gpd(C_2)\times\bbI\{x,y\}$, 
mapping $u,v$ to $x$ and $w$ to $y$, and killing $C_3$, 
defined on generators by 
$$
(u,e,v)\mapsto(x,e,x), ~
(u,e,w)\mapsto(x,e,y), ~
(u,a,u)\mapsto(x,e,x), ~ 
(u,b,u)\mapsto(x,c,x).
$$
The kernel $\Gpd(C_3)\times\bbI\{u,v\} \cup \Gpd(C_3)\times\bbI\{w\}$ 
has two components, and $\bbC/(\ker\phi)\cong\bbD$. 
}\end{example}

\medskip
If $a \in \bbC(u)$ and $c \in \bbC(u,v)$ then the \emph{conjugate} 
$a^c \in \bbC(v)$ is defined to be $c^{-1} \diamond a \diamond c$. 

\bigskip
We now consider actions of a groupoid $\bbC$. 
We restrict to the case when $\bbC$ is connected 
since there is a clear extension to the general case. 

For $\bbC$ a groupoid, a $\bbC$-set-system 
(or, by abuse of language, a $\bbC$-set) 
is a functor $\alpha$  from $\bbC$ to \catSet, mapping arrows to bijections.  
So, for $(a : u \to v) \in \Arr(\bbC)$, there are sets 
$\alpha_0u = X_u,~ \alpha_0v = X_v$ 
and a bijection $\alpha_1a : X_u \to X_v$.  
We also call $\alpha$ an 
\emph{action of $\bbC$ on $\bigsqcup_{u \in \Ob(\bbC)}X_u$}. 
If $(b : v \to w)$ is a second arrow in $\bbC$ and $\alpha_0w = X_w$,
then, since $\alpha$ preserves composition, we have 
$$
\alpha_1(a \diamond b) ~=~ (\alpha_1a)*(\alpha_1b) 
                       ~=~ (\alpha_1b)\circ(\alpha_1a) ~:~ X_u \to X_w.
$$
For $x \in X_u$ we denote, in the usual way, $(\alpha_1a)(x)$ by $x^a$, 
and then the condition becomes $(x^a)^b = x^{a \diamond b}$. 

\begin{figure}[htbp]
\begin{center}
\input{gpd/gpd-sets.pstex_t}
\label{figure:gpd-sets}
\end{center}
\end{figure}

A similar notion applies to sets with structure. 
For example, \emph{$\bbC$-graphs} are functors from $\bbC$ to the groupoid 
of (combinatorial) graphs and their isomorphisms. 

\begin{example}
\emph{
Let $\Gamma$ be a connected graph with automorphism group $A=\Aut\Gamma$. 
Let $\Delta$ be the graph consisting of $n$ copies of $\Gamma$, 
which we may consider as a \emph{graph-system} with $n$ components. 
The appropriate groupoid to consider is $\bbA = \Gpd(A)\times\bbI_n$, 
which has an obvious action on $\Delta$. 
It is reasonable to consider $\bbA$ to be 
the automorphism gadget of $\Delta$, 
rather than its wreath product automorphism group $S_n \wr A$. 
}\end{example}

A \emph{$\bbC$-group-system} (or \emph{$\bbC$-group}) provides, 
for each object $u$ a group $F_u$ and, for each $(a : u \to v)$, 
an isomorphism of groups  $\alpha_1a : F_u \to F_v$. 
As usual, we write $f^a$ for $(\alpha_1a)(f)$ when $f \in F$. 
The group structure has to be preserved so, 
as well as  $(f^a)^b = f^{a \diamond b}$, 
we require $(e_u)^a = e_v$ and $(f_1f_2)^a = (f_1^a)(f_2^a)$. 

A \emph{$\bbC$-module} is a $\bbC$-group in which the $F_u$ are all abelian. 

A \emph{$\bbC$-groupoid-system} is a functor 
$\alpha$ from $\bbC$ to \catGpd, 
where now there are groupoids $\alpha_0u = \bbB_u,~ \alpha_ov = \bbB_v$ 
and an invertible functor $\alpha_1a : \bbB_u \to \bbB_v$. 
As a simple case, note that a $\bbC$-group determines a $\bbC$-groupoid 
on replacing each $F_u$ by $\bbF_u = \Gpd(F_u)$, 
taking $u$ as the single object. 
Thus a $\bbC$-module may be consided as an abelian $\bbC$-groupoid. 
In these cases $f^a$ is defined when $f$ is a loop at $u$ 
and then $f^a$ is a loop at $v$. 
Here is a picture showing part of the structure.

\begin{figure}[htbp]
\begin{center}
\input{gpd/gpd-gps.pstex_t}
\label{figure:gpd-gps}
\end{center}
\end{figure}

A particular example, when $\bbC=\Gpd(G)\times\bbI_n$ and $N \unlhd G$, 
is given by taking $F_u \cong N$ for all $u \in \Ob(\bbC)$ and the action 
to be conjugation:
\begin{equation} \label{eq:gpd-conj}
(p,h,p)^{(p,g,q)} ~=~ (q,g^{-1},p)(p,h,p)(p,g,q) ~=~ (q,g^{-1}hg,q). 
\end{equation}
This will provide our first example of a crossed module of groupoids. 



\newpage
%%%%%%%%%%%%%%%%%%%%%%%%%%%%%%%%%%%%%%%
\subsection{Automorphisms of Groupoids}

An automorphism of a category $\bbC$ is a functor $\alpha : \bbC \to \bbC$ 
which is an isomorphism. 
Let $\bbC$ be the connected groupoid with object set $U=\{u_1,\ldots,u_n\}$ 
and let $\{(a_p : u_1 \to u_p) ~|~ 2 \leqslant p \leqslant n\}$ 
be a generating set for a spanning tree in $\bbC$. 
If $G_1$ is the object group at $u_1$, 
an automorphism of $\bbC$ is obtained on choosing 
\begin{itemize}
\item
$\pi \in \Symm(U)$, permuting the objects in $U$, 
\item
$\gamma \in \Aut\,G$, permuting the elements of $G_1$, 
\item
$\{(b_p : u_1 \to u_p) ~|~ 2 \leqslant p \leqslant n\}$, 
replacing the $a_p$ in the tree. 
\end{itemize}
Thus there are in total $n! \times |\Aut\,G| \times |G|^{n-1}$ 
automorphisms of $\bbC$. 

\bigskip
We now analyse the automorphisms of a standard connected groupoid. 
For $G$ a group, $\bbG = \Gpd(G)$, 
let $\bbC = \bbG \times \bbI_n$ with objects $\{1,\ldots,n\}$; 
arrows $\{(q,g,r) ~|~ g \in G,~ 1 \leqslant q,r \leqslant n\}$; 
composition $(p,h,q)\diamond(q,g,r) = (p,hg,r)$; 
identities $(p,e,p)$ where $e$ is the identity in $G$; 
and inverses $(q,g,r)^{-1} = (r,g^{-1},q)$. 
If $G$ has generating set $\Gamma_G = \{g_1,\ldots,g_{\ell}\}$ 
then the groupoid is generated by sets  
$$
\Gamma_p ~=~ \{(p,g_k,p) ~|~ g_k \in \Gamma_G\} \cup 
             \{(p,e,q) ~|~ q \neq p\}, 
$$
where the second set forms a spanning tree $T_p$ in the underlying 
digraph of $G$. 
The remaining arrows are given as the composites: 
\begin{eqnarray*}
(p,g,p) &=& (p,g_{k_1},p)(p,g_{k_2},p)\ldots(p,g_{k_j},p) \quad\text{when}~ 
             g = g_{k_1}g_{k_2}\ldots g_{k_j} \in G,~ g_{k_i} \in \Gamma_G, \\
(q,g,r) &=& (q,e,p)^{-1}(p,g,p)(p,e,r).
\end{eqnarray*}
An automorphism of $\bbC$ will be specified by giving the images 
of the arrows in one of the $\Gamma_p$. 

\medskip
There are three sets of automorphisms which generate the group 
$A = \Aut(\bbC)$. 
\begin{enumerate}[(1)] 
\item
For $\pi$ a permutation in the symmetric group $S_n$ 
we define an automorphism $\alpha_{\pi}$ by 
$$
\alpha_{\pi}(q,g,r) ~=~ (\pi q, g,\pi r).
$$

\item
We may apply an automorphism $\kappa$ of $G$ to the loops at object $1$, 
giving an automorphism $\alpha_{\kappa}$ of $\bbC$ 
which fixes the objects, where 
$$
\alpha_{\kappa}(1,g,1) ~=~ (1,\kappa g,1), \qquad 
\alpha_{\kappa}(1,e,q) ~=~ (1,e,q).
$$ 
It follows that $\alpha_{\kappa}(q,g,r) = (q,\kappa g,r)$, 
so $\alpha_{\kappa}$ applies $\kappa$ to all the hom-sets at once. 

\item
The hom-set $\bbC(1,q)$ provides a regular representation for $G$ 
with action $(1,g,q)^b = (1,gb,q)$. 
For each $q \neq p$ choose $b_q \in G$ and map the arrow $(p,e,q) \in T_p$ 
to $(p,b_q,q)$. 
The $n$-tuple $\bsyb = (b_1,\ldots,b_p=e,\ldots,b_n)$ determines an 
automorphism $\alpha_{p,\bsyb}$ of $\bbC$, fixing the objects, where 
$$
\alpha_{p,\bsyb}(p,g,p) ~=~ (p,g,p), \qquad 
\alpha_{p,\bsyb}(p,e,r) ~=~ (p,b_r,r). 
$$
\end{enumerate}

\noindent
For $\bsyb \in G^n$, the $n$-fold direct product of $G$ with itself, 
we generalise the $\alpha_{p,\bsyb}$ by defining 
$$
\alpha_{\bsyb} ~:~ \bbC \to \bbC, \quad 
(q,g,r) ~\mapsto~ (q,b_q^{-1}gb_r,r). 
$$
This map is a homomorphism since 
$$
(p,b_p^{-1}hb_q,q)(q,b_q^{-1}gb_r,r) ~=~ (p,b_p^{-1}(hg)b_r,r).
$$
Furthermore, there is a homomorphism 
$\theta : G^n \to \Aut\,\bbC,~ \bsyb \mapsto \alpha_{\bsyb}$, since 
$$
(\alpha_{\bsyb} * \alpha_{\bsyc})(q,g,r) 
~=~ (q,c^{-1}_qb^{-1}_qgb_rc_r,r) 
~=~ \alpha_{\bsyb\bsyc}(q,g,r). 
$$
For $\bsyz \in \ker\theta$ we require 
$gz_r = z_qg$ for all $g \in G,~ 1 \leqslant q,r \leqslant n$. 
It follows that $\bsyz$ is a constant vector $(z,z,\ldots,z)$ 
with $z \in Z(G)$, the centre of $G$. 
When $\bsyg = (g,g,\ldots,g)$ is an arbitrary constant vector in $G^n$, 
we see that $\alpha_{\bsyg}$ is the type (1) conjugation automorphism 
$\alpha_{(\wedge  g)}$. 
We denote by $\hat{G}$ the diagonal subgroup in $G^n$, 
put $\hat{Z} = \ker\theta$, and define $Q = G^n/\hat{Z}$. 

\medskip\noindent
There are actions of both $S_n$ and $\Aut\,G$ on $G^n$, where
\begin{equation} \label{eq:pi-kappa-actions}
\bsyb^{\pi} ~=~ \pi\bsyb ~=~ 
(b_{\pi^{-1}1},\ldots,b_{\pi^{-1}ip},\ldots,b_{\pi^{-1}n}), 
\qquad
\bsyb^{\kappa} ~=~ \kappa\bsyb 
               ~=~ (\kappa b_1,\ldots,\kappa b_p,\ldots,\kappa b_n), 
\end{equation}
and these actions commute, giving an action of $S_n \times \Aut\,G$ on $G^n$. 
We denote by $G^n_p$ the subset $\{\bsyb \in G^n ~|~ b_p=e\}$ 
and note that $\alpha_{\bsyb} = \alpha_{p,\bsyb}$ when $\bsyb \in G^n_p$. 
Note also that $G^n_p$ is closed under multiplication in $G^n$; 
that $G^n_p \cong G^{n-1}$; 
that the kernel of $\theta$ restricted to $G^n_p$ is trivial; 
and that $S_n$ and $\Aut\,G$ act trivially on $\hat{Z}$. 
% $\overline{G^n}$ provided $b_p=c_p=e$ for some $i$, in which case 
% $\bsyb\bsyc = (b_1c_1,\ldots,\overline{b_pc_p},\ldots,b_nc_n)$. 
% 
% We write  $\bsyb \in \overline{G^n}$ 
% as $(b_1,\ldots,\overline{b_p},\ldots,b_n)$ 
% when we wish to specify that $b_p=e$. 
% (In the definition of $\alpha_{p,\bsyb}$ it is this $i$-th component 
% which is marked.) 

\medskip
We now investigate conmposites of the set 
$$
\Gamma_A ~=~ \{\alpha_{\pi} ~|~ \pi \in S_n\} \cup 
             \{\alpha_{\kappa} ~|~ \kappa \in \Aut\,G\} \cup 
             \{\alpha_{\bsyb} ~|~ \bsyb \in G^n\}. 
$$
In keeping with the use of right actions, we write $\alpha*\beta$ 
for the composite mapping $\beta\circ\alpha$. 
It is straightforward to verify the following identities. 
\begin{lem} \label{lem:autopairs} 
Pairs of automorphisms in $\Gamma_A$ compose as follows.
\begin{eqnarray*}
(\alpha_{\pi}*\alpha_{\rho})(q,g,r) 
       ~=~  \alpha_{\pi*\rho}(q,g,r) 
  &=& \left( (\pi*\rho)q,g,(\pi*\rho)r \right), \\
(\alpha_{\kappa}*\alpha_{\lambda})(q,g,r) 
       ~=~  \alpha_{\kappa*\lambda}(q,g,r) 
  &=& (q,(\kappa*\lambda)g,r), \\
(\alpha_{\bsyb}*\alpha_{\bsyc})(q,g,r) 
       ~=~  \alpha_{\bsyb\bsyc}(q,g,r) 
  &=& (q,(b_qc_q)^{-1}g(b_rc_r),r), \\ 
(\alpha_{\pi}*\alpha{_\kappa})(q,g,r) 
       ~=~  (\alpha_{\kappa}*\alpha_{\pi})(q,g,r) 
  &=& (\pi q, \kappa g, \pi r), \\
(\alpha_{\bsyb}*\alpha_{\pi})(q,g,r) 
       ~=~  (\alpha_{\pi}*\alpha_{\pi\bsyb})(q,g,r) 
  &=& (\pi q, b_q^{-1}gb_r, \pi r), \\
(\alpha_{\kappa}*\alpha_{\bsyb})(q,g,r) 
       ~=~  (\alpha_{\kappa^{-1}\bsyb}*\alpha_{\kappa})(q,g,r) 
  &=& (q,b_q^{-1}(\kappa g)b_r,r). 
\end{eqnarray*}
\end{lem}
\begin{pf}
The fifth isomorphism is the least obvious one. 
When $\pi q = q',~ \pi r = r'$ we obtain  
$$
\alpha_{\pi\bsyb}(q',g,r') ~=~ 
\left(q',\left(b_{\pi^{-1}q'}\right)^{-1}g\left(b_{\pi^{-1}r'}\right),r'\right)
~=~ (\pi q, b_q^{-1}gb_r,\pi r). 
$$
\end{pf}

It is clear that the group $A_1$ generated by the $\alpha_{\pi}$ 
is isomorphic to the symmetric group $S_n$; 
that the group $A_2$ generated by the $\alpha_{\kappa}$ 
is isomorphic to $\Aut\,G$; 
and that the group $A_3$ generated by the $\alpha_{1,\bsyb}$ 
is isomorphic to $G^{n-1}$. 

It is clear that the join $A_{1,2}$ of $A_1$ and $A_2$ in the 
automorphism group $\Aut\,\bbC$ of $\bbC$ is isomorphic to $A_1 \times A_2$.  
We denote by $A_{1,3},A_{2,3}$ the joins of $A_1,A_3$ and $A_2,A_3$ 
respectively. 

\begin{prop}
The groups $A_{1,3},~ A_{2,3},~ \Aut\,\bbC$ 
are isomorphic to the following semidirect products.  
\begin{enumerate}[(1)]  
\item 
$A_{2,3} ~\cong~ \Aut\,G \ltimes G^{n-1}$, 
where the action on $G^n_1$ is defined in equation (\ref{eq:pi-kappa-actions}).
\item
$A_{1,3} ~\cong~ S_n \ltimes Q$,  
using the action in (\ref{eq:pi-kappa-actions}), so that  
$(\kappa,\bsyb)^{\pi} = (\kappa,\pi\bsyb)$. 
\item
$\Aut\,\bbC ~\cong~ (S_n \times \Out\,G) \ltimes Q$, 
% (\Aut(G) \ltimes G^{n-1})$, 
using the same action as in (2). 
\end{enumerate} 
\end{prop}
\begin{pf} 
The sixth identity in Lemma \ref{lem:autopairs} shows that every element 
of $A_{2,3}$ has the form $\alpha_{\kappa}*\alpha_{1,\bsyb}$. 
We define an isomorphism, 
$\theta_1 : A_{2,3} \to \Aut\,G \ltimes G^{n-1}$, by 
$\alpha_{\kappa} \mapsto (\kappa,1)$,~ $\alpha_{1,\bsyb} \mapsto (1,\bsyb)$. 
Then we check that 
$\theta_1(\alpha_{\kappa}*\alpha_{\lambda}) 
 = \theta_1(\alpha_{\kappa*\lambda}) 
 = (\kappa*\lambda,1)$, that 
$\theta_1(\alpha_{1,\bsyb}*\alpha_{1,\bsyc}) 
 = \theta_1(\alpha_{\bsyb\bsyc}) 
 = (1,\bsyb\bsyc)$, and that 
$$
\theta_1(\alpha_{1,\kappa^{-1}\bsyb}*\alpha_{\kappa}) 
~=~ (1,\kappa^{-1}\bsyb)(\kappa,1) 
~=~ (\kappa,\kappa^{-1}\bsyb^{\kappa}) 
~=~ (\kappa,\bsyb) 
~=~ \theta_1(\alpha_{\kappa}*\alpha_{1,\bsyb}). 
$$

\medskip\noindent 
For the second isomorphism we use Lemma \ref{lem:autopairs} to obtain  
\begin{equation} \label{eq:pi-b-rho-c}
\alpha_{\pi} * \alpha_{1,\bsyb} * \alpha_{\rho} * \alpha_{1,\bsyc} 
~=~ \alpha_{\pi} * \alpha_{\rho} * \alpha_{\rho\bsyb} * \alpha_{\bsyc} 
~=~ \alpha_{\pi*\rho} * \alpha_{(\rho\bsyb)\bsyc}.
\end{equation}
Iterating this procedure, we see that every word in the generators 
of $A_{1,3}$ has a normal form $\alpha_{\pi}*\alpha_{\bsyb}$. 
We define  
$\theta_2' : S_n \ltimes G^n \to A_{1,3},~ 
             (\pi,\bsyb) \mapsto \alpha_{\pi}*\alpha_{\bsyb}$.  
This is a homomorphism since, by (\ref{eq:pi-b-rho-c}), 
$$
\theta_2'(\pi,\bsyb)\theta_2'(\rho,\bsyc) 
~=~ \alpha_{\pi*\rho} * \alpha_{(\rho\bsyb)\bsyc} 
~=~ \theta_2'(\pi*\rho,\bsyb^{\rho}\bsyc). 
$$
The kernel of $\theta_2'$ is $\{((~),\bsyz) ~|~ \bsyz \in \hat{Z}\}$, 
where $(~)$ denotes the identity permutation, 
so there is an isomorphism $\theta_2 : A_{1,3} \to S_n \ltimes Q$ 
mapping generators $\alpha_{\pi}*\alpha_{1,\bsyb}$ 
to $(\pi,\hat{Z}\bsyb)$. 

\medskip 
To prove the third isomorphism we define 
$\theta_3' : (S_n \times \Aut\,G) \ltimes G^n \to \Aut\,\bbC$ 
mapping $((\pi,\kappa),\bsyb)$ 
to $\alpha_{\pi}*\alpha_{\kappa}*\alpha_{\bsyb}$. 
The formulae in Lemma \ref{lem:autopairs} again show that every 
automorphism can be written in this form. 
The kernel of $\theta_3'$ is generated by elements 
$(((~),\id),\bsyz)$ for $z \in \hat{Z}$ 
and elements $(((~),\wedge  g),\bsyg^{-1})$, 
so there is an isomorphism 
$\theta_3 : \Aut\,\bbC \to (S_n \times \Out\,G) \ltimes Q$ 
mapping $\alpha_{\pi}*\alpha_{\kappa}*\alpha_{1,\bsyb}$ 
to $((\pi,(\Inn\,G)\overline{\kappa}),\hat{Z}\bsyg\bsyb)$ 
where $\kappa = (\wedge  g)\overline{\kappa} \in \Aut\,G$. 
\end{pf}

\bigskip
We conclude this subsection by observing that an automorphism 
$\alpha = (\alpha_1,\alpha_0) : \bbC \to \bbC$ is specified by giving 
\begin{itemize}
\item
the permutation $\alpha_0$ on the objects; 
\item
an automorphism, written $\overline{\alpha}$, of the object group $G_1$, 
so that $\alpha_1(1,g,1) = \overline{\alpha} g$; 
\item
images $\alpha_1(1,e,q) = (\alpha_0 1, \alpha_q, \alpha_0 q)$, 
with $\alpha_q \in G,~ 2 \leqslant j \leqslant n$, 
for the tree $T_1$. . 
\end{itemize}
Then $\alpha$ acts on a typical arrow by 
$$
\alpha_1(q,g,r) ~=~ (\alpha_0 q, \alpha_q^{-1}(\alpha g)\alpha_r, \alpha_0 r). 
$$
It is clear how to replace object $1$ by an arbitrary object $p$ 
in these formulae. 


\bigskip
The next type of groupoid to consider is the disjoint union of $m$ 
copies of a connected groupoid. 
More generally, let $X$ be some structure with an equivalence relation 
$\equiv$ which is preserved by endomorphisms. 
We argue that the the correct automorphism structure for $X$ is not the 
group of automorphisms but rather the \emph{groupoid of automorphisms} 
$\bbA = \bbAut\,X$ whose objects are the $\equiv$-classes $[x]$. 
The object group at class $[x]$ is the group of automorphsisms 
$\Aut\,[x]$ of the elements in $[x]$. 
The hom-set $\bbA([x],[y])$ comprises all isomorphisms from $[x]$ 
to $[y]$ (if there are any). 
A connected component of $\bbA$ has the form $\Aut\,[x] \times \bbI_m$, 
where $m$ is the number of components of $X$ isomorphic to $[x]$. 
A simple example of this situation is a graph $\Gamma$ with $m$ 
connected components all isomorphic to $\Gamma_0$, having 
automorphism groupoid $\Aut\,\Gamma \cong \Aut\,\Gamma_0 \times \bbI_m$. 

So let $\bbB$ be the disjoint union $\bbC \times \bbO_n$ 
of $m$ copies of $\bbC = G \times \bbI_n$, 
and denote the $i$-th copy by $\bbC_i$ and its elements by $(q,g,r)_i$. 
The equivalence relation here is the connectedness of the underlying digraph. 
There should be no confusion if we take $\Ob(\bbAut\,\bbB)$ 
to be $\{1,\ldots,m\}$. 
From our previous discussion we see that 
$\bbAut\,\bbB \cong \Aut\,\bbC \times \bbI_m 
              \cong \Gpd((S_n \times \Out\,G) \ltimes Q) \times \bbI_m$. 
Generators for this groupoid are provided by: 
\begin{itemize}
\item
$\{(1,\alpha_r,1)\}$, where the $\alpha_r$ are generators for $\Aut\,\bbC$; 
\item
$\{(1,\epsilon_i,i) ~|~ 2 \leqslant i \leqslant m\}$, 
where $\epsilon_i : \bbC_1 \to \bbC_i,~ (q,g,r)_1 \mapsto (q,g,r)_i$. 
\end{itemize}

An automorphism of $\bbB$ which does not interchange the components 
is obtained by choosing an automorphism for each component.
These form a group isomorphic to $(\Aut\,\bbC)^m$, 
and the automorphism group of $\bbB$ is the wreath product 
$S_m \wr \Aut\,G$ with action 
$(\kappa_1,\ldots,\kappa_m)^{\pi} ~=~ 
 (\kappa_{(\pi^{-1}1)},\ldots,\kappa_{(\pi^{-1}m)})$. 

\begin{example}
\emph{We shall see groupoids of the form $\bbB = \bbG \times \bbO_m$, 
a disjoint union of groups, when we come to consider crossed modules. 
Clearly $\bbAut\,\bbB \cong (\Aut\,G) \times \bbI_m$ 
and $\Aut\,\bbB \cong S_m \wr \Aut\,G$. 
}\end{example} 

\bigskip 
The final case to consider is that of an arbitrary groupoid $\bbG$, 
whose connected components $\bbG_j$ form $m$ classes $[\bbG_i]$ of 
isomorphic connected groupoids with $m_i$ components in class $[\bbG_i]$. 
The automorphism groupoid $\Aut\,\bbG$ has $\sum_{i=1}^m m_i$ objects 
and $m$ connected components, with the $i$-th component being isomorphic 
to $\Aut\,\bbG_i \times \bbI_{m_i}$. 


\newpage
%%%%%%%%%%%%%%%%%%%%%%%%%%%%%%%%%%%%
\subsection{Natural Transformations}

Functors are related by \emph{natural transformations}. 
\index{natural transformation} \index{automorphism!of a groupoid} 
If $\alpha,\beta : \bbC \to \bbD$ are functors, 
then a natural transformation  $\tau : \alpha \to \beta$ 
is determined by a function 
$\tau : \Ob(\bbC) \to \Arr(\bbD),~ u \mapsto \tau_u$, 
such that for every arrow $(a : u \to v) \in \bbC$ 
the following diagram commutes.
$$
\xymatrix{ 
  \alpha u  \ar[rr]^{\tau_u} \ar[dd]_{\alpha a} \ar[ddrr]^{\tau a}
    &&  \beta u \ar[dd]^{\beta a} \\
    &&  \\
  \alpha v  \ar[rr]_{\tau_v} 
    &&  \beta v  \\
}
$$
Notice that commutativity of the diagram enables us to extend $\tau$ 
to a function $\Arr(\bbC) \to \Arr(\bbD)$, 
where $\tau a$ is this diagonal arrow and 
$\tau 1_u=\tau_u$ for each object $u$. 
Notice also that $[s\tau_{u_1},s\tau_{u_2},\ldots,s\tau_{u_n}]$ 
is a permutation of the list of objects, 
otherwise one or more of these diagrams would be undefined.

\medskip
Restricting to groupoids, so that arrows are invertible, 
we have $\tau_v = (\alpha a)^{-1}\diamond(\tau_u)\diamond(\beta a)$, 
so $\tau$ is defined if we are given, for each component of $\bbC$, 
the image of one object. 
Furthermore, when $\alpha,\beta$ are surjective, 
every transformation is invertible with $(\tau^{-1})_u = (\tau_u)^{-1}$,  
$$
\xymatrix{ 
  \beta u  \ar[rrr]^{(\tau^{-1})_u = \tau_u^{-1}} \ar[dd]_{\beta a} 
    &&&  \alpha u \ar[dd]^{\alpha a} \\
    &&&  \\
  \beta v  \ar[rrr]_{(\tau^{-1})_v = \tau_v^{-1}}  
    &&&  \alpha v  \\
}
$$
In this case the list $[t\tau_{u_1},t\tau_{u_2},\ldots,t\tau_{u_n}]$ 
is a second permutation of the objects in $\bbD$. 

When $\bbC=\bbD$ and $\alpha,\beta$ are isomorphisms, 
we obtain our first example of a \emph{homotopy}, 
\index{homotopy!for a groupoid}
with $\tau$ being considered as a homotopy from $\alpha$ to $\beta$, 
as displayed in the following diagram. 
The significant feature of $\tau$ is that it lifts from one level to the next. 
$$
\xymatrix{ 
  C_1  \ar[rr]^{\alpha_1,\;\beta_1} \ar[dd]<-0.5ex>_s \ar[dd]<0.5ex>^t 
    &&  C_1   \ar[dd]<-0.5ex>_s \ar[dd]<0.5ex>^t \\
    &&  \\
  C_0  \ar[rr]_{\alpha_0,\;\beta_0} \ar[uurr]^(0.55){\tau}
    &&  C_0  \\
}
$$

\medskip
Natural transformations compose in the obvious way. 
If $\kappa$ is a third functor from $\bbC$ to $\bbD$, 
and if $\sigma : \beta \to \kappa$ is a second natural transformation, 
then we obtain the diagrams: 
$$
\xymatrix{ 
  \alpha u  \ar[rr]^{\tau_u} \ar[dd]_{\alpha a} \ar[rrdd]^{\tau a}
    &&  \beta u \ar[dd]_{\beta a} \ar[rr]^{\sigma_u} \ar[rrdd]^{\sigma a} 
        &&  \kappa u \ar[dd]^{\kappa a} \\
    &&  &&  \\
  \alpha v  \ar[rr]_{\tau_v} 
    &&  \beta v  \ar[rr]_{\sigma_v} 
        &&  \kappa v \\
}
\qquad\qquad
\xymatrix{ 
  C_1   \ar[rrr]^{\alpha_1,\;\beta_1,\;\kappa_1} 
        \ar[dd]<-0.5ex>_s \ar[dd]<0.5ex>^t 
    &&&  C_1   \ar[dd]<-0.5ex>_s \ar[dd]<0.5ex>^t \\
    &&&  \\
  C_0   \ar[rrr]_{\alpha_0,\;\beta_0,\;\kappa_0} 
        \ar[uurrr]^(0.55){\tau,\;\sigma}
    &&&  C_0  \\
}
$$ 
We obtain a composite natural transformation 
$\tau\diamond\sigma ~:~ \alpha \to \kappa$ where 
\begin{eqnarray*}
(\tau\diamond\sigma)u 
  &=&  \tau_u \diamond \sigma_u, \\ 
(\tau\diamond\sigma)a 
  &=&  (\tau a)\diamond(\sigma_v) 
       ~=~ (\tau_u)\diamond(\sigma a) 
       ~=~ (\tau a)\diamond(\beta a)^{-1}\diamond(\sigma a).
\end{eqnarray*}
We thus obtain a groupoid whose objects are isomorphisms and whose 
arrows are natural transformations. 
When $\bbC=\bbD$ and we obtain the \emph{automorphism groupoid} of $\bbC$. 
\index{automorphism groupoid}

\begin{thm}
Let $\bbC = G \times \bbI_n$. 
Then the automorphism groupoid $\Aut\,\bbC$ of $\bbC$ has
\begin{itemize}
\item
$n!.|\Aut\,G|.|G|^{n-1}$ objects (automorphisms), 
\item
$(n!)^2.|\Aut\,G|.|G|^{2n-1}$ arrows (natural transformations), 
\item
degree $|Z(G)| = |G|/|\Inn\,G|$, 
\item
$\Out\,G$ connected components, 
with $n!.|\Inn\,G|.|G|^{n-1}$ objects in each component. 
\end{itemize}
\end{thm}
\begin{pf}
An automorphism is specified by choosing a permutation of the objects; 
an automorphism of the group $G$; and, for each arrow in the spanning tree, 
a choice of one of the $G$ arrows between the appropriate vertices.

When specifying a natural transformation $\tau : \alpha \to \beta$, 
the sources of the $\tau_u$ determine one of the $n!$ permutations 
of the objects, and the targets determine a second permutation, 
and there are then $|G|$ choices for each $\tau u_q$. 
The automorphism $\alpha$ is specified on the object group by choosing 
an automorphism from $\Aut\,G$ and on each of the $n-1$ arrows 
$(1,e,q)$ in the tree by choosing one of the $|G|$ 
arrows $(\alpha_0 1,\alpha_q,\alpha_0 q)$. 

The degree is determined by the number of loops at the 
identity automorphism $\id$. 
If $\tau$ is such a loop and $\tau_1=z$, then $z^{-1}az=a$ for every 
generator $a$ of $G$, so $z \in Z(G)$. 
Each $\tau_q$ is then determined by 
$\tau_q = \alpha_q^{-1} \diamond z \diamond \alpha_q$. 

The number of arrows whose source is $\id$ is $n!.|G|^n$
since there are $n!$ choices for the targets of the $\tau_q$, 
and the $|G|$ choices for each $\tau_q$. 
Dividing this number by the degree gives the number of objects in the 
component containing $\id$. 
The automorphism group acts on the objects of the automorphism groupoid 
by right multiplication, permuting the components, so the components 
are isomorphic and their number is the obvious quotient. 
\end{pf}

\begin{cor}
When $\bbC$ is a group $G$ considered as a one-object groupoid, 
the automorphism groupoid has $|\Aut\,G|$ objects; 
$|\Aut\,G|.|G|$ natural transformations; $|\Out\,G|$ components; 
$|\Inn\,G|$ objects in each component; and degree $|Z(G)|$. 
\end{cor}

%% \begin{example}
%% \emph{Our first example is with a group 
%% considerd as a single-object groupoid. 
%% Let $D_8 = \langle a,b ~|~ a^4=b^2=(ab)^2=e \rangle$, where $ba=a^3b$. 
%% The centre of $D_8$ is $\{e,a^2\}$, so $\Inn(D_8) \cong K_4$, 
%% a Klein $4$-group. 
%% It is well-known that $\Out(D_8) \cong C_2$ and $\Aut(D_8) \cong D_8$. 
%% Here is a table of the images of $a,b$ under these automorphisms.} 
%% \begin{center}
%% \begin{tabular}{|r|cccccccc|}
%% \hline
%% $\theta$   & $\id$ & $\alpha$ & $\alpha^2$ & $\alpha^3$ 
%%            & $\beta$ & $\alpha\beta$ & $\alpha^2\beta$ & $\alpha^3\beta$ \\ 
%% \hline
%% $\theta a$ & $a$ &  $a$ &    $a$ &    $a$ 
%%            & $a^3$ &  $a^3$ &  $a^3$ & $a^3$ \\ 
%% $\theta b$ & $b$ & $ab$ & $a^2b$ & $a^3b$ 
%%            &   $b$ & $a^3b$ & $a^2b$ &  $ab$ \\ 
%% \hline
%% \end{tabular}
%% \end{center}
%% \emph{[more to follow]}
%% \end{example}


\newpage
%%%%%%%%%%%%%%%%%%%%%%%%%%%%%%%%%%%%%%%%%%%%%%%%%%%%%%%%%%%%%%%%%%%%%%%%%%%%
\subsection{Admissible and coadmissible sections} \label{subsect:admissible}

These two types of section are related to special cases of 
natural transformations between automorphisms of a groupoid. 

For $\bbC$ a groupoid, an \emph{admissible section} $H_0 : C_0 \to C_1$ 
is a section of the source map which composes with the target map to give 
a bijection on $C_0$, 
$$
H_0 * s ~=~ \id_{C_0}, \qquad 
h_0 ~:=~ H_0 * t ~:~ C_0 \to C_0 \quad\text{is a bijection}. 
$$
Note that if $\tau : \id_{\bbC} \to h$ is a natural transformation, 
then $\tau$ is an admissible section. 

\medskip
The set of admissible sections $M(\bbC)$ of $\bbC$ is a group 
with multiplication 
$$
(H_0 \star J_0)u ~:=~ (H_0 u)(J_0 t H_0u) 
                  ~=~ ((H_0 u)(J_0 h_0 u) : u \to (h_0 * j_0)u)  
$$
where $j_0=J_0*t$. 
It is straightforward to verify that this product is associative, 
$$
(H_0 \star J_0 \star K_0)u 
  ~=~  ((H_0 u)(J_0 h_0 u)(K_0 j_0 h_0 u) : u \to (h_0 * j_0 * k_0)u). 
$$
Here is a sketch showing the situation: 
$$
\xymatrix{
  &&  &&  j_0 u 
          &&  &&  \\
h_0^{-1}u 
  &&  u  \ar[ll]_{H_0^{-1} u}  \ar[rr]^{H_0 u} 
         \ar[urr]^{J_0 u} \ar[drr]^{K_0 u} 
      &&  h_0u  \ar[rr]^{J_0 h_0 u} 
          &&  j_0 h_0 u  \ar[rr]^{K_0 j_0 h_0 u}
              &&  k_0 j_0 h_0 u  \\
  &&  &&  k_0 u 
          &&  &&  
}$$
The identity admissible section is $I_0$ 
where $I_0u = 1_u$ for all $u \in C_0$. 
The inverse of $H_0$ is the admissible section where 
$$
H_0^{-1} u = (H_0 h_0^{-1} u)^{-1}, 
\qquad\text{so}\quad 
H_0^{-1} h_0 u = (H_0 u)^{-1} 
\quad\text{and}\quad 
H_0^{-1}*t = h_0^{-1}. 
$$
Note that the map $M(\bbC) \to \Symm(C_0),~ H_0 \mapsto H_0*t$ 
is a homomorphism.

\bigskip
Similarly, a \emph{coadmissible section} $H_0 : C_0 \to C_1$ 
is a section of the target map which composes with the source map 
to give a bijection of $C_0$. 
For a picture of this situation just reverse all the arrows 
in the diagram above. 
The multiplication is given by 
$$
(H_0 \star J_0)u ~:=~ (J_0 s H_0 u)(H_0 u)
                  ~=~ ((J_0 h_0 u)(H_0 u) : (h_0 * j_0)u \to u)  
$$
where $h_0=H_0*s,~ j_0=J_0*s$. 
Note that if $\tau : h \to \id_{\bbC}$ is a natural transformation, 
then $\tau$ is a coadmissible section. 


\bigskip
We now generalise these notions. 
For $g_0,h_0$ a pair of permutations of the objects of a groupoid $\bbC$, 
a \emph{$(g_0,h_0)$-section} $H_0 : C_0 \to C_1$ 
is a map which composes with the source and target maps to give 
$g_0$ and $h_0$ respectively: 
$$
g_0 ~=~ H_0 * s, \qquad 
h_0 ~=~ H_0 * t.
$$
Note that if $\tau : g \to h$ is a natural transformation 
between automorphisms of $\bbC$, then $\tau$ is an admissible section. 
A $(g_0,h_0)$-section is also called an \emph{admissible $g_0$-section} 
and a \emph{coadmissible $h_0$-section}. 

We have constructed a groupoid $\bbS=\bbS(\bbC)$ 
having the automorphisms of $\bbC$ as objects 
and the $(g_0,h_0)$-sections as the elements of the hom-set $\bbS(g_0,h_0)$. 
Composition in $\bbS$ is defined by 
$$
(H_0 : g_0 \to h_0) \diamond (J_0 : h_0 \to j_0)u 
~:=~ (H_0 u : g_0 u \to h_0 u) \diamond (J_0 u : h_0 u \to J_0 u). 
$$

\medskip
We now define a multiplication on the set of admissible $g_0$-sections 
$M_g(\bbC)$ of $\bbC$. 
Note that there is a multiplication on the permutations of $C_0$ 
given in terms of the standard composition by 
$h_0 \star j_0 := h_0 * g_0^{-1} * j_0$, 
such that $g_0$ is the identity and $h_0$ has inverse $g_0*h_0^{-1}*g_0$. 
We define the product on $M_g(\bbC)$ by 
$$
(H_0 \star J_0)u ~:=~ (H_0 u)(J_0 g_0^{-1} t H_0 u) 
                  ~=~ ((H_0 u)(J_0 g_0^{-1} h_0 u) : u \to (h_0 \star j_0)u)  
$$
where $j_0=J_0*t$. 
It is straightforward to verify that this product is associative, 
$$
(H_0 \star J_0 \star K_0)u 
  ~=~  ((H_0 u)(J_0 g_0^{-1} h_0 u)(K_0 g_0^{-1} j_0 g_0^{-1} h_0 u) 
          : u \to (h_0 \star j_0 \star k_0)u). 
$$
Here is a sketch showing the situation: 
$$
\xymatrix{
  &&  j_0 u 
      &&& &&&   \\
g_0 u  \ar[rr]^{H_0 u} \ar[urr]^{J_0 u} \ar[drr]^{K_0 u} 
  &&  h_0u  \ar[rrr]^{J_0 g_0^{-1} t H_0 u} 
      &&&  j_0 g_0^{-1} h_0 u 
              \ar[rrr]^(0.45){K_0 g_0^{-1} t J_0 g_0^{-1} t H_0 u}
          &&&  k_0 g_0^{-1} j_0 g_0^{-1} h_0 u  \\
  &&  k_0 u 
      &&&  &&&  
}$$
The identity admissible section is $I_0$ 
where $I_0u = 1_{g_0 u}$ for all $u \in C_0$. 
The inverse of $H_0$ is the admissible section where 
$$
H_0^{-1} u = (H_0 h_0^{-1} g_0 u)^{-1}, 
\qquad\text{so}\quad 
H_0^{-1} g_0^{-1} h_0 u = (H_0 u)^{-1} 
\quad\text{and}\quad 
H_0^{-1}*t = g_0 * h_0^{-1} * g_0. 
$$
Note that the map from $M_g(\bbC)$ to $\Symm(C_0)$ with the $\star$ product, 
mapping $H_0$ to $H_0*t$ is a group homomorphism.




\newpage
%%%%%%%%%%%%%%%%%%%%%%%%%%%%%%%%%%%%%%%%%%%%%%%%%%%%%%
\subsection{Groupoid Actions}  \label{subsect:gpd-act}

An action of a groupoid $\bbC$ on a groupoid $\bbB$ 
is usually defined in the case where $\bbB$ is a union of groups 
and has the same objects as $\bbC$. 
Then, when $(c : w \to x) \in \bbC$ and $(b : w \to w) \in \bbB$, 
we have $(b^c : x \to x)$. 
So $c$ does not act by permuting the arrows of $\bbB$, 
but by providing an isomorphism from $\bbB(w)$ to $\bbB(x)$. 
A particular case of this situation is when $\bbB$ is a totally 
discrete subgroupoid of $\bbC$ and the action is conjugation, 
$b^c = c^{-1}bc$. 
We now give an alternative definition of an action, 
using the automorphism groupoid $\bbAut\,\bbB$, 
which does not require $\bbB$ to be totally disconnected, 
and which \emph{does} provide a permutation of the arrows. 

\begin{defn}
An \emph{action} of a groupoid $\bbC$ on a groupoid $\bbB$ 
is a groupoid morphism $\alpha=(\alpha_0,\alpha_1):\bbC\to\bbAut\,\bbB$. 
\end{defn}

{\bf ?????}\quad 
This seems to be saying that $\bbC$ has $m$ objects while $\bbB$ has $mn$ objects, 
so how can $/alpha+_0$ be a bijection? 

This means that, when $\alpha_0$ is a bijection and 
$\bbC = \bbH \times \bbI_m$ with $\bbH = \Gpd(H)$, 
then $\bbB$ has $m$ isomorphic components, 
$\bbB_i \cong \bbG \times \bbI_n$ say, 
and $\alpha_1(p,h,q) = (\alpha_0p, \alpha_{p,q}h, \alpha_0q)$ 
where $\alpha_{p,q} : \bbB_{\alpha_0p} \to \bbB_{\alpha_0q}$ 
is an isomorphism. 




%%%%%%%%%%%%%%%%%%%%%%%%%%%%%%%%%%%%%
\subsection{Conjugation in groupoids}


\begin{defn}
Let $(c : p \to q)$, where $p \neq q$, 
be an arrow in a connected groupoid $\bbC = \bbG \times \bbI_n$. 
Then \emph{conjugation by $c$}, written $\wedge c$, 
is an automorphism of $\bbC$ where: 
\begin{itemize}
\item
$p,q$ are interchanged, and the remaining objects are fixed; 
\item
the loops at $p$ are interchanged with those at $q$, 
$$
(p,g,p) \mapsto (q,c^{-1}gc,q), \qquad 
(q,g,q) \mapsto (p,cgc^{-1},p); 
$$
\item
the hom-set $\bbC(p,q)$ is interchanged with $\bbC(q,p)$, 
$$
(p,g,q) \mapsto (q,c^{-1}gc^{-1},p), \qquad 
(q,g,p) \mapsto (p,cgc,q);
$$
\item
the rest of the costar at $p$ is interchanged with that at $q$, 
$$
(r,g,p) \mapsto (r,gc,q), \qquad 
(r,g,q) \mapsto (r,gc^{-1},p);
$$
\item
the rest of the star at $p$ is interchanged with that at $q$, 
$$
(p,g,r) \mapsto (q,c^{-1}g,r), \qquad 
(q,g,r) \mapsto (p,cg,r);
$$
\item
the remaining arrows are unchanged. 
\end{itemize}
\end{defn}
There are a number of cases to consider when checking that composition 
is preserved by this mapping, for example 
$$
(r,g,p)^c(p,h,q)^c 
~=~ (r,gc,q)(q,c^{-1}hc^{-1},p) 
~=~ (r,(gh)c^{-1},p) 
~=~ (r,gh,q)^c. 
$$

\bigskip 
We now express $\wedge c$ as a word in our standard sets of generators. 

\medskip\noindent
{\bf [to be continued]}

\newpage\noindent
{\bf [Add the corresponding formula for the case $p=q$.]}

\bigskip
It is \emph{not} the case that the map $\wedge  : \bbC \to \bbAut,\bbC$ 
is a groupoid morphism. 
This is clear just by considering the images of the objects, 
where the symmetric group is acting.

\begin{lem}
$$
\wedge(cd) 
~=~ (\wedge  c)*(\wedge  d)*(\wedge  c) 
~=~ (\wedge  d)*(\wedge  c)*(\wedge  d) 
$$
\end{lem}
\begin{pf}
To be added.
\end{pf}


