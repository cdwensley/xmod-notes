% der-sect.tex  (06/05/23)

%%%%%%%%%%%%%%%%%%%%%%%%%%%%%%%%%%%%%%%%%%%%%%%%%%%%%%%
\section{Derivations and Sections} \label{sect:der-sec}


%%%%%%%%%%%%%%%%%%%%%%%%%%%%%%%%%%%%%%%%%%%%%%%%%%%%%%%%%%%%%%%
\subsection{Derivations} \label{subsec:der} \index{derivations}

The Whitehead monoid \index{Whitehead monoid} $\Der(\calX)$  of  
$\calX = (\partial : S \to R)$ was defined in \cite{W-48b} 
to be the monoid of all {\it derivations} from $R$ to $S$, 
that is the set of all maps  $R \to S$, with composition  $\ \star \ $, satisfying
\begin{center}
\begin{tabular}{c r c l }
\textbf{D1:}  &  $\chi(qr)$              &  = 
           & $(\chi q)^{r} \; (\chi r)$  \\
\textbf{D2:}  &  $(\chi_1 \star \chi_2)(r)$  &  =
           & $(\chi_2 r)(\chi_1 r)(\chi_2 \partial \chi_1 r)$. 
\end{tabular}
\end{center}

\noindent
The definition of Whitehead multiplication used here 
differs from that in \cite{alp:wens-ijac} in that it is now 
defined as multiplication on the right rather than on the left, 
which is why we are using `$\star$' in place of `$\circ$'. 
Invertible elements in the monoid are called \emph{regular}. 
\index{regular derivation} \index{derivation!regular} 
The Whitehead group $W = W(\calX)$ is the group of the monoid. 

In Brown and Gilbert \cite{brow:gilb} the notion of derivation 
was extended to that of $\gamma$-derivation, as in the following definition. 
Since ordinary derivations may be obtained from $\gamma$-derivations 
by setting $\gamma$ to be the identity automorphism of $\calX$, 
we shall give properties in terms of the more general case. 

\begin{defn} \index{$\gamma$-derivation}
If $\gamma = (\ddg,\dg)$ is an automorphism of $\calX$,
the Whitehead monoid $\Der_{\gamma}(\calX)$  of  $\calX$ 
is the monoid of all $\gamma$-{\it derivations}
from $R$ to $S$, that is the set of all maps  $R \to S$,
with composition written  $\ \star_{\gamma} \ $, satisfying
\begin{center}
\begin{tabular}{c r c l }
\textbf{\emph{D1:}}  &  $\chi(qr)$              &  = 
           & $(\chi q)^{\dg r}(\chi r)~.$  \\
\textbf{\emph{D2:}}  &  $(\chi_1 \star_{\gamma} \chi_2)(r)$  &  =
           & $(\chi_2 r)(\chi_1 r)(\chi_2 \dg^{-1} \partial \chi_1 r)$. 
\end{tabular}
\end{center}
\end{defn}

The following Lemma verifies that $\Der_{\gamma}(\calX)$ \emph{is} a monoid.
 
\begin{lem} \label{lem:invchir}
\mbox{  }\\
\vspace{-5mm}
\begin{enumerate}[{\rm (a)}]
\item~
$\chi 1 = 1$~,
\item~
$(\chi r)^{-1} \;\;=\;\; (\chi\,r^{-1})^{\dg r}$~,
\item~
the zero map is a derivation and an identity for the Whitehead multiplication,
\item~
the Whitehead multiplication is associative. 
\end{enumerate}
\end{lem}
\begin{pf}
%\vspace{-5mm}
\begin{enumerate}[(a)]
\item
This follows from $\chi(r1) = (\chi r)^1(\chi 1)$.
\item 
This follows from  $1 = \chi(r^{-1}r) = (\chi r^{-1})^{\dg r}(\chi r)$~.
\item
It is clear that $0 : R \to S,~ r \mapsto 1$ is a derivation, and that
$$
(\chi\star_{\gamma}0)r = 1(\chi r)1 = \chi r = (\chi r)11 
= (0\star_{\gamma}\chi)r~.
$$
\item
Expansion by {\bf D2:} using either bracketing 
(though one requires more work!) gives: 
$$
(\chi_1 \star \chi_2 \star \chi_3)r \;\;=\;\;
(\chi_3 r)(\chi_2 r)(\chi_3\dg^{-1}\partial\chi_2 r)
(\chi_1 r)(\chi_3\dg^{-1}\partial\chi_1 r)(\chi_2\dg^{-1}\partial\chi_1 r)
(\chi_3\dg^{-1}\partial\chi_2\dg^{-1}\partial\chi_1 r)~.
$$
\end{enumerate}
\end{pf}

For $\chi$ a $\gamma$-derivation, define 
$\psi=\psi_{\chi} : R \to S$ by $\psi r = \chi \dg^{-1}r$ or, equivalently, 
$\psi\dg r = \chi r$.
\begin{equation} \label{eq:chi-psi}
\vcenter{\xymatrix{
   S \ar[rr]^{\ddg} \ar[dd]_{\partial} 
     && S \ar[dd] <0.5ex>^{\partial} \\
     &&  \\
   R \ar[uurr]^{\chi} \ar[rr]_{\dg} 
     && R \ar[uu] <0.5ex>^{\psi}
}}
\end{equation}

\noindent
Then $\psi$ is a (identity-) derivation since 
$$
\psi(qr)
~=~ \chi((\dg^{-1}q)(\dg^{-1}r)) 
~=~ (\chi\dg^{-1}q)^{\dg(\dg^{-1}r)}(\chi\dg^{-1}r) 
~=~ (\psi q)^r(\psi r).
$$

\begin{lem} \label{lem:wgamma-to-w} 
The map $\ddth : \Der_{\gamma}(\calX) \to \Der(\calX),~ \chi \mapsto \psi_{\chi}$, 
is a monoid isomorphism. 
\end{lem} 
\begin{pf} 
If $\psi_1,\psi_2$ are the derivations corresponding to $\gamma$-derivations 
$\chi_1,\chi_2$, then 
\begin{eqnarray*} 
\left((\ddth\chi_1) \star (\ddth\chi_2)\right)^r 
  &=&  (\psi_1 \star \psi_2)r \\ 
  &=&  (\psi_2r)(\psi_1r)(\psi_2\partial\psi_1r) \\ 
  &=&  \left( \chi_2(\dg^{-1}r) \right) \left( \chi_1(\dg^{-1}r) \right) 
               \left( \chi_2\dg^{-1}\partial\chi_1(\dg^{-1}r) \right) \\ 
  &=&  (\chi_1 \star_{\gamma} \chi_2)(\dg^{-1}r) \\ 
  &=&  \ddth(\chi_1 \star_{\gamma} \chi_2)r. 
\end{eqnarray*}
So $\ddth$ is a homomorphism, and it is invertible since 
$$
\psi_1 = \psi_2  
\quad \Rightarrow \quad 
\chi_1\dg^{-1}r = \chi_2\dg^{-1}r ~~ (\forall r \in R)  
\quad \Rightarrow \quad 
\chi_1 = \chi_2. 
$$
\end{pf} 

\begin{lem} \label{lem:gamma-beta-chi}
Given a $\gamma$-derivation $\chi$ of $\calX$ there is an endomorphism 
$\beta_{\chi} = (\ddb_{\chi}, \db_{\chi})$
of $\calX$ where
$$
\ddb_{\chi} : S \to S, \;\; s \mapsto (\ddg s)(\chi \partial s), \quad
 \db_{\chi} : R \to R, \;\; r \mapsto  (\dg r)(\partial \chi r)
$$
such that 
\begin{enumerate}[{\rm (a)}]
\item~
$\ddb_{\chi}(s^r) 
  \;\;=\;\;  (\ddb_{\chi}s)^{\db_{\chi}r}
  \;\;=\;\;  (\chi r)^{-1}\,(\ddb_{\chi} s)^{\dg r}\,(\chi r)$~,
\item~
$\db_{\chi}(q^r)
  \;\;=\;\;  (\db_{\chi}q)^{\db_{\chi}r}
  \;\;=\;\;  ((\dg r)(\partial\chi r))^{-1}(\dg q)(\partial\chi q)
                     ((\dg r)(\partial\chi r))$~, 
\item~ 
$(\chi_1\star_{\gamma}\chi_2) r 
\;=\;
(\chi_2 r)(\ddb_{\chi_2} \ddg^{-1} \chi_1 r)
\;=\;
(\chi_1 r)(\chi_2 \dg^{-1} \db_{\chi_1} r)$~,
\item~
$\chi\,*\,{\ddg}^{-1}\,*\,\ddb_{\chi}\ 
 \;=\; \db_{\chi}\,*\,{\dg}^{-1}\,*\,\chi
 \;:\; R \to S, \quad r \mapsto (\chi r)(\chi\dg^{-1}\partial\chi r)$\,,~
\emph{so that the following diagram commutes:}
\begin{equation} \label{eq:gamma-sqchi}
\vcenter{\xymatrix{
   S \ar[rrr]^{{\ddg}^{-1}\,*\,\ddb_{\chi}}
     &&& S \\
     &&&  \\
   R \ar[uu]^{\chi} \ar[rrr]_{\db_{\chi}\,*\,{\dg}^{-1}} 
     &&& R \ar[uu]_{\chi}
}}
\end{equation}
\item~
The endomorphism  $\ddb_{\chi} * \ddg^{-1}$  
commutes with  $\partial * \chi * \ddg^{-1}$  
while  $\dg^{-1} * \db_\chi$ commutes with  $\dg^{-1} * \chi * \partial$~. 
\item~
When $\psi = \psi_{\chi}$ as in \emph{(\ref{eq:chi-psi})}, 
then ~$\db_{\chi}r ~=~ \db_{\psi}(\dg r)$~ 
and ~$\ddb_{\chi}s ~=~ \ddb_{\psi}(\ddg s)$, 
so that ~$\beta_{\chi} ~=~ \beta_{\psi} \circ \gamma$. 
\end{enumerate}
\end{lem}
\begin{pf}
We first check that $\db_{\chi}$ and $\ddb_{\chi}$ are homomorphisms.
\begin{eqnarray*}
\db_{\chi}(r_1r_2) 
  &=&   \dg(r_1r_2)\partial((\chi r_1)^{\dg r_2}(\chi r_2)) 
    ~=~ (\dg r_1)(\partial\chi r_1)(\dg r_2)(\partial\chi r_2) 
    ~=~ (\db_{\chi}r_1)(\db_{\chi}r_2), \\
\ddb_{\chi}(s_1s_2)
  &=&   \ddg(s_1s_2)(\chi((\partial s_1)(\partial s_2)) 
    ~=~ (\ddg s_1)(\ddg s_2)(\chi\partial s_1)^{\partial\ddg s_2}
           (\chi\partial s_2) 
    ~=~ (\ddb s_1)(\ddb s_2).
\end{eqnarray*}

\noindent
We now verify the six properties.
\begin{enumerate}[(a)]
\item
\mbox{}\vspace*{-9mm} 
\begin{eqnarray*} 
\ddb_{\chi}(s^r) 
  & = &  (\ddg s^r)(\chi\partial(s^r)) 
    \;=\; (\ddg s)^{\dg r}(\chi(r^{-1}(\partial s)r)) 
    \;=\; (\ddg s)^{\dg r}(\chi(r^{-1}))^{(\partial\ddg s)(\dg r)}
              (\chi\partial s)^{\dg r}(\chi r) \\
  & = &  \{(\chi(r^{-1}))(\ddg s)(\chi\partial s)\}^{\dg r}(\chi r) 
    \;=\;  (\chi r)^{-1}(\ddb_{\chi} s)^{\dg r}(\chi r) 
    \;=\;  (\ddb_{\chi} s)^{(\dg r)(\partial\chi r)} 
    \;=\;  (\ddb_{\chi} s)^{\db_{\chi}r}~.
\end{eqnarray*}
\item
\mbox{}\vspace*{-9mm} 
\begin{eqnarray*}
\db_{\chi}(q^r)
  & = &  (\dg(q^r))(\partial\chi(r^{-1}qr)) 
    \;=\;  (\dg(q^r))\,\partial\{(\chi(r^{-1}))^{\dg(qr)}
                (\chi q)^{\dg r}(\chi r)\} \\
  & = &  (\dg(q^r))(\partial\chi(r^{-1}))^{(\dg r)(\dg q^r)} 
                (\partial\chi q)^{\dg r}(\partial\chi r) 
    \;=\;  (\partial((\chi r)^{-1})(\dg(q^r))
                (\partial\chi q)^{\dg r}(\partial\chi r) \\
  & = &  (\partial\chi r)^{-1}\{(\dg q)(\partial\chi q)\}^{\dg r} 
                (\partial\chi r) 
    \;=\;  (\db_{\chi}q)^{\db_{\chi}r}~.
\end{eqnarray*}
\item 
\mbox{}\vspace*{-9mm} 
\begin{eqnarray*} 
(\chi_1\star_{\gamma}\chi_2)r 
  & = &
    (\chi_2 r)\{(\chi_1 r)(\chi_2\partial(\ddg^{-1}\chi_1 r))\}
    \;=\; (\chi_2 r)(\ddb_{\chi_2} \ddg^{-1} \chi_1 r)~, \\ 
(\chi_2 \dg^{-1} \db_{\chi_1})r
  & = & 
    \chi_2(r(\dg^{-1} \partial \chi_1 r))
  \;=\; (\chi_2 r)^{\partial\chi_1 r}\,(\chi_2 \dg^{-1} \partial \chi_1 r)
  \;=\; (\chi_1 r)^{-1}(\chi_2 r)(\chi_1 r)(\chi_2\dg^{-1}\partial\chi_1 r)~. 
\end{eqnarray*} 
\item
\mbox{}\vspace*{-9mm} 
\begin{eqnarray*}
\ddb_{\chi}(\ddg^{-1}\chi r)
  &=& \ddg(\ddg^{-1}\chi r)(\chi\partial\ddg^{-1}\chi r) 
     ~=~ (\chi r)(\chi\dg^{-1}\partial\chi r)~, \quad \text{and} \\
(\chi\dg^{-1})(\db_{\chi}r) 
  &=& \chi(r(\dg^{-1}\partial\chi r)) 
     ~=~ (\chi r)^{\partial\chi r}(\chi\dg^{-1}\partial\chi r) 
     ~=~ (\chi r)(\chi\dg^{-1}\partial\chi r)~. 
\end{eqnarray*} 
\item
\mbox{}\vspace*{-9mm} 
\begin{eqnarray*} 
\text{By (d)},~~ 
  &&     (\ddb_{\chi} * \ddg^{-1}) * (\partial * \chi * \ddg^{-1}) 
     ~=~ \partial * (\db_{\chi} * \dg^{-1} * \chi) * \ddg^{-1} 
     ~=~ (\partial * \chi * \ddg^{-1}) * (\ddb_{\chi} * \ddg^{-1}), \\ 
  &&     (\dg^{-1} * \db_{\chi}) * (\dg^{-1} * \chi * \partial) 
     ~=~ \dg^{-1} * (\chi * \ddg^{-1} * \ddb_{\chi}) * \partial 
     ~=~ (\dg^{-1} * \chi * \partial) * ( \dg^{-1} * \db_{\chi}). 
\end{eqnarray*} 
\item 
This relationship between $\beta_{\chi}$ and $\beta_{\psi}$ is immediate. 
\vspace*{-9mm} 
\end{enumerate} 
\end{pf}

\bigskip 
Using Lemma \ref{lem:gamma-beta-chi} and the first crossed module axiom,
the identity {\bf D1:} for derivations generalises as follows.
\begin{lem}
\mbox{  }\\
\vspace{-5mm}
\begin{enumerate}[{\rm (a)}]
\item\quad
$\chi(r_1r_2 \ldots r_k) \;=\;
(\chi r_1)^{\dg(r_2 \ldots r_k)}(\chi r_2)^{\dg(r_3 \ldots r_k)}
\,\ldots\,(\chi r_{k-1})^{\dg r_k}(\chi r_k)$~,
\item\quad
$\partial\chi(r_1r_2 \ldots r_k) \;=\;
(\dg(r_1r_2 \ldots r_k))^{-1}\,
(\db_{\chi}r_1)(\db_{\chi}r_2) \ldots (\db_{\chi}r_k)$~,
\item\quad
$\chi\partial(s_1s_2 \ldots s_k) \;=\;
(\ddg(s_1s_2 \ldots s_k))^{-1}\,
(\ddb_{\chi}s_1)(\ddb_{\chi}s_2) \ldots (\ddb_{\chi}s_k)$~.
\end{enumerate}
\end{lem}

\medskip 
It is straightforward to verify that for $g$ an invertible element 
in a monoid $M$, the set $M_g = (M,\ast_g)$ with elements $M$ and 
multiplication $\ast_g$ defined in terms of the usual multiplication by
\begin{equation} \label{eq:star-g-defn} 
m \ast_g n ~:=~ mg^{-1}n, 
\end{equation} 
is a monoid with identity $g$. 
If $m \in M$ is invertible in $M$ then $m$ has $\ast_g$-inverse 
$\overline{m} := gm^{-1}g$. 
The resulting monoids are isomorphic, 
either by $\theta_g : M \to M_g, m \mapsto mg$ 
or by $\theta'_g : M \to M_g, m \mapsto gm$.  
When $M$ is a group the $g$-conjugation automorphisms are the mappings 
\begin{equation} \label{eq:g-conj} 
\wedge_g m \;:\; G \to G,~ 
    n \,\mapsto\, \overline{m} \ast_g n \ast_g m = gm^{-1}ng^{-1}m\,. 
\end{equation} 

This notion generalises to categories and to crossed modules, 
but the application we require here is to the monoid of endomorphisms 
$\End_{\gamma}(\calX)$, 
where $\gamma = (\ddg,\dg)$ is an automorphism of $\calX$, 
with multiplication 
\begin{equation} \label{eq:star-gamma-calX} 
\alpha\ast_{\gamma}\beta ~:=~ 
(\dda\ast_{\ddg}\ddb,~ \da\ast_{\dg}\db). 
\end{equation} 

\newpage
\begin{thm} \label{thm:Delta}
There is a monoid homomorphism 
$\Delta_{\gamma} : \Der_{\gamma}(\calX) \to \End_{\gamma}(\calX),~ 
                   \chi \mapsto \beta_{\chi} = (\ddb_{\chi},\db_{\chi})$.
\end{thm}
\begin{pf}
Since 
\vspace*{-9mm} 
\begin{eqnarray*}
(\ddb_{\chi_1} \ast_{\gamma} \ddb_{\chi_2})s 
 &=& (\ddb_{\chi_1} * \ddg^{-1} * \ddb_{\chi_2})s \\ 
 &=& \ddb_{\chi_2}(s(\ddg^{-1}\chi_1\partial s)  \\ 
 &=& (\ddg s)(\chi_1\partial s)\chi_2((\partial s)
       (\dg^{-1}\partial\chi_1\partial s) \\ 
 &=& (\ddg s)(\chi_1\partial s)(\chi_2\partial s)^{\partial\chi_1\partial s} 
       (\chi_2\dg^{-1}\partial\chi_1\partial s) \\ 
 &=& (\ddg s)(\chi_2\partial s)(\chi_1\partial s)
       (\chi_2\dg^{-1}\partial\chi_1\partial s) \\ 
 &=& \ddb_{\chi_1 \star_{\gamma} \chi_2}\ s, \\
(\db_{\chi_1} \ast_{\gamma} \db_{\chi_2})r 
 &=& (\db_{\chi_1} * \dg^{-1} * \db_{\chi_2})r \\ 
 &=& \db_{\chi_2}(r(\dg^{-1}\partial\chi_1 r)) \\
 &=& (\dg r)(\partial\chi_1 r)\partial\left( 
       (\chi_2 r)^{\partial\chi_1r}(\chi_2\dg^{-1}\partial\chi_1 r)\right) \\ 
 &=& (\dg r)\partial\left( 
       (\chi_2r)(\chi_1r)(\chi_2\dg^{-1}\partial\chi_1 r)\right) \\
 &=& \db_{\chi_1\star\chi_2}\ r, 
\end{eqnarray*}
it follows that~ 
$(\Delta_{\gamma}\chi_1) \ast_{\gamma} (\Delta_{\gamma}\chi_2) 
 = \Delta_{\gamma}(\chi_1 \star_{\gamma} \chi_2)$.
\end{pf}


\begin{lem} \label{lem:gamma-eta_s}
For each $s \in S$ the function $\eta_s$ 
$$
\eta_s : R \to S, \quad r \mapsto (s^{-1})^r s 
$$  
is a derivation, 
\index{derivation!principal} \index{principal derivation} 
called a \emph{principal derivation}, satisfying   
$$
\eta_s(\partial s_0) \,=\, [s_0,s], 
\qquad\mbox{and}\qquad 
\partial(\eta_s r) \,=\, [r, \partial s].
$$ 
Similarly 
$$
\eta_{(s,\gamma)} ~:=~ \ddth^{-1}\eta_s  ~:~ r \mapsto (s^{-1})^{\dg r}\,s
$$
is the corresponding \emph{principal $\gamma$-derivation} satisfying 
$$
\eta_{(s,\gamma)} (\partial s_0) \,=\, [\ddg s_0,s], 
\qquad\mbox{and}\qquad 
\partial(\eta_{(s,\gamma)} r) \,=\, [\dg r, \partial s].
$$
\end{lem}
\begin{pf} 
$$
\eta_s(qr) 
  \;=\;  (s^{-1})^{qr} (s^r(s^{-1})^r) s 
  \;=\;  ((s^{-1})^qs)^r (s^{-1})^r s 
  \;=\;  (\eta_sg)^r (\eta_sr) \qquad\mbox{satisfying D1}, 
$$
$$
\eta_s \partial s_0 \;=\; (s^{-1})^{\partial s_0}\,s 
                    \;=\; (s_0)^{-1}\,s^{-1}\,s_0\,s
                    \;=\; [s_0,s]~,
$$
$$
\partial(\eta_s r)  \;=\; (\partial s^{-1})^r (\partial s) 
                    \;=\; [r, \partial s]~.
$$
The corresponding properties of $\eta_{(s,\gamma)}$ are easily verified in the same way. 
\end{pf}

\medskip
We shall see later that there is a homomorphism from $S$ to $\Der(\calX)$ 
mapping $s$ to the principal derivation $\eta_s$. 


\begin{lem}[More properties of principal $\gamma$-derivations] \label{lem:princ-prop}
\mbox{}\\
\vspace{-5mm}
\begin{enumerate}[{\rm (a)}]
\item~
$\eta_{(1,\gamma)}$ is the zero map,
\item~
$\ddb_{\eta_{(s,\gamma)}}s_0 = (\ddg s_0)^s 
\qquad\mbox{and}\qquad
\db_{\eta_{(s,\gamma)}}r = (\dg r)^{\partial s}$~,
\item~
$\eta_{(s_1,\gamma)} \star_{\gamma} \eta_{(s_2,\gamma)} = \eta_{(s_1s_2,\gamma)}$, 
\item~
the zero map is the identity in $\End_{\gamma}(\calX)$ and 
$\overline{\eta_{(s,\gamma)}} = \eta_{(s^{-1},\gamma)}$~.
\end{enumerate}
\end{lem} 

\begin{pf}
\begin{enumerate}[(a)]
\item~
$\eta_{(1,\gamma)} r ~=~ 1^{\dg r}\; 1 ~=~ 1$~,
\item~
$\ddb_{\eta_{(s,\gamma)}} s_0 ~=~ (\ddg s_0)(\eta_{(s,\gamma)}(\partial s_0))
~=~ (\ddg s_0) [\ddg s_0,s] ~=~ (\ddg s_0)^s$  

and \quad 
$\db_{\eta_{(s,\gamma)}}r ~=~ (\dg r)(\partial \eta_s r)
~=~ (\dg r) [\dg r, \partial s] 
~=~ (\dg r)^{\partial s}$~,
\item~
$(\eta_{(s_1,\gamma)} \star_{\gamma} \eta_{(s_2,\gamma)})r 
~=~ (\eta_{(s_2,\gamma)}r)(\eta_{(s_1,\gamma)}r)
       (\eta_{(s_2,\gamma)}\partial\ddg^{-1}\eta_{(s_1,\gamma)}r)) 
~=~ (\eta_{(s_2,\gamma)}r)(\eta_{(s_1,\gamma)}r)[\eta_{(s_1,\gamma)}r,s_2] $ 

$~=~ (\eta_{(s_2,\gamma)}r)s_2^{-1}(\eta_{(s_1,\gamma)}r)s_2 
~=~ (s_2^{-1})^{\dg r}(s_1^{-1})^{\dg r}s_1s_2 
~=~ \eta_{(s_1s_2,\gamma)}r.$ 
\item~
It follows from (a) and (c) that 
$\eta_{(s,\gamma)} \star_{\gamma} \eta({1,\gamma)} ~=~ \eta_{(s,\gamma)} 
 ~=~ \eta_{(1,\gamma)} \star_{\gamma} \eta({s,\gamma)}$ 
 and also that $\eta_{(s,\gamma)} \star_{\gamma} \eta({s^{-1},\gamma)} = \eta_{(1,\gamma)}.$ 
\end{enumerate}
\end{pf}

%\newpage 
\begin{lem} \label{lem:chi-sigma-rho}
The following statements are equivalent.
\begin{enumerate}[{\rm (i)}]
\item~ 
$\chi$ has a Whitehead $\gamma$-inverse $\overline{\chi}$; 
\item~ 
$\ddb{_\chi} \in \Aut(S)$, 
where $\ddb_{\chi}(s) = (\ddg s)(\chi \partial s)$;
\item~ 
$\db_{\chi} \in \Aut(R)$,
where $\db_{\chi}(r) = (\dg r)(\partial \chi r)$;
\item~
$\beta = (\ddb,\db) \in \Aut_{\gamma}(\calX)$.
\end{enumerate}
When these conditions are satisfied, 
$$
\overline{\chi}r \;=\; ({\ddg\overline{\ddb_{\chi}}}\,\chi\,r)^{-1}
                 \;=\; (\chi\,\overline{\db_{\chi}}\dg\,r)^{-1}\,,~~  
(\chi r)(\overline{\chi}r) \;=\; 
(\chi \dg^{-1} \partial \overline{\chi}r)^{-1}\,,
~~\text{and}~~ 
(\overline{\chi}r)(\chi r) \;=\; 
(\overline{\chi} \dg^{-1} \partial \chi r)^{-1}\,. 
$$
\end{lem}
\begin{pf}
Theorem \ref{thm:Delta} shows that, when $\chi$ is a regular derivation,
both  $\ddb_{\chi}$  and  $\db_{\chi}$  are automorphisms, 
so (i) implies (ii) and (iii), and hence (iv).

\medskip
Now suppose that  $\ddb_{\chi}$  has $\gamma$-inverse $\overline{\ddb_{\chi}}$.
We first show that  $\chi^{\$}$  is a derivation where
$\chi^{\$} r = (\ddg \overline{\ddb_{\chi}} \chi r)^{-1}$~.
Using the equivalent formula, 
$\ddb_{\chi} \ddg^{-1} \chi^{\$}r = (\chi r)^{-1}$~,
\begin{eqnarray*}
\ddb_{\chi} \ddg^{-1}(\,(\chi^{\$} q)^{\dg r} \,(\chi^{\$} r)\,)
  & = &
    (\ddb_{\chi} \ddg^{-1} \chi^{\$} q)^{\db_{\chi}r} 
     (\ddb_{\chi} \ddg^{-1} \chi^{\$} r)
      \quad \mbox{by Theorem \ref{lem:gamma-beta-chi} (a)} \\
  & = &
    ((\chi q)^{-1})^{(\dg r)(\partial \chi r)} (\chi r)^{-1} 
      \quad\quad \mbox{by definition of} \quad \chi^{\$} \\
  & = &
    (\chi r)^{-1}(\,(\chi q)^{-1}~)^{\dg r} 
  \;=\; 
    ((\chi q)^{\dg r}\,(\chi r))^{-1} 
  \;=\;
    (\chi(qr))^{-1}
  \;=\;
    \ddb_{\chi} \ddg^{-1} \chi^{\$}(qr)~.
\end{eqnarray*}

\noindent 
We now show that  $\chi^{\$}$  is the Whitehead $\gamma$-inverse 
$\overline{\chi}$ of  $\chi$, 
using Lemma \ref{lem:gamma-beta-chi} (c), (d) :  
\begin{eqnarray*}
(\chi^{\$} \star_{\gamma} \chi)r 
  & = &  (\chi r)(\ddb_{\chi} \ddg^{-1} \chi^{\$} r) 
  \;=\;  (\chi r)(\chi r)^{-1}~, \\ 
(\chi \star_{\gamma} \chi^{\$})r 
  & = &  (\chi r)(\chi^{\$} \dg^{-1} \db_{\chi} r) 
  \;=\;  (\chi r)(\ddg \overline{\ddb_{\chi}} \chi \dg^{-1} \db_{\chi} r)^{-1} 
  \;=\;  (\chi r)(\chi r)^{-1}~. 
\end{eqnarray*}

\noindent
Thus (ii) implies (i), (iii) and (iv).

\medskip
Similarly, suppose that  $\db_{\chi}$  has an inverse  $ \overline{\db_{\chi}} $.
We show that  $\chi^{\#}$  is a derivation where
$\chi^{\#} r = (\chi \overline{\db_{\chi}} \dg r)^{-1}$\,.
Define  $r' = \overline{\db_{\chi}} \dg r$  so that  
$\chi^{\#}r = (\chi r')^{-1}$, 
and similarly for  $q'$.  Then
\begin{eqnarray*} 
(\chi^{\#}q)^{\dg r}\, (\chi^{\#} r) 
  & = & 
    \left( (\chi q')^{-1} \right)^{\dg r}\, (\chi r')^{-1} 
  \;=\; 
    \left( (\chi r') (\chi q')^{\db_{\chi} r'} \right)^{-1} 
  \;=\; 
    \left( (\chi r') (\chi q')^{(\dg r')(\partial\chi r')} \right)^{-1} \\ 
  & = & 
    \left( (\chi q')^{\dg r'} (\chi r') \right)^{-1}
  \;=\; 
    (\chi(qr)')^{-1}
  \;=\; 
    \chi^{\#}(qr)~. 
\end{eqnarray*}

\noindent
This  $\chi^{\#}$  is another form of $\overline{\chi}$ since, 
again using Lemma \ref{lem:gamma-beta-chi} (c), (d) :  
\begin{eqnarray*}
(\chi^{\#} \star_{\gamma} \chi) r
  & = &  (\chi r)(\ddb_{\chi} \ddg^{-1} \chi^{\#} r) 
  \;=\;  (\chi r)(\ddb_{\chi} \ddg^{-1} \chi \overline{\db_{\chi}} \dg r)^{-1} 
  \;=\;  (\chi r)(\chi r)^{-1} , \\
(\chi \star_{\gamma} \chi^{\#}) r
  & = &  (\chi r)(\chi^{\#} \dg^{-1} \db_{\chi} r)
  \;=\;  (\chi r)(\chi r)^{-1}~.
\end{eqnarray*}

\noindent
Thus (iii) implies (i), (ii) and (iv). 

\medskip\noindent 
Finally, (iv) implies (ii) and (iii), and hence (i). 

\medskip\noindent 
The expressions for $(\chi r)(\overline{\chi}r)$  
and $(\overline{\chi}r)(\chi r)$ are obtained by  
expanding  $(\overline{\chi} \star_{\gamma} \chi) r$  
and  $(\chi \star_{\gamma} \overline{\chi}) r$.  
\end{pf}

\medskip
We shall see in Subsection \ref{subs:AX} that $W_{\gamma}(\calX)$ 
is the source group in the $\gamma$-\emph{actor} of $\calX$, 
$$
\Act_{\gamma}(\calX) 
~=~ 
(\Delta_{\gamma} : W_{\gamma}(\calX) \to \Aut_{\gamma}(\calX))~. 
$$
Lue and Norrie, in \cite{lue1,lue2,norrie-thesis,nor1}, 
showed that $\Act(\calX)$ is the automorphism object of  $\calX$
in the category \catXMod.
Gilbert, in \cite{gilb1}, has discussed a connection between
derivations and group extensions.



\newpage
%%%%%%%%%%%%%%%%%%%%%%%%%%%%%%%%%%%%%%%%%%%%%%%%%%%%%%%%%
\subsection{Sections}  \label{subsec:sec} \index{section}

The construction for a cat$^1$-group $\calC = (e;t,h : G \to R)$ 
equivalent to the $\gamma$-derivation of the corresponding crossed module 
is the \emph{$\gamma$-section}, 
namely a group monomorphism  $\xi : R \to G$  satisfying:
\begin{center}
\begin{tabular}{l l}
\textbf{S1:}  &  $t \xi(r) = \dg r \;\; \mbox{for all} \;\; r \in R$.
\end{tabular}
\end{center}
The equations  
\begin{equation}\label{eq:x2c}
\xi r = (e \dg r)(\epsilon \chi r) = (\dg r, \chi r)\,,
\quad\quad
\chi r = (e \dg r)^{-1} (\xi r) 
\end{equation}
define a section  $\xi$  of  $\calC$  
in terms of a derivation  $\chi$  of  $\calX$,  and conversely.
The automorphism $\gamma = (\ddg,\dg)$ of $\calX = (\partial : S \to R)$ 
determines an automorphism $\barg$ of $R \ltimes S$, 
and hence an automorphism $(\barg,\dg)$ 
of the corresponding cat$^1$-group. 

\medskip
The \emph{principal section} $\kappa_s,\; s \in \ker t$, 
\index{principal section} \index{section!principal} 
and the corresponding principal derivation $\eta_s$
are given by
$$
\eta_s r ~=~ (s^{-1})^{\dg r} s \quad\quad
\kappa_s r ~=~ (e \dg r)^s ~=~ s^{-1}(e \dg r)s\,.
$$
In the semidirect product notation we have 
$$
\kappa_s r ~=~ (\dg r,\eta_s r) ~=~ (\dg r,(s^{-1})^{\dg r} s) ~=~
(1,s^{-1})(\dg r,1)(1,s) ~=~ (\dg r,1)^{(1,s)}\,.
$$

\noindent
Since $(ehg^{-1})(\xi \dg^{-1} hg) \in \ker t$ and $(ehg^{-1})g \in \ker h$ 
we have, in the group groupoid, 
\begin{equation} \label{eq:sect-reversal}
g * \xi \dg^{-1} hg ~=~
g(ehg^{-1})(\xi \dg^{-1} hg) ~=~
(ehg)((ehg^{-1})g)((ehg^{-1})(\xi \dg^{-1} hg)) ~=~
(\xi \dg^{-1} hg)(ehg^{-1})g.
\end{equation}

\noindent
These sections form the monoid  $\rm{Sect}(\calC)$  of  $\calC$,
whose composition rule we determine from the rule  
\textbf{D2:}  for  $\Der(\calX)$  by evaluating:
\begin{eqnarray*}
(\xi_1 \star_{\gamma} \xi_2)r
 & = & (e \dg r)(\epsilon (\chi_1 \star \chi_2) r) \\
 & = & (e \dg r)(\epsilon \chi_2 r)(\epsilon \chi_1 r)
           (\epsilon \chi_2 \dg^{-1} h \epsilon \chi_1 r) \\
 & = & (\xi_2 r)(e \dg r^{-1})(\xi_1 r)(eh(\epsilon \chi_1 r)^{-1})
        (\xi_2 \dg^{-1} h \epsilon \chi_1 r) \\
 & = & (\xi_2 r)(e \dg r^{-1})(\xi_1 r)(eh((\xi_1 r)^{-1}(e \dg r)))
        (\xi_2 \dg^{-1} h ((e \dg r^{-1})(\xi_1 r))) \\
 & = & ((e \dg r)(\xi_2 r^{-1}))^{-1} ((\xi_1 r)(e h \xi_1 r^{-1}))
        ((e \dg r)(\xi_2 r^{-1})) (\xi_2 \dg^{-1} h \xi_1 r).
\end{eqnarray*}
Since  
$(e \dg r)(\xi_2 r^{-1}) \in \ker t$
while  $(\xi_1 r)(e h \xi_1 r^{-1}) \in \ker h$,
we obtain, using (\ref{eq:sect-reversal}),
\begin{equation} \label{eq:section-comp}
\begin{tabular}{l l}
\textbf{S2:}  &  $(\xi_1 \star_{\gamma} \xi_2)r
 ~=~ (\xi_1 r)(e h \xi_1 r^{-1})(\xi_2 \dg^{-1} h \xi_1 r)
 ~=~ (\xi_2 \dg^{-1} h \xi_1 r)(eh \xi_1 r^{-1})(\xi_1 r)$.
\end{tabular}
\end{equation}
(Note that this axiom also differs from that in \cite{alp:wens-ijac}
in that it is converted to a multiplication on the right.) 

The section  $\dg*e$  is the identity for this composition,
and equation (\ref{eq:x2c}) determines a monoid isomorphism
$\Der(\calX) \cong \Sect(\calC)$.
A section is  \emph{regular}  when  $h \xi$  is an automorphism of $R$, 
and the group of regular sections is isomorphic to the Whitehead group.

Each  $\chi$  and its associated  $\xi$  determine endomorphisms of  
$R, S, G, \calX$ and $\calC$, namely 
\begin{eqnarray} \label{eq:five-endos}
   \db_{\chi} ~=~ \db_{\xi}  &  :  &  R \to R, 
                  \quad  r \mapsto (\dg r)(\partial \chi r) = h \xi r,  
\nonumber \\
  \ddb_{\chi} ~=~ \ddb_{\xi}  &  :  &  S \to S,
                  \quad  s \mapsto (\ddg s)(\chi \partial s) \,
                      = \, (\ddg s)(e\partial\ddg s^{-1})(\xi \partial s) \, 
                      = \, (\xi \partial s)(e\partial\ddg s^{-1})(\ddg s), 
\nonumber \\
 \barb_{\chi} ~=~ \barb_{\xi}  &  :  &  G \to G,
                  \quad  g \mapsto (eh \xi tg)(\xi tg^{-1})
                           (\barg g)(eh \barg g^{-1})(\xi hg),  \\
  (\ddb_{\chi}, \db_{\chi})  ~=~  (\ddb_{\xi}, \db_{\xi}) 
    & : & \calX \to \calX, 
\nonumber \\
  (\barb_{\chi}, \db_{\chi})  ~=~  (\barb_{\xi}, \db_{\xi}) 
    & : & \calC \to \calC, 
\nonumber
\end{eqnarray}
and these assignments determine group homomorphisms from the
Whitehead group to these five endomorphism groups.
The accompanying diagram shows the relationship between the various 
groups and homomorphisms.

%\TurnRadius{12pt}
$$\xy
\xymatrix{
   &&&&&&  \\
  \Aut(S)  \ar @{.} [0,2]^(0.4){\curvearrowright}
    &&  S  \ar[rrr] <+0.5ex>^(0.4){\epsilon} 
           \ar[ddd] _{\partial}  
           \ar `u[ul] `[0,-1]_{\ddb_{\chi},\ddg} 
                          `[0,0]<+0.5ex> [0,0]<+0.5ex>
   &&&  G  \ar[ddd] <+0.5ex>^h  
           \ar[ddd] <-0.5ex>_t
           \ar `u[ur] `[r]^{\barb_{\chi},\barg} 
                      `[0,0]<-0.5ex> [0,0]<-0.5ex>
     &    \\
   &&&&&& \\
   &&&&&& \\
    &&  R  \ar `l [ll]<+0.5ex> [uuull]<+0.5ex>^(0.43){\alpha}
           \ar `l[uuul] `[uuu]^{\chi} [uuu]
           \ar[rrr] <-0.5ex>_(0.4){\id_R}
           %\ar`l"2,2"`"1,3"`"2,4"`"3,3"[0,0]
   &&&  R  \ar `l[uuul] `[uuu]^{e}   [uuu]
           \ar `r[uuur]  `[uuu]_{\xi} [uuu]
           \ar `d[dr] `[r]_{\db_{\chi},\dg} 
                      `[0,0]<+0.5ex> [0,0]<+0.5ex>
     &    \\ 
   &&&&&& \\ 
   &&&&&& \\
}
\endxy$$


%%%%%%%%%%%%%%%%%%%%%%%%%%%%%%%%%%%%%%%%%%%%%%%%%%%%%%%%%%%%%%%%%%%%%%
\subsection{The group-groupoid equivalent of derivations and sections}
\label{subs:gpd-sect}

(This Subsection (for now) covers only identity derivations and sections.) 

\medskip\noindent 
The cat$^1-$formula (\ref{eq:section-comp}) 
for Whitehead composition of sections is 
$$ 
\textbf{S2:} \qquad
(\xi_1 \star \xi_2)r
 ~=~ (\xi_1 r)(e h \xi_1 r^{-1})(\xi_2 h \xi_1 r)
 ~=~ (\xi_2 h \xi_1 r)(eh \xi_1 r^{-1})(\xi_1 r)~,
$$
which is rather obscure.  
Considering the group-groupoid $\calG$ 
associated to the cat$^1$-group $\calC$, 
as discussed in Subsection \ref{subs:gpgpd}, 
we see that sections of $\calC$ are associated to automorphisms of $\calG$.

\bigskip\noindent
A section $\xi$ of $\calC$ defines a groupoid endomorphism 
$\lambda = \lambda_{\xi} : \calG \to \calG$ 
as follows.  Consider the diagram
\begin{equation} \label{eq:gpd-section}
\vcenter{\xymatrix{ 
  h\xi tg \ar[rr]^{\lambda g} 
     && h\xi hg  \\
     &&  \\
  tg \ar[rr]_{g} \ar[uu]^{\xi tg}
     && hg \ar[uu]_{\xi hg}
}}
\end{equation}
where $t \lambda g = h\xi tg$ and $h \lambda g = h\xi hg$.
The morphism $\lambda$ is defined on objects and arrows by 
\begin{equation} \label{eq:def_lambda}
\lambda r ~=~ h\xi r, \quad\quad 
\lambda g ~=~ (\widetilde{\xi t g}) * g * \xi h g
~=~ (eh\xi tg)(\xi tg^{-1})g(ehg^{-1})(\xi hg). 
\end{equation}
The product of the first four terms is in $\ker h$, 
while the product of the last four terms is in $\ker t$. 
It is easily verified that $\lambda$ is a groupoid morphism. 
If $r_0 = tg_1,~ r_1 = hg_1 = tg_2$ and $r_2 = hg_2$, then
\begin{eqnarray*}
(\lambda g_1) * (\lambda g_2)
& = & (eh\xi r_0)(\xi r_0^{-1})g_1(er_1^{-1})(\xi r_1)
      .(eh\xi r_1^{-1})
      .(eh\xi r_1)(\xi r_1^{-1})g_2(er_2^{-1})(\xi r_2) \\
& = & (eh\xi r_0)(\xi r_0^{-1})(g_1 * g_2)(er_2^{-1})(\xi r_2) \\
& = & \lambda(g_1 * g_2). 
\end{eqnarray*}

\bigskip\noindent
When we consider $\xi_1$ followed by $\xi_2$ we get 
$$
\xymatrix{ 
 &  h\xi_2 h\xi_1 tg \ar[rr]^{\lambda_2\lambda_1 g}
    & & h\xi_2 h\xi_1 hg 
        & &  \\
 &  & & & &  \\
 &  h\xi_1 tg \ar[rr]^{\lambda_1 g} \ar[uu]^(0.6){\xi_2 h\xi_1 tg} 
            \ar `dl[0,-1] `[0,-1]^{eh\xi_1tg} `[0,0] [0,0] 
            %% \ar@(ul,dl)_{eh\xi_1tg}
    & & h\xi_1 hg \ar[uu]_(0.6){\xi_2h\xi_1 hg} 
            \ar `ur[0,1] `[0,1] `[0,0]^{eh\xi_1hg} [0,0] 
            %% \ar@(ur,dr)^{eh\xi_1hg}
        & &  \\
 &  & & & &  \\
 &  tg \ar[rr]^{g} \ar[uu]^(0.4){\xi_1 tg}
    & & hg \ar[uu]_(0.4){\xi_1 hg}
        & &
}
$$
and the composite on the left-hand side is 
$$
(\xi_1 tg)*(\xi_2 h\xi_1 tg) ~=~
(\xi_1 tg)(eh\xi_1 tg^{-1})(\xi_2 h\xi_1 tg)
$$
in agreement with {\bf S2:}, 
and similarly for the right-hand side.
Thus $\lambda_{\xi_1\star\xi_2} ~=~ \lambda_{\xi_1} * \lambda_{\xi_2}$  
and we have the following result. 
\begin{lem}
There is a monoid homomorphism 
$$
\Sect(\calC) \rightarrow \End(\calG), \qquad \xi \mapsto \lambda_{\xi} 
$$
which restricts to a homomorphism $W(\calC) \to \Aut(\calG)$.
\end{lem}

\bigskip\noindent
Associated to a principal derivation $\eta_sr = (s^{-1})^rs$, 
and the corresponding principal section $\kappa_sr = (er)^s$, 
there is a \emph{principal endomorphism} \index{principal endomorphism} 
$\lambda_s$ of $\calG$. 

\begin{prop}
The principal endomorphism $\lambda_s = \lambda_{\kappa_s} : \calG \to \calG$ 
is given by
$$
\lambda_s r ~=~ r^{hs}, \qquad  \lambda_s g ~=~ g^{hs}.
$$ 
\end{prop}
\begin{pf}
Applying the formulae in equation (\ref{eq:def_lambda}), 
and $[\ker t,\ker s] = 1$,  
\begin{eqnarray*}
\lambda_s r 
  &=& h \kappa_s r ~=~ (hs^{-1})r(hs) ~=~ r^{hs}, \\
\lambda_s g
  &=& (\widetilde{\kappa_stg})*g*(\kappa_shg) \\
  &=& (\widetilde{(etg)^s})*g*(ehg)^s \\
  &=& (eh(s^{-1}(etg)s)(s^{-1}(etg)s)^{-1}(et(s^{-1}(etg)s)(etg^{-1})g(ehg)^{-1}(ehg)^s \\
  &=& (ehs^{-1})(etg)(ehs)s^{-1}(etg^{-1})[s][g(ehg)^{-1}]s^{-1}(ehg)s \\
  &=& (ehs^{-1})(etg)[(ehs)s^{-1}][(etg^{-1})g]s \\
  &=& (ehs^{-1})g(ehs) \\
  &=& g^{hs}. 
\end{eqnarray*}
\end{pf}

