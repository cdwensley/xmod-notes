%% dbl-gpd-dev.tex,  version 02/05/17
\newpage
\noindent
{\bf [This is now only in the development version.]}

\bigskip
%%%%%%%%%%%%%%%%%%%%%%%%%%%%%%%%%%%%%%%%%%%%%%%%%%%%%%%%%
\subsection{From actor crossed square to group-double groupoid}

{\bf [Why start with the actor? -- Because we know it works!]}

\bigskip\noindent
The correspondence from actor crossed square to cat$^2$-group is
$$
\xymatrix @C=4.0pc @R=2.0pc{
S \ar[r]^{\eta} \ar[dd]_{\partial}
  & W \ar[dd]^{\Delta} 
    & & (A \ltimes R)\ltimes(W \ltimes S) 
        \ar[r]<1.0ex>^(0.65){\ddtto,\ddhho} \ar[r]<0.2ex>
        \ar[dd]<-1.0ex>_{\ddttt,\ddhht} \ar[dd]<-0.2ex>
        & A \ltimes W  \ar[l]<1.0ex>^(0.35){\ddeeo} 
                       \ar[dd]<1.0ex>^{\dttt,\dhht} \ar[dd]<0.2ex>\\
  & & \longrightarrow
      & & \\
R \ar[r]_{\iota}
  & A
    & & A \ltimes R \ar[r]<-1.0ex>_{\dtto,\dhho} \ar[r]<-0.2ex>
                    \ar[uu]<-1.0ex>_{\ddeet}
        & R \ar[l]<-1.0ex>_{\deeo}  \ar[uu]<1.0ex>^{\deet}
}
$$

We saw in Subsection \ref{subs:cat1} (\ref{eq:sdpcat1}) that the maps on 
$(e;t,h: R \ltimes S \to R)$ are 
$t(r,s)=r, h(r,s)=r(\partial s)$ and $er = (r,1)$.
These may be used immediately to define $\ddttt,\ddhht$ and $\ddeet$.
They may also be used to define $\ddtto, \ddhho$ and $\ddeeo$ 
after applying the isomorphism $\tau$
between $G = (A \ltimes R)\ltimes(W \ltimes S)$ 
and $\tilde{G} = (A \ltimes W)\ltimes(R \ltimes S)$ 
as shown in the following diagram
$$
\xymatrix @C=4.0pc @R=4.0pc{
G \ar[rr]<0.6ex>^{\tau}
  \ar[dr]<-0.2ex> \ar[dr]<-1.0ex>_{\ddtto,\ddhho}
  & & \tilde{G} \ar[dl]<0.2ex> \ar[dl]<1.0ex>^{\tilde{t}_1,\tilde{h}_1} 
                \ar[ll]<0.2ex>^{\tilde{\tau}} \\
  & A \ltimes W \ar[ul]<-1.0ex>_{\ddeeo} \ar[ur]<1.0ex>^{\tilde{e}_1}
    & \\
}
$$

\bigskip\noindent
Recall from Proposition \ref{prop:tau} that the formulae for $\tau$ 
and its inverse $\tilde{\tau}$ are
\begin{eqnarray*}
\tau((\alpha,r),(\chi,s))
 & = &  ((\alpha,\chi),(r,(r \bt \chi)s)), \\
\tilde{\tau}((\alpha,\chi),(r,s)) 
 & = &  ((\alpha,r),(\chi,(r \bt \chi)^{-1}s)). 
\end{eqnarray*}

\noindent
Applying these formulae, using $r \bt \chi = \chi r$, we obtain
\begin{eqnarray*}
g 
    &=& ((\alpha,r),(\chi,s)), \\
\tau g 
    &=& ((\alpha,\chi),(r,(\chi r)s)), \\
\ddtto g ~=~ 
  \tilde{t}_1 \tau g 
    &=& (\alpha,\chi), \\
\tilde{e}_1 \tilde{t}_1 \tau g 
    &=& ((\alpha,\chi),(1,1)), \\
\ddeeo\ddtto g ~=~
  \tilde{\tau} \tilde{e}_1 \tilde{t}_1 \tau g 
    &=& ((\alpha,1),(\chi,1)), \\
\ddhho g ~=~ 
  \tilde{h}_1 \tau g 
    &=& (\alpha,\chi)(\iota r, \eta((\chi r)s))  \\
    &=& (\alpha,\chi)(\beta_r, \eta_{\chi r} \star \eta_s) \\
    &=& (\alpha * \beta_r, \chi^{\beta_r} \star \eta_{\chi r} \star \eta_s),\\
\tilde{e}_1 \tilde{h}_1 \tau g 
    &=& ((\alpha*\beta_r,\chi^{\beta_r}\star\eta_{\chi r}\star\eta_s),(1,1)),\\
\ddeeo\ddhho g ~=~ 
\tilde{\tau} \tilde{e}_1 \tilde{h}_1 \tau g 
    &=& ((\alpha*\beta_r,1),(\chi^{\beta_r}\star\eta_{\chi r}\star\eta_s,1)).\\
\end{eqnarray*}

\noindent
Thus the images of the various maps in the cat$^2$-group are given by
\begin{eqnarray*}
\ddtto((\alpha,r),(\chi,s)) 
 & = &  (\alpha,\chi), \\
\ddhho((\alpha,r),(\chi,s)) 
 & = &  (\alpha * \beta_r,~ \chi^{\beta_r}\star\eta_{\chi r}\star\eta_s), \\
\ddttt((\alpha,r),(\chi,s)) 
 & = &  (\alpha,r), \\
\ddhht((\alpha,r),(\chi,s)) 
 & = &  (\alpha * \beta_{\chi},~ r(\partial\chi r)(\partial s)), \\
\dtto(\alpha,r)
 & = &  \alpha, \\
\dhho(\alpha,r)
 & = &  \alpha * \beta_r, \\
\dttt(\alpha,\chi) 
 & = &  \alpha, \\ 
\dhht(\alpha,\chi) 
 & = &  \alpha * \beta_{\chi}. 
\end{eqnarray*}

\medskip\noindent
{\bf From crossed square to group-double groupoid:}

Here is the same calculation, but starting with a general crossed square 
rather than an actor crossed square.

$$
\xymatrix @C=3.0pc @R=5.0pc{
  & (P \ltimes M)\ltimes(N \ltimes L) 
    \ar[rr]<0.5ex>^{\tau} 
    \ar[dl]<0.2ex> \ar[dl]<+0.8ex>^{\ddttt,\ddhht}
    \ar[dr]<-0.2ex>\ar[dr]<-1.0ex>_{\ddtto,~\ddhho}
    & & (P \ltimes N)\ltimes(M \ltimes L)  
        \ar[ll]<0.5ex>^{\tilde{\tau}} 
        \ar[dl]<0.2ex> \ar[dl]<+0.8ex>^{\tilde{t}_1,~\tilde{h}_1} \\
P \ltimes M \ar[ur]<1.0ex>^{\ddeet}
\ar[dr]<-0.2ex>\ar[dr]<-1.0ex>_{\dtto,~\dhho}
      & & P \ltimes N 
      \ar[ul]<-1.0ex>_{\ddeeo} \ar[ur]<1.0ex>^{\tilde{e}_1}
      \ar[dl]<0.2ex> \ar[dl]<+0.8ex>^{\dttt,\dhht} 
        & \\
  & P \ar[ur]<1.0ex>^{\deet} \ar[ul]<-1.0ex>_{\deeo} \\
}
$$

\bigskip\noindent
Again we evaluate some images. 
The formulae for $\ddtto,\ddhho$ are obtained by first applying 
$\tilde{\tau}$ and then using formulae equivalent to $\ddttt,\ddhht$.
\begin{eqnarray*}
g &=& ((p,m),(n,\ell)), \\
\ddttt g
 &=&  (p,m), \\
\ddhht g
 &=&  (p,m)(\dbdyt n, \ddbdyt\ell) \\
 &=& (p(\dbdyt n), m^n (\ddbdyt\ell)), \hspace*{37mm}
        \mbox{{\bf (**)}} \\
\tau g
 &=&  ((p,n),(m,(m \bt n)\ell)), \\
\ddtto g ~=~ \tilde{t}_1 \tau g 
 &=&  (p,n), \\
\ddhho g ~=~ \tilde{h}_1 \tau g
 &=&  (p,n)(\dbdyo m, \ddbdyo((m \bt n)\ell)) \\
 &=& (p(\dbdyo m), n^m((n^{-1})^mn)(\ddbdyo\ell)) \\
 &=& (p(\dbdyo m), n(\ddbdyo\ell)) \hspace*{40mm} 
        \mbox{{\bf (***)}} \\
\dtto(p,m) 
 &=& p, \\
\dhho(p,m) 
 &=&  p(\dbdyo m), \\
\dttt(p,n) 
 &=&  p,\\
\dhht(p,n) 
 &=&  p(\dbdyt n). 
\end{eqnarray*}
{\bf [Note the minor difference betwen (**) and (***),
which will be very important!]}

\newpage
We now check the axioms  in (\ref{eq:cat-commute}) for a cat$^2$-group, 
where we put $g = ((p,m),(n,\ell))$, 
\begin{eqnarray*}
\dtto\ddhht g
 &=& \dtto(p(\dbdyt n), m^n(\ddbdyt\ell)) 
 ~=~ p(\dbdyt n) 
 ~=~ \dhht(p,n)
 ~=~ \dhht\ddtto g, \\
\dttt\ddhho g
 &=& \dttt(p(\dbdyo m), n(\ddbdyo\ell)) 
 ~=~ p(\dbdyo m)  
 ~=~ \dhho(p,m)
 ~=~ \dhho\ddttt g, \\
\ddttt\ddeeo (p,n) 
 &=& \ddttt((p,1),(n,1)) 
 ~=~ (p,1) 
 ~=~ \deeo p 
 ~=~ \deeo\dttt (p,n), \\
\ddtto\ddeet (p,m) 
 &=& \ddtto((p,m),(1,1)) 
 ~=~ (p,1) 
 ~=~ \deet p 
 ~=~ \deet\dtto (p,m), \\
\ddhht\ddeeo (p,n) 
 &=& \ddhht((p,1),(n,1)) 
 ~=~ (p(\dbdyt n),1) 
 ~=~ \deeo(p(\dbdyt n)) 
 ~=~ \deeo\dhht (p,n), \\
\ddhho\ddeet (p,m) 
 &=& \ddhho((p,m),(1,1)) 
 ~=~ (p(\dbdyo m),1) 
 ~=~ \deet(p(\dbdyo m)) 
 ~=~ \deet\dhho (p,m). \\
\dttt\ddtto g 
 &=& \dttt(p,n)
 ~=~ p 
 ~=~ \dtto(p,m) 
 ~=~ \dtto\ddttt g, \\ 
\dhho\ddhht g
 &=& \dhho (p(\dbdyt n), m^n(\ddbdyt\ell)) 
 ~=~ p(\dbdyt n)(\dbdyo m^n)(\barbdy\ell) 
 ~=~ p(\dbdyo m)(\dbdyt n)(\barbdy\ell) \hspace*{12mm}\mbox{{\bf (****)}} \\
 &=& \dhht(p(\dbdyo m), n(\ddbdyo\ell)) 
 ~=~\dhht\ddhho g, \\
\ddeeo\deet p 
 &=& \ddeeo(p,1) 
 ~=~ ((p,1),(1,1)) 
 ~=~ \ddeet(p,1) 
 ~=~ \ddeet\deeo p, \\
\end{eqnarray*}

\vspace*{-6mm}\noindent
{\bf [Note that the third equality in (****) is the crucial step!]}

\medskip\noindent
Vertical composition is easy to evaluate.
\begin{lem}
If we let  $(p',m') = (p(\dbdyt n),m^n(\ddbdyt\ell))$  
then a vertical composite square evaluates as 
$$
((p,m),(n,\ell)) \ddastt ((p',m'),(n',\ell'))
~=~
((p,m),(nn',\ell^{n'}\ell')).
$$
\end{lem}
\begin{pf}
Applying formula (\ref{eq:gpgpd-comp}) for the composite,
\begin{eqnarray*}
 & &  ((p,m),(n,\ell)) \ddasto ((p',m'),(n',\ell')) \\
 &=&  ((p,m),(n,\ell))((p',m')^{-1},(1,1))((p',m'),(n',\ell')) \\
 &=&  ((p,m),(n,\ell)(n',\ell')).
\end{eqnarray*}
\end{pf}

\noindent
Horizontal composition is more complicated.
\begin{lem}
If we let  
$(p^{\prime\prime},n^{\prime\prime}) = (p(\dbdyo m),n(\ddbdyo\ell))$ 
then a horizontal composite square evaluates as 
$$
((p,m),(n,\ell)) \ddasto ((p^{\prime\prime},m^{\prime\prime}),
                          (n^{\prime\prime},\ell^{\prime\prime}))
~=~
((p,mm^{\prime\prime}),(n,\ell\ell^{\prime\prime})).
$$
\end{lem}
\begin{pf}
Applying formula (\ref{eq:gpgpd-comp}) again,
\begin{eqnarray*}
 & &  ((p,m),(n,\ell))~ 
       (({p^{\prime\prime}}^{-1},1),
        (({n^{\prime\prime}}^{-1})^{{p^{\prime\prime}}^{-1}},1))~ 
           ((p^{\prime\prime},m^{\prime\prime}),
             (n^{\prime\prime},\ell^{\prime\prime})) \\
 &=&  ((p,m)(1,m^{\prime\prime}),\; 
      (n,\ell)^{(1,m^{\prime\prime})}
      (({n^{\prime\prime}}^{-1})^{{p^{\prime\prime}}^{-1}},1)
         ^{(p^{\prime\prime},m^{\prime\prime})}
      (n^{\prime\prime},\ell^{\prime\prime})) \\
 &=&  ((p,mm^{\prime\prime}),
       (n,(m^{\prime\prime} \bt n)^{-1}\ell^{m^{\prime\prime}})
       ({n^{\prime\prime}}^{-1},
         ({m^{\prime\prime}} \bt {n^{\prime\prime}}^{-1})^{-1})  
         (n^{\prime\prime},\ell^{\prime\prime})) \\
 &=&  ((p,mm^{\prime\prime}),
       (n,(m^{\prime\prime} \bt n)^{-1}\ell^{m^{\prime\prime}} 
     (({m^{\prime\prime}} \bt {n^{\prime\prime}}^{-1})^{-1})^{n^{\prime\prime}}
          \ell^{\prime\prime})) \\
 &=&  ((p,mm^{\prime\prime}),
       (n,(m^{\prime\prime} \bt n)^{-1}\ell^{m^{\prime\prime}} 
         (m^{\prime\prime} \bt n(\ddbdyo\ell))\ell^{\prime\prime})) 
      \hspace{25mm}\mbox{(by Proposition \ref{prop:xpair}(b))}     \\
 &=&  ((p,mm^{\prime\prime}),
       (n,(m^{\prime\prime} \bt n)^{-1}\ell^{m^{\prime\prime}} 
         (m^{\prime\prime} \bt \ddbdyo\ell) 
         (m^{\prime\prime} \bt n)^{\ddbdyo\ell}\ell^{\prime\prime})) \\
 &=&  ((p,mm^{\prime\prime}),
       (n,(m^{\prime\prime} \bt n)^{-1}\ell^{m^{\prime\prime}} 
         ((\ell^{-1})^{m^{\prime\prime}}\ell) 
         (\ell^{-1}(m^{\prime\prime} \bt n)\ell)  
          \ell^{\prime\prime})) \\
 &=&  ((p,mm^{\prime\prime}), (n,\ell\ell^{\prime\prime}). 
\end{eqnarray*}
\end{pf}\\
{\bf [The difference between these two composites should be very significant!]}

\bigskip
Now consider the two composites of four squares with elements as shown: 
$$
\xymatrix  @C=0.8pc @R=0.8pc{
\cdot \ar[rr]^{(p_1,m_1)} \ar[dd]_{(p_1,n_1)} 
  &   & \cdot \ar[rr]^{(p_2,m_2)} \ar[dd] 
          &   & \cdot \ar[dd]^{(p_2(\dbdyo m_2),n_2(\ddbdyo\ell_2))}  \\
  & ((p_1,m_1),(n_1,\ell_1)) 
      &   & ((p_2,m_2),(n_2,\ell_2)) 
              &  \\
\cdot \ar[rr] \ar[dd]_{(p_3,n_3)}  
  &   & \cdot \ar[rr] \ar[dd]
          &   & \cdot \ar[dd]^{(p_4(\dbdyo m_4),n_4(\ddbdyo\ell_4))}  \\
  & ((p_3,m_3),(n_3,\ell_3))  
      &   & ((p_4,m_4),(n_4,\ell_4)) 
              &  \\
\cdot \ar[rr]_{(p_3(\dbdyt n_3),m_3^{n_3}(\ddbdyt\ell_3))}  
  &   & \cdot \ar[rr]_{(p_4(\dbdyt n_4),m_4^{n_4}(\ddbdyt\ell_4))}  
          &   & \cdot  \\
}
$$
For the horizontal composites to be defined we require 
$$
p_2 = p_1(\dbdyo m_1), \quad
n_2 = n_1(\ddbdyo\ell_1), \quad
p_4 = p_3(\dbdyo m_3), \quad
n_4 = n_3(\ddbdyo\ell_3).
$$
Similarly, for the vertical composites, we require
$$
p_3 = p_1(\dbdyt n_1), \quad
m_3 = {m_1}^{n_1}(\ddbdyt\ell_1), \quad
p_4 = p_2(\dbdyt n_2), \quad
m_4 = {m_2}^{n_2}(\ddbdyt\ell_2).
$$
The two requirements on $p_4$ are consistent since
\begin{eqnarray*}
p_4 &=&  p_2(\dbdyt n_2)
    ~=~  p_1(\dbdyo m_1)(\dbdyt n_1)(\dbdyt\ddbdyo\ell_1) 
    ~=~  p_1(\dbdyt n_1)(\dbdyo({m_1}^{n_1}))(\dbdyo\ddbdyt\ell_1) 
    ~=~  p_3(\dbdyo m_3). \\
\end{eqnarray*}
Composing horizontally, we obtain 
$$
\xymatrix  @C=0.8pc @R=0.8pc{
\cdot \ar[rr] \ar[dd] 
  &   & \cdot \ar[dd]  \\
  & ((p_1,m_1m_2),(n_1,\ell_1\ell_2)) 
      &  \\
\cdot \ar[rr] \ar[dd] 
  &   & \cdot \ar[dd]  \\
  & ((p_3,m_3m_4),(n_3,\ell_3\ell_4))  
      &  \\
\cdot \ar[rr] 
  &   & \cdot  \\
}
$$
The vertical composite is then
$$
((p_1,m_1m_2),(n_1,\ell_1\ell_2)(n_3,\ell_3\ell_4))
~=~ ((p_1,m_1m_2),(n_1n_3,(\ell_1\ell_2)^{n_3}(\ell_3\ell_4)). 
$$

\noindent
When the four squares are composed vertically then horizontally, 
we obtain
$$
\xymatrix  @C=0.8pc @R=0.8pc{
\cdot \ar[rr] \ar[dd] 
  &   & \cdot \ar[rr] \ar[dd]  
         &   & \ar[dd]  \\
  & ((p_1,m_1),(n_1n_3,{\ell_1}^{n_3}\ell_3)) 
      &  &  ((p_2,m_2),(n_2n_4,{\ell_2}^{n_4}\ell_4)) 
             &  \\
\cdot \ar[rr] 
  &   & \cdot \ar[rr]  
         &   &  \\
}
$$
and the horizontal composite is~  
$$
((p_1,m_1m_2), 
  (n_1n_3,{\ell_1}^{n_3}\ell_3{\ell_2}^{n_3(\ddbdyo\ell_3)}\ell_4))
~=~ 
((p_1,m_1m_2), 
  (n_1n_3,{\ell_1}^{n_3}{\ell_2}^{n_3}\ell_3\ell_4))
$$
as before. 
We have therefore verified the interchange law for this double groupoid.



\newpage
%%%%%%%%%%%%%%%%%%%%%%%%%%%%%%%%%%%%%%%%%%%%%%%%%%%%%%%%%%%%%
\subsection{From cat$^2$-group to group-double groupoid ~(1)}

Recall the cat$^1$-group semidirect factorisation of 
Subsection \ref{subs:cat1} (\ref{eq:cat1-sdp-fact})
$$
(\phi',\id_R) ~:~ \calC = (e;t,h : G \to R) 
                ~\to~ \calC' = (e';t',h' : R \ltimes S \to R),
                \quad g \mapsto (tg,ug),~~ ug = (etg^{-1})g, 
$$
and the properties of the projection $u : G \to \ker t$ 
in Lemma \ref{lem:u-props},
$$
tug = 1_R, \quad
hug = (tg^{-1})(hg), \quad 
uer = 1_G, \quad
u(g_1g_2) = (ug_2)(ug_1)^{g_2}, \quad
(ug)^{-1} = g^{-1}(etg) = (u(g^{-1}))^g.
$$

\noindent
Generalising this isomorphism to cat$^2$-groups, the relevant diagram is: 
$$
\xymatrix @C=5.0pc @R=2.0pc{
G \ar[r]<1.0ex>^{\ddtto,\ddhho} \ar[r]<0.2ex> 
  \ar[dd]<-1.0ex>_{\ddttt,\ddhht} \ar[dd]<-0.2ex> 
  & K \ar[dd]<1.0ex>^{\dttt,\dhht} \ar[dd]<0.2ex> 
      \ar[l]<1.0ex>^{\ddeeo}
    & & (J \ltimes F) 
        \ar[r]<1.0ex>^{{\ddtto}^{\,\prime},{\ddhho}^{\,\prime}} 
        \ar[r]<0.2ex>
        \ar[dd]<-1.0ex>_{{\ddttt}^{\,\prime},{\ddhht}^{\,\prime}} 
        \ar[dd]<-0.2ex>
        & P \ltimes N  \ar[l]<1.0ex>^{{\ddeeo}^{\,\prime}} 
                       \ar[dd]<1.0ex>^{{\dttt}^{\,\prime},{\dhht}^{\,\prime}} 
                       \ar[dd]<0.2ex>\\
  & & \stackrel{\phi}{\longrightarrow} 
      & & \\
J \ar[r]<-1.0ex>_{\dtto,\dhho} \ar[r]<-0.2ex>
  \ar[uu]<-1.0ex>_{\ddeet} 
  & P \ar[l]<-1.0ex>_{\deeo} \ar[uu]<1.0ex>^{\deet} 
    & & P \ltimes M \ar[r]<-1.0ex>_{{\dtto}^{\,\prime},{\dhho}^{\,\prime}} 
                    \ar[r]<-0.2ex>
                    \ar[uu]<-1.0ex>_{{\ddeet}^{\,\prime}}
        & P \ar[l]<-1.0ex>_{{\deeo}^{\,\prime}}  
            \ar[uu]<1.0ex>^{{\deet}^{\,\prime}}
}
$$
where
$$
J \cong P \ltimes M, \quad
K \cong P \ltimes N, \quad
E \cong M \ltimes L, \quad
F \cong N \ltimes L.
$$

\noindent
The components of $\phi$ and the fourth semidirect product isomorphism are: 
\begin{eqnarray*}
\phi_{[2]} &:& G \to K \ltimes E, \quad
  g \mapsto (\ddtto g, (\ddeeo\ddtto g^{-1})g) ~=~ (\ddtto g, \dduuo g),\\
\phi_{\{1\}} &:& J \to P \ltimes M, \quad
  j \mapsto (\dtto j, (\deeo\dtto j^{-1})j) ~=~ (\dtto j, \duuo j),\\
\phi_{\{2\}} &:& K \to P \ltimes N, \quad
  k \mapsto (\dttt k, (\deet\dttt k^{-1})k) ~=~ (\dttt k, \duut k),\\
\phi_{\emptyset} &:& P \to P, \hspace{13mm} p \mapsto p,\\
\phi_{[2']} &:& F \to N \ltimes L, \quad
  f \mapsto (\ddttt f, (\ddeet\ddttt f^{-1})f) ~=~ (\ddttt f, \dduut f),\\
\mbox{or}~~ ? &:& G \to J \ltimes F, \quad
  g \mapsto (\ddttt g, (\ddeet\ddttt g^{-1})g) ~=~ (\ddttt g, \dduut g).
\end{eqnarray*}

\noindent
Recall the following identities from (\ref{eq:cat2gp-ids}), 
$$
\duuo\ddttt = \ddttt\dduuo, \quad
\duuo\ddhht = \ddhht\dduuo, \quad
\dduuo\ddeet = \ddeet\duuo, \quad
\duut\ddtto = \ddtto\dduut, \quad
\duut\ddhho = \ddhho\dduut, \quad
\dduut\ddeeo = \ddeeo\duut. 
$$

\noindent
Combining these isomorphisms (as in Lemma \ref{lem:cat1iso-prime}) we obtain
\begin{eqnarray*}
\phi ~:~ G & \to & (P \ltimes M)\ltimes(N \ltimes L), \\
    g & \mapsto & (~(t_{[2]}g,~ \duut\ddtto g),~
        (\duuo\ddttt g,~ \dduut\dduuo g)~), \\
\mbox{where}\quad \dduut\dduuo g
   & = & (\ddeet\ddttt g^{-1})(e_{[2]}t_{[2]}g)(\ddeeo\ddtto g^{-1})g.
\end{eqnarray*}

\noindent
For the alternative semidirect product decomposition of $G$ we obtain: 
\begin{eqnarray*}
\tilde{\phi} ~:~ G & \to & (P \ltimes N)\ltimes(M \ltimes L), \\
    g & \mapsto & (~(t_{[2]}g,~ \duuo\ddttt g),~
        (\duut\ddtto g,~ \dduuo\dduut g)~).\\ 
\end{eqnarray*}
The difference between the two fourth coordinates 
(corresponding to $m \bt n$ in the crossed square version) 
is given by~ 
$(\dduut\dduuo g)(\dduuo\dduut g)^{-1}$. 
Thus the crossed pairing corresponds to a mapping
\begin{eqnarray} \label{eq:uboxt}
u_{[2]} ~:~ G \to G, \quad g 
  & \mapsto & (\dduut\dduuo g)(\dduuo\dduut g)^{-1} \\
  &=&  (\ddeet\ddttt g^{-1})(e_{[2]}t_{[2]}g)(\ddeeo\ddtto g^{-1})
       (\ddeet\ddttt g)(e_{[2]}t_{[2]}g^{-1})(\ddeeo\ddtto g), \nonumber
\end{eqnarray}
which measures the difference between the two composite projections onto
$\ker\ddtto \cap \ker\ddttt$.

\begin{lem}~  $u_{[2]}g ~\in~ [\ker\ddtto,\ker\ddttt]$.
\end{lem}
\begin{pf}
Put  $y = (e_{[2]}t_{[2]}g^{-1})(\ddeet\ddttt g) \in \ker \ddtto$
and  $z = (e_{[2]}t_{[2]}g^{-1})(\ddeeo\ddtto g) \in \ker \ddttt$.
Then, by (\ref{eq:uboxt}),  
$u_{[2]}g = [y,z] ~\in~ [\ker\ddtto,\ker\ddttt]$.
\end{pf}

\bigskip
We are now ready to define the associated group-double groupoid.
There is a square for each element  $g \in G$, 
and the tail and head cat$^2$ homomorphisms give the two groupoid 
source and target functions.
$$
\xymatrix{
\dtto\ddttt g \ar[rr]^{\ddttt g} \ar[dd]_{\ddtto g} 
  &   & \dtto\ddhht g \ar[dd]^{\ddhho g} \\
  & g & \\
\dhho\ddttt g \ar[rr]_{\ddhht g} 
  &   & \dhho\ddhht g \\
}
$$
Vertical composition is given as in (\ref{eq:gpgpd-comp}),
$$
g_1 *_1 g_3 ~=~ g_1(\ddeet\ddhht{g_1}^{-1})g_3 
\qquad\mbox{when}\quad
\ddhht g_1 = \ddttt g_3,
$$
while horizontal composition first applies $\phi_{M,N}$ 
and then (\ref{eq:gpgpd-comp}),
$$
\ldots~\ldots ~~\mbox{{\bf to be completed}} ~~\ldots~\ldots
$$



\newpage
%%%%%%%%%%%%%%%%%%%%%%%%%%%%%%%%%%%%%%%%%%%%%%%%%%%%%%%%%
\subsection{From cat$^2$-group to group-double groupoid ~(2)}

Given a cat$^2$-group $\calG$,  define a douple groupoid 
$\calD = \calD(\calG)$ to have as squares the elements of $\calG$, 
with the head and tail maps in $\calG$ providing the 
boundary maps in $\calD$.

Consider the following composite square of cat$^1$-group elements
$$
\xymatrix  @C=1.2pc @R=0.8pc{
\cdot \ar[rr] \ar[dd]
  &   & \cdot \ar[rr] \ar[dd]|{\phantom{gZ}w\phantom{gZ}}
          &   & \cdot \ar[dd] 
                 &   &  \\
  & a &   & b &  &   &  w = \ddhho a = \ddtto b,\quad
                        z = \ddhho c = \ddtto d, \hspace{6mm}\\
\cdot \ar[rr]|{~x~} \ar[dd] 
  &   & p \ar[rr]|{~y~} \ar[dd]|{\phantom{gZ}z\phantom{gZ}}
          &   & \cdot \ar[dd] 
                 &   &  x = \ddhht a = \ddttt c,\quad
                        y = \ddhht b = \ddttt d, \hspace{6mm}
\\
  & c &   & d &  &   &  p = \dhho\ddhht a = \dhht\ddtto b 
                          = \dttt\ddhho c = \dtto\ddttt d. \\
\cdot \ar[rr] 
  &   & \cdot \ar[rr] 
          &   & \cdot 
                 &   &  \\
}
$$

\noindent
Define horizontal and vertical composition as in 
$$
a \ddastt b ~=~ a(\ddeeo w^{-1})b, \qquad
a \ddasto c ~=~ a(\ddeet x^{-1})c,
$$
{\bf [N.B.~ Either one of these is incorrect, 
or else some boundary maps should be changed!]}

\bigskip\noindent
Note that squares such as $\ddeeo w$ and $\ddeet x$ are identities for 
the groupoid structures $\ddastt,\ddasto$ respectively:
$$
a \ddastt (\ddeeo w) ~=~ a(\ddeeo w^{-1})(\ddeeo w) ~=~ a, 
\qquad
a \ddasto (\ddeet x) ~=~ a(\ddeet x^{-1})(\ddeet x) ~=~ a, 
$$
while $e_{[2]}p = \deeo\ddeet p$ is an identity for both groupoid structures. 

\bigskip\noindent
Note also that
$$
\ddhht(a \ddastt b) ~=~ (\ddhht a)(\ddhht\ddeeo w^{-1})(\ddhht b) 
                    ~=~ x(\deeo p^{-1})y 
                    ~=~ \ddttt( c \ddastt d).
$$

\bigskip\noindent
Here is an expanded view of the composite square using nine squares: 
$$
\xymatrix  @C=1.0pc @R=1.4pc{
\cdot \ar[rr] \ar[dd]
  &   & \cdot \ar[rr] \ar[dd]|{\phantom{gZ}w\phantom{gZ}}
          &   & \cdot \ar[rr] \ar[dd]|{\phantom{gZ}w\phantom{gZ}} 
                  &   & \cdot \ar[dd] \\
  & a &   & \ddeeo w 
              &   & b &  \\
\cdot \ar[rr]|{~x~} \ar[dd] 
  &   & p \ar[rr]|{~\deeo p~} 
          \ar[dd]|{\phantom{gZ} \deet p \phantom{gZ}}
          &   & p \ar[rr]|{~y~} 
                  \ar[dd]|{\phantom{gZ} \deet p \phantom{gZ}}
                  &   & \cdot \ar[dd] \\
  & \ddeet x 
      &   & e_{[2]} p 
              &   & \ddeet y 
                      &  \\
\cdot \ar[rr]|{~x~} \ar[dd]
  &   & p \ar[rr]|{~\deeo p~} 
          \ar[dd]|{\phantom{gZ}z\phantom{gZ}}
          &   & p \ar[rr]|{~y~} \ar[dd]|{\phantom{gZ}z\phantom{gZ}}
                  &   & \cdot \ar[dd] \\
  & c &   & \ddeeo z 
              &   & d &  \\
\cdot \ar[rr] 
  &   & \cdot \ar[rr] 
          &   & \cdot \ar[rr]
                  &   &  \cdot \\
}
$$

\bigskip\noindent
Composing in two ways, we obtain
\begin{eqnarray*}
(a \ddastt b) \ddasto (c \ddastt d) 
 &=& (a(\ddeeo w^{-1})b) \ddasto (c(\ddeeo z^{-1})d) \\
 &=& (a(\ddeeo w^{-1})b) 
        ~(\ddeet (y^{-1}(\deeo p)x^{-1})) 
           ~(c(\ddeeo z^{-1})d) \\
 &=& a(\ddeeo w^{-1})b
         (\ddeet y^{-1})(e_{[2]} p)(\ddeet x^{-1})
              c(\ddeeo z^{-1})d, \\
(a \ddasto c) \ddastt (b \ddasto d) 
 &=& (a(\ddeet x^{-1})c) \ddastt (b(\ddeet y^{-1})d) \\
 &=& (a(\ddeet x^{-1})c) 
        ~(\ddeeo (z^{-1}(\deet p)w^{-1})) 
           ~(b(\ddeet y^{-1})d) \\
 &=& a(\ddeet x^{-1})c
         (\ddeeo z^{-1})(e_{[2]} p)(\ddeeo w^{-1})
              b(\ddeet y^{-1})d, \\
\end{eqnarray*}

\noindent
For these to be equal we require
$$
a(\ddeeo w^{-1})b
         (\ddeet y^{-1})(e_{[2]} p)(\ddeet x^{-1})
              c(\ddeeo z^{-1})d 
\quad=\quad
a(\ddeet x^{-1})c
         (\ddeeo z^{-1})(e_{[2]} p)(\ddeeo w^{-1})
              b(\ddeet y^{-1})d 
$$
which we may rewrite as
$$
a(\dduuo b)(\dduuo\ddeet y)^{-1}(\dduut c)(\dduuo d)
\quad=\quad
a(\dduut c)(\dduut\ddeeo z)^{-1}(\dduuo b)(\dduut d).
$$

\bigskip\noindent
This equation does not appear to be satisfied, 
so the interchange law is not satisfied.
On the other hand, the group-double groupoid obtained earlier from 
a crossed square does satisfy the interchange law.

\bigskip\noindent
{\bf [This contradiction still needs to be sorted out!]}







%%%%%%%%%%%%%%%%%%%%%%%%%%%%%%%%%%%%%%%%%%%%%%%%%%%%%%%%%%%%%%%
%\subsection{The double groupoid associated to a crossed module}

