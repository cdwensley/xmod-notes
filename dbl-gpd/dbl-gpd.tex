%% dbl-gpd.tex,  version 22/04/17

%%%%%%%%%%%%%%%%%%%%%%%%%%%%%%%%%%%%%%%%%%%%%%%%%%%%%%%%%%%%%%%%%%%%
\section{Double Categories and Double Groupoids} \label{sect:dblgpd}

Our interest here is in double groupoids and their connection with 
crossed squares and cat$^2$-groups.

%%%%%%%%%%%%%%%%%%%%%%%%%%%%%%%%%%%%%%%%%%%%%%%%%%
\subsection{Double Categories} \label{subs:dblcat}

A \emph{double category} $\calD$ consists of four sets and 
four category structures, and satisfies an interchange law:
\begin{itemize}
\item~ $D_{[2]}$, a set of \emph{squares},
\item~ $D_{\{2\}}$, a set of \emph{vertical} 
or \emph{up-down arrows},
\item~ $D_{\{1\}}$, a set of \emph{horizontal} 
or \emph{left-right arrows},
\item~ $D_{\emptyset}$, a set of \emph{objects}, 
\item~ maps $\ddtto,\ddhho : D_{[2]} \to D_{\{2\}}$ 
       and $\ddeeo : D_{\{2\}} \to D_{[2]}$ 
       and a composition $\ddasto$ giving a category structure 
       $\ddcalDo$ on squares displayed horizontally, 
\item~ maps $\ddttt,\ddhht : D_{[2]} \to D_{\{1\}}$
       and $\ddeet : D_{\{1\}} \to D_{[2]}$ 
       and a composition $\ddastt$ giving a category structure 
       $\ddcalDt$ on squares displayed vertically, 
\item~ maps $\dtto,\dhho : D_{\{1\}} \to D_{\emptyset}$, 
       and $\deeo : D_{\emptyset} \to D_{\{1\}}$ 
       and a composition $\dasto$ 
       giving a category structure $\dcalDo$ on horizontal arrows, 
\item~ maps $\dttt,\dhht : D_{\{2\}} \to D_{\emptyset}$, 
       and $\deet : D_{\emptyset} \to D_{\{2\}}$ 
       and a composition $\dastt$ 
       giving a category structure $\dcalDt$ on vertical arrows, 
\item~ the tail and head maps commute as follows:
$$
\dttt\ddtto = \dtto\ddttt, \quad
\dttt\ddhho = \dhho\ddttt, \quad
\dhht\ddtto = \dtto\ddhht, \quad
\dhht\ddhho = \dhho\ddhht,
$$
\item~ for all squares $A,B,C,D$ such that the compositions are defined,
$$
\xymatrix  @C=0.8pc @R=0.8pc{
\cdot \ar[rr] \ar[dd] 
  &   & \cdot \ar[rr] \ar[dd]
          &   & \cdot \ar[dd] 
                 &   &  \\
  & A &   & B &  &   &  \\
\cdot \ar[rr] \ar[dd] 
  &   & \cdot \ar[rr] \ar[dd]
          &   & \cdot \ar[dd] 
                 &   &  (A \ddasto B) \ddastt (C \ddasto D) ~=~
                        (A \ddastt C) \ddasto (B \ddastt D). \\
  & C &   & D &  &   &  \\
\cdot \ar[rr] 
  &   & \cdot \ar[rr] 
          &   & \cdot 
                 &   &  \\
}
$$
\end{itemize}

For a square $A$ the four arrows and the four objects are displayed as follows.
$$
\xymatrix{
\dtto\ddttt A \ar[rr]^{\ddttt A} \ar[dd]_{\ddtto A} 
  &   & \dhho\ddttt A \ar[dd]^{\ddhho A} \\
  & A & \\
\dtto\ddhht A \ar[rr]_{\ddhht A} 
  &   & \dhho\ddhht A \\
}
$$

\medskip
The structure of a double category is shown in the following diagram: 
\begin{equation} \label{eq:dblcat}
\vcenter{\xymatrix @C=2pc @R=2pc{
 & D_{[2]} \ar[dd] <-1.2ex>  \ar[dd] <-2.0ex>_{\ddttt,\ddhht}
     \ar[rr] <+1.2ex>  \ar[rr] <+2.0ex>^{\ddtto,\ddhho}
    &&  D_{\{2\}}  \ar[ll]^{\ddeeo}
            \ar[dd]<+1.2ex>  \ar[dd] <+2.0ex>^{\dttt,\dhht}  \\
\calD \quad = \quad
 &  &&   \\
 & D_{\{1\}} \ar[uu]_{\ddeet}
     \ar[rr] <-1.2ex>  \ar[rr] <-2.0ex>_{\dtto,\dhho} 
    &&  D_{\emptyset} \ar[uu]^{\deet}   \ar[ll]_{\deeo} 
 \\
}} 
\end{equation}

Horizontal, vertical and double identity squares 
($e_{[2]} = \ddeet\deeo = \ddeeo\deet$) are represented by 
$$
\xymatrix @C=1.5pc @R=1.5pc{
\cdot \ar[rr]^{\deeo\dtto a} \ar[dd]_{a}
  &  &  \cdot \ar[dd]^{a} \\
  &  \ddeeo a
     & \\
\cdot \ar[rr]_{\deeo \dhho a}
  &  &  \cdot
}
\qquad\qquad
\xymatrix @C=1.5pc @R=1.5pc{
\cdot \ar[dd]_{\deet\dttt c} \ar[rr]^{c}
  &  &  \cdot \ar[dd]^{\deet \dhht c} \\
  &  \ddeet c
     & \\
\cdot \ar[rr]_{c}
  &  &  \cdot
}
\qquad\qquad
\xymatrix @C=1.3pc @R=1.5pc{
r \ar[dd]_{\deet r} \ar[rr]^{\deeo r}
  &  &  r \ar[dd]^{\deet r} \\
  &  e_{[2]} r
     & \\
r \ar[rr]_{\deeo r}
  &  &  r
}
$$


\begin{defn} \index{morphism!of double categories} 
A morphism of double categories is a \emph{double functor}
$\lambda : \calD \to \calE$,  given by a quadruple 
$\lambda=(\lambda_{[2]},\lambda_{\{2\}},\lambda_{\{1\}},\lambda_{\emptyset})$ 
of maps 
$$
      \lambda_{[2]} : D_{[2]} \to E_{[2]},\quad
    \lambda_{\{2\}} : D_{\{2\}} \to E_{\{2\}},\quad
    \lambda_{\{1\}} : D_{\{1\}} \to E_{\{1\}},\quad
\lambda_{\emptyset} : D_{\emptyset} \to E_{\emptyset},
$$
which combine to give functors from 
$\ddcalDo,\ddcalDt,\dcalDo,\dcalDt$ to
$\ddcalEo,\ddcalEt,\dcalEo,\dcalEt$ respectively, 
and which preserve all compositions. 
\end{defn}

\medskip\noindent
{\bf [Expand on this?]}


%%%%%%%%%%%%%%%%%%%%%%%%%%%%%%%%%%%%%%%%%%%%%%%%%%%%%%%%%%%
\subsection{Double Groupoids and Group -- Double Groupoids} 
\label{subs:dblgpd}

A \emph{double groupoid} is a double category in which the four 
category structures are all groupoids.\\
When the two sets of arrows coincide, $D_{\{1\}}=D_{\{2\}}$, 
we have an \emph{arrow-symmetric double groupoid}.
These are the subject of Chapter 6 of Brown and Sivera \cite{brow:siv}.

\medskip
By analogy with the equivalence between cat$^1$-groups and group-groupoids, 
we attempt to describe an equivalence between cat$^2$-groups and
group-double groupoids.

\begin{defn} \index{group-double groupoid}
A \emph{group-double groupoid} is a double groupoid $\calD$ 
where each of 
$\calD_{[2]}, \calD_{\{2\}}, \calD_{\{1\}}, \calD_{\emptyset}$ are groups 
and the twelve structure maps in (\ref{eq:dblcat}) are group homomorphisms, 
so that the four category structures 
$\ddcalDo, \dcalDo, \ddcalDt, \dcalDt$ are all group-groupoids.
\end{defn}

\bigskip\noindent
{\bf [Is the diagonal a group-groupoid?]}



%\newpage
%%%%%%%%%%%%%%%%%%%%%%%%%%%%%%%%%%%%%%%%%%%%%%%%%%%%%%%%%%%%%%%%%%%%%%%%
\subsection{Horizontal, Vertical and Double Sections} \label{subs:dsect}

We saw in Subsections \ref{subs:gpgpd} and \ref{subs:gpd-sect} 
that a section of a group-groupoid is a group monomorphism 
$\xi : G_0 \to G_1$ such that $t\xi = 1_{G_0}$.
In order to generalise this to a horizontal section of a 
group-double groupoid $\calD$, we require compatible monomorphisms 
from up-down arrows to squares and from points to left-right arrows.
In order to see what `compatible' means in this context 
we note that, in the following diagrams, the left-right boundaries of
$\ddxo g$ should be the images of the points of $g$ under $\dxo$, 
while $\ddxo\deet r$ should be the vertical identity square for $\dxo r$.
We show up-down arrows as dashed in the diagrams in this Subsection.

$$
\xymatrix @C=3.0pc @R=1.5pc{
\dttt g \ar@{-->}[dd]_{g} \ar[rr]^{\dxo\dttt g = \ddttt\ddxo g}
  &  &  \dttt\ddhho\ddxo g \ar@{-->}[dd]^{\ddhho\ddxo g} \\
  &  \ddxo g
     & \\
\dhht g \ar[rr]_{\dxo\dhht g = \ddhht\ddxo g}
  &  &  \dhht\ddhho\ddxo g
}
\qquad\qquad
\xymatrix @C=1.5pc @R=1.5pc{
r \ar@{-->}[dd]_{\deet r} \ar[rr]^{\dxo r}
  &  &  \dhho\dxo r \ar@{-->}[dd]^{\ddhho\ddxo\deet r} \\
  &  \ddxo\deet r = \ddeet\dxo r
     & \\
r \ar[rr]_{\dxo r} 
  &  &  \dhho\dxo r
}
$$

\begin{defn} \index{section!horizontal} \index{horizontal section} \mbox{}\\
{\rm (a)}~
A \emph{horizontal section} of a group-double groupoid $\calD$ 
is a pair $\xi_1 = (\ddxo,\dxo)$ of monomorphisms 
$\ddxo : D_{1} \to D_{[2]}$ and $\dxo : D_{\emptyset} \to D_{\{2\}}$ 
such that 
$$
\ddtto\ddxo = 1_{D_{\{1\}}}, \quad
\dtto\dxo = 1_{D_{\emptyset}}, \quad
\ddttt\ddxo g = \dxo\dttt g, \quad
\ddhht\ddxo g = \dxo\dhht g, \quad
\ddxo\deet r = \ddeet\dxo r.
$$
{\rm (b)}~
A \emph{vertical section} of a group-double groupoid $\calD$ 
is a pair $\xi_2 = (\ddxt,\dxt)$ of monomorphisms 
$\ddxt : D_{2} \to D_{[2]}$ and $\dxt : D_{\emptyset} \to D_{\{1\}}$ 
such that 
$$
\ddttt\ddxt = 1_{D_{\{2\}}}, \quad
\dttt\dxt = 1_{D_{\emptyset}}, \quad
\ddtto\ddxt g = \dxt\dtto g, \quad
\ddhho\ddxt g = \dxt\dhho g, \quad
\ddxt\deeo r = \ddeeo\dxt r.
$$
{\bf [Maybe we should swap $1$ and $2$ in $\xi_1, \xi_2$~?]}
\end{defn}

\bigskip
Given a section $\xi_1 = (\ddxo,\dxo)$ of $\ddcalDo$ 
we may apply the construction in Subsection \ref{subs:gpd-sect} 
to obtain a groupoid automorphism $\lambda_1$ of $\ddcalDo$, 
which extends to a double groupoid automorphisms 
$\lambda_1 : \calD \to \calD$.

$$
\xymatrix @C=0.5pc @R=2pc{
\dhho\dxo\dtto\ddttt A 
\ar[rr]^{\dlo\ddttt A} 
\ar@{-->}[dd]_{\ddhho\ddxo\ddtto A}
 & & \dhho\dxo\dhho\ddttt A 
     \ar@{-->}[dd]^{\ddhho\ddxo\ddhho A} 
     & & & & \dhho\dxo\dtto\ddttt A 
             \ar@{-->}[dd]_{\ddhho\ddxo\ddtto A} 
             & & \dtto\ddttt A 
                 \ar[ll]_(0.4){\dxo\dtto\ddttt A}
                 \ar[rr]^{\ddttt A} 
                 \ar@{-->}[dd]^{\ddtto A} 
                 & & \dhho\ddttt A 
                     \ar@{-->}[dd]_{\ddhho A} 
                     \ar[rr]^(0.4){\dxo\dhho\ddttt A}
                     & & \dhho\dxo\dhho\ddttt A 
                         \ar@{-->}[dd]^{\ddhho\ddxo\ddhho A} \\
 & \ddlo A 
   & & & = 
         & & & \ddxo\ddtto A 
               & & A 
                   & & \ddxo\ddhho A
                       & \\ 
\dhho\dxo\dtto\ddhht A 
\ar[rr]_{\dlo\ddhht A}
 & & \dhho\dxo\dhho\ddhht A 
     & & & & \dhho\dxo\dtto\ddhht A 
             & & \dtto\ddhht A 
                 \ar[ll]^(0.4){\dxo\dtto\ddhht A} 
                 \ar[rr]_{\ddhht A}
                 & & \dhho\ddhht A 
                     \ar[rr]_(0.4){\dxo\dhho\ddhht A} 
                     & & \dhho\dxo\dhho\ddhht A \\
}
$$
The defining equations for $\lambda_1$ are:
\begin{eqnarray} \label{eq:lambdao}
(\lambda_1)_{[2]}\,A 
  &=& (\widetilde{\ddxo\ddtto A}) \ddasto A \ddasto (\ddxo\ddhho A)\,,\\
(\lambda_1)_{\{2\}}\,a 
  &=& \ddhho\ddxo\,a\,, \nonumber \\
(\lambda_1)_{\{1\}}\,c 
  &=& (\widetilde{\dxo\dtto c}) \dasto c \dasto (\dxo\dhho c), \nonumber \\
(\lambda_1)_{\emptyset}\,p 
  &=& \dhho\dxo\,p\,. \nonumber
\end{eqnarray}

\medskip\noindent
Similarly, given a section $\xi_2 = (\ddxt,\dxt)$ of $\ddcalDt$, 
applying the same construction we obtain
$$
\xymatrix @C=0.8pc @R=2pc{
 & & & & & & \dhht\dxt\dtto\ddttt A 
             \ar[rr]^{\ddhht\ddxt\ddttt A} 
             & & \dhht\dxt\dhho\ddttt A \\
 & & & & & & & \ddxt\ddttt A 
               & \\
\dhht\dxt\dtto\ddttt A 
\ar[rr]^{\ddhht\ddxt\ddttt A} 
\ar@{-->}[dd]_{\dlt\ddtto A} 
 & & \dhht\dxt\dhho\ddttt A 
     \ar@{-->}[dd]^{\dlt\ddhho A}
     & & & & \dtto\ddttt A 
             \ar[rr]_{\ddttt A} 
             \ar@{-->}[dd]_{\ddtto A} 
             \ar@{-->}[uu]^{\dxt\dtto\ddttt A} 
             & & \dhho\ddttt A 
                 \ar@{-->}[dd]^{\ddhho A} 
                 \ar@{-->}[uu]_{\dxt\dhho\ddttt A}\\
 & \ddlt A 
   & & & = 
         & & & A 
               & \\ 
\dhht\dxt\dtto\ddhht A 
\ar[rr]_{\ddhht\ddxt\ddhht A} 
 & & \dhht\dxt\dhho\ddhht A 
     & & & & \dtto\ddhht A 
             \ar[rr]^{\ddhht A} 
             \ar@{-->}[dd]_{\dxt\dtto\ddhht A} 
             & & \dhho\ddhht A 
                 \ar@{-->}[dd]^{\dxt\dhho\ddhht A}\\
 & & & & & & & \ddxt\ddhht A 
               & \\
 & & 
     & & & & \dhht\dxt\dtto\ddhht A 
             \ar[rr]_{\ddhht\ddxt\ddhht A} 
             & & \dhht\dxt\dhho\ddhht A \\
}
$$
This determines a double groupoid automorphism $\lambda_2$ of $\calD$ where
\begin{eqnarray} \label{eq:lambdat}
(\lambda_2)_{[2]}\,A 
  &=& (\widetilde{\ddxt\ddttt A}) \ddastt A \ddastt (\ddxt\ddhht A)\,,\\
(\lambda_2)_{\{2\}}\,a 
  &=& (\widetilde{\dxt\dttt a}) \dastt a \dastt (\dxt\dhht a)\,, \nonumber \\
(\lambda_2)_{\{1\}}\,c 
  &=& \ddhht\ddxt\,c\,, \nonumber \\
(\lambda_2)_{\emptyset}\,p 
  &=& \dhht\dxt\,p\,. \nonumber
\end{eqnarray}

\bigskip
Given a pair of sections $\xi = (\xi_1,\xi_2)$, 
one horizontal and one vertical, 
if we apply both constructions we obtain an automorphism 
$\lambda = \lambda_{\xi}$ of $\calD$ 
provided $\lambda_1 A = \lambda_2 A$.
We call $\xi$ a \emph{double section} of $\calD$.
The requirement  $\lambda_1 A = \lambda_2 A$  implies
four equations at the four levels of $\calD$, 
\begin{eqnarray*}
\dhho\dxo\,p
  &=&  \dhht\dxt\,p\,, \\
\ddhho\ddxo\,a
  &=&  (\widetilde{\dxt\dttt a}) \dastt a \dastt (\dxt\dhht a)
      ~=~ \dlt\,a\,, \\
\dlo\,c ~=~  (\widetilde{\dxo\dtto c}) \dasto c \dasto (\dxo\dhho c)
  &=&  \ddhht\ddxt\,c\,, \\
\ddlo\,A ~=~ (\widetilde{\ddxo\ddtto A}) \ddasto A \ddasto (\ddxo\ddhho A) 
  &=&  (\widetilde{\ddxt\ddttt A}) \ddastt A \ddastt (\ddxt\ddhht A)
       ~=~ \ddlt\,A\,.
\end{eqnarray*}

\bigskip
The previous two diagrams may be combined into the cube-like figure 
of equation (\ref{eq:cube-like}), where the six square faces are denoted 
$F$=front, $L$=left, $R$=right, $U$=up, $D$=down and $B$=back, and
$$
F = A, \quad 
L = \ddxo \ddtto A, \quad 
R = \ddxo \ddhho A, \quad
U = \ddxt \ddttt A, \quad 
D = \ddxt \ddhht A, \quad
B = \lambda A. 
$$


\begin{equation} \label{eq:cube-like}
\vcenter{\xymatrix  @C=1.3pc %%% @C = 0.5pc @R = 3pc 
{
  && && \dhho\dxo\dtto\ddttt A 
        \ar[rrrrrr]^{\ddhht\ddxt\ddttt A ~=~ \dlo\ddttt A}
        \ar@{-->}'[dddd][dddddd]^(-0.7){\ddhho\ddxo\ddtto A}
            ^(-0.5){~=~} ^(-0.35){\dlt\ddtto A}
        && && && \dhho\dxo\dhho\ddttt A 
                 \ar@{-->}[dddddd]^(0.38){\ddhho\ddxo\ddhho A}
                   ^(0.45){~=~}^(0.5){\dlt\ddhho A} \\ %1
  && && && && && \\ %2
  && && && && && \\ %3
  && && && && && \\ %4
\dtto\ddttt A 
\ar[rrrrrr]_{\ddttt A} \ar@{-->}[dddddd]_{\ddtto A} 
\ar@/^3ex/[rrrruuuu]^(0.45){\dxo\dtto\ddttt A}
\ar@{-->}@/^-3ex/[rrrruuuu]_(0.4){\dxt\dtto\ddttt A} 
  && && && \dhho\ddttt A 
           \ar@{-->}[dddddd]_{\ddhho A} 
           \ar@{-->}@/^3ex/[rrrruuuu]^(0.45){\dxt\dhho\ddttt A}
           \ar@/^-3ex/[rrrruuuu]_(0.45){\dxo\dhho\ddttt A} 
         && && \\ %5
  && && && && && \\ %6
  && && \dhho\dxo\dtto\ddhht A 
        \ar'[rr][rrrrrr]^(0.0){\ddhht\ddxt\ddhht A ~=~ \dlo\ddhht A}
        && && && \dhho\dxo\dhho\ddhht A \\ %7
  && && && && && \\ %8
  && && && && && \\ %9
  && && && && && \\ %10
\dtto\ddhht A 
\ar[rrrrrr]_{\ddhht A} 
\ar@/^3ex/[rrrruuuu]^(0.6){\dxo\dtto\ddhht A}
\ar@{-->}@/^-3ex/[rrrruuuu]_(0.55){\dxt\dtto\ddhht A} 
  && && && \dhho\ddhht A 
           \ar@{-->}@/^3ex/[rrrruuuu]^(0.55){\dxt\dhho\ddhht A}
           \ar@/^-3ex/[rrrruuuu]_(0.55){\dxo\dhho\ddhht A} 
         && && \\ %11
  && && && && && \\ %12 - blank
}}
\end{equation}

