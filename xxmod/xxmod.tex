%% xxmod.tex,  version 22/09/17

%%%%%%%%%%%%%%%%%%%%%%%%%%%%%%%%%%%%%%%%%%%%%%%%%%%%%%%%%%%%%%%%%%%%%%%%%
\section{Crossed squares and $2$-fold crossed modules} \label{sect:xxmod}

We now show how the action of one crossed module on another 
leads to a crossed square.
As a first step, we consider alternative formulations of the
crossed module axioms in order to find one which can most easily
be generalised.

%% \subsection{First alternative formulation} moved to notes9_1_old.tex ??

%%%%%%%%%%%%%%%%%%%%%%%%%%%%%%%%%%%%%%%%%%%%%%%%%%%%%%%%%%%
\subsection{An alternative formulation for crossed modules} 
\label{subs:alt-viewt}

Let $\calX = (\partial : S \to R)$  be a crossed module.
Define maps $\iota_S$ (similarly $\iota_R$) and $\bara$  by
$$
\bari_S : S \times S \to S,~ (s_1,s) \mapsto {s_1}^s = s^{-1}s_1s,
\qquad
\bara : S \times R \to S,~ (s,r) \mapsto s^r.
$$
From these we may define
\begin{eqnarray*}
\iota_s ~=~ \bari_S(-,s) ~:~ S \to S,~ s_1 \mapsto s^{-1}s_1s
& ~\mbox{and}~ &
\iota_S ~:~ S \to \Aut(S),~ s \mapsto \iota_s,  \\
\alpha_r ~=~ \bara(-,r) ~:~ S \to S,~ s \mapsto s^r
& ~\mbox{and}~ &
\alpha ~:~ R \to \Aut(S),~ r \mapsto \alpha_r,
\end{eqnarray*}

\medskip\noindent
The first crossed module axiom says that the following diagram commutes: 
\begin{equation} 
\vcenter{\xymatrix{
S \times R \ar[rr]^(0.4){1_S \times \alpha}
           \ar[dd]_{\partial \times \iota_R}
  && S \times \Aut S \ar[rr]^(0.6){{\rm eval}}
    && S \ar[dd]^{\partial} \\
  && && \\
R \times \Aut R \ar[rrrr]_{{\rm eval}}
  && && R
}} 
\end{equation}
mapping $(s,r)$ to $\alpha_r(\partial s) = r^{-1}(\partial s)r$ 
and to $\partial(s^r)$.

\medskip\noindent
The corresponding diagram for $2$-fold crossed modules, 
where $\calY = (\delta : Q \to P)$, is 
\begin{equation}
\vcenter{\xymatrix{
\calX \times \calY \ar[rr]^(0.4){1_{\calX} \times \alpha}
           \ar[dd]_{\mu \times \iota_{\calY}}
  && \calX \times \Act \calX \ar[rr]^(0.6){{\rm eval}}
    && \calX \ar[dd]^{\mu} \\
  && && \\
\calY \times \Act \calY \ar[rrrr]_{{\rm eval}}
  && && \calY
}} 
\end{equation}

\medskip\noindent
Applying these maps to $s \in S,~ r \in R,~ q \in Q,~ p \in P$ 
clockwise around the rectangle we obtain:
$$
\left( \begin{tabular}{c} $s$ \\ $r$ \\ \end{tabular} \right),
\left( \begin{array}{c} q \\ p \\ \end{array} \right)
\quad\mapsto\quad
\left( \begin{array}{c} s \\ r \\ \end{array} \right),
\left( \begin{array}{c} \chi_q \\ \alpha_p \\ \end{array} \right)
\quad\mapsto\quad
\left( \begin{array}{c} \chi_q r \\ 
         \left( \begin{array}{c} \dda_p s \\ \da_p r \\ \end{array} \right) \\ 
       \end{array} \right),
\left( \begin{array}{c} \dda\ddm s \\ \da\dm r \\ \end{array} \right)
\quad\mapsto\quad
\left( \begin{array}{c} \ddm(\chi_q r) \\ 
        \left( \begin{array}{c} \ddm(s^p) \\ \dm(r^p) \\ \end{array} \right) \\
       \end{array} \right).
$$

\noindent
In the other direction,
$$
\left( \begin{tabular}{c} $s$ \\ $r$ \\ \end{tabular} \right),
\left( \begin{array}{c} q \\ p \\ \end{array} \right)
\quad\mapsto\quad
\left( \begin{array}{c} \ddm s \\ \dm r \\ \end{array} \right),
\left( \begin{array}{c} \eta_{q} \\ \beta_{p} \\ \end{array} \right)
\quad\mapsto\quad
\left( \begin{array}{c} \eta_q(\dm r) \\ 
        \left( \begin{array}{c} \ddb_p(\ddm s) \\ \db_p(\dm r) \\ 
               \end{array} \right) \\
       \end{array} \right). 
$$

\noindent
Hence we obtain
\begin{eqnarray}
\begin{tabular}{rcccl}
$\eta_q(\dm r)$   & $=$ & $(q^{-1})^{\dm r}q$  & $=$ & $\ddm(\chi_q r)$ \\
$\ddb_p (\ddm s)$ & $=$ & $(\ddm s)^p$         & $=$ & $\ddm(s^p)$ \\
$\db_p(\dm r)$    & $=$ & $p^{-1}(\dm r)p$     & $=$ & $\dm(r^p)$ \\
\end{tabular}
\end{eqnarray}

\medskip\noindent
From this we may deduce (for example)
\begin{equation} \label{eq:mu-of-q-on-s}
\ddm(s^q) = \ddm(s^{\delta q}) = (\ddm s)^{\delta q} = q^{-1}(\ddm s)q~.
\end{equation}

\bigskip
Now we have to do a similar development for the second crossed module axiom.
Corresponding to the diagram
\begin{equation}
\vcenter{\xymatrix{
S \times S \ar[rr]^{1_S \times \partial}
           \ar[dd]_{1_S \times \iota_S}
  && S \times R \ar[rr]^{1_S \times \alpha}
    && S \times \Aut S \ar[dd]^{{\rm eval}} \\
  && && \\
S \times \Aut S \ar[rrrr]_{{\rm eval}}
  && && S
}} 
\end{equation}
\medskip\noindent
mapping $(s_0,s)$ to ${s_0}^s$ and to ${s_0}^{\partial s}$, 
the diagram for $2$-fold crossed modules is :
\begin{equation}
\vcenter{\xymatrix{
\calX \times \calX \ar[rr]^{1_{\calX} \times \mu}
           \ar[dd]_{1_{\calX} \times \iota_{\calX}}
  && \calX \times \calY \ar[rr]^{1_{\calX} \times \alpha}
    && \calX \times \Act \calX \ar[dd]^{{\rm eval}} \\
  && && \\
\calX \times \Act \calX \ar[rrrr]_{{\rm eval}}
  && && \calX
}} 
\end{equation}

\medskip\noindent
Applying these maps to $s,s_0 \in S,\; r,r_0 \in R$ 
clockwise around the rectangle we have:
$$
\left( \begin{tabular}{c} $s_0$ \\ $r_0$ \\ \end{tabular} \right),
\left( \begin{array}{c} s \\ r \\ \end{array} \right)
\quad\mapsto\quad
\left( \begin{array}{c} s_0 \\ r_0 \\ \end{array} \right),
\left( \begin{array}{c} \ddm s \\ \dm r \\ \end{array} \right)
\quad\mapsto\quad
\left( \begin{array}{c} s_0 \\ r_0 \\ \end{array} \right),
\left( \begin{array}{c} \dda\ddm s \\ \da\dm r \\ \end{array} \right)
\quad\mapsto\quad
\left( \begin{array}{c} \chi_{\ddm s} r_0 \\ 
                        {s_0}^{\dm r} \\
                        {r_0}^{\dm r} \\ \end{array} \right).
$$

\noindent
In the other direction,
$$
\left( \begin{tabular}{c} $s_0$ \\ $r_0$ \\ \end{tabular} \right),
\left( \begin{array}{c} s \\ r \\ \end{array} \right)
\quad\mapsto\quad
\left( \begin{array}{c} s_0 \\ r_0 \\ \end{array} \right),
\left( \begin{array}{c} \eta_{s} \\ \beta_{r} \\ \end{array} \right)
\quad\mapsto\quad
\left( \begin{array}{c} \eta_s r_0 \\ 
                        {s_0}^r \\
                        {r_0}^r \\ \end{array} \right). 
$$

\noindent
Thus we obtain
\begin{eqnarray}
\chi_{\ddm s} r_0 & = & (s^{-1})^{r_0}s~, \nonumber\\
    {s_0}^{\dm r} & = & {s_0}^r~, \\
    {r_0}^{\dm r} & = & r^{-1} r_0 r~. \nonumber
\end{eqnarray}



\newpage
%%%%%%%%%%%%%%%%%%%%%%%%%%%%%%%%%%%%%%%%%%%%%%%
\subsection{Crossed Modules of Crossed Modules} 

Using the generalisation of the crossed module axioms given 
in the previous section we obtain the following definition.

\begin{defn} \index{crossed module!of crossed modules} 
\index{$2$-fold crossed module}
A \emph{crossed module of crossed modules} 
or \emph{$2$-fold crossed module}  has the form 
$\calM = (\mu : \calX \to \calY)$
where  $\calX = (\partial : S \to R)$,~ $\calY = (\delta : Q \to P)$,
and $\mu = (\ddm,\dm) : \calX \to \calY$
is a morphism of crossed modules, 
together with an action of  $\calY$  on  $\calX$,  
as in Subsection \ref{subs:xmod-action}, satisfying:

\begin{center}
\begin{tabular}{rccc}
{\bf CC1:}\quad  &
 $\ddm({s}^p) \,=\, (\ddm s)^p\,$, &
  $\dm({r}^p) \,=\, p^{-1}(\dm r)p\,$,  &
  $\ddm \circ \chi_q \,=\, \eta_q\circ \dm\,$, \\
 & & \\
{\bf CC2:}\quad  &
  ${s_0}^{\dm r} \,=\, {s_0}^r\,$, &
  ${r_0}^{\dm r} \,=\, r^{-1}\,r_0\,r\,$, &
  $\chi_{\ddm s} \,=\, \eta_s\,$. \\
 & & \\
\end{tabular}
\end{center}
\end{defn}
\begin{equation} \label{eq:xxsq}
\vcenter{\xymatrix{ 
     &&  && \\
  S \ar[rr]^(0.4){\ddi} \ar[dd]_{\partial}
    \ar `u[-1,1] `[0,4]_{\ddm} [0,4]
     && W(\calX) \ar[dd]^{\Delta}
     && Q \ar[ll]_(0.4){\dda} \ar[dd]^{\delta}  \\
     &&  &&  \\
  R \ar[rr]_(0.4){\di}
    \ar `d[1,1] `[0,4]^{\dm} [0,4]
     && A(\calX) 
     && P \ar[ll]^(0.4){\da} \\
     &&  &&
}} 
\end{equation}

\bigskip\noindent
Axioms \textbf{CC1:} and \textbf{CC2:} are sufficient 
to make the diagram commute.

\begin{lem}
$$
\alpha\mu ~=~ \iota\,.
$$
\end{lem}
\begin{pf}
$$
\dda\ddm s \;=\;
\chi_{\ddm s} \;=\; \eta_s \;=\; \ddi s
\qquad\mbox{and}\qquad  
\da\dm r \;=\;
\beta_{\dm r} \;=\; \beta_r \;=\; \di r~.
$$
\end{pf}

\bigskip\noindent
In order to show that $\calX$ and $\calY$ form a crossed square 
we have to define a crossed pairing, 
and there is one obvious choice.


\begin{prop} \label{prop:xxmod-is-xsq}
The $2$-fold crossed module $\calM$ is a crossed square with crossed pairing
$$
\bt ~:~ R \times Q \to S, \quad 
(r,q) \mapsto r \bt q \,=\, (\dda q)r \,=\, \chi_q r.
$$
\end{prop}

\medskip\noindent
\begin{pf}
Axiom (a) for crossed squares requires  $\ddm(s^p) = (\ddm s)^p$ 
and  $\partial(s^p) = (\partial s)^p$.
The first of these is given by \textbf{CC1:} 
while $(\partial s)^p = \db_p\partial s = \partial\ddb_p s = \partial(s^p)$.

\medskip\noindent
For (b) we require that  $(\ddm : S \to Q)$ and $(\dm : R \to P)$ 
are crossed modules:
$$
\ddm(s^q) = q^{-1}(\ddm s)q \quad\mbox{by (\ref{eq:mu-of-q-on-s})\quad and}
\quad {s_0}^{\ddm s} = {s_0}^{\delta\ddm s} = {s_0}^{\dm \partial s}
                     = {s_0}^{\partial s} = s^{-1}s_0s~,
$$
$$
\dm(r^p) = p^{-1}(\dm r)p \quad\mbox{by \textbf{CC1:}\quad and}
\quad {r_0}^{\dm r} = r^{-1}r_0r \quad \mbox{by \textbf{CC2:}}~.
$$

\medskip\noindent
We then verify that ~$\bt$~ is a crossed pairing.
\begin{enumerate}[(i)]
\item
$\quad
(r_1r_2) \bt q  ~~=~~ (\dda q)(r_1r_2) ~~=~~
((\dda q)(r_1))^{r_2}\,((\dda q)(r_2)) ~~=~~ (r_1 \bt q)^{r_2}\,(r_2 \bt q) $
\item
$\quad
r \bt (q_1q_2)  ~~=~~
\dda(q_1q_2)(r) ~~=~~ (\dda q_1 \star \dda q_2)(r) ~~=~~
(\chi_{q_2}r)(\chi_{q_1}r)( \chi_{q_2} \partial \chi_{q_1} r) $

\hspace*{23mm}
$=~~ (\chi_{q_2} r)\,(\chi_{q_1} r)^{q_2} 
\quad\mbox{(using Lemma \ref{lem:ids5} (b))}
~~=~~ (r \bt q_2)\,(r \bt q_1)^{q_2}\,.$
\item
$\quad
r^{p} \bt q^{p}
  ~ = ~  (\dda q^{p})\,(r^{p})
  ~ = ~  (\dda q)^{\da p}\,(r^{p})
  ~ = ~  (\chi_q)^{\beta_p}\,(r^{p}) 
  ~ = ~  \ddb_{p} \chi_q {\db_{p}}^{-1} (r^{p})
  ~ = ~  \ddb_{p} \chi_q r
  ~ = ~  (r \bt q)^{p}\,.$
\end{enumerate}

\medskip\noindent
To verify axiom (d) for a crossed square, we check:
$$
\ddm(r \bt q) = \ddm(\chi_q r) = (q^{-1})^{\dm r}q = (q^{-1})^rq \;,
$$
$$
r \partial(r \bt q) = r(\partial\chi_q r) = \db_{\chi_q}r = \Delta \chi_q r
= (\Delta \dda q)r = (\da \delta q)r  
= \db_{\delta q}r = r^{\delta q} = r^q \;.
$$

\medskip\noindent
To verify axiom (e) we check the following: 
$$
(r \bt \ddm s) = \chi_{\ddm s}r = \eta_s r = (s^{-1})^r s \;,
$$
$$
s(\partial s \bt q) = s(\chi_q \partial s) = \ddb_{\chi_q}s = \Delta \chi_q s
= (\Delta \dda q)s = (\da \delta q)s 
= \ddb_{\delta q}s = s^{\delta q} = s^q\;.
$$
\end{pf}

\medskip
\begin{defn} \index{morphism!of $2$-fold crossed modules}
A \emph{morphism of crossed module of crossed modules}
$$
\theta ~~:~~ \calM_1 = (\mu_1 : \calX_1 \to \calY_1) 
       ~\to~ \calM_2 = (\mu_2 : \calX_2 \to \calY_2)
$$
is a pair of crossed module morphisms
$$
\ddth : \calX_1 \to \calX_2, \qquad 
 \dth : \calY_1 \to \calY_2,
$$
plus some extra condition(s) which say that the action is preserved.

\noindent
{\bf [Complete this definition.]}
\end{defn}

\bigskip
%%%%%%%%%%%%%%%%%%%%%%%%%%%%%%%%%%%%%%%%%%%%%%%%%%%%%%%%%%%%
\subsection{A $2$-fold crossed module from a crossed square}

In order to show that the category {\sf XXMod} of $2$-fold 
crossed modules and their morphisms is equivalent to the
category {\sf XSq} of crossed squares and their morphisms 
we require a construction in the reverse direction.

\begin{thm}  \label{thm:xxmod-from-square}
Let  $\calR$  be a crossed square with crossed pairing
\;$\bt : R_{\{1\}} \times R_{\{2\}} \to R_{[2]}$, 
$$
\xymatrix{
\calR \quad = 
  & R_{[2]} \ar[rr]^{\ddbdyo} \ar[dd]_{\ddbdyt}
     && R_{\{2\}} \ar[dd]^{\dbdyt}  \\
  &  &&  \\
  & R_{\{1\}} \ar[rr]_{\dbdyo}
     && R_{\emptyset}
}
$$
so that
$\ddcalRt = (\ddbdyt: R_{[2]} \to R_{\{1\}})$ and 
$\dcalRt = (\dbdyt:R_{\{2\}} \to R_{\emptyset})$  are crossed modules, 
$(\ddbdyo, \dbdyo): \ddcalRt \to \dcalRt$  is a crossed module morphism, 
and  $R_{\{1\}}$ and $R_{\{2\}}$ act on $R_{[2]}$ via $R_{\emptyset}$. 
Then there is an action of  $\dcalRt$ on $\ddcalRt$ which makes 
$(\partial_1 : \ddcalRt \to \dcalRt)$ a crossed module of crossed modules.
\end{thm}
\begin{pf}
We first demonstrate an action of $\dcalRt$ on $\ddcalRt$ by showing 
that there exists a morphism of crossed modules 
$\alpha \,:\, \dcalRt \to \calA(\ddcalRt)$ as in the following diagram:

\begin{equation}  \label{eq:xxmod-from-square-diag}
\vcenter{\xy
\xymatrix{ 
  &&&&&& \\
  &      R_{[2]} \ar[rr]^(0.4){\ddi} \ar[dd] <1.0ex>^{\ddbdyt}
           \ar `u[-1,1] `[0,4]_{\ddbdyo} [0,4]
           \ar `u[-1,-1] `[0,-1] `[0,0]_{\ddb_m, \ddb_p} [0,0]
   &&    W(\ddcalRt) \ar[dd]^{\Delta}
     &&  R_{\{2\}} \ar[ll]_(0.4){\dda} \ar[dd]^{\dbdyt} 
       & \\
  &&&&&& \\
  &      R_{\{1\}} \ar[rr]_(0.4){\di}  \ar[uu] <1.0ex>^{\chi_{\ell},\chi_n}
           \ar `d[1,1] `[0,4]^{\dbdyo} [0,4]
           \ar `d[1,-1] `[0,-1] `[0,0]^{\db_m, \db_p} [0,0]
   &&    A(\ddcalRt)
     &&  R_{\emptyset} \ar[ll]^(0.4){\da}
       & \\
  &&&&&& \\
  & \ddcalRt && \calA(\ddcalRt) && \dcalRt & \\
}
\endxy} 
\end{equation}

\medskip\noindent
Define 
$$
\da : R_{\emptyset} \to \calA(\ddcalRt), \quad
p \mapsto \beta_p : \ddcalRt \to \ddcalRt 
\quad\mbox{for all}\quad 
p \in R_{\emptyset}~,
$$
where  
$$
\ddb_p \ell = \ell^p \quad\mbox{for all}\quad  \ell \in R_{[2]}
\qquad\mbox{and}\qquad  
\db_p m = m^p \quad\mbox{for all}\quad  m \in R_{\{1\}}~.
$$ 
We see that $\da p$ is a crossed module morphism since $\ddb$ and $\db$ are
group homomorphisms and 
%$$
%\ddb_{p_1p_2} \ell  \;=\;
%\ell^{p_1p_2}  \;=\;
%(\ddb_{p_1} \ell)^{p_2}  \;=\;
%(\ddb_{p_2} \circ \ddb_{p_1})\,\ell \;=\;
%(\ddb_{p_1} * \ddb_{p_2})\,\ell ~,
%$$
%and similarly for the action on  $R_{\{2\}}$.
$$
(\ddb_p\ell)^{\db_pm} ~=~ (\ell^p)^{m^p} ~=~ \ell^{pp^{-1}mp} 
~=~ \ell^{mp} ~=~ \ddb_p(\ell^m)~.
$$

\vspace*{12mm}\noindent
The map  $\dda : R_{\{2\}} \to W(\ddcalRt)$  is defined by
\begin{equation} \label{eq:chi-n-m}
\dda n \;=\; \chi_n : R_{\{1\}} \to R_{[2]},\quad 
\chi_n m = m \bt n, \quad 
\mbox{for all} ~~ m \in R_{\{1\}}~.
\end{equation}
Each  $\chi_n$  is a derivation of  $\ddcalRo$   since
$$
\chi_n(m_1m_2) \;=\;
(m_1m_2 \bt n) \;=\;
(m_1 \bt n)^{m_2}\;(m_2 \bt n) \;=\;
(\chi_n m_1)^{m_2}\,(\chi_n m_2)~.
$$

\noindent
That  $\dda$  is a homomorphism is verified by 
\begin{eqnarray*}
(\chi_{n_1} \star \chi_{n_2})\,m
  & = &  (\chi_{n_2}m)\,(\chi_{n_1}m)\, \chi_{n_2} \ddbdyt (m \bt n_1) \\
  & = &  (m \bt n_2)\,(m \bt n_1)\,\chi_{n_2}(m^{-1}\,m^{n_1}) \\
  & = &  (m \bt n_2)\,(m \bt n_1)\,
          (\chi_{n_2}m^{-1})^{m^{n_1}}\,(\chi_{n_2}m^{n_1}) \\
  & = &  (m \bt n_2)\,(m \bt n_1)\,(m^{-1} \bt n_2)^{n_1^{-1}mn_1}\,
           (m^{n_1} \bt n_2) \\
  & = &  (m \bt n_2)\,(m \bt n_1)\,((m \bt n_2)^{[m,n_1]})^{-1}\, 
          (m \bt n_1)^{-1}\,(m \bt n_2)\,(m \bt n_1)^{n_2} \\
  & = &  (m \bt n_1)\,\{(m \bt n_1)^{-1}\,(m \bt n_2)\}\,(m \bt n_1)^{n_2} \\
  & = &  (m \bt n_2)\,(m \bt n_1)^{n_2}
  \;=\;  (m \bt n_1n_2)
  \;=\;  \chi_{n_1n_2}\,m ~.
\end{eqnarray*}

\noindent
To show that  $\alpha$  preserves actions, we calculate
\begin{eqnarray*}
\dda(n^p) \;=\; \chi_{n^p} 
  & : &  m \mapsto (m \bt n^p) ~, \\
(\dda n)^{\da p} 
  \;=\;  \beta_p^{-1} \ast \chi_n \ast \beta_p
  & : &  m \mapsto  \ddb_p \chi_n \db_p^{-1} m
  \;=\;  \ddb_p \chi_n(m^{p^{-1}})
  \;=\;  (m^{p^{-1}} \bt n)^p
  \;=\;  (m \bt n^p)~.
\end{eqnarray*}

\noindent
To show that the right-hand square in diagram (\ref{eq:xxsq}) commutes,
$\da \dbdyt \,=\, \Delta \dda\,$,
note that
$$
\Delta \dda \,n \;=\; \Delta \chi_n \;=\; (\ddb_{\chi_n},\db_{\chi_n})
$$
where
$$
\db_{\chi_n}m \;=\; m(\ddbdyt \chi_n m) \;=\; m (\ddbdyt (m \bt n))
\;=\; m(m^{-1}m^n) \;=\; m^n \;=\; \db_{\dbdy_2 n}m ~,
$$
and
$$
\ddb_{\chi_n}\ell \;=\; \ell(\chi_n \ddbdyt \ell)
\;=\; \ell(\ddbdyt \ell \bt n) \;=\; \ell^n \;=\; \ddb_{\dbdy_2 n}\ell ~.
$$
So  
$$
\Delta \dda n \;=\; (\ddb_{\dbdy_2 n}, \db_{\dbdy_2 n}) 
\;=\; \beta_{\dbdy_2 n} \;=\; \da \dbdy_2 n ~.
$$
This completes the proof that
$\alpha : \dcalRt \to \calA(\ddcalRt)$ is a morphism of crossed modules. 

\medskip\noindent
Thus the crossed square  $\calR$
gives rise to a semidirect crossed module
$\dcalRt \ltimes \ddcalRt$~.

\medskip
Of course we could obtain a second semidirect product crossed module 
$\dcalRo \ltimes \ddcalRo$  for the transpose of  $\calR$
by reversing the roles of $R_{\{1\}}$ and $R_{\{2\}}$
and using the crossed pairing  $\tilde{\bt}$~.

\bigskip
The morphism $\iota = (\di, \ddi)$ in diagram (\ref{eq:xxmod-from-square-diag})
is given (see Subsection \ref{subs:inner-morphism}) by:

$$
\di \,:\, R_{\{1\}} \to A, \quad
\di m = \beta_m = (\ddb_m,\db_m) : \ddcalRt \to \ddcalRt,
\quad\mbox{for all}\quad m \in R_{\{1\}}~,
$$
where\quad
$\ddb_m \ell = \ell^m$ \quad
for all\quad 
$\ell \in R_{[2]}$\quad
and\quad
$\db_m m_0 = {m_0}^m = m^{-1}m_0m$\quad
for all\quad 
$m_0 \in R_{\{1\}}$, 
and
$$
\ddi \,:\, R_{[2]} \to W, \quad 
\ell \,\mapsto\, \eta_{\ell} : R_{\{1\}} \to R_{[2]},\; 
m \mapsto (\ell^{-1})^m\,\ell \quad\mbox{for all}\quad m \in R_{\{1\}}~.
$$

\bigskip
The $2$-fold crossed module axioms for $(\partial_1 : \ddcalRt \to \dcalRt)$ 
are easily verified.
\begin{enumerate}[{\bf CC1:}]
\item
\begin{enumerate}[{\rm (i)}]
\item
$\ddbdyo(\ell^p) = (\ddbdyo \ell)^p$ by (a),
\item
$\dbdyo(m^p) = p^{-1}(\dbdyo m)p$ by \textbf{M1:} for $\dcalRo$,
\item
$\ddbdyo(\chi_n m) = \ddbdyo(m \bt n) = (n^{-1})^mn 
                  = (n^{-1})^{\dbdyo m}n = \eta_n(\dbdyo m)$~.
\end{enumerate}
\item
\begin{enumerate}[{\rm (i)}]
\item
$\ell^{\dbdyo m} = \ell^m$ by definition of the action,
\item
${m_0}^{\dbdyo m} = m^{-1}m_0m$ by \textbf{M2:} for $\dcalRo$,
\item
$\chi_{\ddbdyo\ell}m = m \bt \ddbdyo\ell = (\ell^{-1})^m\ell = \eta_{\ell}m$. 
\end{enumerate}
\end{enumerate}
\end{pf}

%%%%%%%%%%%%%%%%%%%%%%%%%%%%%%%%%%%%%%%%%%%%%%%%%%%%%%%%%%%%%%%%%%%%%%%%%%%%%
%%%%%%%%%%%%%%%%%%%%%%%%%%%%%%%%%%%%%%%%%%%%%%%%%%%%%%%%%%%%%%%%%%%%%%%%%%%%%
\newpage
%%%%%%%%%%%%%%%%%%%%%%%%%%%%%%%%%%%%%%%%%%%%%%%%%%
%\subsection{$2$-fold derivations and sections} 
%\label{subs:dder-ssec}
%% {\bf [(19-21/04/04) - big changes to this section]}

\subsection{Derivations of a $2$-fold crossed module} 
\label{subs:-der-xxmod}

The following definition (b) of an $\calR_1$-derivation \emph{appears}
to be correct, but we shall see that an extra axiom is appropriate when 
we consider the corresponding version of a section.
The missing information is an expansion for $\ddch(n^p)$.
A revised version will be given as Definition \ref{def:chi-b}.

\begin{defn} \label{def:chi-a}
Let ${\calR_1}$ be the usual crossed square $\calR$ of (\ref{eq:Ssquare}) 
considered as a crossed module of crossed modules  
$(\partial_1 = (\ddbdyo,\dbdyo) ~:~ \ddcalRt \to \dcalRt)$.
\begin{equation} \label{eq:Rsqder}
\vcenter{\xymatrix{
        &   &    R_{[2]} \ar[rr] <+0.8ex>^{\ddbdyo} 
                         \ar[dd]_{\ddbdyt}
             &&  R_{\{2\}} \ar[ll] <+0.8ex>^{\ddph,\,\ddch} 
                           \ar[dd]^{\dbdyt} \\
\calR_1 & = &&&  \\
        &   &    R_{\{1\}} \ar[rr] <-0.8ex>_{\dbdyo}
             &&  R_{\emptyset} \ar[ll] <-0.8ex>_{\dph,\,\dch}
}} 
\end{equation}
\begin{enumerate}[\rm (a)]
\item 
An $\calR_1$-map is a pair of maps 
$\phi = (\ddph, \dph) : \dcalRt \to \ddcalRt$ which commute with the two 
boundaries:
$$ 
\ddph \ast \ddbdyt = \dbdyt \ast \dph 
$$
\item 
An $\calR_1$-derivation is an $\calR$-map $\chi = (\ddch, \dch)$ 
such that $\ddch$ is a derivation of $\ddcalRo$ 
and $\dch$ is a derivation of $\dcalRo$.
\end{enumerate}

Similarly, let $\calR_2$ be $\calR$ considered as
$(\partial_2 = (\ddbdyt,\dbdyt) : \ddcalRo \to \dcalRo)$,
giving notions of $\calR_2$-map and $\calR_2$-derivation.
\end{defn}

\begin{lem}
To check that a pair of derivations 
$\chi = (\ddch,\dch)$ is an $\calR_1$-derivation
it is sufficient to check that $\ddch\ast\ddbdyt = \dbdyt\ast\dch$
on a generating set of $R_{\{2\}}$.
\end{lem}
\begin{pf}
If $\ddbdyt\ddch n_1 = \dch\dbdyt n_1$ 
and  $\ddbdyt\ddch n_2 = \dch\dbdyt n_2$ then
$$
\ddbdyt\ddch(n_1n_2)
= \ddbdyt((\ddch n_1)^{n_2}(\ddch n_2))
= (\ddbdyt\ddch n_1)^{\dbdyt n_2}(\ddbdyt\ddch n_2)
= (\dch\dbdyt n_1)^{n_2}(\dch\dbdyt n_2)
= \dch\dbdyt(n_1n_2).
$$ 
\end{pf}

\begin{lem}
The set of $\calR_1$-derivations has a Whitehead multiplication 
$$ 
\chi_1 \star  \chi_2 = (\ddch_1 \star  \ddch_2, \: \dch_1 \star  \dch_2).
$$
\end{lem}
\begin{pf}
Since $\ddch_1 \star \ddch_2$ and $\dch_1 \star \dch_2$ are derivations of 
$\ddcalRo$ and $\dcalRo$ respectively, 
we only have to show that $\chi_1 \star \chi_2$ is an $\calR_1$-map.
\begin{eqnarray*}
\ddbdyt (\ddch_1 \star \ddch_2)(n)  
  & = & (\ddbdyt \ddch_2 n) (\ddbdyt \ddch_1 n) 
         (\ddbdyt \ddch_2 \ddbdyo \ddch_1 n) \\
  & = & (\dch_2 \dbdyt n) (\dch_1 \dbdyt n) 
         (\dch_2 \dbdyt \ddbdyo \ddch_1 n) \\
  & = & (\dch_2 \dbdyt n) (\dch_1 \dbdyt n) 
         (\dch_2 \dbdyo \ddbdyt \ddch_1 n) \\
  & = & (\dch_2 \dbdyt n) (\dch_1 \dbdyt n) 
         (\dch_2 \dbdyo \dch_1 \dbdyt n) \\
  & = & (\dch_2 \star \dch_1)(\dbdyt n)~.
\end{eqnarray*}
\end{pf}

\bigskip\noindent
{\bf [Not sure if the following Lemma is true!]}
\vspace{3mm}

\begin{lem}
A pair of principal derivations $\eta = ( \dde_{\ell}, \de_m)$ 
is an $\calR_1$-derivation if and only if 
$(\ddbdyt\ell)m^{-1}$  is fixed by the action of  $\dbdyt R_{\{2\}}$.
\end{lem}
\begin{pf}
$$
\ddbdyt \dde_{\ell} n ~=~ \de_{m} \dbdyt n
~\Leftrightarrow~
(\ddbdyt \ell^{-1})^{\dbdyt n} (\ddbdyt \ell) ~=~ (m^{-1})^{\dbdyt n} m 
~\Leftrightarrow~
(\ddbdyt\ell)m^{-1} ~=~ ((\ddbdyt\ell)m^{-1})^{\dbdyt n}~.
$$
{\bf [But what about equation (\ref{eq:extra-axiom})?]}
\end{pf}

\bigskip
\begin{defn}  \index{Principal derivation!of $2$-fold crossed modules}
A \emph{principal $\calR_1$-derivation} is an $\calR_1$-derivation 
$\eta = ( \dde_{\ell}, \de_m)$ such that $\dde_{\ell}, \de_m$ 
are principal derivations.
\end{defn}

\medskip
Given an $\calR_1$-derivation $\chi = (\ddch,\dch)$ 
we have automorphisms
$\beta_{\ddch} = (\ddb_{\ddch},\db_{\ddch})$  of  $\ddcalRo$ 
and
$\beta_{\dch} = (\ddb_{\dch},\db_{\dch})$  of  $\dcalRo$ 
given by
$$ 
\ddb_{\ddch} \ell = \ell(\ddch \ddbdyo \ell), \qquad
    \db_{\ddch} n = n(\ddbdyo \ddch n),  \qquad
    \ddb_{\dch} m = m(\dch \dbdyo m),  \qquad
     \db_{\dch} p = p(\dbdyo \dch p), 
$$
such that
$$
\ddb_{\ddch} \ddch ~=~ \ddch \db_{\ddch},~ 
n \mapsto (\ddch n)(\ddch \ddbdyo \ddch n)
\qquad\mbox{and}\qquad
\ddb_{\dch} \dch ~=~ \dch \db_{\dch},~
p \mapsto (\dch p)(\dch \dbdyo \dch p).
$$

\medskip\noindent
These automorphisms will be important for our construction 
of the actor crossed square.

Since these maps are automorphisms, we know that
$$
\ddb_{\ddch} (\ell^n) ~=~ (\ddb_{\ddch} \ell)^{\db_{\ddch} n}
\qquad\mbox{and}\qquad
    \ddb_{\dch} (m^p) ~=~ (\ddb_{\dch} m)^{\db_{\dch} p}.
$$

\begin{lem}
These four group automorphisms combine to give automorphisms 
$$
\beta_{\ddch} = (\ddb_{\ddch}, \ddb_{\dch}) ~\mbox{of}~ \ddcalRt
\qquad\mbox{and}\qquad
 \beta_{\dch} = (\db_{\ddch}, \db_{\dch}) ~\mbox{of}~ \dcalRt.
$$
\end{lem}
\begin{pf}
We first check commutation:
$$
\ddbdyt \ddb_{\ddch} \ell 
  ~=~ (\ddbdyt \ell)(\ddbdyt \ddch \ddbdyo \ell) 
  ~=~ (\ddbdyt \ell)(\dch \dbdyt \ddbdyo \ell) 
  ~=~ (\ddbdyt \ell) (\dch \dbdyo(\ddbdyt \ell)) 
  ~=~ \ddb_{\dch} \ddbdyt \ell,\quad
$$
$$
\dbdyt \db_{\ddch} n 
  ~=~ (\dbdyt n)(\dbdyt \ddbdyo \ddch n) 
  ~=~ (\dbdyt n) (\dbdyo \ddbdyt \ddch n) 
  ~=~ (\dbdyt n)(\dbdyo \dch (\dbdyt n))  
  ~=~ \db_{\dch} \dbdyt n.
$$

We now require to prove
$$
\mbox{(a)}~~
\ddb_{\ddch}(\ell^m) ~=~ (\ddb_{\ddch} \ell)^{\ddb_{\dch} m}
\qquad\mbox{and}\qquad 
\mbox{(b)}~~
    \db_{\ddch}(n^p) ~=~ (\db_{\ddch} n)^{\db_{\dch} p}.
$$
We prove (b) as follows.
\begin{eqnarray*}
\db_{\ddch}(n^p)
 &=&  n^p (\ddbdyo \ddch (n^p)) \\
 &=&  n^p~\ddbdyo[(\dch p \bt n^p)^{-1} (\ddch n)^{p(\dch p)}] \\
 &=&  n^p~(n^p)^{-1}~(n^p)^{\dch p}~(\ddbdyo \ddch n)^{p(\dbdyo \dch p)} \\
 &=&  (n(\ddbdyo \ddch n))^{p(\dbdyo \dch p)} \\
 &=&  (\db_{\ddch} n)^{\db_{\dch} p}.
\end{eqnarray*}

\medskip\noindent
{\bf [We still need to check (a) !~]}
\end{pf}

\bigskip
\begin{lem}
The endomorphisms $\ddb_{\ddch}, \db_{\ddch}, \ddb_{\dch}, \db_{\dch}$ 
commute with 
$\ddch \ddbdyo, \ddbdyo \ddch, \dch \dbdyo, \dbdyo \dch$ respectively.
\end{lem}
\begin{pf}
This follows immediately from Lemma \ref{lem:gamma-beta-chi}(d).
\end{pf}

\begin{lem}
Given a composite $\calR_1$-derivation $\chi = \chi_1 * \chi_2$, 
$$
\ddb_{\ddch} ~=~ \ddb_{\ddch_1} * \ddb_{\ddch_2} 
\qquad \mbox{and} \qquad
 \db_{\dch}  ~=~ \db_{\dch_1} * \db_{\dch_2}.
$$
\end{lem}
\begin{pf}
This is immediate from Corollary \ref{lem:wgamma-to-w}. \\
{\bf [Result on derivations changed to $gamma$-derivations 
-- what happens here?]}
\end{pf}



\newpage
%%%%%%%%%%%%%%%%%%%%%%%%%%%%%%%%%%%%%%%%%%%%%%%%%%%%%%%%%%%
\subsection{Sections of a crossed module of cat$^1$-groups}

Replacing the crossed modules $\ddcalRo, \dcalRo$  by the corresponding
cat$^1$-groups, we obtain the following 
\emph{crossed module of cat$^1$-groups},
(equivalently a \emph{cat$^1$-group of crossed modules} 
as in Section \ref{sect:cat2-xsq}), where
$\barbdyt : 
 R_{\{2\}} \ltimes R_{[2]} \to R_{\emptyset} \ltimes R_{\{1\}}, \: 
 (n, \ell) \mapsto (\dbdyt n, \ddbdyt \ell)$.

\begin{equation} 
\vcenter{\xymatrix{
 & R_{\{2\}} \ltimes R_{[2]} \ar[dd]_{\barbdyt}
     \ar[rr] <+1.2ex>  \ar[rr] <+2.0ex> ^(0.55){\ddtto,\ddhho}
     &&  R_{\{2\}}   \ar[dd]^{\dbdyt}  
             \ar[ll] \ar[ll] <+0.8ex> ^(0.45){\ddeeo,\ddx} \\
\calC_1 ~~=
 &   &&   \\
 & R_{\emptyset} \ltimes R_{\{1\}}  
     \ar[rr] <-1.2ex>  \ar[rr] <-2.0ex> _(0.55){\dtto,\dhho}
     &&  R_{\emptyset} \ar[ll]  \ar[ll] <-0.8ex> _(0.45){\deeo,\dx} 
 \\
}}
\end{equation}

\begin{lem}
The pair $(\barbdyt,\dbdyt)$ is a morphism of cat$^1$-groups.
\end{lem}
\begin{pf}
We first show that $\barbdyt$ is a group homomorphism.
\begin{eqnarray*}
\barbdyt ((n_1, \ell_1)(n_1,\ell_2))  
   & = & \barbdyt(n_1n_2, \ell_1^{n_2} \ell_2) \\  
   & = & (\dbdyt(n_1 n_2), \ddbdyt(\ell_1^{n_2} \ell_2)) \\
   & = & ((\dbdyt n_1)(\dbdyt n_2), 
          (\ddbdyt \ell_1)^{\dbdyt n_2} \ddbdyt \ell_2) \\
   & = & (\dbdyt n_1, \ddbdyt \ell_1)(\dbdyt n_2, \ddbdyt \ell_2) 
\end{eqnarray*}
Next we check that $(\barbdyt,\dbdyt)$ commutes with 
$t_1 = (\ddtto,\dtto),~ h_1 = (\ddhho,\dhho)$  and  $e_1 = (\ddeeo,\deeo)$.
$$
(\barbdyt * \dtto)(n,\ell)
 ~=~ \dtto(\dbdyt n, \ddbdyt\ell)
 ~=~ \dbdyt n
 ~=~ (\ddtto * \dbdyt)(n,\ell)
$$
$$
(\barbdyt * \dhho)(n,\ell)
 ~=~ \dhho(\dbdyt n, \ddbdyt\ell)
 ~=~ (\dbdyt n)(\dbdyo\ddbdyt\ell)
 ~=~ (\dbdyt n)(\dbdyt\ddbdyo\ell)
 ~=~ \dbdyt(n(\ddbdyo\ell))
 ~=~ (\ddhho * \dbdyt)(n,\ell)
$$
$$
(\deeo * \barbdyt)n
 ~=~ \barbdyt(n,1)
 ~=~ (\dbdyt n, 1)
 ~=~ \deeo(\dbdyt n)
 ~=~ (\dbdyt * \deeo) n
$$
\end{pf}

\vspace*{-8mm}
\begin{defn} \mbox{}\\
\begin{enumerate}[{\rm (a)}]
\item
\vspace*{-6mm}
A \emph{$\calC_1$-map} is a pair of maps 
$\phi = (\ddph,\dph) : \dcalRt \to \dcalRt \ltimes \ddcalRt$ 
which commute with the two boundaries.
\item
A \emph{$\calC_1$-section} is a pair  $\xi = (\ddx,\dx)$  such that
\begin{itemize}
\item
$\xi$ is a $\calC_1$-map:~
$\barbdyt \circ \ddx ~=~ \dx \circ \dbdyt$, 
\item
$\ddx$ is a section of $\ddcalRo$ and $\dx$ is a section of $\dcalRo$,
\item
$t \circ \xi = 1_{\dcalRt}$,
\item
$\xi$ is a crossed module morphism:~ $\ddx(n^p) = (\ddx n)^{\dx p}$.
\end{itemize}
\end{enumerate}
\end{defn}

\begin{lem}
If $\xi$ is an $\calC_1$-section, then there is an $\calR_1$-derivation $\chi$
such that $\dx p = (p, \dch p)$  and  $\ddx n = (n, \ddch n)$.
\end{lem}
\begin{pf}
The sections $\ddx, \dx$ determine derivations 
$\ddch : R_{\{2\}} \to R_{\emptyset}$ and 
$\dch : R_{\emptyset} \to R_{\{1\}}$, so we have to verify
that $\chi = (\ddch,\dch)$ is an $\calR_1$-map.
Since
$$
(\ddx * \barbdyt) n 
    ~=~  \barbdyt(n, \ddch n) 
    ~=~ (\dbdyt n, \ddbdyt \ddch n) 
\qquad\mbox{and}\qquad
(\dbdyt * \dx) n
    ~=~  \dx(\dbdyt n)
    ~=~(\dbdyt n, \dch \dbdyt n),
$$
commuting sections imply commuting derivations.
\end{pf}

\begin{defn} \index{Whitehead group}
A Whitehead multiplication for $\calR_1$-sections is defined as follows 
$$
\xi_1 \star \xi_2 ~=~ ( \ddx_1 \star \ddx_2,\, \dx_1 \star \dx_2). 
$$ 
\end{defn}


\medskip
Let us investigate whether the requirement that $\xi$ is a morphism 
of crossed modules gives further information about $\chi$.
Since
\begin{eqnarray*}
\ddx(n^p)
& = &  (n^p,\ddch(n^p))~, \\
(\ddx n)^{\dx p}
& = &  (n, \ddch n)^{(p, \dch p)} 
~ = ~  (n^p,~(\dch p \bt n^p)^{-1}~(\ddch n)^{p(\dch p)})~,
\end{eqnarray*}
we obtain the additional identity
\begin{equation} \label{eq:extra-axiom}
\ddch(n^p) ~=~ (\dch p \bt n^p)^{-1}~(\ddch n)^{p(\dch p)})~.
\end{equation}

We can now provide the modified version of Definition \ref{def:chi-a}
as promised earlier.

\begin{defn} \label{def:chi-b}
An $\calR_1$-derivation is an $\calR$-map $\chi = (\ddch, \dch)$ 
such that $\ddch$ is a derivation of $\ddcalRo$, 
$\dch$ is a derivation of $\dcalRo$,
and equation (\ref{eq:extra-axiom}) is satisfied.
\end{defn}

\bigskip
{\bf The hope is that this extra information (\ref{eq:extra-axiom}) 
will prove useful in simplifying formulae to found later.
On the other hand, the right-hand side is rather a messy expression,
so perhaps we should stick to sections?

\medskip
Here is a tricky task: prove that a principal derivation
$\eta = (\dde_{\ell},\de_m)$ satisfies (\ref{eq:extra-axiom}).\\
Perhaps it is easier to do for principal sections?}


\vspace*{15mm}
%%%%%%%%%%%%%%%%%%%%%%%%%%%%%%%%%%%%%%%%%%%%%%%%%%%%%%%%%%%%%%%%%%%%%%%%%%%%
\subsection{$2$-fold derivations of a crossed square} \label{subs:xsq-der}
\index{derivation!of a crossed square}

The maps $\theta : R_{\emptyset} \to R_{[2]}$ which we shall use to form
the group of $2$-derivations can be considered as the final part of a map
$$
C_{\emptyset} \to C_{[2]},\quad
p \mapsto (~(p, \chi p),~(\phi p, \theta p)~)\,,
$$
where $C_{\emptyset} = R_{\emptyset}$ and $C_{[2]} = 
(R_{\emptyset} \ltimes R_{\{1\}}) \ltimes (R_{\{2\}} \ltimes R_{[2]})$
as in diagram (\ref{eq:cat2-sdp}).

Applying the formula for semidirect product multiplication 
(Proposition \ref{eq:xmod-of-cat1s}(b) ???) 
and the semidirect product action (\ref{eq:action2}), we obtain
\begin{eqnarray*}
  &   & ((p, \chi p),(\phi p, \theta p))((q, \chi q),(\phi q, \theta q)) \\
  & = & ((p, \chi p)(q, \chi q),
         (\phi p, \theta p)^{(q, \chi q)}(\phi q,\theta q)) \\
  & = & ((pq, (\chi p)^q \chi q),
         ((\phi p)^q, (\chi q \boxtimes (\phi p)^q)^{-1} 
                      (\theta p)^{q (\chi q)}) (\phi q, \theta q)) \\
  & = & ((pq, (\chi p)^q \chi q),
         ((\phi p)^q, ((\phi p)^q \boxtimes \chi q) 
                      (\theta p)^{q (\chi q)}) (\phi q, \theta q)) \\
  & = & ((pq, (\chi p)^q \chi q),
         ((\phi p)^q \phi q, ((\phi p)^q \boxtimes \chi q)^{\phi q} 
                      (\theta p)^{[q (\chi q)(\phi q)]} (\theta q))) \\
\end{eqnarray*}



%%%%%%%%%%%%%%%%%%%%%%%%%%%%%%%%%%%%%%%%%%%%%%%%%%%%%%%%%%%%%%%%%%%%%%%%%%%%%
\subsection{Sections of a cat$^2$-group}
\index{section!of a cat$2$-group}

Recall from Section \ref{subs:cattwo} that a cat$^2$-group  $\calC$  
comprises $4$ groups and $15$ homomorphisms,
as shown in the following diagram,
$$
\xymatrix{
 & C_{[2]} \ar[ddd] <-1.2ex>  \ar[ddd] <-2.0ex>_{\ddttt,\ddhht}
     \ar[rrr] <+1.2ex>  \ar[rrr] <+2.0ex>^{\ddtto,\ddhho}
     \ar[dddrrr] <-0.2ex>  \ar[dddrrr] <-1.0ex>_(0.55){\ttot,\hhot}
    &&&  C_{\{2\}}  \ar[lll]^{\ddeeo, \ddxo}
            \ar[ddd]<+1.2ex>  \ar[ddd] <+2.0ex>^{\dttt,\dhht}  \\
\calC \quad = \quad
 &  &&&   \\
 &  &&&   \\
 & C_{\{1\}} \ar[uuu]_{\ddeet, \ddxt}
     \ar[rrr] <-1.2ex>  \ar[rrr] <-2.0ex>_{\dtto,\dhho} 
    &&&  C_{\emptyset} \ar[uuu]^{\deet, \dxt}   \ar[lll]_{\deeo, \dxo} 
           \ar[uuulll] <-1.0ex>_(0.55){\eeot, \xot}
 \\
}
$$
where the four sides of the square are all cat$^1$-groups,
and
\begin{equation} \label{eq:cat-commute}
\left\{ \begin{array}{l}
 \dtto\ddhht = \dhht\ddtto, \quad
 \dttt\ddhho = \dhho\ddttt, \quad
 \deeo\dttt = \ddttt\ddeeo, \quad
 \deet\dtto = \ddtto\ddeet, \quad
 \deeo\dhht = \ddhht\ddeeo, \quad
 \deet\dhho = \ddhho\ddeet,  \\
 \dtto\ddttt = \dttt\ddtto = \ttot, \quad 
 \dhho\ddhht = \dhht\ddhho = \hhot, \quad
 \deeo\ddeet = \deet\ddeeo = \eeot, \\
\end{array} \right.
\end{equation}
while the diagonal is only a pre-cat$^1$-group.

\medskip
We name the four cat$^1$-groups as 
$\ddcalCo$, $\dcalCo$, $\ddcalCt$, $\dcalCt$ 
and the diagonal as $\calCot$, 
so that 
$t_1 = (\ddtto,\dtto),~ h_1 = (\ddhho,\dhho) : \ddcalCo \to \dcalCo$ 
and
$e_1 = (\ddeeo,\deeo) : \dcalCo \to \ddcalCo$ 
are horizontal morphisms of cat$^1$-groups, while 
$t_2 = (\ddttt,\dttt),~ h_2 = (\ddhht,\dhht) : \ddcalCt \to \dcalCt$ 
and
$e_2 = (\ddeet,\deet) : \dcalCt \to \ddcalCt$ 
are vertical morphisms.
Pairs such as $(\ddtto,\id_{C_{\emptyset}})$ are pre-cat$^1$-morphisms.

\begin{lem} \label{lem:xio-is-sect}
For $\ddxo$ a section of $\ddcalCo$, the map $\dxo = \ddttt\ddxo\deet$ 
is a section of $\dcalCo$. 
\end{lem}
\begin{pf}
$$
\dtto\dxo ~=~ \dtto\ddttt\ddxo\deet ~=~ \dttt\ddtto\ddxo\deet 
~=~ \dttt\deet ~=~ \id_{C_{\emptyset}}.
$$
\end{pf}

We now consider $\calC_1 = (e_1;t_1,h_1 : \ddcalCo \to \dcalCo)$ 
as a \emph{cat$^1$-group of cat$^1$-groups} and define the
notion of section in this situation.

\begin{defn}
A section of $\calC_1$ is a pair of maps $\xi_1 = (\ddxo,\dxo)$ 
where $\ddxo$ is a section of $\ddcalCo$ and
\begin{equation} \label{eq:two-section-axioms}
%\dxo = \ddttt\ddxo\deet, \quad 
\ddttt\ddxo = \dxo\dttt, \quad 
\ddhht\ddxo = \dxo\dhht  \quad\mbox{and}\quad
\ddeet\dxo = \ddxo\deet.
\end{equation}
\end{defn}

\noindent
By Lemma \ref{lem:xio-is-sect}, $\xi_1$ a morphism of cat$^1$-groups.  
Since both $\ddxo$ and $\dxo$ are sections, 
the usual formula {\bf S1:} $t_1\xi_1 = \id_{\dcalCt}$ holds.

\bigskip
Recall that the Whitehead multiplication for sections of a 
cat$^1$-group applied to two sections of $\ddcalCo$ gives 
$$
(\ddxo \star {\ddzo})k
 ~=~ (\ddxo k)(\ddeeo\ddhho\ddxo k^{-1})({\ddzo}\ddhho\ddxo k)
 ~=~ ({\ddzo}\ddhho\ddxo k)(\ddeeo\ddhho\ddxo k^{-1})(\ddxo k)~.
$$


\begin{lem}
The section of $\dcalCo$ associated to $\ddxo \star {\ddzo}$ 
is $\dxo \star {\dzo}$.
\end{lem}
\begin{pf}
The associated section is 
$\ddttt(\ddxo \star {\ddzo})\deet$ which maps $p \in C_{\emptyset}$ to
\begin{eqnarray*}
(\ddttt\ddxo\deet p)~(\ddttt\ddeeo\ddhho\ddxo\deet p^{-1})~
(\ddttt{\ddzo}\ddhho\ddxo\deet p) 
& = & 
(\dxo\dttt\deet p)~(\deeo\dhho\dxo\dttt\deet p^{-1})~
({\dzo}\dhho\dxo\dttt\deet p) \\
& = & 
(\dxo p)~(\deeo\dhho\dxo p^{-1})~({\dzo}\dhho\dxo p) 
~ = ~
(\dxo \star {\dzo})p~,
\end{eqnarray*}
where identities (\ref{eq:cat-commute}) and (\ref{eq:two-section-axioms}) 
are frequently used.
\end{pf}

\bigskip
Recall that a section $\xi$ of a cat$^1$-group 
$\calC = (e;t,h : C \to R)$ determines a cat$^1$-group endomorphism 
$(\gamma,\rho)$ of $\calC$ where
\begin{eqnarray*}
  \rho : R \to R, && r \mapsto h \xi r, \\
\gamma : C \to C, && g \mapsto (eh \xi tg)(\xi tg^{-1})g(ehg^{-1})(\xi hg). \\
\end{eqnarray*}
The equivalent definitions for our $\dcalCo$ are
\begin{eqnarray*}
\rho_1 = (\ddro,\dro): \dcalCt \to \dcalCt, 
  & &  k \mapsto \ddhho \ddxo k, \quad
       p \mapsto \dhho \dxo p, \\
\gamma_1 = (\ddgo,\dgo) : \ddcalCt \to \ddcalCt, 
  & &  j \mapsto (\deeo\dhho\dxo\dtto j)(\dxo\dtto j^{-1})j
                 (\deeo\dhho j^{-1})(\dxo\dhho j), \\
  & &  g \mapsto (\ddeeo\ddhho\ddxo\ddtto g)(\ddxo\ddtto g^{-1})g
                 (\ddeeo\ddhho g^{-1})(\ddxo\ddhho g).
\end{eqnarray*}
It is easy to check that $\rho_1$ and $\gamma_1$ are endomorphisms 
of cat$^1$-groups.  For example,
$$
\dttt\ddro k ~=~
\dttt\ddhho\ddxo k ~=~
\dhho\ddttt\ddxo k ~=~
\dhho\dxo\dttt k ~=~
\dro\dttt k~.
$$

Of course we may also consider 
$\calC_2 = (e_2;t_2,h_2 : \ddcalCt \to \dcalCt)$ 
as a second cat$^1$-group of cat$^1$-groups, and define 
sections $\xi_2 = (\ddxt,\dxt) : \dcalCt \to \ddcalCt$.



%%%%%%%%%%%%%%%%%%%%%%%%%%%%%%%%%%%%%%%%%%%%%%%%
\subsection{More on sections of a cat$^2$-group}

A first attempt at defining a section of $\calC$ 
might be as a homomorphism $\xot : C_{\emptyset} \to C_{[2]}$ 
which composes with the tail maps in various ways to give identities
$$
t_{[2]}\xot ~=~ \id_{C_{\emptyset}}, \qquad
\ddtto\xot\dttt ~=~ \id_{C_{\{2\}}}, \qquad
\ddttt\xot\dtto ~=~ \id_{C_{\{1\}}}.
$$
This will not do because, when $\dtto = 0$ (say)
the last identity puts to many restrictions of $\xi_{[2]}$.

\noindent
Since $t_{[2]} = \dtto\ddttt = \dttt\ddtto$, the first identity gives 
$$
\dtto(\ddttt\xot)  ~=~ \id_{C_{\emptyset}}
\quad\mbox{and}\quad
\dttt(\ddtto\xot)  ~=~ \id_{C_{\emptyset}}
$$
so that $\ddttt\xot$ is a section of $\dcalCo$ 
and $\ddtto\xot$ is a section of $\dcalCt$.

\bigskip
An alternative approach is to consider commuting pairs of sections.

\begin{defn}
A \emph{cat$^2$-group section} of $\calC$ is a pair 
$\xi = (\xi_1,\xi_2)$ 
where $\xi_1 = (\ddxo,\dxo)$ is a section of $\calC_1$, 
and $\xi_2 = (\ddxt,\dxt)$ is a section of $\calC_2$, such that 
$$
\ddxt\dxo ~=~ \ddxo\dxt ~:~ C_{\emptyset} \to C_{[2]}.
$$
\end{defn}

We denote the common composite by $\xot : C_{\emptyset} \to C_{[2]}$ 
and, since
$$
\ttot\xot ~=~ \dtto\ddttt\ddxo\dxt ~=~ \dtto\dxo\dttt\dxt ~=~ 1,
$$
the map $\xot$ is a section of $\calCot$.

\bigskip
We expect Whitehead multiplication of cat$^2$-sections to be defined by 
$$
\xi\star\zeta ~=~ (\ddxo\star\ddzo)(\dxt\star\dzt)
~:~ C_{\emptyset} \to C_{[2]},
$$
but for this to make sense we must first show that a 
product of commuting sections commutes.

\begin{lem}
$$
(\ddxo\star\ddzo)(\dxt\star\dzt) ~=~ (\ddxt\star\ddzt)(\dxo\star\dzo).
$$
\end{lem}
\begin{pf}
\begin{eqnarray*}
&   &  (\ddxo\star\ddzo)(\dxt\star\dzt)p \\
& = &  (\ddxo\dxt p)(\ddxo\deet\dhht\dxt p^{-1})(\ddxo\dzt\dhht\dxt p)
       (\ddeeo\ddhho\ddxo\dzt\dhht\dxt p^{-1})
       (\ddeeo\ddhho\ddxo\deet\dhht\dxt p)(\ddeeo\ddhho\ddxo\dxt p^{-1})\\
&   &  \qquad(\ddzo\ddhho\ddxo\dxt p)(\ddzo\ddhho\ddxo\deet\dhht\dxt p^{-1})
       (\ddzo\ddhho\ddxo\dzt\dhht\dxt p)\\
& = &  (\ddxt\dxo p)(\ddeet\ddhht\ddxt\dxo p^{-1})(\ddxo\dzt\dhht\dxt p)
       (\ddeeo\ddhho\ddxo\dzt\dhht\dxt p^{-1})
       (\ddeeo\ddhho\ddeet\ddhht\ddxt\dxo p)(\ddeeo\ddhho\ddxt\dxo p^{-1})\\
&   &  \qquad(\ddzo\ddhho\ddxo\dxt p)(\ddzo\ddhho\ddeet\ddhht\ddxt\dxo p^{-1})
       (\ddzo\ddhho\ddxo\dzt\dhht\dxt p)\\
& = & ?????
\end{eqnarray*}

\noindent
This seems to be going nowhere fast!

\noindent
Perhaps the group-groupoid or double groupoid approaches are better 
(see Subsection 2.3 of the Notes)  -- and the next Section.
\end{pf}


\bigskip\noindent
{\bf What needs doing next?}
\begin{itemize}
\item
Prove that the definition of Whitehead multiplication of sections
makes sense.
\item
Investigate the corresponding $(\gamma,\rho)$ pairs.
\item
{\bf Eventually:} convert to crossed squares and see what
the corresponding formulae are for $2$-fold derivations.
\end{itemize}

%%%%%%%%%%%%%%%%%%%%%%%%%%%%%%%%%%%% Gareth's Diagram
%$$\xymatrix @C = 0.5pc @R = 3pc {
%& N_{1} \cap N_{2} \cap N_{3}
%  \ar[ld] \ar[rr] \ar'[d][dd]
%&& N_{2} \cap N_{3} \ar[dd] \\
%  N_{1} \cap N_{3} \ar[dd] \ar[rr]
%&& N_{3} \ar[dd] \ar[ur] \\
%& N_{1} \cap N_{2} \ar'[r][rr] \ar[ld]
%&& N_{2} \ar[ld] \\
%N_{1} \ar[rr] && G
%}$$
