%% 2gpd.tex,  version 01/05/17


%%%%%%%%%%%%%%%%%%%%%%%%%%%%%%%%%%%%%
\section{2-groupoid enrichments}

\noindent
{\bf How does a group act on a set?}

Let $G$ be a group (thought of as a $1$-object groupoid) 
and $X$ a set (with $0$ groupoid structures). 
Construct the ($1$-)groupoid $\Symm(X)$, 
whose arrows form the group of permutations of $X$.
An action of $G$ on $X$ is then determined by a groupoid homomorphism 
$\theta : G \to \Symm(X)$.

We now form a cat$^0$-group $X_G$ (which is just a group) from $G$, 
by throwing away the single object, 
leaving a set with extra structure (the group multiplication). 
We then require the subgroupoid of $\Symm(X_G)$ whose arrows 
are the permutations which preserve the multiplication. 
This subgroupoid is of course $\Aut(G)$. 
The inner morphism $\iota$, mapping $g \in G$ to conjugation by $g$, 
gives an action of $G$ on itself. 
The actor crossed module of $G$ is then $(\iota : G \to \Aut(G))$, 
which can be thought of as a cat$^1$-group, 
or group-groupoid, or $2$-groupoid.

\bigskip\noindent
{\bf How does a crossed module act on a groupoid?} 

Let $\calX$ be a crossed module, 
thought of as a $2$-groupoid $\calG$ with one object, 
and let $\Gamma$ be a groupoid. 
Construct the $2$-groupoid $\Symm(\Gamma)$, with one object $\bullet$, 
the automorphisms of $\Gamma$ as arrows, 
thought of as functors $\Gamma \to \Gamma$, 
and natural transformations between these functors as $2$-cells. 
An action of $\calX$ on $\Gamma$ is then determined by a 
$2$-groupoid homomorphism $\theta : \calG \to \Symm(\Gamma)$.

We now form a cat$^1$-group (group-groupoid) $\Gamma_{\calX}$ from $\calX$ 
by throwing away the single object, 
leaving a groupoid with extra structure (again group multiplication).
We then require the sub-$2$-groupoid $\Act(\calX)$ 
of $\Symm(\Gamma_{\calX})$ whoses arrows 
are the automorphims which preserve the cat$^1$-structure 
(and whose $2$-cells preserve ???). 
The inner morphism $\iota : \calX \to \Act(\calX)$ then has to be determined.  
The actor crossd square of $\calX$ is then 
$(\iota : \calX \to \Act(\calX))$, 
which can be thought of as a cat$^2$-group, 
or $3$-groupoid, or group-double groupoid (is this correct?), or whatever.

\bigskip\noindent 
{\bf How does a crossed square act on a double groupoid?} 

Let $\calS$ be a crossed square, thought of as a $3$-groupoid, 
and let $\calD$ be a double groupoid. 
Construct the $3$-groupoid $\Symm(\calD)$ 
(or should this be a $2$-double groupoid?)
with one object, whose arrows are automorphisms of $\calD$ (double functors), 
$2$-cells are homotopies, or double natural transformations (?), 
and $3$-cells are $2$-homotopies of some sort? 
An action of $\calS$ on $\calD$ is then determined by a 
$3$-groupoid homomorphism $\theta : \calS \to \Symm(\calD)$.

We now form a cat$^2$-group (group-double groupoid) 
$\calD_{\calS}$ from $\calS$ by throwing away the single object, 
leaving a double groupoid with extra structure (again group multiplication).
We then require the sub-$3$-groupoid $\Act(\calS)$ 
of $\Symm(\calD_{\calS})$ whoses arrows 
are the automorphims which preserve the cat$^2$-structure 
(and whose $2$-cells and $3$-cells preserve ???). 
The inner morphism $\iota : \calS \to \Act(\calS)$ then has to be determined.  
The actor crossd cube of $\calS$ is then 
$(\iota : \calS \to \Act(\calS))$, 
which can be thought of as a cat$^3$-group, 
or $4$-groupoid, or group-triple groupoid (is this also correct?) 
and, no doubt, lots of other gadgets in the crossed menagerie!


\newpage
%%%%%%%%%%%%%%%%%%%%%%%%%%%%%%%%%%%%%%%%%%%%%%%%%%%%%
\subsection{The automorphism 2-category of a 2-group}

This material is adapted from \cite{kamps:port}.

Let $\calG$ be the ``$2$-group associated to the crossed module 
$\calX = (\partial : S \to R)$'', 
as described in Subsection \ref{subs:twogps}, 
with $2$-cells

$$
\xy
\xymatrix{
  && && \\
  \bullet  \ar@/^5ex/[rrrr]^{r} 
           \ar@/_5ex/[rrrr]_{r(\partial s)} 
  && \Downarrow (r,s)
     && \bullet \\
  && && \\
}
\endxy
$$

\medskip\noindent
We now seek to describe $\calF = \calF(\calG,\calG)$, 
the automorphism $2$-category of $\calG$.

\medskip\noindent
{\large{\bf objects of $\calF$}}\\ 
These are strict $2$-functors $\beta,\gamma,\ldots : \calG \to \calG$, 
where we write $\beta = (\id_{\bullet},\db, \barb=(\db,\ddb))$, 
mapping $\bullet$ to itself, $r$ to $\db r$, 
and $(r,s)$ to $(\db r, \ddb s)$. 
Of course $\db,\ddb$ are automorphisms of $R,S$ respectively. 
(Perhaps generalise to endomorphisms.)
Since $\beta$ is a $2$-functor, tail and head maps are preserved, 
so $\partial\ddb = \db\partial$.

\medskip\noindent
{\large{\bf $1$-arrows of $\calF$}}\\
These have the form $\barch : \beta \to \gamma$, 
and are $2$-natural transformations which make the following assignments. 
To the object $\bullet$ we assign an element 
$q = q_{\beta}^{\gamma} = \barch(\bullet) \in R$.

\noindent
To $r \in R$ we assign a square $2$-cell
$$
%\vcenter{
{\xy
\xymatrix{
  \bullet \ar[rr]^{\db r} 
    & \ar @{=>} [dr]_{\barch r}
      & \bullet 
        & & \\
    & & & \equiv
          & \\
  \bullet \ar[uu]^q \ar[rr]_{\dg r} 
    & & \bullet \ar[uu]_q
        & & \\
}
\endxy}
{\xy
\xymatrix{
  && && \\
  \bullet  \ar@/^5ex/[rrrr]^{q(\db r)} 
           \ar@/_5ex/[rrrr]_{(\dg r)q} 
  && \Downarrow\, \barch r
     && \bullet \\
  && && \\
}
\endxy}
%}
$$
where 
$\barch r = \barch_{\beta}^{\gamma} r = (q(\db r),\ddch r) \in R \ltimes S$, 
say.
Since $\barch r$ has head $(\dg r)q$, we have
\begin{equation} \label{eq:partial-ddch-r} 
(\db r)(\partial \ddch r) ~=~ (\dg r)^q \quad\mbox{in}~R,
\end{equation}
so that $\partial \ddch r$ measures the amount by which the square 
does not commute.

\medskip\noindent
The squares $\barch r$, for $r \in R$, are required to satisfy three axioms.
\begin{enumerate}[(i)]
\item
For the identity in $R$ we have $\barch 1_R = \Downarrow(1,q)$, 
the vertical identity $2$-cell at $q$.
\item
Product in $R$ is preserved: $\barch(r_1 \sharpz r_2)$ corresponds to 
the vertical composite $\barch r_1 \star_0 \barch r_2$ 
shown in the following diagram.
$$
{\vcenter 
{\xy
\xymatrix{
  \bullet \ar[rr]^{\db r_1} 
    & \ar @{=>} [dr]_{\barch r_1} 
      & \bullet \ar[rr]^{\db r_2} 
        & \ar @{=>} [dr]_{\barch r_2}  
          & \bullet \\
    & & & & \\
  \bullet \ar[uu]^q \ar[rr]_{\dg r_1}  
    & & \bullet \ar[uu]_q \ar[rr]_{\dg r_2}  
        & & \bullet \ar[uu]_q \\
}
\endxy}
\quad=\quad
{\xy
\xymatrix @C=1.5pc{
    & & & & & \\
  \bullet  \ar@/^5ex/[rr]^{q(\db r_1)} 
           \ar@/_5ex/[rr]_{(\dg r_1)q} 
    & \Downarrow\, (q(\db r_1), \ddch r_1)
       & \bullet \ar[r]^{q^{-1}} 
          & \bullet \ar@/^5ex/[rr]^{q(\db r_2)} 
                    \ar@/_5ex/[rr]_{(\dg r_2)q} 
              & \Downarrow\, (q(\db r_2), \ddch r_2) 
                   & \bullet \\
    & & & & & \\
}
\endxy}
}$$
The composite of these is
$$
\xy
\xymatrix @C=1.8pc{
  && && \\
  \bullet  \ar@/^5ex/[rrrr]^{q(\db r_1)(\db r_2)} 
           \ar@/_5ex/[rrrr]_{(\dg r_1)(\dg r_2)q} 
  && \Downarrow\, (q(\db r_1)(\db r_2), (\ddch r_1)^{\db r_2}(\ddch r_2))
      && \bullet \\
  && && \\
}
\endxy
$$
This composite $2$-cell is required to be be equal to
$$
{\vcenter 
{\xy
\xymatrix{
  \bullet \ar[rr]^{\db(r_1r_2)} 
    &  \ar @{=>} [dr]_{\barch(r_1r_2)} 
      & \bullet \\
    & & \\
  \bullet \ar[uu]^q \ar[rr]_{\dg(r_1r_2)} 
    & & \bullet \ar[uu]_q \\
}
\endxy} 
\quad\equiv\quad 
{\xy
\xymatrix{
    && && \\
    \bullet \ar@/^5ex/[rrrr]^{q\db(r_1r_2)} 
            \ar@/_5ex/[rrrr]_{\dg(r_1r_2)q} 
    && \Downarrow\, (q\db(r_1r_2),\ddch(r_1r_2))
       && \bullet \\
    && && \\
}
\endxy}
}$$
We deduce that $\ddch$ is a $\beta$-derivation:
$$
\ddch(r_1r_2) ~=~ (\ddch r_1)^{\db r_2}\;(\ddch r_2).
$$

\item
If $(r,s) : r \to r(\partial s)$ then $(\barb(r,s),\barg(r,s))$ forms a 
vertical homotopy from $\barch r$ to $\barch(r(\partial s))$. 
This means that the following composite $2$-cells are equal. 
$$
{\vcenter 
{\xy
\xymatrix @C=1.6pc{ 
    & & & & \\
  \bullet \ar@/^5ex/[rrrr]^{\db r}
    & & & \ar@{=>} [dr]_{\barch r}
          & \bullet \\
    & & & & \\
  \bullet \ar[uu]^q \ar@/^5ex/[rrrr]^{\dg r}
          \ar@/_5ex/[rrrr]_{\dg(r(\partial s))}
    & & \Downarrow(\dg r,\ddg r)
        & & \bullet \ar[uu]_q \\
    & & & & \\
}
\endxy}
\qquad=\qquad
{\xy 
\xymatrix @C=1.6pc@R=0.9pc{ 
    & & & & \\ 
  \bullet \ar@/^5ex/[rrrr]^{\db r} \ar@/_5ex/[rrrr]_{\db(r(\partial s))}
    & & \Downarrow(\db r,\ddb r)
        & & \bullet \\
    & & & \ar@{=>} [ddr]_{\barch(r(\partial s))}
          & \\ 
    & & & & \\
    & & & & \\ 
  \bullet \ar[uuuu]^q \ar@/_5ex/[rrrr]_{\dg(r(\partial s))}
    & & & & \bullet \ar[uuuu]_q \\ 
    & & & & \\ 
}
\endxy}
}$$
It follows that
\begin{eqnarray*}
       (\barch r) \sharpo ((\dg r,\ddg s) \sharpz q)
 & = & (q \sharpz (\db r,\ddb s)) \sharpo \barch(r(\partial s)), \\
       (q(\db r),\ddch r) \sharpo ((\dg r)q,(\ddg s)^q)
 & = & (q(\db r),\ddb s) \sharpo (q\db(r(\partial s)),\ddch(r(\partial s))), \\
       (q(\db r), (\ddch r)(\ddg s)^q) 
 & = & (q(\db r),(\ddb s)(\ddch r)^{\db\partial s}(\ddch\partial s)), \\ 
       (\ddch r)(\ddg s)^q 
 & = & (\ddb s)(\ddch r)^{\partial\ddb s}(\ddch\partial s), \\
       (\ddch r)(\ddg s)^q 
 & = & (\ddch r)(\ddb s)(\ddch\partial s), 
\end{eqnarray*}
which gives the identity (compare with equation (\ref{eq:partial-ddch-r}))
\begin{equation} \label{eq:ddch-partial-r} 
(\ddg s)^q  ~=~  (\ddb s)(\ddch\partial s),~~ 
\end{equation}
which shows that $\ddch \partial s$ measures the amount by which 
$(\ddg s)^q$ differs from $\ddb s$.
\end{enumerate}

\medskip
The composite $\barch_{\beta}^{\delta}$ 
of $\barch_{\beta}^{\gamma} : \beta \to \gamma$ 
with $\barch_{\gamma}^{\delta} : \gamma  \to \delta$, 
is shown in the following diagram.
$$
{(\barch_{\beta}^{\gamma} r) \star_1 (\barch_{\gamma}^{\delta} r)}
{\quad=\quad}
{\xy
\xymatrix{
  \bullet \ar[rr]^{\db r} 
    & \ar @{=>} [dr]_{\barch_{\beta}^{\gamma} r}
      & \bullet \\
    & & \\
  \bullet \ar[uu]^{q_{\beta}^{\gamma}} 
          \ar[rr]^(0.3){\dg r} 
    & \ar @{=>} [dr]_{\barch_{\gamma}^{\delta} r} 
      & \bullet \ar[uu]_{q_{\beta}^{\gamma}} \\
    & & \\
  \bullet \ar[uu]^{q_{\gamma}^{\delta}} \ar[rr]_{\dd r} 
    & & \ar[uu]_{q_{\gamma}^{\delta}} \\
}
\endxy}
{\quad=\quad} 
{(q_{\gamma}^{\delta} \sharpz (\barch_{\beta}^{\gamma} r)) 
\sharpo
((\barch_{\gamma}^{\delta} r) \sharpz (q_{\beta}^{\gamma})),} 
$$
so that
$$
\barch_{\beta}^{\delta}(\bullet) \;=\; q_{\gamma}^{\delta}q_{\beta}^{\gamma}
\qquad\mbox{and}\qquad
\barch_{\beta}^{\delta}(r) \;=\; \Downarrow 
(q_{\gamma}^{\delta}q_{\beta}^{\gamma}(\db r), 
 (\chi_{\beta}^{\gamma}r)(\chi_{\gamma}^{\delta}r)^{q_{\beta}^{\gamma}}).
$$


\medskip
Suppose there exists $s \in S$ such that $\partial s = q$. 
Then, using the principal $\gamma$-derivation $\eta_s$, 
we may define the \emph{principal $2$-cell} 
$\bare_{\gamma}^{\gamma} r = (q(\dg r),\eta_s r)$ 
with tail $q(\dg r)$ and head 
$$
q(\dg r) \partial((s^{-1})^{\dg r}s)
 \;=\; q(\dg r)(q^{-1})^{\dg r}q 
 \;=\; (\dg r)q~.
$$


\bigskip\noindent
{\large{\bf $2$-arrows of $\calF$}}\\
To $\barch,\barch' \in \calF$

